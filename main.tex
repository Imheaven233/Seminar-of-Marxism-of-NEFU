\documentclass[a4paper,twoside,12pt,AutoFakeBold]{ctexart}
\newcommand\specialsectioning{\setcounter{secnumdepth}{-2}}
\specialsectioning
\usepackage[center]{titlesec}
\usepackage{indentfirst}
\setlength{\parindent}{2em}
\usepackage{fancyhdr}
\pagestyle{fancy}%fancy style
\usepackage{caption}
\fancyhf{}%清空页眉页脚
\fancyhead[LE,RO]{\thepage}%页码位置:偶数页居左,奇数页居右
\fancyfoot[RO,RE]{\textit{NEFU's seminar of Marxism}}% 设置页脚:在每页的右下脚以斜体显示书名
\usepackage{graphicx}
\setlength{\headheight}{15pt}%解决页眉warnings
\usepackage{amsmath}
\renewcommand{\headrulewidth}{0pt} % 页眉与正文之间的水平线粗细
\renewcommand{\footrulewidth}{0pt}
\usepackage{tcolorbox}%文本框宏包
\usepackage{changepage}%设置引用段落左右侧缩进

\usepackage{tabularx}
\usepackage{hyperref}%设置超链接
\usepackage{float}
\title{读书会记录}
\author{东北林业大学马克思主义研讨会\\
指导教师:李光玉~许婕\\
 吴云飞~编}
\date{2023年12月}
\usepackage{perpage}
\MakePerPage{footnote}
\begin{document}
\maketitle
\newpage



\tableofcontents%目录

\newpage

\section{序言}

\begin{adjustwidth}{2em}{2em}
\qquad\fangsong 
本书是东北林业大学马克思主义经典著作研究与讨论会的一个内部记录,诸多观点与看法可能会欠缺专业性,因而这是一个非教学性质的内容记录。本书全部内容均来自我们的研讨会中的发言与讨论以及后续编者所进行的思考,如有雷同,纯属巧合,本书最终解释权归本研讨会与本书编者\footnote{编者注:我的邮箱是1360540699@qq.com,如果您在阅读本书之后有相应的意见与建议,抑或是想进行一些学术方面的交流与探讨,请通过该邮箱与我联系。}所有。

本书采用\LaTeX{}编写,\TeX{}源代码暂由编者吴云飞托管至其本人的Github仓库之中,供各位成员获取:\url{https://github.com/Imheaven233/Seminar-of-Marxism-of-NEFU}。



\end{adjustwidth}




\newpage

\section{第一期:《哲学的贫困》选读(1)}

\subsection{学习提示}\label{sec:1}

《哲学的贫困》是马克思针对蒲鲁东的《贫困的哲学》一书而写的一部论战性著作,以法文写成于1847年上半年,并于同年7月在布鲁塞尔和巴黎出版。

该著作分为两个部分,即第一章和第二章。第一章的讨论针对蒲鲁东为“工资平等”的社会主义所作的经济学论证,揭示这种论证尚未达到李嘉图经济学理论的水准。第二章批判了蒲鲁东经济学理论的哲学基础。

本期讨论会我们选的内容是第二章“\textbf{政治经济学的形而上学}”的第一节“\textbf{方法}”,这部分内容在我们编排的讲义中的3—17页\footnote{编者注:这部分内容收录于马恩选集第1卷。}。

在本次讨论会中,您将会了解到(或带着以下问题去阅读):

\textbf{1.}马克思对黑格尔的辩证法的一个简要概述。

\textbf{2.}蒲鲁东的经济学理论所遵循的“辩证法”同黑格尔的辩证法之间的差异。

\textbf{3.}蒲鲁东的经济学理论的哲学基础之实质是什么,犯了什么错误?

\textbf{4.}对蒲鲁东的经济学理论的哲学基础的批判对于我们有何启示?

\subsection{读书会记录}
2023年10月16日18:00—20:00,我们在东北林业大学奥林学院203教室开展了第一期的读书会。本期读书会我们阅读了《哲学的贫困》第二章第一节的部分内容(截止到“\textbf{第五个说明}”之前),并做了充分的讨论。具体讨论内容可简要地概括如下:

\subsubsection{1.前言部分}\label{sec:3}

首先,马克思在前言部分指出:
\begin{adjustwidth}{2em}{2em}
\qquad\fangsong
“如果说有一个英国人把人变成帽子,那末,有一个德国人就把帽子变成观念。这个英国人就是李嘉图,一位银行巨子,杰出的经济学家;这个德国人就是黑格尔,柏林大学的一位专任哲学教授。”
\end{adjustwidth}

这段话中的“帽子”、“观念”指的是什么?在这里我们认为,“帽子”指的是李嘉图的经济学中的诸多“经济范畴”,体现着物与物之间的关系;而“观念”则指的是黑格尔哲学中的“概念”,更确切地说,是黑格尔哲学中的形而上学传统\footnote{编者注:这在后面的讨论中会详细的说明。}。

古典经济学家们\footnote{编者注:这里指的是英国古典经济学家斯密、李嘉图等人。}通过对资本主义社会的外部经验现象的考察总结出了一些基本的经济规律,但是他没有意识到这些规律本身并非是永恒不变的存在物,以至于他们将人与人之间关系转变为物与物之间的关系,企图通过不变的经济范畴(帽子)解释人类社会的运行本身\footnote{编者注:按照马克思的话就是:“把人变成帽子。”}。而持有着形而上学传统的哲学家们(尤指黑格尔)则将这些“帽子”进行了更进一步的“抽象”,将其转化为观念本身\footnote{编者注:按照马克思的话就是:“把帽子变成观念。”}。

\subsubsection{2.第一个说明}

第一个说明所讨论的核心内容是黑格尔的辩证法,而黑格尔的辩证法在整体上也是建筑于形而上学这一传统的基础之上的。马克思指出,经济学家们只是向我们解释了生产如何在现存的关系下进行,但是没有向我们解释这些“关系”本身是何以存在的。古典经济学家们面对的是活生生的现实,他们所做的工作是对“活生生的现实”的直接抽象;而蒲鲁东面对的则是古典经济学家们提出的诸多范畴,以及隐藏在这些范畴背后的诸多古典经济学家们的教条与偏见,蒲鲁东在探寻这些“关系”得以产生的原因的过程中,他仅仅从这些范畴本身出发,因而他只能在观念的领域兜圈子。

黑格尔哲学的第一步便是形而上学式的抽象,将一切事物都抽象成为逻辑范畴,并将一切事物的运动也抽象成纯粹形式的运动,这样一来,正如马克思所言:

\begin{adjustwidth}{2em}{2em}
    \qquad\fangsong
    “既然把任何一种事物都归结为逻辑范畴,任何一个运动、任何一种生产行为都归结为方法,那末,由此自然得出一个结论,产品和生产、对象和运动的任何总和都可以归结为应用的形而上学。黑格尔为宗教、法等做过的事情,蒲鲁东先生也想在政治经济学上如法炮制。”
\end{adjustwidth}

由此,黑格尔找到了一种“绝对的方法”来描绘现实世界的运动,即纯粹理性的运动。事实上,这是一种颠倒,但在这里我们就不赘述了\footnote{编者注:唯心主义与唯物主义之间的差异。}。在这里不难理解,当事物的一切“偶性”都被抽掉之后,剩下的就是纯粹的“概念”、纯粹的“理性”,但需要指出的是,这种纯粹的“概念”与“理性”是脱离个别主体而存在的,是一种绝对的、客观的东西。因此,对于黑格尔的辩证法而言,辨证运动是“绝对精神”的自我运动,是“概念”的自我规定,因而这是一种客观的唯心主义。


\subsubsection{3.第二个说明}

在这部分内容中的末尾,马克思表明了他的历史唯物主义哲学思想:

\begin{adjustwidth}{2em}{2em}
    \qquad\fangsong
    “经济学家蒲鲁东先生非常明白,人们是在一定的生产关系范围内制造呢绒、麻布和丝织品的。但是他不明白,这些一定的社会关系同麻布、亚麻等一样,也是人们生产出来的。社会关系和生产力密切相联。随着新生产力的获得,人们改变自己的生产方式,随着生产方式即保证自己生活的方式的改变,人们也就会改变自己的一切社会关系。手工磨产生的是封建主为首的社会,蒸汽磨产生的是工业资本家为首的社会。”
\end{adjustwidth}

事实上,上面这段话的第一句可以转译为:

\begin{adjustwidth}{2em}{2em}
    \qquad\fangsong
    蒲鲁东明白,人们在一定的“关系”内制造“物”。但是他不明白,这些“关系”同这些“物”一样,也是人们生产出来的。
\end{adjustwidth}
这充分说明了马克思的历史唯物主义哲学与那种纯粹被动的、毫无生机的机械唯物主义哲学之间的差异,因为马克思告诉我们,“关系”可不是什么神秘主义式的东西,“关系”本身就是通过人类的实践活动所建构出来的。

\subsubsection{4.第三个说明}

在“第三个说明”中,马克思揭示了蒲鲁东经济理论的矛盾。马克思是这么说的:

\begin{adjustwidth}{2em}{2em}
    \qquad\fangsong
    “每一个社会中的生产关系都形成一个统一的整体。蒲鲁东先生把种种经济关系看做同等数量的社会阶段,认为这些阶段一个产生一个,一个来自一个,正如反题来自正题一样;认为这些阶段在自己的逻辑顺序中实现着人类的无人身的理性。

这个方法的唯一短处就是:蒲鲁东先生在考察其中任何一个阶段时,都不能不靠其它一些社会关系来说明,可是当时这些社会关系尚未被他用辩证运动产生出来。当蒲鲁东先生后来借助纯粹理性使其它阶段产生出来时,却又把它们当成初生的婴儿,忘记它们和第一个阶段是同样年老了。

因此,要构成被他看做一切经济发展基础的价值,就非有分工、竞争等等不可。然而当时这些关系在一定的系列中、在蒲鲁东先生的理性中以及逻辑顺序中根本还不存在。”
\end{adjustwidth}
在这里可以看到,蒲鲁东的“辩证法”很有意思,当他运用“辩证法”推出某些新的概念时,他总是需要依靠一些尚未被他的“辩证法”所生成出来的东西去说明。例如,蒲鲁东通过A和B推出了C,再通过C和I推出了D,但是I本身并没有通过他的辩证法被创造出来,而当他通过E和F推出I的时候,他也忘记了,I本身在C到D的“辩证运动”中就已经作为前提条件生成了。因此,马克思在这里隐约地想表明,蒲鲁东的“辩证法”似乎是一种非辩证的主观臆想。

\subsubsection{5.第四个说明}

通过对前面部分的阅读与讨论,“第四个说明”这部分内容便显得容易理解了。马克思在这一部分中说明了蒲鲁东是如何将他的“辩证法”应用到政治经济学领域的。马克思是这么叙述的:

\begin{adjustwidth}{2em}{2em}
    \qquad\fangsong
“蒲鲁东先生认为,任何经济范畴都有好坏两个方面。他看范畴就象小资产者看历史伟人一样:拿破仑是一个大人物;他行了许多善,但是也作了许多恶。

蒲鲁东先生认为,好的方面和坏的方面,益处和害处加在一起就构成每个经济范畴所固有的矛盾。

应当作的是:保存好的方面,消除坏的方面。”
\end{adjustwidth}

可见,当蒲鲁东的“辩证法”应用到政治经济学的领域时便成了一种机械式的“保存好的方面,消除坏的方面”。

因此,马克思会指出:

\begin{adjustwidth}{2em}{2em}
\qquad\fangsong
    “黑格尔没有需要提出任务。他只有辩证法。蒲鲁东先生从黑格尔的辩证法那里只学到了术语。而蒲鲁东先生自己的辨证运动只不过是机械地划分出好、坏两面而已。
    
    ……
    
    两个矛盾方面的共存、斗争以及融合成一个新范畴,就是辩证运动的实质。谁要给自己提出消除坏的方面的任务,就是立即使辩证运动终结。我们看到的已经不是由于矛盾本性而自我安置和自相对置的范畴,而是在范畴的两个方面中间激动、挣扎和冲撞的蒲鲁东先生。”

\end{adjustwidth}
    
\subsubsection{简短的总结}

通过本期的研讨会,我们可以对“\textbf{\nameref{sec:1}}”提出的四个问题中的前两个进行简要地回答。

对于\textbf{问题1“马克思对黑格尔的辩证法的一个简要概述”}而言,我们可以认为黑格尔的辩证法表明的是概念的自我规定(或自我运动),且这种辩证运动的前提基础是形而上学式的抽象。

对于\textbf{问题2“蒲鲁东的经济学理论所遵循的“辩证法”同黑格尔的辩证法之间的差异”}而言,我们可以认为黑格尔的辩证法所揭露的是脱离于单个主体之外的纯粹理性的自我运动,是一种客观唯心主义;而蒲鲁东的“辩证法”仅仅是借用了黑格尔哲学的诸多词句,他并没有领会黑格尔辩证法的实质,他将黑格尔的辩证法下降到了简单的二元式的机械运动,更具体地说,可以认为蒲鲁东的“辩证法”是一种主观唯心主义。
\newpage

\section{第二期:《哲学的贫困》选读(2)}

\subsection{读书会记录}
\begin{figure}[h]
    \centering
    \includegraphics[width=1\linewidth]{10.23.jpg}
    \caption{第二期读书会实拍图(2023.10.23)}
    \label{fig1}
\end{figure}
2023年10月23日18:00—20:00,我们在东北林业大学奥林学院203教室开展了第二期的读书会。本期读书会我们阅读了《哲学的贫困》第二章第一节中的\textbf{“第五个说明”}与\textbf{“第六个说明”}。我们本期做了非常丰富的讨论,现场情况见\textbf{\nameref{fig1}}\footnote{编者注:在这里非常抱歉,由于本期读书会讨论的过于热情,以至于我在中途拍照的时候漏掉了一些同学与老师。},主要讨论内容可简要地概括如下:
\subsubsection{1.第五个说明}

在“第五个说明”中,马克思首先指出:
\begin{adjustwidth}{2em}{2em}
    \qquad\fangsong
    “如果把辩证运动的全部过程归结为简单地对比善和恶,归结为提出任务来消除恶并且把一个范畴用作另一个范畴的消毒剂,那末范畴就失去自己的独立运动;观念就“不再发生作用”;他就没有内在的生命。它既不能把自己安置为范畴,也不能把自己分解为范畴。范畴的顺序成了一种脚手架。辩证法已不是绝对理性的运动了。辩证法没有了,代替它的至多不过是最纯粹的道德而已。”
\end{adjustwidth}

这里马克思似乎是想表达:“辨证运动”并不是蒲鲁东所认为的那种“简单地对比善和恶,并提出任务来消除恶”。马克思认为蒲鲁东的“辩证运动”不过是借助他自己的搭建出的一种“脚手架”而进行的,这意味着蒲鲁东将黑格尔所认为的“范畴的自我运动”变成了通过他(指蒲鲁东)自己搭建起来的“脚手架”而进行的运动。因此,马克思会说,在蒲鲁东的“辩证运动”下,范畴本身“就没有内在的生命”了。

之后,马克思揭示了蒲鲁东自己的逻辑混乱。马克思认为,蒲鲁东在谈论“历史”的时候,他是从他自认为的“范畴的顺序”\footnote{编者注:根据前文的分析,不难理解,这种“范畴的顺序”是通过蒲鲁东自己构建的“脚手架”而展开的。}去出发的。但是,当蒲鲁东将他自认为的“范畴的顺序”应用到现实中时,便同现实产生了矛盾。因此马克思指出:
\begin{adjustwidth}{2em}{2em}
    \qquad\fangsong
    “……于是蒲鲁东先生只得承认,他用以说明经济范畴的次序和这些经济范畴在其中相互产生的次序是不相适应的。经济的进化不再是理性本身的进化了。”
\end{adjustwidth}

最后,马克思在“第五个说明”中的最后一部分的内容中强调,如果按照蒲鲁东所设想的“现实的历史是观念、范畴和原理在其中出现的那种历史顺序”的话,那末就必须对此进行进一步的追问。也即是说,如果说是某一时代的原理创造了这一时代的历史的话,那末,为什么这一时代的原理没有创造别的时代的历史呢?为什么这一时代的原理恰好地就出现在这一时代了呢?为了回答上述疑问,就必须进一步考察那一时代的人们是怎样进行生产生活的、那一时代的人们之间的关系是怎么样的。但只要进入到对具体时代的人们的生产生活的考察之中,那末就是背弃了那种认为“观念创造历史”的观点,就是回到了历史发生的真正出发点。

因此,马克思在最后是这么论述的:
\begin{adjustwidth}{2em}{2em}
    \qquad\fangsong
“我们暂且和蒲鲁东先生一同假定:现实的历史,适应时间次序的历史是观念、范畴和原理在其中出现的那种历史顺序。

每个原理都有其出现的世纪。例如,与权威原理相适应的是11世纪,与个人主义原理相适应的是18世纪,推其因果,我们应当说,不是原理属于世纪,而是世纪属于原理。换句话说,不是历史创造原理,而是原理创造历史。但是,如果为了顾全原理和历史我们再进一步自问一下,为什么该原理出现在11世纪或者18世纪,而不出现在其它某一世纪,我们就必然要仔细研究一下:11世纪的人们是怎样的,18世纪的人们是怎样的,在每个世纪中,人们的需求、生产力、生产方式以及生产中使用的原料是怎样的;最后,由这一切生存条件所产生的人与人之间的关系是怎样的。难道探讨这一切问题不就是研究每个世纪中人们的现实的、世俗的历史,不就是把这些人既当成剧作者又当成剧中人物吗?但是,只要你们把人们当成他们本身历史的剧中人物和剧作者,你们就是迂回曲折地回到真正的出发点,因为你们抛弃了最初作为出发点的永恒的原理。”
\end{adjustwidth}

\subsubsection{2.第六个说明}
马克思在这部分内容中开篇提到:
\begin{adjustwidth}{2em}{2em}
    \qquad\fangsong
    “我们已经看到,在这一切一成不变的、停滞不动的永恒下面没有历史可言,即使有,至多也只是观念中的历史,即反映在纯理性的辩证运动中的历史。蒲鲁东先生谈到辩证运动中的各种观念不能自相‘区分’时,把运动的一切影子和影子(它们可以造成某种类似历史的东西)的一切运动一概抹熬。”
\end{adjustwidth}

其中,\textbf{“运动的影子”}和\textbf{“影子的运动”}分别代表的是什么意思呢?在这里,我们认为前者指的是“感性历史”的理论表现,后者指的是黑格尔的那种“纯粹理性”的运动。因此,我们可以认为,马克思在这里想表明:蒲鲁东所谈论的“辨证运动”既不是对感性历史发展过程的理论表达,也不是对黑格尔式的纯粹理性的运动的表达。因此,正如前面我们所讨论的那样,事实上,蒲鲁东自己构建了一个范畴的“脚手架”,范畴遵循着蒲鲁东自己构建的“脚手架”向上爬行。而范畴为什么要踏着“脚手架”向上运动呢?因为蒲鲁东认为范畴就得踏着他所构建的“脚手架”向上运动,这似乎是蒲鲁东先生的一厢情愿。

既然蒲鲁东的“辨证运动”既不是对感性历史的理论表达,也不是对黑格尔式的“纯粹理性”运动的表达,那末,蒲鲁东是如何从底层逻辑层面说明他的“辩证运动”的合理性的呢?很简单,既然蒲鲁东自己的“辨证运动”既不是“运动的影子”也不是“影子的运动”,那末,蒲鲁东先生完全可以“大言不惭地”构建出一个新的规律,并用一种“唬人的词句”宣称这种规律的至高性地位。因此,蒲鲁东先生自己构建出了所谓的\textbf{“人类理性”}、\textbf{“社会天才”}等等的概念。然而事实上,我们不难看出,所谓的\textbf{“人类理性”}、\textbf{“社会天才”}等的概念不过是蒲鲁东自己虚构出来的东西,不过是蒲鲁东自己的主观设想。

因此,当蒲鲁东的理论同现实的历史之间发生矛盾的时候,他就将这些\textbf{“矛盾”}看作是\textbf{“人类理性”}、\textbf{“社会天才”}的任务,他认为\textbf{“人类理性”}、\textbf{“社会天才”}就是需要解决这些矛盾的,进而使得范畴朝着\textbf{“人类理性”}、\textbf{“社会天才”}所期待的目标发展。

因而我个人\footnote{编者注:指本书编者,下同。}认为,在马克思的视角下,蒲鲁东的逻辑进路应可分为如下的步骤:


\begin{tcolorbox}[colback=gray!20, colframe=gray!100, sharp corners, leftrule={3pt}, rightrule={0pt}, toprule={0pt}, bottomrule={0pt}, left={2pt}, right={2pt}, top={3pt}, bottom={3pt}] 
\textbf{step1.}假定有一个“先在”的真理:例如“纯粹的平等”、“纯粹的自由”等最终的概念。

\textbf{step2.}假定一种“规则”:可以通过这种“规则”发现“真理”,并把这种“规则”命名为“人类理性”、“社会天才”。

\textbf{step3.}因此,当理论同现实相违背时,蒲鲁东就可以借着“人类理性”、“社会天才”的名义说:这种矛盾正是要通过“辩证运动”被解决的!


\end{tcolorbox}

因此,马克思会说:
\begin{adjustwidth}{2em}{2em}
    \qquad\fangsong
    “假设只是为了某种特定的目的而设立的。通过蒲鲁东先生之口讲话的社会天才首先给自己提出的目的,就是消除每个经济范畴的一切坏的东西,使它只保留好的东西。他认为,好的东西,最高的幸福,真正的实际目的就是平等。为什么社会天才只要平等,而不要不平等或友爱、不要天主教或别的什么原理呢?因为‘人类之所以实现这么多特殊的假设,正是由于考虑到一个最高的假设’,这个最高的假设就是平等。换句话说,因为平等是蒲鲁东先生的理想。他以为分工、信用、工厂,一句话,一切经济关系都仅仅是为了平等的利益才被发明的,但是结果它们往往对平等不利。由于历史和蒲鲁东先生的臆测步步发生矛盾,所以他得出结论说,有矛盾存在。即使是有矛盾存在,那也只存在于他的固定观念和现实运动之间。”
\end{adjustwidth}

在上面马克思的这段论述中,我们可以挑几句关键的句子来看,例如:\begin{fangsong}
    “通过蒲鲁东之口讲话的社会天才”
\end{fangsong},以及\begin{fangsong}
    “为什么社会天才只要平等,而不要不平等或友爱、不要天主教或别的什么原理呢?……因为平等是蒲鲁东先生的理想”
\end{fangsong}
等。不难看出,前面这几句话处处体现出了我在前文所概括的\textbf{“蒲鲁东的逻辑进路”}中的三个步骤。因此,马克思会指出:\begin{fangsong}
    “即使是有矛盾存在,那也只存在于他\footnote{编者注:指蒲鲁东。}的固定观念和现实运动之间。”
\end{fangsong}

因此我们可以看到,马克思在这里最终想要表达的是:\textbf{蒲鲁东的思想既不唯物,也不客观。}在这里,我个人认为,蒲鲁东的思想既没有超越古典经济学家们,也没有超越黑格尔。因为,相较于古典经济学家们而言,蒲鲁东缺少了这些理论家们的唯物主义倾向;相较于黑格尔而言,蒲鲁东也没有领悟“辩证法”或“辨证运动”的实质,因而蒲鲁东的思想更像一种拙劣的“折衷主义”\footnote{编者注:事实上,我认为在“第七个说明”中马克思对蒲鲁东的这种“折衷主义”的批判会体现地更为明显,但本期讨论会并没有讨论到“第七个说明”,因此这里先按下不表。}。

在“第六个说明”的最后,马克思这样说道:
\begin{adjustwidth}{2em}{2em}
    \qquad\fangsong
    “当然,平等趋势是我们这个世纪所特有的。但是,说以往各世纪及其完全不同的需求、生产资料等等都是为实现平等而遵照天命行事,这首先就是把我们这个世纪的人和生产资料当做过去世纪的人和生产资料看待,否认世世代代不断改变前代所获得的成果的历史运动。经济学家们很清楚,同是一件东西对甲说来是成品,对乙说来只是从事另一种生产的原料。
    
    如果你们同蒲鲁东先生一道假定:社会天才制造出,或者更确切些说随兴制造出封建主,是为了达到把耕者变为负有义务的和彼此平等的劳动者这一天命的目的,那末,你们就是把目的和人换了一下,这种做法和为了达到恶意的满足(即羊群赶走人)而在苏格兰确立土地私有制的天命比较起来,毫不逊色。”
\end{adjustwidth}

马克思这两段论述同样地体现出了满满的历史唯物主义原则。马克思承认,“平等”确实是当时那个世纪所特有的思想倾向,但他想进一步表明的是,并不是自古以来就一直存在着这种“平等”的思想倾向的。如果将“平等”这种仅仅在当时那个世纪所特有的思想倾向看作是永恒不变的思想倾向的话,就意味着,当时那个世纪的人的物质生产活动同以往一切世纪的物质生产活动之间并无差别了。而这种观念是反历史唯物主义的,在这里就不赘述了。

\subsubsection{本期小结:以及一些余论}
本期我们讨论的比较热烈,同时一些新同学也在本期加入了我们的讨论会。在本期讨论会的最后几分钟,我们初步探讨了一下古典国民经济学(特别是李嘉图的经济学)同马克思主义经济学之间的差异。我们探讨了“李嘉图难题”的产生,以及马克思是如何解决李嘉图难题的(但我们只是初步地提了一嘴,并没有深入讨论下去)。我们计划在下期的讨论会中留一段时间和大家讨论一下李嘉图的利润理论与马克思之间的差异,以及马克思是如何系统地批判李嘉图经济学的。总之,这是一期收获很满的讨论会。
\newpage
\section{第三期:《哲学的贫困》选读(完结)+《雇佣劳动与资本》(1)}

\subsection{读书会记录}
2023年11月6日18:00—20:00,我们在东北林业大学奥林学院203教室开展了第三期的读书会\footnote{编者注:由于2023年10月30日大家感冒的比较多,我们就将第三期读书会推迟到了11月6日。}。本期读书会我们结束了《哲学的贫困》第二章第一节的讨论,由于前两期的积累,我们对于《哲学的贫困》第二章第一节中\textbf{“第七即最后一个说明”}这部分内容的讨论比较顺利。之后,我们开启了马克思的另一篇政治经济学经典著作《雇佣劳动与资本》的讨论。总之,和之前几期一样,我们进行了充分而深刻的讨论,具体内容可概括如下:

\subsubsection{1.第七即最后一个说明}
“第七个说明”这部分内容更像是对前面六个说明的总结,整体读下来不难理解,不过还是存在一些值得深入讨论的细节之处。
例如,马克思的这两段话是这么说的:

\begin{adjustwidth}{2em}{2em}
    \qquad\fangsong
 “这样,为了正确地判断封建的生产,必须把它当做以对抗为基础的生产方式来考察。必须指出,财富怎样在这种对抗中间形成,生产力怎样和阶级对抗同时发展,这些阶级中一个代表着社会上坏的、否定的方面的阶级怎样不断地成长,直到它求得解放的物质条件最后成熟。这难道不是说,生产方式、生产力在其中发展的那些关系并不是永恒的规律,而是同人们及其生产力发展的一定水平相适应的东西,人们生产力的一切变化必然引起他们的生产关系的变化吗?由于最重要的是不使文明的果实(已经获得的生产力)被剥夺,所以必须粉碎生产力在其中产生的那些传统形式。从此以后,从前的革命阶级将成为保守阶级。

资产阶级开始自己的历史发展时就有一个本身是封建时期无产阶级残存物的无产阶级存在。资产阶级在其历史发展过程中不可避免地要发展它的对抗性质,起初这种性质或多或少是掩饰起来的,只是处于隐蔽状态。随着资产阶级的发展,在它的内部发展着一个新的无产阶级,即现代无产阶级。无产阶级同资产阶级之间展开了斗争,在双方尚未感觉、注意、重视、理解、承认并公开宣告以前,这个斗争最初仅表现为局部的暂时的冲突,表现为一些破坏行为。另一方面,如果说现代资产阶级的全体成员由于组成一个与另一个阶级相对立的阶级而有共同的利益,那末,由于他们互相对立,他们的利益又是对立的,对抗的。这种利益上的对立是由他们的资产阶级生活的经济条件产生的。资产阶级运动在其中进行的那些生产关系的性质绝不是一致的单纯的,而是两重的;在产生财富的那些关系中也产生贫困;在发展生产力的那些关系中也发展一种产生压迫的力量;只有在不断消灭资产阶级个别成员的财富和形成不断壮大的无产阶级的条件下,这些关系才能产生资产者的财富,即资产阶级的财富;这一切都一天比一天明显了。”
\end{adjustwidth}

这里马克思主要是在阐述他的历史唯物主义思想。我们可以看到,在上面的第一段中,马克思说:
\begin{adjustwidth}{2em}{2em}
    \qquad\fangsong
    “这难道不是说,生产方式、生产力在其中发展的那些关系并不是永恒的规律,而是同人们及其生产力发展的一定水平相适应的东西,人们生产力的一切变化必然引起他们的生产关系的变化吗? ”
\end{adjustwidth}

在这里便有一个疑问,什么叫\textbf{“生产方式在其中发展的那些关系”}呢?我们知道,在马克思看来,“生产方式”本身就应包涵着“生产关系”与“生产力”两个维度,但马克思在这里又说“生产方式在其中发展的关系”,这是否有同义反复的嫌疑了呢?我们认为,这是一种很微妙的说法,马克思在这里想表示的应该就是“生产关系”本身,或者认为,马克思这里说的“生产方式”是一种更为具体的“生产形式”,也是生产力的一种表现。总之,这并不影响整段所论述的思想,这一细节之处可留给读者们细细思索。

此外,在上面的第二段中,马克思论述了无产阶级的生成以及无产阶级同资产阶级二者的斗争。马克思在这里对无产阶级的描述同样比较微妙,马克思既不是从经济关系出发,将无产阶级表述为不占有生产资料的阶级,也不是从哲学层面出发,即像在《<黑格尔法哲学批判>导言》中所表述的那样:“一个被戴上彻底的锁链的阶级,一个并非市民社会阶级的市民社会阶级,形成一个表明一切等级解体的等级”\footnote{编者注:见马克思《<黑格尔法哲学批判>导言》。}。

我们认为,马克思在这里更想说明的是无产阶级的自我发展过程,即从自在到自为的过程。马克思在这里首先指明了无产阶级最早生成于资本主义发展的初期,从某种意义上说,无产阶级是伴随着资产阶级的发展而发展的。然而在资本主义发展的前期,无产阶级与资产阶级之间的阶级对抗不明显,因而在那一时期,资本主义社会的对抗性质“或多或少是掩饰起来的,只是处于隐蔽状态。”因此,在这种对抗性不明显的社会下,无产阶级的阶级意识\footnote{编者注:这里借用了卢卡奇的术语“阶级意识”。}同样是不明显的,这便是一种“自在”的状态。而随着资本主义社会的进一步发展,阶级对抗也愈来愈明显,如马克思所言,资产阶级的运动由于资本主义生关系的作用所表现出了双重性的结果,即资产阶级的历史运动一方面制造着大量的财富(但这种财富是个别性的、属于个别资产阶级成员的财富),另一方面同时又制造着大量的贫困(这种贫困是一种普遍性的贫困)。这种直接在物质利益上的对抗促使了无产阶级的阶级意识的提高,也意味着无产阶级从“自在”向“自为”的转变\footnote{编者注:这里用的词是“转变”,并没有明确指出无产阶级必然会随着物质利益的对抗而成为“自为阶级”,因为其实在这里有一些疑问,当代资本主义国家的工人阶级是否可以被称作是“自为阶级”呢?}。

马克思在“第七个说明”后面部分的论述就很像《共产党宣言》了。不过马克思在《共产党宣言》中是对形形色色的社会主义者进行了划分,而在“第七个说明”中则是对政治经济学语境下的形形色色的改良主义学派进行了划分。在这里,按照马克思在“第七个说明”中的划分,我给大家做一个简短的归纳总结:
\begin{tcolorbox}
    \textbf{1.宿命论的经济学家:}斯密、李嘉图等代表的古典政治经济学,可以被看作是一种无批判的实证主义。

    \textbf{2.浪漫派:}对宿命论经济学家的直接继承,但他们甚至认为无产阶级的贫困是“理所应当”的。

    \textbf{3.人道学派:}企图在道德层面约束资产阶级。

    \textbf{4.博爱学派:}人道学派的进一步完善,幻想着人人都能变成资产者。

    \textbf{5.空想社会主义者:}在无产阶级解放的物质存在条件不足的情况下,企图寻得无产阶级解放的理论。
\end{tcolorbox}

最后,回到出发点,马克思对蒲鲁东做了一个总结性的评论,在这里我们就不过多解释了,我仅把原文放出来,根据前面几期的内容读者们应该能够体会到马克思的意思了。

马克思在最后写道:
\begin{adjustwidth}{2em}{2em}
    \qquad\fangsong
    “现在再来谈谈蒲鲁东先生。

每一种经济关系都有其好的一面和坏的一面;只有在这一点上蒲鲁东先生没有背叛自己。他认为好的方面由经济学家来揭示,坏的方面由社会主义者来揭发。他从经济学家那里借用了永恒经济关系的必然性这一看法;从社会主义者那里借用了使他们在贫困中只看到贫困的那种幻想。他对两者都表示赞成,企图拿科学权威当靠山。而科学在他的观念里已成为某种微不足道的科学公式了;他无休止地追逐公式。正因为如此,蒲鲁东先生自以为他既批判了政治经济学,也批判了共产主义;其实他远在这两者之下。说他在经济学家之下,因为他作为一个哲学家,自以为有了神秘的公式就用不着深入纯经济的细节;说他在社会主义者之下,因为他既缺乏勇气,也没有远见,不能超出(哪怕是思辨地也好)资产者的眼界。

他希望成为一种合题,结果只不过是一种总合的错误。

他希望充当科学泰斗,凌驾于资产者和无产者之上,结果只是一个小资产者,经常在资本和劳动、政治经济学和共产主义之间摇来摆去。”
\end{adjustwidth}

\subsubsection{2.雇佣劳动与资本(1)}
本期讨论会我们把《雇佣劳动与资本》开了个头,本期具体讨论的内容在我们所编排的讲义的18—21页。在进一步展开我们的讨论内容之前,我有必要跟读者们交代一下《雇佣劳动与资本》这部著作的写作背景以及后续恩格斯在1891对此做出的一些修改与补充。

\paragraph{一个简要的背景概述}\begin{fangsong}
《雇佣劳动与资本》这部著作是马克思根据1847年12月在布鲁塞尔德意志工人协会发表的演说写成的,最初以社论形式于 1849年4月5—8日和11日在《新莱茵报》陆续发表。后来由于《新莱茵报》被迫停刊,这部著作的连载遂告中断。 
时间来到1891年,为适应工人群众学习科学社会主义理论的需要,在恩格斯的关心下,这部著作的新单行本在柏林印行。 恩格斯根据《资本论》的基本观点和科学论述,对马克思这部著作进行了适当的修改和补充,并为该著作的单行本写了一篇《导言》。 恩格斯在《导言》中指出:“我所作的全部修改,都归结为一点。 在原稿上是,工人为取得工资向资本家出卖自己的劳动,在现在这一版本中则是出卖自己的劳动力。”\footnote{编者注:见马恩选集第一卷(中央编译局2012年版)第318页。}恩格斯阐明了修改的理由,论述了马克思主义政治经济学的科学价值,揭露了资本主义制度的本质,指出工人阶级不仅是社会财富的生产者,而且是新的社会制度的创造者。
\end{fangsong}

事实上,就马克思在1849年在《新莱茵报》发表的原版内容而言,我们不得不承认其中存在着不可忽视的局限性,当时的马克思并没有完成他的政治经济学批判工作,虽然当时他能够在整体性层面洞察出资本主义经济社会的固有局限,但是就一些关键的细节之处,例如对“劳动”与“劳动力”之间的区分,马克思的观点还是处于一个模糊的状态。然而经过恩格斯修改之后的版本就好多了。

 \vspace{0.5cm} %设置垂直距
\textbf{在把背景交代清楚之后,接下来便可以展开我们的讨论了。}
首先,按照原文的顺序,我们跟着马克思的思路讨论了“什么是工资?”的问题。对于这一问题的回答我们首先需要从现实的经验现象中去寻找答案。马克思说,假如去问工人们什么是他们的工资,他们会回答:“工资是工作的一定时间所换来的报酬。”可见,工资似乎表现为是工人同资本家之间进行的交换所获得的价值物,更确切地说,似乎是工人工作的一定时间(或一定量的工作)同资本家之间进行的交换所获得的价值物。但事实上并非如此,因为当我们考虑一下这个“交换”本身是否同其他商品之间的交换存在着差异时,我们便会发现,工人为获取工资同资本家之间进行的交换同他为获取生活资料同卖主进行的交换没有什么不同:这种交换总是遵循着“等价交换”的原则的\footnote{编者注:因为如果不是这样的话,商品经济本身便会出现混乱。}。那为什么工人所获得的工资与工人所生产出的商品的价值量之间存在着不一致呢?事实上,问题的关键则在于,工人同资本家所付给他的工资之间所进行的等价交换的那个“物”并不是劳动本身,而是劳动力商品。事实上,劳动是无法成为商品的,劳动本身没有价值,关于这一点的详细论述,大家可以去看《资本论》第一卷第四章的“劳动力的买和卖”以及第六章的“工资”\footnote{编者注:事实上,在《雇佣劳动与资本》中,即使是恩格斯修改之后的版本,也并没有详细地论述劳动力商品同劳动之间的差异性,因为恩格斯是直接根据《资本论》第一卷对其进行的修改与增补的,因而在其中有些概念的出场会显得缺少一些前提性的阐述,因此,我还是推荐大家直接去看《资本论》第一卷,这是一个苦功夫,需要静下心来好好研读,在这里我就不赘述其具体内容了。}。

\textbf{因此,仅仅就劳动力商品与工资之间的交换而言,这一过程不存在剥削!}\footnote{编者注:请读者们记住这一点,在不考虑生产价格理论的前提下,这一点是完全正确的,我和光玉、许婕老师都是赞成的,但是在后面关于劳动与自由之间的关系时,我们之间发生了一定的分歧。}如果有人固执地认为劳动力商品同工资之间的交换过程发生了剥削,我想他大概率是不懂马克思主义政治经济学的。

接下来的这一段引起了我们的广泛讨论:

\begin{adjustwidth}{2em}{2em}
    \qquad\fangsong
“可是,劳动是工人本身的生命活动,是工人本身的生命的表现。工人正是把这种生命活动出卖给别人,以获得自己所必需的生活资料。可见,工人的生命活动对于他不过是使他能以生存的一种手段而已。他是为生活而工作的。他甚至不认为劳动是自己生活的一部分;相反地,对于他来说,劳动就是牺牲自己的生活。劳动是已由他出卖给别人的一种商品。因此,他的活动的产物也就不是他的活动的目的。工人为自己生产的不是他织成的绸缎,不是他从金矿里开采出的黄金,也不是他盖起的高楼大厦。他为自己生产的是工资,而绸缎、黄金、高楼大厦对于他都变成一定数量的生活资料,也许是变成棉布上衣,变成铜币,变成某处地窖的住所了。一个工人在一昼夜中有十二小时在织布、纺纱、钻孔、研磨、建筑、挖掘、打石子、搬运重物等等,他能不能认为这十二小时的织布、纺纱、钻孔、研磨、建筑、挖掘、打石子是他的生活的表现,是他的生活呢?恰恰相反,对于他来说,在这种活动停止以后,当他坐在饭桌旁,站在酒店柜台前,睡在床上的时候,生活才算开始。在他看来,十二小时劳动的意义并不在于织布、纺纱、钻孔等等,而在于这是挣钱的方法,挣钱使他能吃饭、喝酒、睡觉。假如说蚕儿吐丝作茧是为了维持自己的生存,那末它就可算是一个真正的雇佣工人了。”
\end{adjustwidth}

马克思在这里说:“可见,工人的生命活动对于他不过是使他能以生存的一种手段而已。他是为生活而工作的。他甚至不认为劳动是自己生活的一部分;相反地,对于他来说,劳动就是牺牲自己的生活。”
关于这一方面的论述,可以引申到对\textbf{劳动与自由之间的关系}的讨论。\textbf{在关于这一点的看法上,我们出现了分歧。}下面我将尽我所能还原当时我们争论的情况\footnote{编者注:遗憾的是,正如马克思的好朋友、著名的诗人海涅曾在其所著的《论德国宗教和哲学的历史》一书中所言的那样:“箭一离开弦便不再属于射手了,言论一离开说话人的口……便不再属于他了。”即使我全程极度认真地参与了这场讨论,但我也无法完全准确地描绘当事人的思想观点,请原谅我可能在某些地方对当事人的观点存在着的误解。}:

\paragraph{观点1:}\begin{fangsong}
以往似乎有一种误解,认为自由王国只有在共产主义社会才能达到,但其实在资本主义社会,自由王国与必然王国二者是以一种二分的形式所存在着的。诚然,当工人将自己的劳动力商品出卖给资本家之后,在资本家使用劳动力商品进行价值生产的这一过程中,工人事实上是不自由的,是处在一种“必然王国”之中。但是,一旦工人脱离生产过程,也即是说,当工人跨入到对于其自身的劳动力商品的恢复过程之中时,他实际上是暂时性地脱离了必然王国,进入到自由王国之中了。这应该如何理解呢?可以认为对于“工人恢复其自身劳动力商品”的这一过程而言,这一过程不涉及生产的领域,换句话说,这一过程仅仅是工人的纯粹消费的领域,不涉及价值与剩余价值的创造。因而这一领域不具有任何剥削性,换言之,这一领域是工人真正的生活领域——对于工人而言的仅仅纯粹的消耗使用价值的领域,完全是一个自然的过程。举个例子,比如说某工人一天工作了八个小时,赚了100块钱,他下班了给自己买了一个烤鸭吃,从这一行为中是无法分析出什么样的社会关系的,这一行为仅仅是工人消耗使用价值的过程,因而从这个意义上来说,这一个过程是“自由”的(至少相对于生产过程而言是自由的)。
\end{fangsong}
\paragraph{观点2:}\begin{fangsong}
我们认为“观点1”中所认为的那种“工人的纯消费领域属于自由王国”的看法是存在局限性的。我们同样承认剩余价值产生于生产领域,这即意味着在工人的纯消费领域不存在着剥削。但我们认为对于问题的分析不能止步于此。因为事实上,工人的消费可以被看作是资本主义生产过程的延申。诚然,工人在纯消费领域进行的活动仅仅是对使用价值的消耗,但这并不意味着工人阶级脱离了必然王国,我们认为工人阶级仍然是非自由的。这是因为,工人阶级的消费事实上具有两重性:一方面满足了工人阶级自身的物质需求\footnote{编者注:我们甚至对这种“满足工人阶级自身物质需求”的观点还是持有保留意见的,因为处于贫困水平的工人阶级似乎连这一点也无法被满足。};另一方面,工人阶级的消费事实上为资本主义的再生产创造了条件,而这一方面则是我们批判“观点1”的关键。我们认为,工人阶级的消费实际上处于一种“知其不可而为之”的境遇,工人阶级为了生存(生活)他必须进行消费,但工人阶级的这种消费同时又为下一次资产阶级对其的剥削创造了物质条件。在这里我们做一个简单的政治经济学分析:对于资本家而言,雇佣工人进行生产这一过程诚然是至关重要的,因为这一过程直接涉及到了价值与剩余价值的生成、涉及到了资本家对工人阶级的剥削,但是生产过程的结束对于资本家而言并不是直接的结果,资本家必须要使得剩余价值转换为最终的利润\footnote{编者注:这涉及到马克思的生产价格理论,在这里就不赘述了。有兴趣的读者可以阅读《资本论》第三卷的前10章。},而这种转换恰恰是通过工人阶级的消费而得以实现的。因此,在这个意义上,我们认为工人阶级的纯消费领域依然是受到着资本主义制度的规制,并没有脱离必然王国。    
\end{fangsong}
\vspace{0.5cm} %设置垂直距

我们之间的分歧主要是“观点1”和“观点2”之间的分歧。光玉老师的看法倾向于“观点1”,我和许婕老师的看法倾向于“观点2”。并且,从某种意义上而言,我们似乎都能够理解彼此观点的侧重之处。事实上,我认为,在政治经济学原理部分,我们都能清晰地认识到剩余价值的生产与剩余价值的实现之间的区别,我们也都能理解商品经济的等价交换原则。因此,相较于政治经济学原理而言,我们之间更像是对于“自由”这一范畴的看法出现了分歧,而对于“自由”的理解,在我看来是一个哲学问题,因而我们之间的分歧,确切的说是一个哲学层面的分歧。

事实上,马克思早年就在其博士论文《论德谟克利特的自然哲学与伊壁鸠鲁的自然哲学之差别》中探讨了“自由”的问题。当时的马克思由于受到青年黑格尔派的自我意识哲学的影响,他致力于通过诉诸一种“抽象个别性的自我意识”的形式去对抗当时存在着的普遍的宗教威权。(关于这部分的内容,我就不在这里赘述了,有兴趣的读者可以自行去阅读马克思的博士论文,或者在将来的读书会中继续探讨这部分的内容\footnote{编者注:估计这学期不会讨论到这部分内容了,这学期的主题是马克思的政治经济学。}。)当然,现在我们知道,当时的马克思的自由思想是存在着局限性的,因为对于“抽象个别性的自我意识”的形式而言,其代表的是一种“脱离定在的自由”,马克思甚至在当时也意识到了这种“自由”是无法“在定在之光中发亮”的。

让我们顺着这个思路继续下去,我们会意识到,不存在绝对的自由。进而再次回到我们之间争论的出发点,“工人阶级的纯消费领域”的自由\footnote{编者注:我们暂且假定这种自由存在。}便是一种“定在”之中的自由,而这个“定在”本身就是资本主义社会,更具体地说,是资本主义市场机制。我想关于我在上面的那种诠释,我们之间应该都是认同的。如果大家都能够接受这一点的话,那末争论的焦点便又一次地被转移了,现在的问题变成了\textbf{工人阶级在资本主义市场经济下的“消费自由”是否是“真实的自由”?}关于这一问题,我想我在这里还是不要做出回答的好,这部分留给读者们细细思索了。

\subsubsection{本期小结:以及一些余论}
本期讨论会的人数比较少,但却是讨论的最为热烈的一期。当天哈尔滨下了一天的大雪,不过恶劣的天气却丝毫没有减少大家学习与讨论的热情,我想这就是马克思主义政治经济学的魅力吧!此外,在本期讨论会中,我们还提到了20世纪60年代出现的“斯拉法体系”\footnote{编者注:详见皮埃罗·斯拉法《用商品生产商品》。}、马克思的“地租理论”、恩格斯对农民问题的研究、日本马克思经济学家置盐信雄提出的“置盐定理(Okishio Theorem)”、森岛通夫等人提出的“马克思主义基本定理(Fundamental Marxian Theorem,FMT)”等。这些内容大多是在讨论某一具体点的时候顺便提的一嘴,没有具体的深入展开。事实上,上述提到的每个部分都蕴含着极为丰富的内容,限于篇幅限制,我就不把这些具体内容一一编排到本书中了\footnote{编者注:可能在之后我会以附录的形式编排一些内容。},如果有成员对其中某一方面感兴趣的话,可以在群里提出来,我会给大家发相关的学习资料。总而言之,这是一期收获满满的讨论会。
\newpage
\section{第四期:《雇佣劳动与资本》(2)}
\subsection{读书会记录}
2023年11月13日18:00—20:00,我们在东北林业大学奥林学院203教室开展了第四期的读书会。本期读书会我们继续探讨了马克思《雇佣劳动与资本》中的部分内容。接下来我将对本期讨论内容进行简要概述。

\subsubsection{1.劳动力并不向来就是商品}
我们对马克思的这段话进行了一定的讨论:
\begin{adjustwidth}{2em}{2em}
    \qquad\fangsong
    “劳动\footnote{在1891年的版本中,“劳动”改为“劳动力”。——编者注}并不向来就是商品。劳动并不向来就是雇佣劳动、即自
由劳动。奴隶就不是把他自己的劳动 \footnote{同上}出卖给奴隶主,正如耕牛不
是向农民卖工一样。奴隶连同自己的劳动 \footnote{同上} 一次而永远地卖给自
己的主人了。奴隶是商品,可以从一个所有者手里转到另一个所有
者手里。奴隶本身是商品,但劳动  \footnote{同上} 却不是 他的商品。农奴只出卖
自己的一部分劳动  \footnote{同上} 。不是他从土地所有者方面领得报酬;相反
地,土地所有者从他那里收取贡赋。农奴是土地的附属品,替土地
所有者生产果实。相反地,自由工人自己出卖自己,并且是零碎地
出卖。他每天把自己生命中的八小时、十小时、十二小时、十五小时
拍卖给出钱最多的人,拍卖给原料、劳动工具和生活资料的所有者,即拍卖给资本家。工人既不属于私有者,也不属于土地,但是他
每日生命的八小时、十小时、十二小时、十五小时却属于它的购买
者。工人只要愿意,就可以离开雇用他的资本家,而资本家也可以
随意辞退工人,只要工人使他不能再获得利益或者不能使他获得
预期的利益,他就可以辞退。但是,工人是以出卖劳动 \footnote{同上} 为其工资
的唯一来源的,如果他不愿饿死,就不能离开整个购买者阶级即资
本家阶级。工人不是属于某一个资产者,而是属于整个资产阶
级\footnote{在1891年的版本中,“不是属于某一个资产者,而是属于整个资产阶级”改为“不是属于某一个资本家,而是属于整个资本家阶级”。——编者注};至于工人给自己寻找一个雇主,即在资产阶级\footnote{在1891年的版本中, “资产阶级”改为“资本家阶级”。——编者注} 中间寻找一
个买主,那是工人自己的事情了。”
\end{adjustwidth}

马克思在这里说“劳动力并不向来就是商品。劳动并不向来就是雇佣劳动、即自由劳动。”这种说法实际上便是同资产阶级经济学家所认为的“经济范畴永恒存在”的观点相区别了。我们认为,马克思在这里想表达的还是他的历史唯物主义的哲学观点,即一种透过现象看本质的哲学观。这事实上是不难理解的,因为“劳动力”作为工人自身所内蕴的一种自然属性,从发生学机制层面来看,它最初显然不可能是作为体现着社会关系的“商品”而发生的。因此,对于“劳动力商品”这一规定——即劳动力是商品——而言,这种规定性一定是在历史中生成的。马克思在这里举了个例子,他说,“奴隶向奴隶主出卖自身”实际上与“资本主义社会下的工人向资本家出卖劳动力商品”是截然不同的两种情况。因为在奴隶制社会里,奴隶本身就是商品,但奴隶自身的劳动力却不是奴隶自己的商品,即对于奴隶自身而言,他是丧失了全部自主性(主体性)的;而在资本主义社会里,工人的劳动力却是他自己的商品,即工人在一定程度上具有自主性(主体性),因而这便是这两种情况的差异所在。

我们认为,这两种情况的差异实际上就是由不同历史阶段上的不同社会关系所造成的,资本主义的雇佣劳动形式打破了奴隶社会存在的直接人身束缚,这是一种进步,但是资本主义雇佣劳动本身又是一种对于工人而言的无形的束缚。就像马克思在《<黑格尔法哲学批判>导言》里面带有讽刺意味地谈论马丁·路德的宗教改革时所说的那样:
\begin{adjustwidth}{2em}{2em}
   \qquad\fangsong
   “的确, 路德战胜了虔信造成的奴役制,是因为他用 信念 造成的
奴役制代替了它。 他破除了对权威的信仰,是因为他恢复了信仰
的权威。 他把僧侣变成了世俗人,是因为他把世俗人变成了僧侣。
他把人从外在的宗教笃诚解放出来,是因为他把宗教笃诚变成了
人的内在世界。 他把肉体从锁链中解放出来,是因为他给人的心
灵套上了锁链。”
\end{adjustwidth}

我就不再继续解读了,请读者们自行领会。但总的来说,无论如何,奴隶制社会与资本主义社会由于整个社会的生产方式——进而是生产关系——之间的差异,导致了经济形式的差异。资本主义社会的经济关系并非永恒不变的自然产物,恰恰相反,这种经济关系是在社会历史运动中生成。并且,还需注意的是,由于这些“关系”需要依靠物质载体得以表现自身,因而人们在面对社会现实时,由于面对的是直接的物质载体,因而往往会错把历史中生成的“关系”看作是永恒的自然法则,这便是一种\textbf{遮蔽},事实上,马克思在后面展开他的意识形态批判理论的时候便是要对这种遮蔽进行\textbf{祛魅}。关于这一点,我也不再这里继续赘述了,下期讨论会的时候大家可以线下交流。

\subsubsection{2.商品的价格是由什么决定的?}
说实话,我认为《雇佣劳动与资本》的这部分内容是稍微有些混乱的,尽管恩格斯修改之后的版本会好一点,然而其中的某些概念的论述在现在看来还是有些模糊不清的,特别是马克思在这里没有刻意区分“价值”和“价格”\footnote{编者注:实际上,对于“价值”和“价格”的区分是一件非常重要的事情,甚至不亚于对“劳动力”与“劳动”之间的区分。\textbf{在这里请读者们务必要弄清,“价格”是一个纯粹现象层面的表现,而“价值”则是透过现象看本质的那个“本质”,这是理解马克思主义政治经济学的一个关键!}}。马克思在这部分首先说,商品的价格\footnote{编者注:这里的价格指的就是现象层面的价格。}是由“买主和卖主之间的竞争即供求关系决定的”。转而马克思分析了这一“竞争即供求关系”本身。我们认为,事实上,马克思在这里的分析是处在一个现象的层面的,因而也是较为贴近现实的、令人容易理解的。马克思在这里指出,影响商品价格的竞争涉及三个方面,即卖主之间的竞争、买主之间的竞争、卖主同卖主之间的竞争。

我在这里给大家做一个简要概述。在马克思看来,卖主之间的竞争压低了商品的“供给价格”,买主之间的竞争提高了商品的“需求价格”,卖主与买主之间的竞争促使了最终“市场价格”的形成。看完上面的概述,我想读者们应该就不难理解马克思的这几段话了:

\begin{adjustwidth}{2em}{2em}
    \qquad\fangsong
    “同一种商品,有许多不同的卖主供应。谁以最便宜的价格出卖
同一质量的商品,谁就一定会战胜其他卖主,从而保证自己有最大
的销路。于是,各个卖主彼此间就进行争夺销路、争夺市场的斗争。
他们每一个人都想出卖商品,都想尽量多卖,如果可能,都想由他
一个人独卖,而把其余的卖主排挤掉。因此,一个人就要比另一个
人卖得便宜些。于是卖主之间就发生了竞争,这种竞争降低他们所
供应的商品的价格。

但是买主之间也有竞争,这种竞争反过来提高所供应的商品
的价格。

最后,买主和卖主之间也有竞争。前者想买得尽量便宜些,后
者却想卖得尽着贵些。买主和卖主之间的这种竞争的结果怎样,要
依上述竞争双方的对比关系怎样来决定,就是说要看是买主阵营
里的竞争激烈些呢还是卖主阵营里的竞争激烈些。产业把两支军
队抛到战场上对峙,其中每一支军队内部又发生内讧。战胜敌人
的是内部冲突较少的那支军队。”
\end{adjustwidth}

此外,这里还有一个思考点,即对于劳动力这一特殊商品而言,它的价格(即工资)是否也决定于以上三个层面的竞争呢?我倾向认为也是这样的。影响劳动力商品的这种竞争的具体情况同样可以分为三个方面:资本家之间的竞争会提高工人阶级的整体工资水平(但往往资本家会发现,相较于竞争,他们更倾向于联合),工人阶级内部的竞争会压低工人阶级整体的工资水平(工人阶级的内卷),工人阶级同资本家阶级之间的博弈会影响一定历史时期的工资水平。因而从某种意义上说,我认为,工资的大小在马克思看来似乎是一种外生给定的因素,即一定社会生产力水平下的一定历史时期存在着一个特定的工资水平。

\textbf{但上述那种竞争似乎又不是决定商品价格的根本因素。}\footnote{编者注:竞争当然不是决定商品价格的根本因素,价值才是,但这里马克思对于政治经济学的一些概念的认识还是较为模糊。}我认为马克思在这里是想表明,竞争、供求等因素会导致商品价格的上下波动,但商品的价格不会一直“虚高”也不会一直“虚低”,这里隐约地表明着商品的价格是由另一种因素所决定的,马克思发现了这个因素,在本部分内容中马克思将其称作“生产费用”\footnote{编者注:遗憾的是,马克思在这里的叙述还是处在一个模棱两可的状态,根据他的论述,生产费用似乎指的既是生产成本,又是生产价格,甚至还表示着价值。但生产成本、生产价格、价值是截然不同的东西,因而这是一种矛盾的叙述。}。马克思在这里是这么论述的:

\begin{adjustwidth}{2em}{2em}
    \qquad\fangsong
    “我们刚才说过,需求和供应的波动,每次都把商品的价格引导
到生产费用的水平。固然,商品的实际价格始终不是高于生产费
用,就是低于生产费用;但是,上涨和下降是相互抵销的,因此,在
一定时间内,如果把工业中的资本流入和流出总合起来看,就可看
出各种商品是依其生产费用而互相交换的,所以它们的价格是由
生产费用决定的。

价格由生产费用决定这一点,不应当了解成像经济学家们所
了解的那种意思。经济学家们说,商品的平均价格等于生产费用;
在他们看来,这是一个规律。他们把价格的上涨被价格的下降所抵
销,而下降则被上涨所抵销这种无政府状态的变动看作偶然现象。
那末,同样也可以(另一些经济学家就正是这样做的)把价格的波
动看作规律,而把价格由生产费用决定这一点看作偶然现象。可是
实际上,只有在这种波动的进程中,价格才是由生产费用决定的;
我们细加分析时就可以看出,这种波动起着极可怕的破坏作用,并像地震一样震撼资产阶级社会的基础。这种无秩序状态的总运动
就是它的秩序。在这种产业无政府状态的进程中,在这种循环运转
中,竞争可以说是拿一个极端去抵销另一个极端。

由此可见,商品价格是由生产费用这样来决定的:某些时期,
某种商品的价格超过它的生产费用,另一些时期,该商品的价格却
下跌到它的生产费用以下,而抵销以前超过的时期,反之亦然。当
然,这不是就个别产业的产品来说的,而只是就整个产业部门来说
的。所以,这同样也不是就个别产业家来说的,而只是就整个产业
家阶级来说的。”
\end{adjustwidth}
上面那几段论述中的“生产费用”事实上可以理解为“生产价格”。但随之马克思的这段论述又让人陷入了迷惑:
\begin{adjustwidth}{2em}{2em}
    \qquad\fangsong
    “价格由生产费用决定,就等于说价格由生产商品所必需的劳
动时间决定,因为构成生产费用的是:(1)原料和劳动工具\footnote{在1891年的版本中, “劳动工具”改为“劳动工具损耗部分”。——编者注} ,即产
业产品,它们的生产耗费了一定数量的工作日,因而也就是代表一
定数量的劳动时间;(2)直接劳动,它也是以时间计量的。”
\end{adjustwidth}
上面这段话中的“生产费用”更应该被理解为“价值”。因为马克思用生产商品的“必要劳动时间”代替了“生产费用”,这显然指的是商品的价值。不过值得欣慰的是,若是将上面的“生产费用”理解为“商品的价值”的话,事实上在这里马克思已经超越了斯密教条。因为马克思意识到了,商品的价值的实际上是由两个部分组成的,即生产资料的价值的转移(即死劳动的转移、物化劳动的转移)和活劳动的对象化(或者称做活劳动的物化)。

之后,马克思还论述了工人工资一般会处在一个水平,即:
\begin{adjustwidth}{2em}{2em}
\qquad\fangsong
“单个工人所得,千百万工人所得,不足以维持生存和延续后代,
但整个工人阶级的工资在其波动范围内则是和这个最低额相等
的。”
\end{adjustwidth}
这里可以对照《1844年经济学哲学手稿》中的部分内容来看,这里我就不赘述了。
\subsubsection{3.资本是一种社会生产关系}
马克思说:
\begin{adjustwidth}{2em}{2em}
    \qquad\fangsong
“资本包括原料、劳动工具和各种生活资料,这些东西是用以生产新的原料、新的劳动工具和新的生活资料的。资本的所有这些组成部分都是劳动的创造物,劳动的产品,积累起来的劳动。作为进行新生产的手段的积累起来的劳动就是资本。

经济学家们就是这样说的。

什么是黑奴呢?黑奴就是黑种人。上面的说明和这个说明是一样的。

黑人就是黑人。只有在一定的关系下,他才成为奴隶。纺纱机是纺棉花的机器。只有在一定的关系下,它才成为资本。脱离了这种关系,它也就不是资本了,就像黄金本身并不是货币,沙糖并不是沙糖的价格一样。”
\end{adjustwidth}
马克思在这里是想表明,如果仅仅把资本看作是“作为新生产的手段的积累起来的劳动”是不够的。这种解释同那种认为“黑奴就是黑种人”的观点并无差异。借用复旦大学吴晓明老师的说法,这种解释只能被称作是“不错”,但“不错”不能被称为“真”,譬如我问你水果是什么?你告诉我是苹果和香蕉,那么我应该说“不错”,但如果我是一个生物学教授,我要你回答的是水果的定义,而不要你告诉我苹果、香蕉和橘子是水果,但我也不能说你不对,只能说“不错”。因此,当我们在考察某一事物时,绝不可以仅仅止步于外部的观察层面,必须要将对事物所处的社会关系的考察同对事物本身的考察联系起来。这是进入马克思主义政治经济学研究的一项基本要求。同样,关于这一点,请读者们细细思索,我就不再赘述了。

因此,不难理解,马克思会说:

\begin{adjustwidth}{2em}{2em}
    \qquad\fangsong
    “资本也是一种社会生产关系。这是资产阶级的生产关系,是资产阶级社会的生产关系。构成资本的生活资料、劳动工具和原料,难道不是在一定的社会条件下,不是在一定的社会关系下生产出来和积累起来的吗?难道这一切不是在一定的社会条件下,在一定的社会关系内被用来进行新生产的吗?并且,难道不正是这种一定的社会性质把那些用来进行新生产的产品变为资本的吗?”
\end{adjustwidth}

\subsubsection{本期小结}
本期读书会同样见到了许多新面孔,大家讨论的也是较为热烈的。本期我们讨论的内容比较多,这一方面是因为我们想加快一下阅读的速度,另一方面则是因为马克思的这部分内容的叙述相较于其他文本而言算是较为容易理解的。此外,在本期读书会我们又讨论了一下置盐定理,并且在读书会之后我在群里给大家发了我临时编写的简化版的置盐定理概述,我将在本期读书会记录的附录页中附上我编写的更为严谨的置盐定理概述\footnote{编者注:当然,我依然会省略大部分的数学推导过程。},请大家按需查阅。总之,这同样是一期收获满满的讨论会。

\newpage
\subsection{本期附录}
\subsubsection{1.置盐定理概述(Okishio Theorem)}
\noindent\textbf{这是相较于我之前在群里发的更为严谨的置盐定理概述。}\footnote{编者注:其实也不是特别严谨,因为我省略了大部分的数学推导。} 接下来我将更为严谨地概述置盐信雄在1961年发表的《技术变革与利润率》\footnote{编者注:该文于2010年被我国学者骆桢、李怡乐和孟捷等人翻译成中文,刊登在《教学与研究》期刊2010年第7期。}一文。

\vspace{0.5cm} %设置垂直距

我们知道,马克思关于\textbf{一般利润率下降趋势规律}的逻辑可以被概括为3个方面的内容:(1)资本家之间的竞争迫使他们引进新的生产技术以提高劳动生产率;(2)劳动生产率的提高通常会提高资本的有机构成;(3)资本有机构成的提高会导致一般利润率的下降,虽然剩余价值率的提高会阻碍一般利润率的下降,但总的来说一般利润率还是处于下降的趋势。

针对以上内容,置盐信雄提出了3方面的疑问:
\begin{tcolorbox}[colback=gray!20, colframe=gray!100, sharp corners, leftrule={3pt}, rightrule={0pt}, toprule={0pt}, bottomrule={0pt}, left={2pt}, right={2pt}, top={3pt}, bottom={3pt}] 
\textbf{(1)}资本家引入的新生产技术一定会提高劳动生产率吗?

\textbf{(2)}提高劳动生产率的生产技术通常会提高资本的有机构成吗?

\textbf{(3)}为什么有机构成提高对一般利润率产生的影响会大于剩余价值率的提高对一般利润率产生的影响呢?

\end{tcolorbox}


对于上述3点疑问,置盐信雄分别做出了回答。

\paragraph{对于问题(1)。}置盐认为资本家引入新生产技术遵循的并不是马克思所认为的生产率准则,而是成本准则。并且他指出,“生产率准则”不同于“成本准则”。置盐用数学的形式对两者做出了区分,首先一单位商品中所蕴涵的劳动量可以表示为:
\begin{equation}\tag{1}
    t_i=\Sigma a_{ij}t_j+\tau_i \quad (i=1,2,...,n)
\end{equation}
其中,$t_i$为生产一单位第$i$种商品所耗费的直接或间接地必要劳动量,$a_{ij}$为生产一单位第$i$种商品所必须的第$j$种商品的直接投入量,$\tau_i$表示生产一单位第$i$种商品所需的直接劳动量。

因此,对于第$k$产业而言,新技术能提高生产第$k$中商品的劳动生产率的条件是:
\begin{equation}\tag{2}
    \Sigma a_{kj}t_j+\tau_k >\Sigma a'_{kj}t_j+\tau'_k
\end{equation}
其中$(a'_{k1},a'_{k2},...,a'_{kn},\tau'_k)$表示第$k$产业中的新技术,\textbf{式(2)即为“生产率准则”}。

另一方面,成本准则可表示为:
\begin{equation}\tag{3}
    \Sigma a_{kj}q_j+\tau_k >\Sigma a'_{kj}q_j+\tau'_k
\end{equation}
其中,$q_j=\frac{p_j}{w}$,$p_j$和$w$分别为第$j$种商品的价格和货币工资率。可见,只有对于所有的$i$而言$q_i=t_i$时,“生产率准则”和“成本准则”才等价。而对于资本主义经济而言,每个行业都必须存在正的利润,从而必须满足下列不等式:
\begin{equation}\tag{4}
    q_i> \Sigma a_{ij}+\tau_i
\end{equation}
因此,对于所有的$i$而言,都有$q_i>t_i$。那末,对于式(2)和式(3)而言,二者不是等价的,也即是说,“生产率准则”不同于“成本准则”。

\paragraph{对于问题(2)。}置盐认为这是一个经验统计意义上的研究,即提高劳动生产率的技术是否会提高有机构成需要通过统计研究来说明,换言之,置盐认为劳动生产率与有机构成之间的关系需要通过实证研究来说明。

\paragraph{对于问题(3)。}置盐认为大多数的结论和通常的回答如下:
\begin{equation}\tag{5}
    \frac{m}{c+v} \leq \frac{m+v}{c}
\end{equation}
只有当$v=0$时,上式才能取等号,换句话说,此时工人完全无偿劳动。可见,一般利润率存在着一个上界,这个上界$\frac{m+v}{c}$便是活劳动与物化劳动的比值。按照马克思的观点,生产的活劳动与物化劳动的比值会逐渐降低,因此虽然平均利润率会上下波动,但它的整体趋势是伴随着其上界的下降而下降的。

但是置盐认为,一般利润率不应该通过\textbf{价值形式}来表示,也即是说他认为$r=\frac{m}{c+v}$这一式子是错误的。
置盐通过斯拉法体系对此进行了修正,置盐认为一般利润率$r$应是由下列方程组决定的:
\begin{equation}\tag{6}
    \begin{cases}
        q_i=(1+r)(\Sigma a_{ij}q_j+\tau_i)\quad(i=1,2,...,n)\\
        1=\Sigma b_iq_i
    \end{cases}
\end{equation}

事实上,对于置盐给出的方程组(6)而言,每一个等式两边同时乘工资率$w$,便是斯拉法体系对于生产价格的规定\footnote{编者注:不过这里和传统斯拉法体系中的后付工资不同,这里的工人的工资被纳入到成本之中了,是预付的。},即:
\begin{equation}\tag{7}
    \begin{cases}
         p_i=(1+r)(\Sigma a_{ij}p_j+w\tau_i)\quad(i=1,2,...,n)\\
        w=\Sigma b_ip_i
    \end{cases}
\end{equation}
这里的$b_i$可以理解为工人消耗一单位的劳动时间所能够以工资形式换来的第$i$种消费资料的实物量。

将方程(7)与方程(1)联立,可得\footnote{编者注:此处省略数学推导过程}:
\begin{equation}\tag{8}
    r<\frac{\tau_i}{\Sigma a_{ij}t_j}
\end{equation}
置盐指出对于某些$i$而言,式(8)的右边表示的含义同式(5)的右边表示的含义相同,即生产过程中活劳动与物化劳动的比值。因此,虽然置盐认为式(5)不正确,但式(5)表示的结论同式(8)相比仍具有一定的合理性,然而置盐认为这种合理性只有对于某些$i$而言才是成立的,因此还需进一步考察资本家引进新生产技术的类型。

置盐认为,资本家在进行新技术引进时所遵循的是“成本准则”。也即是说对于第$k$行业的生产技术而言,新生产技术向量$(a'_{k1},...,a'_{kn},\tau'_k)$和原生产技术向量$(a_{k1},...,a_{kn},\tau_k)$满足式(3)中的不等式,那末便有如下结论\footnote{编者注:此处省略数学推导过程。}:
\begin{tcolorbox}
(1)在实际工资率\footnote{编者注:实际工资率其实就是$\Sigma b_i$,或者把它理解为工人的实物工资量。}不变的情况下,如果引入新技术的行业是“非基本品行业”,则一般利润率不受影响。

(2)在实际工资率不变的情况下,如果引入新技术的行业是“基本品行业”,则一般利润率必然上升。  
\end{tcolorbox}


\vspace{0.5cm} %设置垂直距

\textbf{上述便是置盐定理的的基本内容。}

\vspace{0.5cm} %设置垂直距

这里补充说明一下基本品行业,在置盐看来,基本品行业指的是工资品行业以及与工资品行业不可分的行业,即$b_i>0$的行业。在置盐信雄看来,当给定工资率时,剩余价值率取决于基本品行业的生产技术,这是因为
\begin{equation}\tag{9}
\frac{m}{v}=\frac{\tau_i-\tau_i\Sigma b_jt_j}{\tau_i\Sigma b_jt_j}=\frac{1-\Sigma b_jt_j}{\Sigma b_jt_j}    
\end{equation}
可见,在式(9)中,剩余价值率只受到$b_j>0$的部门(即基本品部门)的生产技术的影响;而对于$b_j=0$的部门(即非基本品部门)而言,无论$t_j$如何变化,对于整体剩余价值率的变化是没有影响的。


因而在置盐看来,一般利润率下降趋势并不是资本主义制度内生的规律,相反,利润率下降在置盐眼里是阶级斗争(工人阶级争取提高实际工资率)的外生结果。

事实上,若是将置盐定理用矩阵的形式表示,那末置盐定理在形式上便是\textbf{Perron-Frobenius定理}的一个推论,这一形式是能够被数学手段严格证明的。但形式逻辑的严密,并不意味与现实情况相契合,因此,对于置盐定理的前提假设及其结论而言,仍然是值得进一步商榷的。
\newpage
\section{第五期:《雇佣劳动与资本》(3)}
\subsection{读书会记录}
2023年11月20日18:00—20:00,我们在东北林业大学奥林学院203教室开展了第五期的读书会。本期读书会我们依然是继续探讨了马克思的《雇佣劳动与资本》中的部分内容。本期讨论我们从老版本《雇佣劳动与资本》第“三”节中的“一些商品即一些交换价值的总和究竟是怎样成为资本的呢?”开始,一直读到老版本中的第“五”节之前。接下来我将对本期讨论会的内容进行简要概述。
\subsubsection{1.“资本的必要前提”与“必要的资本前提”:一种异化的关系}
事实上,马克思在他写作的后期很少用“异化”这个词了,并且在本部分内容\footnote{编者注:老版本的第“三”节的剩余内容,具体在我编排的讲义的29—31页。}中马克思也没有提到“异化”这个词。但是我思来想去,并结合我们本期讨论的记录,我认为用“异化”这个词来概括这部分的核心思想是较为合适的。我们记得,在上期读书会记录的最后一小节中,马克思谈论了“资本之实质是一种社会生产关系”的观点,本期的这部分内容实质上是顺着上述观点深入考察了资本这种“社会生产关系”的具体细节之处。在我看来,即使在这部分的内容中马克思的政治经济学批判理论还未成熟,但是经由这部分的论述所带给我们的哲学思考依然是极为重要的。

马克思在这里说:
\begin{adjustwidth}{2em}{2em}
    \qquad\fangsong
“它成为资本,是由于它作为一种独立的社会力量,即作为一种属于社会一部分的力量,借交换直接的、活的劳动\footnote{在1891年的版本中,“劳动”改为“劳动力”。——编者注}而保存下来并增殖起来。除劳动能力以外一无所有的阶级的存在是资本的必要前提。

只是由于积累起来的、过去的、物化的劳动支配直接的、活的劳动,积累起来的劳动才变为资本。

资本的实质并不在于积累起来的劳动是替活劳动充当进行新生产的手段。它的实质在于活劳动是替积累起来的劳动充当保存自己并增加其交换价值的手段。”
\end{adjustwidth}
我们在这里可以看到,马克思认为“除劳动能力以外一无所有的阶级的存在是资本的必要前提”,这是完全正确的。为什么这么说呢?因为资本的一个关键(本质)属性就是增殖,而这种“增殖”只有通过对劳动力商品的使用(即活劳动的对象化、物化)才能够得以实现。因此,在这个意义上说来,工人阶级的劳动是资本得以发生的必要前提。这即意味着,资本一旦脱离工人阶级的劳动,就无法实现自身。

上述这些观点是不难理解的。然而现实情况却表现出了一种颠倒的形式——似乎资本不再以雇佣劳动为前提,而是雇佣劳动以资本为前提。我们可以看到马克思在之后是这么论述的:
\begin{adjustwidth}{2em}{2em}
    \qquad\fangsong
“资本和雇佣劳动\footnote{在1891年的版本中,“资本和雇佣劳动”改为“资本家和雇佣工人”。——编者注}是怎样进行交换的呢?

工人拿自己的劳动\footnote{在1891年的版本中,“劳动”改为“劳动力”。——编者注}换到生活资料,而资本家拿归他所有的生活资料换到劳动,即工人的生产活动,亦即创造力量。这种力量不仅能补偿工人所消费的东西,并且还使积累起来的劳动具有比以前更大的价值。工人从资本家那里得到一部分现有的生活资料。这些生活资料对工人有什么用处呢?用于直接消费。可是,如果我不把靠这些生活资料维持我的生活的一段时间用来生产新的生活资料,即在消费的同时用我的劳动创造新价值来补偿那些因消费而消失了的价值,那末我一把这些生活资料消费完,它们对于我就算是完全白耗费了。但是,工人为了换到生活资料,正是把这种贵重的再生产力量让给了资本家。因此,对于工人本身来说,这种力量是白耗费了。

举一个例子来说吧。有个农场主每天付给他的一个短工五银格罗申。这个短工为得到这五银格罗申,就整天在农场主的田地上干活,保证农场主能得到十银格罗申的收入。农场主不但收回了他付给短工的价值,并且还把它增加了一倍。可见,他有成效地、生产性地使用和消费了他付给短工的五银格罗申。他拿这五银格罗申买到的正是一个短工的能生产出双倍价值的农产品并把五银格罗申变成十银格罗申的劳动和力量。短工则拿他的生产力(他正是把这个生产力让给了农场主)换到五银格罗申,并用它们换得迟早要消费掉的生活资料。所以,这五银格罗申的消费有两种方法:对资本家来说,是有生产性的,因为他用这五银格罗申换来的劳动力使他得到了十银格罗申;对工人来说,是非生产性的,因为他用这五银格罗申换来的生活资料永远消失了,他只有再和农场主进行同样的交换才能重新取得这些生活资料的价值。这样,资本以雇佣劳动为前提,而雇佣劳动又以资本为前提。两者相互制约;两者相互产生。”
\end{adjustwidth}
上面这一大段话想表达的观点简要说来就是,工人为了生存(即换取他所必须的生活资料),他就必须得把自己的劳动力出卖给资本家(劳动力同资本进行交换),他就不得不使自己被资本支配。这样一来,似乎“雇佣劳动”又是以资本为前提的了。

我们认为,上述那种近乎矛盾的情况是资本主义生产关系所造成的异化现象。关于这一点,实际上我们在讨论会中已经很细致地讨论过了,我在这里就进行一个简要地回顾。“异化”这个范畴我们可以将其理解为这样的一种机制:主体在创造客体的过程中丧失了自己的主体性。工人阶级的劳动使资本得以发生,然而工人阶级的劳动却无时无刻不受制于资本,这种情况便是一种典型的“异化”关系。并且这种“异化”关系又随着资本主义生产方式\footnote{编者注:完全可以认为,资本主义生产方式本身就内蕴着一种异化的关系。}的进一步发展而使其遮蔽性日益得到稳固,以至于马克思带着一种反讽的语气如是说道:
\vspace{0.5cm} %设置垂直距
\begin{adjustwidth}{2em}{2em}
\qquad\fangsong
“千真万确呵!工人若不受雇于资本家就会灭亡。”
\end{adjustwidth}

\vspace{0.5cm} %设置垂直距
然而让我们进行进一步地反思,真的是“工人离开了资本家就会灭亡”吗?若是从现实情况来看,似乎是这样的。等等!我们切不要轻易地做出结论。正如我们在上期读书会所强调的那样,对问题的考察必须要将其置于一定的社会关系之中,正是由于资本主义生产关系所体现出的异化性质,因而建筑于这种生产关系之上的现实表象必然会产生一种颠倒的形式。也即是说,在资本主义生产关系下,仿佛资本家阶级与工人阶级是互为补充、不可分割的:资本家离不开工人,工人也离不开资本家,二者失去了对方都无法过活。然而这是一种非常荒谬的看法,这种看法就好比说:仿佛寄生虫与寄主是不可分割的,仿佛寄主同寄生虫一样,离开了对方就无法过活了!事实上,认为资本家与工人二者无法分离的观点恰恰是由于资本主义生产方式所产生的意识形态之幻象,这是一种资本主义的遮蔽。

因此,马克思最后会说:
\begin{adjustwidth}{2em}{2em}
    \qquad\fangsong
    “断言资本的利益和劳动的利益\footnote{注:在1891年的版本中,“劳动的利益”改为“工人的利益”。——编者注}是一致的,事实上不过是说资本和雇佣劳动是同一种关系的两个方面罢了。一个方面制约着另一个方面,就如同高利贷者和挥霍者相互依存一样。

当雇佣工人仍然是雇佣工人的时候,他的命运是取决于资本的。所谓工人和资本家的利益一致就是这么一回事。”
\end{adjustwidth}
关于这一点,我也同样不再赘述了,留给读者们细细思索了。

\subsubsection{2.名义工资、实际工资与相对工资}
在老版本《雇佣劳动与资本》的第“四”节中,马克思较为细致地探讨了工人阶级的\textbf{名义工资
、实际工资与相对工资}三者的差别,并着重强调了相对工资所具有的不同于前两者的重要性地位。

马克思论述了当名义工资不变时,实际工资可能会发生的变化。具体内容我就不赘述了,归结起来马克思想表达这样一种意思:当名义工资(即劳动力商品的货币价格)不变时,实际工资(即用名义工资换取的实物量)可能会发生变化。因而马克思会说:
\begin{adjustwidth}{2em}{2em}
    \qquad\fangsong
 “因此,我们谈到工资的增加或降低时,不应当仅仅注意到劳动的货币价格,仅仅注意到名义工资。”
\end{adjustwidth}
但这是否意味着工人只需关注他自身的实际工资就好了呢?显然不是这样的,因为在马克思看来,工人的实际工资也是可以随着资本主义生产方式的发展而提高的\footnote{编者注:其实这是一个社会现实,随着劳动生产率的发展,工人所占有的消费资料的绝对量事实上是在不断提升的。}。事实上,我认为马克思在这里是想强调,对于工人阶级而言,真正有原则性意义的“工资”所代表的应是一种“对比关系”。而为什么这种“对比关系”是如此重要的呢?在我看来,是因为这种“对比关系”反映了资本家和工人阶级之间的斗争。

同样,为了更为清晰地说明在工资与利润之间存在着的“斗争因素”,让我们做一个简单的政治经济学分析\footnote{编者注:为了使对问题的分析更加简洁,这里先不考虑马克思的生产价格理论,也即是说,在这里我们认为工资=可变资本,利润=剩余价值。}:\begin{fangsong}
    在资本主义实际生产中,劳动过程与价值增殖过程事实上是不可分割的,二者统一于资本主义生产过程本身。这即是说,在生产过程中创造的新价值事实上是一个整体,这个整体的所有权是归资本家所有的,且这个整体的总量等于可变资本与剩余价值的和。资本家会在生产结束之后抽调这一整体其中的一部分来补偿他预先支付给工人的工资量,而剩下的那部分,便是剩余价值。
\end{fangsong}

可见,工人的工资和资本家的利润之间具有一种对抗性的关系。因此,无论是名义工资还是实际工资,都无法体现出这种对抗关系,它们二者都是处在现象层面,还没有深入到问题的实质。就像我们当时所讨论的那样,当劳动的强度与复杂程度不变时,无论劳动生产率发生怎样的变化,一小时的劳动创造的价值量是不会发生任何变化的\footnote{编者注:在这里有必要提醒一下读者,在进入马克思劳动价值论的过程中,请务必突破实物主义的思考范式,弄清使用价值与价值之间的辩证关系。}。并且这一小时劳动所创造的价值量在量的层面可以被划分为可变资本与剩余价值,而按照何种比例划分,便体现了资本家和工人阶级之间的博弈。

因此马克思会说:
\begin{adjustwidth}{2em}{2em}
    \qquad\fangsong
    “所谓资本迅速增加对工人有好处的论点,实际上不过是说:工
人把他人的财富增殖得愈迅速,落到工人口里的残羹剩饭就愈多,
能够获得工作和生活下去的工人就愈多,依附资本的奴隶人数就
增加得愈多。
这样我们就看出:

即使最有利于工人阶级的情势,即使资本的尽快增加如何改
善了工人的物质生活状况,也不能消灭工人的利益和资产者即资
本家的利益之间的对立状态。利润和工资仍然是互成反比的。

假如资本增加得迅速,工资是可能提高的;可是资本家的利润
增加得更迅速无比。工人的物质生活改善了,然而这是以他们的社
会地位的降低为代价换来的。横在他们和资本家之间的社会鸿沟扩大了。”
\end{adjustwidth}
同样,我在这里就不继续赘述了,请读者们多读几遍原文,细细思索。
\subsubsection{本期小结}
本期读书会人数不是太多,但是讨论的依旧非常热情。本期讨论结束之后,我们本学期的读书计划已经顺利完成一半了。经过这五期的讨论,我想各位师友们应该会认同,马克思主义政治经济学实际上是一门深奥且非常富有研究价值的学问,对于马克思的经典著作,我们是有必要静下心来仔细研读的。在后续几期读书会中我们会尽量完成余下的讨论内容\footnote{编者注:我们还剩《雇佣劳动与资本》的末尾部分、《政治经济学批判》导言与序言、《哥达纲领批判》这些内容。}。总而言之,本期读书会同样是收获满满的。

\newpage
\section{第六期:《雇佣劳动与资本》(完)+《<政治经济学批判>导言》(1)}
\subsection{读书会记录}
2023年11月23日18:00—20:00,我们在东北林业大学奥林学院203教室开展了第六期的读书会。本期读书会我们结束了马克思《雇佣劳动与资本》的讨论,并开启了马克思的另一篇政治经济学的经典著作《<政治经济学批判>导言》的讨论。接下来我将对本期讨论会的内容进行简要概述。

\subsubsection{1.资本主义的“进步强制”}
在老版本《雇佣劳动与资本》第“五”节中,马克思论述了资本家之间的竞争导致社会平均劳动生产率提高的作用。大致的逻辑可以表述为:一个资本家为了将其他资本家逐出市场,他就必须使得本部门商品的价格低于其他部门商品的价格,但是,商品的价格是受到生产成本的制约的,因而他若是要降低本部门商品的价格,就需要降低本部门商品的生产成本,而商品生产成本的降低又意味着本部门劳动生产力(率)的提高。

可见,对于单个资本家而言,他若是要在市场竞争中获得优胜,他就需要将自己内部的个别劳动生产力(率)提高至社会平均劳动生产力(率)之上。在资本主义市场经济中\footnote{编者注:其实这里蕴含着一个假设,即资本主义处于自由竞争时期。},对这个资本家是这样,对其他资本家也是一样的道理,因此从整体来看,资本主义市场竞争会促使社会生产力的进步。因此马克思会说:
\begin{adjustwidth}{2em}{2em}
    \qquad\fangsong
    “一个资本家只有在自己更便宜地出卖商品的情况下,才能把另一个资本家逐出战场,并占有他的资本。可是,要能够贱卖而又不破产,他就必须廉价生产,就是说,必须尽量增加劳动的生产力。而增加劳动的生产力的首要办法是更细地分工,更全面地运用和经常地改进机器。内部实行分工的工人大军愈庞大,应用机器的规模愈广大,生产费用相对地就愈迅速缩减,劳动就更有效率。因此,资本家之间就发生了各方面的竞争:他们竭力设法扩大分工和增加机器,并尽可能大规模地使用机器。”
\end{adjustwidth}

可见,对于资本主义社会制度而言,它似乎内在地包涵着一种运动机制,这一机制的内在驱动力就是\textbf{资本贪婪的本性、资本的逐利性}。并且这种机制表明了:资本只要不增殖,它就会灭亡,因而它必须为自己创造增殖的条件。事实上,我们认为马克思上面的那段论述很像《共产党宣言》中的部分内容,我在这里给大家截取出来:
\begin{adjustwidth}{2em}{2em}
    \qquad\fangsong
    “资产阶级除非对生产工具,从而对生产关系,从而对全部社会关系不断地进行革命,否则就不能生存下去。反之,原封不动地保持旧的生产方式,却是过去的一切工业阶级生存的首要条件。生产的不断变革,一切社会状况不停的动荡,永远的不安定和变动,这就是资产阶级时代不同于过去一切时代的地方。一切固定的僵化的关系以及与之相适应的素被尊崇的观念和见解都被消除了,一切新形成的关系等不到固定下来就陈旧了。一切等级的和固定的东西都烟消云散了,一切神圣的东西都被亵渎了。人们终于不得不用冷静的眼光来看他们的生活地位、他们的相互关系。”\footnote{编者注:见《共产党宣言》。}
\end{adjustwidth}
\vspace{0.5cm}
让我们从这一点继续引申下去,资本主义的这种内在机制表明了一个事实,即资本主义社会存在着一种“进步强制”。即以资本为原则的社会必须不断取得进步以维持价值的增殖。换句话说,资本主义社会的生产力提高\textbf{往往}并不是因为资本家非常高尚、想为人类做出贡献。恰恰相反,社会生产力的提高事实上是资本家们不得已而为之的结果,即资本家们为了追逐自己利益的最大化,必须要提高劳动生产力(率)。

\subsubsection{2.机器排挤工人?资本排挤工人!}
让我们顺着前面的资本主义的“进步强制”继续讨论下去。我们已经知道了,资本家之间的竞争会内在地促进社会生产力的提高,而这首先又表现为机器和分工的扩大化。接下来,马克思探讨了机器和分工的扩大化对于工人阶级而言造成了什么样的影响。我们可以看到马克思在下面是这么说的:
\begin{adjustwidth}{2em}{2em}
    \qquad\fangsong
“其次,分工愈细,劳动就愈简单化。工人的特殊技巧失去任何价值。工人变成了一种简单的、单调的生产力,就不需要体力上或智力上的特别本事和技能了。他的劳动成为人人都能从事的劳动了。因此,工人受到四面八方的排挤;我们还要提醒一下,一种工作愈简单,就愈容易学会,为学会这种工作所需要的生产费用愈少,工资也就愈降低,因为工资像一切商品的价格一样,是由生产费用决定的。

总之,劳动愈是不能给人以乐趣,愈是令人生厌,竞争也就愈激烈,工资也就愈减少。工人想维持自己的工资总额,就得多劳动:多工作几小时或者在一小时内造出更多的产品。这样一来,工人为贫困所迫,就愈加重分工的极危险的后果。结果就是:他工作得愈多,他所得的工资就愈少。这里的原因很简单:他工作得愈多,他给自己的工友们造成的竞争就愈激烈,因而就使自己的工友们变成他自己的竞争者,这些竞争者也像他一样按同样恶劣的条件出卖自己。所以,原因同样很简单:他归根到底是自己给自己,即自己给作为工人阶级一员的自己造成竞争。

机器也发生同样的影响,而且影响的规模更大得多,因为机器用不熟练的工人代替熟练工人,用女工代替男工,用童工代替成年工;因为在最先使用机器的地方,机器就把大批手工工人抛到街头上去,而在机器日益完善、改进或为生产效率更高的机器所替换的地方,机器又把一批一批的工人排挤出去。我们在前面大略地描述了资本家相互间的产业战争。这种战争有一个特点,就是致胜的办法与其说是增加劳动大军,不如说是减少劳动大军。统帅们即资本家们相互竞赛,看谁能解雇更多的产业士兵。”
\end{adjustwidth}
不难看出,马克思在这里认为机器和分工的扩大化导致了工人阶级整体利益的削减。劳动生产率的提高缩减了工人的数量,越来越多的工人沦为无业游民,以前需要10个人的工作现在可能只需要5个人了,剩下的那5个人便成为了资本主义制度下“生产力进步”的牺牲品。此外,在这里还产生了一个现象:工人的竞争愈发激烈了。这在现实情况中表现为\textbf{机器排挤工人,工人之间内卷程度的加深}。

从人类历史发展的脉络来看,机器的使用、生产分工的科学化……总之一句话,劳动生产率的提高显然是一件具有积极意义的事情。然而对于资本主义社会的现实而言,劳动生产率的提高不仅没有令工人感到欣喜,反而让工人变得更加麻木、更加失去自我。在这种状况下,工人感到自己被机器排挤了,进而不得不同自己阶级内部的其他成员之间进行竞争,在这一竞争过程中,个别工人为了取得胜利,则必须要更为努力地向资本家献殷勤,让自己的劳动力商品更为廉价。而这种现象又表明了,束缚着工人阶级的锁链变得更为沉重了。

这种情况下,对于工人而言,他们往往看到的是自身同机器之间的矛盾,早期的卢德运动\footnote{编者注:卢德运动(Luddite Movement)是英国工人以破坏机器为手段反对工厂主压迫和剥削的自发工人运动。首领称为卢德王,故名。相传,莱斯特郡一个名叫卢德(Ludd)的工人,为抗议工厂主的压迫,第一个捣毁织袜机。1811年初卢德运动开始形成高潮。其中心是诺丁汉郡,1811年,诺丁汉郡的袜商不顾行业规矩,生产一种劣质长筒袜,压低袜子价格,严重冲击了织袜工人的正常收入。一些织工秘密组织起来,以“路德将军”的名义捣毁商人的织袜机。1812年,英国国会通过《保障治安法案》,动用军警对付工人。1813年政府颁布《捣毁机器惩治法》,规定可用死刑惩治破坏机器的工人。1813年在约克郡绞死和流放破坏机器者多人。1814年企业主又成立了侦缉机器破坏者协会,残酷迫害工人。但运动仍继续蔓延。1816年这类运动仍时有发生。在当代,“卢德分子”一词用于描述工业化、自动化、数字化或一切新科技的反对者。他们也被称为“新卢德分子”。}便是工人联合起来对机器进行反抗的典型例子。但实际上,让我们深入思考一下便不难发现,工人同机器之间的对立仅仅是一种假象,这只不过是工人同资本之间的矛盾一种外在表现。具体内容就不在这里深入讨论了。总的来说,我们认为,说“机器排挤工人”是不完全正确的,究其实质,应是“资本排挤工人”。事实上,机器的广泛运用、生产力的提高等因素都是无产阶级解放的必要条件,按其本性而言是有利于工人阶级的,只不过资本主义社会的生产关系使得生产力的发展展现出了一种颠倒的迹象,生产力的提高仅仅成为了资本主义实现其自身要求的手段。

因此,总的来说,马克思的观点认为,资本家(其只不过是资本的人格化)与工人之间的利益是对立的。资本的迅速增殖是以更大规模地牺牲工人阶级的整体利益为前提的。资本主义社会无论处在繁荣或萧条时期,受苦的总是工人。因此马克思会在最后说道:
\begin{adjustwidth}{2em}{2em}
    \qquad\fangsong
    “但是,资本不光靠剥削劳动来生活。像显贵的野蛮的奴隶主一样,资本也要他的奴隶们陪葬,即在危机时期要使大批的工人死亡。由此可见:如果说资本增长得迅速,那末工人之间的竞争就增长得更迅速无比,就是说,资本增长得愈迅速,工人阶级的就业手段即生活资料就相对地缩减得愈厉害”
\end{adjustwidth}
\vspace{0.4cm}
到这里,我们对于《雇佣劳动与资本》的讨论就暂告一段落了,读者们若是还觉得意犹未尽的话,就多读几遍原文吧。

\subsubsection{3.《政治经济学批判》导言(1)}
本期讨论会我们把马克思的《政治经济学批判》导言(下文简称《导言》)开了个头,我们一致认为,这部分内容的难度相较于前几期要更高一些。在进一步展开我们的讨论内容之前,我首先跟读者们交代一下《导言》的写作背景与内容概述。

\paragraph{背景及内容概述}\begin{fangsong}
《 导言 》 产生于 1857 年 8 月底 。 这一手稿是
马克思为自己计划中的政治经济学巨著而写的,但后来没有发表。马
克思在《政治经济学批判》第 1 分册的《序言》中说明了这个手稿没
有写完和没有发表的原因:“我把已经起草好的一篇总的导言压下
了,因为仔细想来,我觉得预先说出正要证明的结论总是有妨害的。”

在这篇具有重要科学价值的手稿中 , 马克思比在任何别的地方
都更详细地论述了他关于政治经济学的对象和方法的观点 。 马克思
指出 , 资产阶级经济学家把生产与分配 、 交换 、 消费的内在联系割裂
开来和并列起来 , 认为发生变化的只是分配方式 , 往往把分配提到首
位 , 把它当作政治经济学的研究对象 , 并把资本主义说成是历史上永恒的制度 。 马克思同他们相反 , 说明生产不是某种抽象的永恒不变的
东西 , 它是由特定的社会历史条件决定的 。 他阐明了生产 、 分配 、 交
换 、 消费的辩证统一和相互作用 , 指出它们是一个总体的各个环节 。
他得出结论说 , 生产不仅是这种统一的出发点 , 而且是决定因素 , 而
分配形式不过是生产形式的另一种表现 。 马克思认识到生产是一定
社会性质的生产 , 并把它当作自己的研究对象 。

马克思阐述了政治经济学的从抽象上升到具体的方法 , 指出它
是 “ 科学上正确的方法 ” , 同时批评了黑格尔对这一
方法的唯心主义观点 。 按照马克思对从抽象上升到具体的方法的辩
证唯物主义的解释 , 作为理论分析的出发点的具体 , 在研究的结果中
表现为多样性的统一 、 许多规定的综合 。 马克思理论中的科学抽象是
同作为它们的前提的具体现实不可分割地联系在一起的 , 而从简单
到复杂的抽象思维的进程 , 总的说来是同现实的历史过程相一致的。

马克思从他对政治经济学的对象和方法的见解出发 , 在 《 导言 》
中拟定了他的未来的经济学巨著的结构的最初计划 。 这一结构包括
资产阶级社会一切重要的方面 , 拟分为五篇: “ (1) 一般的抽象的规
定 , 因此它们或多或少属于一切社会形式 … … (2) 形成资产阶级社会
内部结构并且成为基本阶级的依据的范畴。 资本 、 雇佣劳动 、 土地所
有制 。 它们的相互关系 。 城市和乡村 。 三夫社会阶级 。 它们之间的
交换 。 流通 。 信用事业 ( 私人的 ) 。 (3) 资产阶级社会在国家形式上的
概括 。 就它本身来考察 。 ‘ 非生产 ’阶级 。 税 。 国债 。 公共信用 。 人
口 。 殖民地 。 向国外移民 。 (4)生产的国际关系 。 国际分工 。 国际交
换 。 输出和输入 。 汇率 。 (5) 世界市场和危机 。 ”

在 《 导言 》 的最后一节中 , 马克思从社会发展的经济基础出发 , 也
研究了属于政治的和意识形态的上层建筑领域的某些过程 , 探究这些过程对经济基础的依赖关系和反作用 , 论述了艺术作为社会意识
的一种形式的特点 。 他指出 , 物质生产在社会生活中的决定作用并不
排除艺术和文学这样一些上层建筑要素的相对独立性 。 他以古希腊
的艺术和莎士比亚的创作为例 , 说明艺术的兴盛并不是必然同经济
和社会的发展完全一致的 。 这是由错综复杂的情况决定的 。 上层建
筑对基础的依赖关系 , 是不能简单化地加以阐述的 。\footnote{编者注:以上内容摘选自马恩选集第二版第30卷。}
\end{fangsong}

\paragraph{接下来正式进入我们的讨论。}首先,马克思在《导言》的开篇说:

\begin{adjustwidth}{2em}{2em}
    \qquad\fangsong
    “在社会中进行生产的个人,因而,这些个人的一定社会性质的生产,自然是出发点。”
\end{adjustwidth}

我们认为,马克思在这里强调的是社会性。这是一个老生常谈的话题,简言之,不能抽离历史关系(社会关系)去孤立地考察个人。马克思认为鲁滨逊的故事是一种“毫无想象力的虚构”,这仅仅是美学上的错觉。马克思在这里批判了斯密、李嘉图、卢梭等人的自然主义式的错觉,即认为存在一个通过自然形成的、且合乎自然的“人类天性”,并幻想着达到这种“人类天性”。这些人的错误在于他们没有从社会历史关系中去考察他们所认为的“人类本性”,他们没有意识到他们所认为的这种“人类本性”恰恰是历史的结果,而误以为这种所谓的“人类本性”是历史的起点。关于这一点,我就不再继续赘述了。
\vspace{0.2cm}

之后,马克思的这句话引起了我们的讨论:
\vspace{0.2cm}
\begin{adjustwidth}{2em}{2em}
    \qquad\fangsong
    “只有到十八世纪,在‘市民社会’中,社会结合的各种形式,对个人说来,才只是达到他私人目的手段,才是外在的必然性。”
\end{adjustwidth}
\vspace{0.3cm}

如何理解“外在的必然性”呢?我们认为\footnote{编者注:这是我们讨论会的观点,可能不准确。}可以将其理解为“个人的目的只有通过他人才能实现”,也即是说,在资本主义社会中,对于个人而言,他人变成了自己的手段。关于这一点,我同样不在这里过多的阐释了,大家可以联系马克思的《黑格尔法哲学批判》中的内容去理解。这里附上马克思《黑格尔法哲学批判》中有关“外在必然性”的论述:
\begin{adjustwidth}{2em}{2em}
    \qquad\fangsong
“对家庭和市民社 会这两个领域来说,一方面,国家是‘外在必然性’,是一种权力,由于这种权力,‘法律’和‘利益’都‘从属并依存于’国家。国家对家庭和市民社会来说是‘外在必然性’,这已经部分地包含在‘过渡’这一范畴中,部分地包含在家庭和市民社会对国家的被意识到的关系中。对国家的‘从属’完全符合这种‘外在必然性’的关系。”\footnote{编者注:见马恩全集第二版第三卷《黑格尔法哲学批判》。}
    
\end{adjustwidth}

\vspace{0.3cm}
在本期讨论的最后,我们读到了马克思关于“生产一般”的论述。马克思在这里是这么说的:
\begin{adjustwidth}{2em}{2em}
    \qquad\fangsong
    “因此,说到生产,总是指在一定社会发展阶段上的生产——社会个人的生产。因而,好象只要一说到生产,我们或者就要把历史发展过程在它的各个阶段上一一加以研究,或者一开始就要声明,我们只的是某个一定的历史时代,例如,是现代资产阶级生产——这种生产事实上是我们研究的本题。可是,生产的一切时代有某些共同标志,共同规定。生产一般是一个抽象,但是只要它真正把共同点提出来,定下来,免得我们重复,它就是一个合理的抽象。”
\end{adjustwidth}

\vspace{0.8cm}
我们认为,从方法论角度上说,“生产一般”是马克思政治经济学批判的“从抽象上升到具体”辩证法的第一个必要环节,即马克思说的“合理的抽象”。比方说,小孩子在认识水果的过程中,他总是从具体的水果开始认识的,当他面对香蕉的时候,我告诉他这是香蕉,当他面对苹果的时候,我告诉他这是苹果,当他面对梨的时候,我告诉他这是梨……,但是这仅仅是一种感性直观,并没有上升到合理的抽象,只有当小孩子从这些苹果、香蕉、梨等物中抽象出“水果”概念的时候,他才算进入到了“从抽象上升到具体”辩证法的第一个环节,即合理的抽象。

但是,绝不能仅仅停留在最初的抽象(尽管这种抽象对于马克思的政治经济学批判而言是极为重要的一个环节)。我们可以看到马克思接下来是这么说的:
\begin{adjustwidth}{2em}{2em}
    \qquad\fangsong
    “如果说最发达语言的有些规律和规定也是最不发达语言所有的,但是构成语言发展的恰恰是有别于这一般和共同点的差别,那末,对生产一般适用的种种规定所以要抽出来,也正是为了不致因见到统一(主体是人,客体是自然,这总是一样的,这里已经出现了统一)就忘记了本质的差别。而忘记这种差别,正是那些证明现存社会关系永存与和谐的现代经济学家的全部智慧所在。例如,他们说,没有生产工具,哪怕这种生产工具不过是手,任何生产都不可能。没有过去的,累积下来的劳动,哪怕这种劳动不过是由于反复操作而累聚在野蛮人手上的技巧,任何生产都不可能。资本,别的不说,也是生产工具,也是过去的,客体化了的劳动。可见资本是一种一般的,永存的自然关系;这就是说,如果我们恰好抛开了正是使‘生产工具’,‘累积下来的劳动’成为资本的那个特殊的话。因此,生产关系的全部历史,例如在凯里看来,是历代政府的恶意篡改。”
\end{adjustwidth}
马克思在这里想表达的意思是,如果我们对于社会历史的观察仅仅止步于“共有的规定”的话,那末,历史在人们眼里就是不发展的、静止的。这是不难理解的,因为如果我们仅仅从“共有的规定”去考察包涵着历史关系的对象的话,我们便会发现这一对象在不同的历史时代并无什么差异。

比如说,在资本主义社会下,资本表现为“积累起来的劳动”,但是对于积累起来的劳动(客体化的劳动)而言,哪一个历史时代没有“积累起来的劳动”呢?难道说在其他的时代,“积累起来的劳动”也可以称作是资本吗?对于那些仅仅处在那种最初的抽象的理论家们而言,他们难免会认为“资本”同“积累起来的劳动”一样,是一种永存的自然关系,从而认为“生产关系的全部历史”是“历代政府的恶意篡改”。然而,在我们看来,这些理论家们犯了形而上学的错误。关于这一点,我同样不再继续赘述了,请读者们细细思索。

\subsubsection{本期小结}
本期读书会是在周四举办的,这个礼拜一共举办了两次。本期参加读书会的师友人数比上期多一些,都是一些较为熟悉的面孔。本期结束了《雇佣劳动与资本》的讨论,开启了《导言》的讨论。我编排的讲义中《导言》的内容或许存在着一些问题\footnote{编者注:我是从中文马克思主义文库上面摘取的文章,可能部分内容会和马恩全集上面的不一致。},大家可以直接参阅马恩全集第二版第30卷中的内容。《导言》的文本是要比我们之前讨论过的内容难理解一些的,因而我估计之后的几期读书会将会是非常高质量的。此外,最近感冒的比较多,在这里我呼吁大家一定要多注意防护,带好口罩,生病要及时吃药,健康的身体是搞好学术的物质基础。
\newpage
\section{第七期:《<政治经济学批判>导言》(2)}
\subsection{读书会记录}
2023年11月30日18:00—20:00,我们在东北林业大学奥林学院203教室开展了第七期的读书会。本期读书会我们继续了对马克思《<政治经济学批判>导言》的讨论。接下来我将对本期读书会的讨论内容进行简要概述。
\subsubsection{1.生产一般}
在上期读书会中我们讨论了马克思关于“生产一般”的规定,但仅仅是开了个头。我们认为,“生产一般”在马克思的政治经济学理论中属于最初的抽象,且这种抽象是合理的与必要的。但我们又进一步意识到(马克思同样也意识到),仅仅通过这个源初的抽象去考察事物的话,往往是片面的,甚至会得出荒谬的结论。究其原因是由于事物本身是在历史中不断发展着的,且对于事物的发展起关键性作用的是那个“特殊”,而不是普遍的“一般”,因而抽象必须要上升到具体。关于这一点,我不再继续赘述,请读者回顾上期的内容。

在本期的讨论内容中,马克思批判了资产阶级经济学家们的研究范式。马克思是这么说的:
\begin{adjustwidth}{2em}{2em}
    \qquad\fangsong
“现在时髦的做法,是在经济学的开头摆上一个总论部份——就是标题为《生产》的那部份(参看约翰,斯图亚特,穆勒的著作),用来论述一切生产的一般条件。
    
这个总论部份包括或者好像应当包括:

(1)进行生产所必不可缺少的条件。因此,这实际上不过是要说明一切生产的基本要素。可是,我们将会知道,实际上归纳起来不过是几个十分简单的规定,却扩展成浅薄的同义反复。

(2)或多或少促进生产的条件,如像亚当·斯密所说的前进的和停滞的社会状态。要把这些在斯密那里作为提示而具有价值的东西提升到科学意义上来,就得研究各个民族的发展过程终生产率程度不同的各个时期——这种研究超出本题应有的范围,但就属于本题范围来说,在叙述竞争,累积等等时是要谈到的。照一般的提法,答案总是这样一个一般的说法:一个工业民族,当它一般地达到它的历史高峰的时候,也就达到它的生产高峰。实际上,一个民族的工业高峰是在它还不是以既得利益为要务,而是以争取利益为要务的时候。在这一点上,美国人胜过英国人。或者是这样的说法:例如,某一些种族,素质,气候,自然条件如离海远近,土地肥沃程度等等,比另外一些更有利于生产。这又是同义反复,即财富的主客观因素越是在更高的程度上具备,财富就越容易创造。”
\end{adjustwidth}
\vspace{0.3cm}

在这里我们认为,马克思想表达的观点是,资产阶级经济学家们对于生产的考察脱离了一定的社会历史关系,而是仅仅对在不同历史阶段所共有的生产本身(即生产一般)进行考察。这种仅仅对生产一般的物质条件的考察并没有触及到社会历史关系层面,因而这种考察往往会使人们受到遮蔽,即认为资本主义时代的生产关系是一种永恒的“自然产物”。

我个人认为,在这里马克思并没有肯定地说对“生产一般”的物质条件的考察就是不正确的,我更倾向于认为马克思在这里想表达的意思是资产阶级经济学家们仅仅止步于“从抽象上升到具体”的第一个环节\footnote{编者注:即最初的合理抽象。},并在这第一个环节内部兜圈子,以至于他们\footnote{编者注:这里指马克思所批判的资产阶级经济学家们。}看不清历史的发展。

此外,我们认为,在马克思看来,资产阶级经济学家们似乎也并没有完全地领悟“从抽象上升到具体”的第一个环节,即“抽象”本身。我们可以看到马克思随后是这么说的:



\begin{adjustwidth}{2em}{2em}
    \qquad\fangsong
    “但是,经济学家在这个总论部分所真正要谈的并不是这一切。相反,照他们的意见,生产不同于分配等等(参看穆勒的著作),应当被描写成局限在脱离历史而独立的永恒自然规律之内的事情,于是资产阶级关系就被乘机当作社会一般的颠扑不破的自然规律偷偷地塞了进来。这是整套手法的多少有意识的目的。反之,在分配上,好像人们事实上可以随心所欲。”
\end{adjustwidth}
\vspace{0.8cm}

我们认为,马克思在上面那段话中想表达的意思是,资产阶级经济学家认为生产不同于分配。生产和分配之间的区别在资产阶级经济学家们眼里表现为:\textbf{生产是具有普遍性的,而分配则是脱离于生产的纯粹偶然的规定。} 这种说法在现实社会中看起来似乎是这样的,但是让我们继续深入思考一下,就会发现资产阶级理论家们的观点实际上是非常令人费解的。因而马克思随后会说:

\begin{adjustwidth}{2em}{2em}
    \qquad\fangsong
    “即使根本不谈生产和分配的这种粗暴割裂与生产与分配的现实关系,下面这一点总应当是一开始就明白的:无论在不同社会阶段上分配如何不同,总是可以像在生产中那样提出一些共同的规定来,可以把一切历史差别混合和融化在一般人类规律之中。例如,奴隶,农奴,雇佣工人都得到一定量的食物,使他们能够作为奴隶,农奴和雇佣工人来生存。靠贡赋生活的征服者,靠税收生活的官吏,靠地租生活的土地占有者,靠施舍生活的僧侣,或者靠什一税生活的教士,都得到一份社会产品,而决定这一份产品的规律不同于决定奴隶等等那一份产品的规律。”
\end{adjustwidth}
\vspace{0.2cm}
这就是说,当资产阶级经济学家们从不同历史时代的生产中抽象出一个共性的“生产一般”的时候,他们是先验地将“生产”与“分配”割裂开来的。这是因为:\textbf{既然资产阶级经济学家们承认不同历史时代存在着共性的“生产一般”的话,那末资产阶级经济学家们就没有理由不承认不同历史时代同样也存在着一个共性的“分配一般”了。}然而对于资产经济学家们而言,显然,他们只承认共性的“生产一般”,但不承认共性的“分配一般”,这便是理论的前后不一致。因此,对于那种资产阶级经济学家们所认为的将“生产”看作永恒的、而仅仅将“分配”看作偶然的情况而言,是一种割裂。



因此,我们认为,马克思在这里想表达的是,资产阶级经济学家们所考察的“分配”并不是纯粹偶然性的规定,而是处在一定的社会历史关系下的“分配”,换言之,资产阶级经济学家们所认为的那种所谓的具有偶然性的“分配”不过同样是被赋予了一定“特殊”的“分配一般”。

因而在这里不难看出,资产阶级经济学家们是很有趣的。他们看不到生产形式的变化,却能看到分配形式的变化。于是他们会天真地认为生产是永恒的自然规律,而分配是偶然的社会形式。于是乎,资产阶级经济学家们将注意力集中在分配领域——按照他们的理解,生产是一个永恒的自然规律,因而是无法改变的,因而他们只能在分配领域“大显身手”——并企图通过在分配问题上做文章,从而幻想着实现他们所期待的美好愿景。但这是不现实的,关于这一点,马克思在之后会谈到,在这里就不赘述了。

在本节最后,马克思是这么说的:

\begin{adjustwidth}{2em}{2em}
    \qquad\fangsong
    “总之:一切生产阶段所共同的,被思维当作一般规定而确定下来的规定,是存在的,但是所谓一切生产的一般条件,不过是这些抽象要素,用这些抽象要素不可能理解任何一个现实的历史的生产阶段。”
\end{adjustwidth}
这段话在一定程度上印证了我们之前的观点,就是说“生产一般”这个抽象是存在的,但是仅仅通过这个源初的抽象去理解什么东西是不现实的,因而抽象必须要上升到具体。在这里同样不再赘述了。

\subsubsection{2.生产与消费之间的关系}

这里讨论的内容是《导言》的第二节“生产与分配,交换,消费的一般关系”。在这里马克思首先指出,在进一步分析生产之前,有必要观察一下资产阶级经济学家们拿来与生产并列的几个项目,即分配、交换与消费。马克思认为将以上三者(分配、交换与消费)同生产相并列实际上是一种“肤浅的表象”,因而在这里我们能够体会到,生产、分配、交换与消费之间应该存在着某种关联,换句话说,应该存在着某种统一性。

因此在这里,马克思首先论述了生产与消费之间的同一性,为了不引起误解,我尽量用《导言》中的原文来概述这种同一性关系。
马克思在这里是这么说的:

\begin{adjustwidth}{2em}{2em}
    \qquad\fangsong
“生产直接也是消费。双重的消费,主体的和客体的:个人在生产当中发展自己的能力,也在生产行为中支出和消耗这种能力,同自然的生殖是生命力的一种消耗完全一样。第二,生产资料的消费,生产资料被使用,被消耗,一部分(如在燃烧中)重新分解为一般元素。原料的消费也是这样,原料不再保持自己的自然形状和特性,这种自然形状和特性倒是消耗掉了。因此,生产行为本身就它的一切要素来说也是消费行为。不过,这一点是经济学家所承认的,他们把直接与消费同一的生产,直接与生产合一的消费,称作生产的消费。生产和消费的这种同一性,归结起来是斯宾诺莎的命题:“规定即否定”。但是,提出生产的消费这个规定,只是为了把与生产同一的消费跟原来意义上的消费区别开来,后面这种消费被理解为起消灭作用的与生产相对的对立面,我们且观察一下这个原来意义上的消费。

消费直接也是生产,正如自然界中的元素和化学物质的消费是植物的生产一样。例如,吃喝是消费形式之一,人吃喝就生产自己的身体,这是明显的事。而对于以这种或那种形式从某一方面来生产人的其它任何消费形式也都可以这样说。消费的生产。可是,经济学却说,这种与消费同一的生产是第二种生产,是靠消灭第一种生产的产品引起的。在第一种生产中,生产者物化,在第二种生产中,生产者所创造的物人化。因此,这种消费的生产,——虽然它是生产和消费的直接统一——是与原来意义上的生产根本不同的。生产同消费合而为一和消费同生产合而为一的这种直接统一,并不排斥它们的直接两立。”
\end{adjustwidth}
\vspace{0.3cm}
上面这两段话是容易理解的,在这里,我们认为马克思侧重于从哲学层面\footnote{编者注:这里我更倾向于认为马克思是在本体论层面论述了生产与消费之间的同一性。}论述了生产与消费之间的同一性。我来简单概括一下就是:
\begin{tcolorbox}[colback=gray!20, colframe=gray!100, sharp corners, leftrule={3pt}, rightrule={0pt}, toprule={0pt}, bottomrule={0pt}, left={2pt}, right={2pt}, top={3pt}, bottom={3pt}] 
  1.生产过程本身就是参与生产的主体和客体的消费过程。
  
  2.主体的消费过程就是主体自身的生产过程。
  
\end{tcolorbox}

在这之后,马克思又从另外的角度分别论述了“消费生产着生产”与“生产生产着消费”,限于篇幅,我就不做具体说明了,读者可以自行阅读原文。

最后,马克思总结出了生产与消费之间的同一性,并做了三方面的概括,这里我把原文截取出来,大家可以自行体会,我同样不再赘述了。

\begin{adjustwidth}{2em}{2em}
    \qquad\fangsong
“因此,消费和生产之间的同一性表现在三方面:

(1)直接的同一性:生产是消费;消费是生产。消费的生产。生产的消费。政治经济学家把两者都称为生产的消费,可是还做了一个区别。前者表现为再生产,后者表现为生产的消费。关于前者的一切研究是关于生产的劳动或非生产的劳动的研究;关于后者的研究是关于生产的消费或非生产的消费的研究。

(2)每一方表现为对方的手段;以对方为媒介;这表现为他们的相互依存;这是一个运动,它们通过这个运动彼此发生关系,表现为互不可缺,但又各自处于对方之外。生产为消费创造作为外在对象的材料;消费为生产创造作为内在对象,作为目的的需要。没有生产就没有消费;没有消费就没有生产。这在经济学中以多种多样的形式表现出来。

(3)生产不仅直接是消费,消费也不仅直接是生产;而且生产不仅是消费的手段,消费不仅是生产的目的,——就是说,每一方都为对方提供对象,生产为消费提供外在的对象,消费为生产提供想象的对象;两者的每一方不仅直接就是对方,不仅媒介着对方,而且,两者的每一方当自己实现时也就创造对方,把自己当作对方创造出来。消费完成生产行为,只是在消费使产品最后完成其为产品的时候,在消费把它消灭,把它的独立的物体形式毁掉的时候;在消费使得在最初生产行为中发展起来的素质通过反复的需要达到完美的程度的时候;所以,消费不仅是使产品成为产品的最后行为,而且也是使生产者成为生产者的最后行为。另一方面,生产生产出消费,是在生产创造出消费的一定方式的时候,然后是在生产把消费的动力,消费能力本身当作需要创造出来的时候。这和第三项所说的这个最后的同一性,经济学在论述需求和供给,对象和需要,社会创造的需要和自然需要的关系时,曾多次加以解释。
” 
\end{adjustwidth}

\subsubsection{3.生产和分配之间的关系(1)}
在本期读书会中,我们把马克思关于生产与分配之间关系的讨论开了个头。在进行对生产与分配之间关系的讨论之前,马克思提出了一个疑问\footnote{编者注:当然,马克思心里面是已经有了答案的。},马克思是这么说的:“分配是否作为独立的领域,处于生产之旁和生产之外呢?”事实上,根据前文的讨论,我们是可以猜到的,马克思应该不会觉得分配是独立于生产的领域的,它们二者之间应该具有某种联系。

马克思指出对于“普通的经济学著作”\footnote{编者注:这里指的应该就是资产阶级经济学著作。}而言,这些著作里“什么都被提出两次”。这是什么意思呢?马克思给我们举了个例子,比方说对于“地租、工资、利息和利润”而言,它们处在分配领域,而对于“土地、劳动与资本”而言,它们则处在生产领域。这也即是说,在“普通的经济学著作”中,相应的生产要素对应着相应的收入源泉,换句话说,生产的规定对应着分配的规定\footnote{编者注:若是有熟悉马克思在《资本论》中对萨伊经济学的“三位一体”公式批判的读者在这里一定会感到奇怪,按照劳动价值论,劳动是价值的唯一源泉,因而所谓不同的生产要素对应不同的收入源泉这个观点本质上是错误的。但我在这里想提醒读者的是,马克思在这里讨论的生产规定与分配规定之间的“对应”更倾向于在形式层面的“对应”,事实上,它们也确实是在形式层面对应着的,并且如果进行进一步思索的话,这种“形式”似乎表现为一种外在性的强制。}。

因此,马克思会说:
\begin{adjustwidth}{2em}{2em}
    \qquad\fangsong
    “同样,工资也是在另一个项目中被考察的雇佣劳动:在一处作为生产要素的劳动所具有的规定性,在另一处表现为分配的规定。如果劳动不是规定为雇佣劳动,那末,它参与产品分配的方式,也就不表现为工资,如在奴隶制度下就是这样。”
\end{adjustwidth}
我认为,马克思在这里想说明的是,生产与分配似乎是同一个东西的不同侧面。也即是说,生产的性质同时决定了分配的性质。在这里,按照马克思的例子说来就是,工人以“雇佣劳动”的形式参与生产,那末,他就必然以“工资”的形式参与分配。而“雇佣劳动”本身又是资本主义社会关系的表现,关于这一点,我同样不在这里继续深入解读了,读者们可以自行体会。

因此,我们不难理解,马克思会如是说道:
\begin{adjustwidth}{2em}{2em}
    \qquad\fangsong
    “所以,分配关系和分配方式只是表现为生产要素的背面。个人以雇佣劳动的形式参与生产,就以工资形式参与产品,生产成果的分配。分配的结构完全取决于生产的结构,分配本身就是生产的产物,不仅就对象说是如此,而且就形式说也是如此。就对象说,能分配的只是生产的成果,就形式说,参与生产的一定形式决定分配的特定形式,决定参与分配的形式。把土地放在生产上来谈,把地租放在分配上来谈,等等,简直是幻觉。”
\end{adjustwidth}

\vspace{0.5cm}

事实上,我认为上面这段论述可以和“生产资料之间的分配”联系起来来看,但我在这里就不继续赘述了,我在读书会中已经叙述过我的观点了,有感兴趣的同学可以在下期继续探讨。

\subsubsection{本期小结}
本期读书会同样是在周四举办的,参加本期读书会的师友都是比较熟悉的面孔。本期我们讨论的内容非常丰富,我只是挑了其中的一小部分内容记录了下来。事实上,《导言》是学习马克思的政治经济学批判理论的一部非常重要的文本,这里面的许多内容是很难通过短时间的阅读与讨论而完全弄懂的,因而还需反复阅读,在这部分内容中可以深入探究的点有很多。此外,在本期结束之后,我们会暂停两个礼拜,在第16周之后继续举行剩下几期的读书会。在我编写本期读书会记录的时候,已经到12月份,临近放假了,希望大家能够坚持下去,搞好学术、收获知识。总而言之,本期读书会同样是收获满满的。
\newpage
\section{第八期:《<政治经济学批判>导言》(3)}
\subsection{读书会记录}
2023年12月19日14:00—16:00,我们在东北林业大学丹青楼608教室开展了第八期的读书会。本期读书会我们同样地继续进行对马克思《<政治经济学批判>导言》的讨论,本期讨论内容截止到\textbf{“政治经济学的方法”}一章中的开头部分。接下来我将对本期读书会的讨论内容进行简要概述。
\subsubsection{1.生产与分配的关系(2)}
在本期的讨论内容中,我们可以看到,马克思首先指出,无论是对于\textbf{“单个的个人”}还是\textbf{“整个社会”}而言,分配似乎都是先于生产而存在的,并且似乎是分配决定着生产。马克思的叙述逻辑总体上是这样的:

\begin{tcolorbox}
\textbf{1.对于单个的个人。}~\begin{fangsong}
    他在出生之前所具有的分配情况决定了他的生产地位。例如,若是一个人出生下来就不占有资本或地产的话,那末,他一出生就由社会分配指定专门从事雇佣劳动,也即是说,他在生产中充当着劳动力商品的地位。
\end{fangsong}

\textbf{2.对于整体社会。}~\begin{fangsong}
    一个社会整体的生产方式在经验层面往往是决定于这个社会形成之初的资本、地产的分配形式的。例如一个征服者民族在征服者之间分配土地,造成了地产的一定的分配与形式,从而决定了生产。又或者,一个民族经过革命把大地产粉碎成小块,从而通过这种新的地产分配形式使得生产有了一种新的性质。这些例子表明,似乎对于整体社会而言,也是分配决定于生产的。
\end{fangsong}
\end{tcolorbox}

\vspace{0.5cm}
但上述的那种理解仅仅是对于表象的理解。事实上在上期的讨论会中,我们便隐约地意识到了,生产和分配二者之间是有着极为密切的联系的。说分配先于生产,其实就是在无意中将分配与生产之间的关联给割裂开来了。我们可以看到马克思接下来是这么说的:

\vspace{0.3cm}
\begin{adjustwidth}{2em}{2em}
    \qquad\fangsong
    “照最浅薄的理解,分配表现为产品的分配,因而它仿佛离开生产很远,对生产是独立的。但是,在分配是产品的分配之前,它是(1)生产工具的分配,(2)社会成员在各类生产之间的分配(个人从属于一定的生产关系)——这是同上述同一关系的进一步规定。这种分配包含在生产过程本身中并且决定着生产的结构。”
\end{adjustwidth}

马克思的这段话意味着,生产本身便包涵着分配的因素。诚然,分配在一定程度上表现为产品的分配,并且这种“产品的分配”使得分配本身具有了一定的独立形式,但若仅仅将分配看作是产品的分配的话,这便是一种非常浅薄的理解。

事实上,生产本身便内在地包涵着生产资料的分配。换句话说,生产之所以是具体的生产,便是因为其内在地包涵着分配的要素。按照我的理解,这即是说,“生产资料的分配”是生产得以实现其自身的必要条件。我认为马克思下面的这段话很关键,马克思说:
\begin{adjustwidth}{2em}{2em}
    \qquad\fangsong
    “如果在考察生产时把包含在其中的这种分配撇开,生产显然只是一个空洞的抽象;反过来说,有了这种本来构成生产的一个要素的分配,力求在一定的社会结构中来理解现代生产并且主要是研究生产的经济学家李嘉图,不是把生产而是把分配说成现代经济学的本题。从这里,又一次显出了那些把生产当作永恒真理来论述而把历史限制在分配范围之内的经济学家是多么荒诞无稽。”
\end{adjustwidth}

我认为这段话是想表明,当我们在考察生产的过程中,不应该机械地将生产与分配割裂开来,去寻求二者之间的某种“机械式的纽带关系”(即寻求究竟是谁先谁后的关系)。因为真实的情况往往是辩证的而非机械的。因此,我认为在这个意义上便不难理解马克思的这句话:\textbf{“如果在考察生产时把包含在其中的这种分配撇开,生产显然只是一个空洞的抽象”}。

因此,对于马克思而言,分配作为生产的一个内部因素,它们二者之间的关系统一于生产自身内部。马克思是这么说的:
\begin{adjustwidth}{2em}{2em}
    \qquad\fangsong
    “这种决定生产本身的分配究竟和生产处于怎么样的关系,这显然是属于生产本身内部的问题。”
\end{adjustwidth}

但是,或许有人会存在着这样的疑问:既然承认了生产本身包含着分配的要素,即生产必须从生产资料的分配出发才能展开自身,那末,至少在这个意义上我们有理由认为分配是先于生产的,因而在这个意义上说分配先于生产是并无问题的。事实上,马克思是承认生产在\textbf{最初}可能存在着一个自然的分配前提的,但是马克思在这里想强调的是,在这个最初的自然前提之后,通过生产过程本身的作用,影响生产的那种分配便成为了一种历史性的结果。关于这一点,我不再赘述了,在这里我把马克思的原文放上来,大家可以再仔细思考一下。

\begin{adjustwidth}{2em}{2em}
    \qquad\fangsong
    “如果有人说,既然生产必须从生产工具的一定分配出发,至少在这个意义上分配先于生产,成为生产的前提,那末就应该答复他说,生产实际上有它的条件和前提,这些条件和前题构成生产的要素。这些要素最初可能表现为自然发生的东西。通过生产过程本身,它们就从自然发生的东西变成历史的东西了,如果它们对于一个时期表现为生产的自然前提,对于另一个时期就是生产的历史结果了。它们在生产内部不断地改变。例如,机器的应用既改变了生产工具的分配,也改变了产品的分配。现代大土地所有制本身既是现代商业和现代工业的结果,也是现代工业在农业上应用的结果。
    
    ……
    
    例如,蒙古人把俄罗斯弄成一片荒凉,这样做是适合于他们的生产,畜牧的,大片无人居住的地带是畜牧的主要条件。在日耳曼蛮族,用农奴耕作是传统的生产,过的是乡村的孤独生活,他们能够非常容易地让罗马各省服从于这些条件,因为那里发生的土地所有权的集中已经完全推翻了旧的农业关系。

有一种传统的观念,认为在某些时期人们只靠劫掠生活。但是要能够劫掠,就要有可以劫掠的东西,因此就要有生产。而劫掠方式本身又决定生产方式。例如,劫掠一个从事证券投机的民族就不能同劫掠一个游牧民族一样。

奴隶直接被剥夺了生产工具。但是奴隶受到剥夺的国家的生产必须安排得容许奴隶劳动,或者必须建立一种适于使用奴隶的生产方式(如在南美等)。
”
\end{adjustwidth}

\subsubsection{2.交换和流通}
关于交换和流通,我们认为,马克思在这里并没有刻意地给这二者做出一个区分。从某种意义上说,“交换和流通”这部分内容的关注度相较于前面的生产与分配而言略有下降,至少就《导言》中关于“交换和流通”这部分内容的写作篇幅而言,是较少的。造成这种现象的原因可能是因为仅仅就交换和流通这一过程本身而言,它们很难体现出一定的社会关系。就商品经济——进而是资本主义商品经济——而言,一方面,流通过程不涉及价值的增减;另一方面,在交换过程中反而是体现出了某种平等的要素。

但实际上,虽然交换与流通表现为独立于生产过程之外的因素,但其实它们还都是生产的内部因素,都是由生产方式本身所决定着的。马克思这里列举了三点来说明交换与生产之间的密切联系。

第一,如果没有分工,便没有交换。这实际上是很好理解的,分工作为生产方式的一种表现,导致了某一分工领域的群体无法生产出他们所需要的全部商品,这就需要同其他分工领域的产品进行交换。

第二,私人交换以私人生产为前提。这同样不难理解,对于个人而言,他若是要同别人进行交换,便需要进行私人生产,否则他就没有用以交换的东西了。

第三,交换的深度、广度和方式都是由生产的发展和结构决定的。马克思在这里举了城乡之间的交换、乡村中的交换以及城市中的交换等的例子。比如货币出现以前的简单商品交换与资本主义商品交换显然在交换方式上存在着差异,而这种差异恰恰是由于这两个时期的生产方式的差异所造成。

最后,马克思对生产、分配、交换和消费这四者的关系做了一个总结性的论述,同样我不再继续赘述了,我把原文放在这里,请读者们细细思索。

\begin{adjustwidth}{2em}{2em}
    \qquad\fangsong
“我们得到的结论并不是说,生产,分配,交换,消费是同一的东西,而是说,它们构成一个总体的各个环节,一个统一体内部的差别。生产既支配着生产的对立规定上的自身,也支配着其它要素。过程总是从生产重新开始。交换和消费是不能支配作用的东西,那是自明之理。分配,作为产品的分配,也是这样。而作为生产要素的分配,它本身就是生产的一个要素。因此,一定的生产决定一定的消费,分配,交换和这些不同要素相互间的一定关系。当然,生产就其片面形式来说也决定于其它要素。例如,当市场扩大,即交换范围扩大时,生产的规模也就增大,生产也就分得更细。随着分配的变动,例如,随着资本的集中,随着城乡人口的不同的分配等等,生产也就发生变动。最后,消费的需要决定着生产。不同要素之间存在着相互作用。每一个有机整体都是这样。”
\end{adjustwidth}

\subsubsection{3.政治经济学的方法(1)}

这一部分的内容是《导言》的关键之处,马克思的从“抽象上升到具体”的方法论思想理解起来是较为困难的,但同样又是对于整个政治经济学研究而言极为重要的。为了尽可能地减少歧义之处,我尽量多引用马克思原文中的语句来进行说明。

马克思在本部分的开篇说:
\begin{adjustwidth}{2em}{2em}
    \qquad\fangsong
    “从实在和具体开始,从现实的前提开始,因而,例如在经济学上从做为全部社会生产行为的基础和主体开始,似乎是正确的。但是,更仔细地考察起来,这是错误的。如果我抛开构成人口的阶级,人口就是一个抽象。如果我不知道这些阶级所依据的因素,如雇佣劳动,资本等等,阶级又是一句空话。而这些因素是以交换,分工,价格等等为前提的。比如资本,如果没有雇佣劳动,价值,货币,价格等等,它就什么也不是。因此,如果我从人口着手,那末这就是一个混沌的关于整体的表象,经过更切进的规定之后,我就会在分析中达到越来越简单的概念;从表象中的具体达到越来越稀薄的抽象,直到我达到一些最简单的规定。于是行程又得从那里回过头来,直到我最后又回到人口,但是这回人口已不是一个混沌的关于整体的表象,而是一个具有许多规定和关系的丰富的总体了。”
\end{adjustwidth}

我们认为,在这段叙述中,马克思已经阐述了他的“从抽象上升到具体”的哲学方法论思想。马克思说,从实在和具体开始,从现实的前提开始,似乎是正确的,但更仔细地考察起来,这是错误的。\textbf{这句话往往会引起歧义},似乎会觉得马克思是这样认为的:从实在和具体开始考察事物是错误的,也即是说,实在和具体不是考察事物的科学出发点。

我们认为,上面的那种看法是一种误解。事实上,实在和具体、现实的前提作为考察事物的出发点而言是没有问题的。如果对于事物的考察不从感性的实在和具体出发,那末就会陷入唯心主义的错误。那为什么马克思接下来会说“更加仔细地考察起来,这是错误的”呢?我们认为,马克思在这里想表明的是,虽然要从感性的实在和具体出发考察事物,但是对于事物的考察绝对不能止步于此,换句话说,若是要对事物进行科学的考察,便不能仅仅外在地停留于对实在与具体的笼统的抽象概括。

马克思在这里举了人口的例子,他说:“如果我从人口着手,那末这就是一个混沌的关于整体的表象,经过更切进的规定之后,我就会在分析中达到越来越简单的概念;从表象中的具体达到越来越稀薄的抽象,直到我达到一些最简单的规定。”这看似是很合理的对于事物的分析过程,但是我认为,更为关键的是后面的一句话,马克思的叙述进行了一次\textbf{“转折”},他接着说:“于是行程又得从那里\textbf{回过头来},直到我最后又回到人口,但是这回人口已不是一个混沌的关于整体的表象,而是一个具有许多规定和关系的丰富的总体了。”

让我们来分析一下上面这个认识过程。我们要首先要明确一点,这整个认识过程的主体是人,因此只要这个认识过程仅仅是在主体的头脑中进行的话,那末这一过程就并不会对认识对象产生什么实质性的影响。因此,就整个认识过程而言,它的起点是“混沌的关于整体的表象”,它的终点是“具有许多规定和关系的丰富的总体”,而这二者的区别是仅仅对于认识主体而言的区别,这即意味着,对于认识对象而言,其自身是没有什么变化的。按照“人口”的例子来说,这种科学认识过程从总体性层面来看就是从“人口”到“人口”的回归,前一个“人口”是“混沌的关于整体的表象”,后一个“人口”是“具有许多规定和关系的丰富的总体”,但“人口”在现实中所对应着的那个客体是不会发生变化的,变化仅仅存在于对于“人口”这一范畴的认识。

马克思指出,从“混沌的表象”出发得到“最简单的规定”这是“第一条道路”,而从“最简单的规定”回到范畴本身,以达到“许多规定和关系的丰富的总体”,这便是“第二条道路”。马克思指出,“在第一条道路上,完整的表象蒸发为抽象的规定;在第二条道路上,抽象的规定在思维行程中导致具体的再现。”马克思认为,第二条道路是科学上正确的方法。但这并不意味着第一条道路就是毫无意义的。我们认为,第二条道路需要通过第一条道路,换句话说,第二条道路是对第一条道路的进一步扬弃。

\textbf{接下来马克思批判了黑格尔式的唯心主义。}马克思指出:
\begin{adjustwidth}{2em}{2em}
    \qquad\fangsong
    “因而黑格尔陷入幻觉,把实在理解为自我综合,自我深化和自我运动的思维的结果,其实,从抽象上升到具体的方法,只是思维用来掌握具体并把它当作一个精神上的具体再现出来的方式。但决不是具体本身的产生过程。”
\end{adjustwidth}

这便正如我在前面所叙述的那样,认识过程仅仅是一个主体自身的思维过程,就这一过程而言,它是不会影响认识对象自身的,在整个认识过程的起点与终点发生变化的不是认识对象,而是认识主体的思维。而唯心主义则将这一过程颠倒,将主体认识世界的过程等于世界自身的生成过程,把具体思维的生产等同于客观具体事物的生产,这是极其荒谬的。马克思接下来是这么说的:
\begin{adjustwidth}{2em}{2em}
    \qquad\fangsong
“在意识看来(而哲学意识就是被这样规定的:在它看来,正在理解着的思维是现实的人,而被理解了的世界本身才是现实的世界),范畴的运动表现为现实的生产行为(只可惜它从外界取得一种推动),而世界是这种生产行为的结果;这——不过又是一个同义反复——只有在下面这个限度内才是正确的:具体总体作为思想总体、作为思想具体,事实上是思维的、理解的产物;但是,决不是处于直观和表象之外或驾于其上而思维着的、自我产生着的概念的产物,而是把直观和表象加工成概念这一过程的产物。”
\end{adjustwidth}

可见,在黑格尔看来,思维是真正具有“现实性”的主体,思维的运动是“现实”的生产行为。既然黑格尔这类唯心主义的哲学家们认为思维是具有“现实性”的主体,那末,他们便会很自然的认为是这种真正具有“现实性”的思维主体正在塑造着他们所认为的“现实”,“现实世界”便是这种“现实的”思维运动的结果。但他们没有意识到,这种思维主体的运动是受到客观世界的外部推动的,这种思维的运动无非是真正的现实主体通过他的头脑认识世界的一种机制,这种运动本身究其实质是以真正现实的客观世界作为前提的,只要这种思维运动仅仅局限于思维领域,那末它就不会对真正的现实世界产生任何影响,从而现实世界相较于这种思维运动而言便是保持着独立性的。因此马克思会在最后说道:
\begin{adjustwidth}{2em}{2em}
    \qquad\fangsong
“整体,当它在头脑中作为思想整体而出现时,是思维着的头脑的产物,这个头脑用它所专有的方式掌握世界,而这种方式是不同于对于世界的艺术精神的,宗教精神的,实践精神的掌握的。实在主体仍然是在头脑之外保持着它的独立性;只要这个头脑还仅仅是思辨地、理论地活动着。因此,就是在理论方法上,主体,即社会,也必须始终作为前提浮现在表象面前。”
\end{adjustwidth}


\subsubsection{本期小结}
时隔两个礼拜,我们的读书会又再次举办了,参加本期读书会的师友都是非常熟悉的面孔。本期我们讨论的内容较为晦涩难理解,特别是《导言》的“政治经济学的方法”一章中的内容。我们在本期近乎以一种“句读”的形式讨论了马克思的“从抽象上升到具体”的方法论。我在这里浅说一下我个人的看法,关于“从抽象上升到具体”的具体机制是如何的,在本期读书会的内容记录中我已经表达了很多我的观点,大家可以向上翻阅;另一方面,对于这一思想的存在论意义而言,我倾向于认为马克思的“从抽象上升到具体”的哲学方法论思想是一种对于科学认识论的描述性(解释性)的分析,而不是一种规定性的分析。也即是说,我更倾向于认为,马克思只是揭露了人类的科学认识过程的机制究竟是怎么样的,而并不是提供了一种“放之四海而皆准”的方法,因而对于具体问题还需具体分析,关于这一点,限于篇幅,我就不在这里继续赘述了。总之,本期的讨论是热情而充实的,这是一期收获满满的读书会。



\end{document}
