\documentclass[a4paper,twoside,12pt,AutoFakeBold]{ctexart}
\newcommand\specialsectioning{\setcounter{secnumdepth}{-2}}
\specialsectioning
\usepackage[center]{titlesec}
\usepackage{indentfirst}
\setlength{\parindent}{2em}
\usepackage{fancyhdr}
\pagestyle{fancy}%fancy style
\usepackage{caption}
\fancyhf{}%清空页眉页脚
\fancyhead[LE,RO]{\thepage}%页码位置:偶数页居左,奇数页居右
\fancyfoot[RO,RE]{\textit{NEFU's seminar of Marxism}}% 设置页脚:在每页的右下脚以斜体显示书名
\usepackage{graphicx}
\setlength{\headheight}{15pt}%解决页眉warnings
\usepackage{amsmath}
\renewcommand{\headrulewidth}{0pt} % 页眉与正文之间的水平线粗细
\renewcommand{\footrulewidth}{0pt}
\usepackage{tcolorbox}%文本框宏包
\usepackage{changepage}%设置引用段落左右侧缩进

\usepackage{tabularx}
\usepackage{hyperref}%设置超链接
\usepackage{float}
\title{读书会记录}
\author{东北林业大学马克思主义研讨会\\
指导教师:李光玉~许婕\\
 吴云飞~编}
\date{2023年12月}
\usepackage{perpage}
\MakePerPage{footnote}
\begin{document}
\maketitle
\newpage



\tableofcontents%目录

\newpage

\section{序言}

\begin{adjustwidth}{2em}{2em}
\qquad\fangsong 
本书是东北林业大学马克思主义经典著作研究与讨论会的一个内部记录,诸多观点与看法可能会欠缺专业性,因而这是一个非教学性质的内容记录。本书全部内容均来自我们的研讨会中的发言与讨论以及后续编者所进行的思考,如有雷同,纯属巧合,本书最终解释权归本研讨会与本书编者\footnote{编者注:我的邮箱是1360540699@qq.com,如果您在阅读本书之后有相应的意见与建议,抑或是想进行一些学术方面的交流与探讨,请通过该邮箱与我联系。}所有。

本书采用\LaTeX{}编写,\TeX{}源代码暂由编者吴云飞托管至其本人的Github仓库之中,供各位成员获取:\url{https://github.com/Imheaven233/Seminar-of-Marxism-of-NEFU}。



\end{adjustwidth}




\newpage

\section{第一期:《哲学的贫困》选读(1)}

\subsection{学习提示}\label{sec:1}

《哲学的贫困》是马克思针对蒲鲁东的《贫困的哲学》一书而写的一部论战性著作,以法文写成于1847年上半年,并于同年7月在布鲁塞尔和巴黎出版。

该著作分为两个部分,即第一章和第二章。第一章的讨论针对蒲鲁东为“工资平等”的社会主义所作的经济学论证,揭示这种论证尚未达到李嘉图经济学理论的水准。第二章批判了蒲鲁东经济学理论的哲学基础。

本期讨论会我们选的内容是第二章“\textbf{政治经济学的形而上学}”的第一节“\textbf{方法}”,这部分内容在我们编排的讲义中的3—17页\footnote{编者注:这部分内容收录于马恩选集第1卷。}。

在本次讨论会中,您将会了解到(或带着以下问题去阅读):

\textbf{1.}马克思对黑格尔的辩证法的一个简要概述。

\textbf{2.}蒲鲁东的经济学理论所遵循的“辩证法”同黑格尔的辩证法之间的差异。

\textbf{3.}蒲鲁东的经济学理论的哲学基础之实质是什么,犯了什么错误?

\textbf{4.}对蒲鲁东的经济学理论的哲学基础的批判对于我们有何启示?

\subsection{读书会记录}
2023年10月16日18:00—20:00,我们在东北林业大学奥林学院203教室开展了第一期的读书会。本期读书会我们阅读了《哲学的贫困》第二章第一节的部分内容(截止到“\textbf{第五个说明}”之前),并做了充分的讨论。具体讨论内容可简要地概括如下:

\subsubsection{1.前言部分}\label{sec:3}

首先,马克思在前言部分指出:
\begin{adjustwidth}{2em}{2em}
\qquad\fangsong
“如果说有一个英国人把人变成帽子,那末,有一个德国人就把帽子变成观念。这个英国人就是李嘉图,一位银行巨子,杰出的经济学家;这个德国人就是黑格尔,柏林大学的一位专任哲学教授。”
\end{adjustwidth}

这段话中的“帽子”、“观念”指的是什么?在这里我们认为,“帽子”指的是李嘉图的经济学中的诸多“经济范畴”,体现着物与物之间的关系;而“观念”则指的是黑格尔哲学中的“概念”,更确切地说,是黑格尔哲学中的形而上学传统\footnote{编者注:这在后面的讨论中会详细的说明。}。

古典经济学家们\footnote{编者注:这里指的是英国古典经济学家斯密、李嘉图等人。}通过对资本主义社会的外部经验现象的考察总结出了一些基本的经济规律,但是他没有意识到这些规律本身并非是永恒不变的存在物,以至于他们将人与人之间关系转变为物与物之间的关系,企图通过不变的经济范畴(帽子)解释人类社会的运行本身\footnote{编者注:按照马克思的话就是:“把人变成帽子。”}。而持有着形而上学传统的哲学家们(尤指黑格尔)则将这些“帽子”进行了更进一步的“抽象”,将其转化为观念本身\footnote{编者注:按照马克思的话就是:“把帽子变成观念。”}。

\subsubsection{2.第一个说明}

第一个说明所讨论的核心内容是黑格尔的辩证法,而黑格尔的辩证法在整体上也是建筑于形而上学这一传统的基础之上的。马克思指出,经济学家们只是向我们解释了生产如何在现存的关系下进行,但是没有向我们解释这些“关系”本身是何以存在的。古典经济学家们面对的是活生生的现实,他们所做的工作是对“活生生的现实”的直接抽象;而蒲鲁东面对的则是古典经济学家们提出的诸多范畴,以及隐藏在这些范畴背后的诸多古典经济学家们的教条与偏见,蒲鲁东在探寻这些“关系”得以产生的原因的过程中,他仅仅从这些范畴本身出发,因而他只能在观念的领域兜圈子。

黑格尔哲学的第一步便是形而上学式的抽象,将一切事物都抽象成为逻辑范畴,并将一切事物的运动也抽象成纯粹形式的运动,这样一来,正如马克思所言:

\begin{adjustwidth}{2em}{2em}
    \qquad\fangsong
    “既然把任何一种事物都归结为逻辑范畴,任何一个运动、任何一种生产行为都归结为方法,那末,由此自然得出一个结论,产品和生产、对象和运动的任何总和都可以归结为应用的形而上学。黑格尔为宗教、法等做过的事情,蒲鲁东先生也想在政治经济学上如法炮制。”
\end{adjustwidth}

由此,黑格尔找到了一种“绝对的方法”来描绘现实世界的运动,即纯粹理性的运动。事实上,这是一种颠倒,但在这里我们就不赘述了\footnote{编者注:唯心主义与唯物主义之间的差异。}。在这里不难理解,当事物的一切“偶性”都被抽掉之后,剩下的就是纯粹的“概念”、纯粹的“理性”,但需要指出的是,这种纯粹的“概念”与“理性”是脱离个别主体而存在的,是一种绝对的、客观的东西。因此,对于黑格尔的辩证法而言,辨证运动是“绝对精神”的自我运动,是“概念”的自我规定,因而这是一种客观的唯心主义。


\subsubsection{3.第二个说明}

在这部分内容中的末尾,马克思表明了他的历史唯物主义哲学思想:

\begin{adjustwidth}{2em}{2em}
    \qquad\fangsong
    “经济学家蒲鲁东先生非常明白,人们是在一定的生产关系范围内制造呢绒、麻布和丝织品的。但是他不明白,这些一定的社会关系同麻布、亚麻等一样,也是人们生产出来的。社会关系和生产力密切相联。随着新生产力的获得,人们改变自己的生产方式,随着生产方式即保证自己生活的方式的改变,人们也就会改变自己的一切社会关系。手工磨产生的是封建主为首的社会,蒸汽磨产生的是工业资本家为首的社会。”
\end{adjustwidth}

事实上,上面这段话的第一句可以转译为:

\begin{adjustwidth}{2em}{2em}
    \qquad\fangsong
    蒲鲁东明白,人们在一定的“关系”内制造“物”。但是他不明白,这些“关系”同这些“物”一样,也是人们生产出来的。
\end{adjustwidth}
这充分说明了马克思的历史唯物主义哲学与那种纯粹被动的、毫无生机的机械唯物主义哲学之间的差异,因为马克思告诉我们,“关系”可不是什么神秘主义式的东西,“关系”本身就是通过人类的实践活动所建构出来的。

\subsubsection{4.第三个说明}

在“第三个说明”中,马克思揭示了蒲鲁东经济理论的矛盾。马克思是这么说的:

\begin{adjustwidth}{2em}{2em}
    \qquad\fangsong
    “每一个社会中的生产关系都形成一个统一的整体。蒲鲁东先生把种种经济关系看做同等数量的社会阶段,认为这些阶段一个产生一个,一个来自一个,正如反题来自正题一样;认为这些阶段在自己的逻辑顺序中实现着人类的无人身的理性。

这个方法的唯一短处就是:蒲鲁东先生在考察其中任何一个阶段时,都不能不靠其它一些社会关系来说明,可是当时这些社会关系尚未被他用辩证运动产生出来。当蒲鲁东先生后来借助纯粹理性使其它阶段产生出来时,却又把它们当成初生的婴儿,忘记它们和第一个阶段是同样年老了。

因此,要构成被他看做一切经济发展基础的价值,就非有分工、竞争等等不可。然而当时这些关系在一定的系列中、在蒲鲁东先生的理性中以及逻辑顺序中根本还不存在。”
\end{adjustwidth}
在这里可以看到,蒲鲁东的“辩证法”很有意思,当他运用“辩证法”推出某些新的概念时,他总是需要依靠一些尚未被他的“辩证法”所生成出来的东西去说明。例如,蒲鲁东通过A和B推出了C,再通过C和I推出了D,但是I本身并没有通过他的辩证法被创造出来,而当他通过E和F推出I的时候,他也忘记了,I本身在C到D的“辩证运动”中就已经作为前提条件生成了。因此,马克思在这里隐约地想表明,蒲鲁东的“辩证法”似乎是一种非辩证的主观臆想。

\subsubsection{5.第四个说明}

通过对前面部分的阅读与讨论,“第四个说明”这部分内容便显得容易理解了。马克思在这一部分中说明了蒲鲁东是如何将他的“辩证法”应用到政治经济学领域的。马克思是这么叙述的:

\begin{adjustwidth}{2em}{2em}
    \qquad\fangsong
“蒲鲁东先生认为,任何经济范畴都有好坏两个方面。他看范畴就象小资产者看历史伟人一样:拿破仑是一个大人物;他行了许多善,但是也作了许多恶。

蒲鲁东先生认为,好的方面和坏的方面,益处和害处加在一起就构成每个经济范畴所固有的矛盾。

应当作的是:保存好的方面,消除坏的方面。”
\end{adjustwidth}

可见,当蒲鲁东的“辩证法”应用到政治经济学的领域时便成了一种机械式的“保存好的方面,消除坏的方面”。

因此,马克思会指出:

\begin{adjustwidth}{2em}{2em}
\qquad\fangsong
    “黑格尔没有需要提出任务。他只有辩证法。蒲鲁东先生从黑格尔的辩证法那里只学到了术语。而蒲鲁东先生自己的辨证运动只不过是机械地划分出好、坏两面而已。
    
    ……
    
    两个矛盾方面的共存、斗争以及融合成一个新范畴,就是辩证运动的实质。谁要给自己提出消除坏的方面的任务,就是立即使辩证运动终结。我们看到的已经不是由于矛盾本性而自我安置和自相对置的范畴,而是在范畴的两个方面中间激动、挣扎和冲撞的蒲鲁东先生。”

\end{adjustwidth}
    
\subsubsection{简短的总结}

通过本期的研讨会,我们可以对“\textbf{\nameref{sec:1}}”提出的四个问题中的前两个进行简要地回答。

对于\textbf{问题1“马克思对黑格尔的辩证法的一个简要概述”}而言,我们可以认为黑格尔的辩证法表明的是概念的自我规定(或自我运动),且这种辩证运动的前提基础是形而上学式的抽象。

对于\textbf{问题2“蒲鲁东的经济学理论所遵循的“辩证法”同黑格尔的辩证法之间的差异”}而言,我们可以认为黑格尔的辩证法所揭露的是脱离于单个主体之外的纯粹理性的自我运动,是一种客观唯心主义;而蒲鲁东的“辩证法”仅仅是借用了黑格尔哲学的诸多词句,他并没有领会黑格尔辩证法的实质,他将黑格尔的辩证法下降到了简单的二元式的机械运动,更具体地说,可以认为蒲鲁东的“辩证法”是一种主观唯心主义。
\newpage

\section{第二期:《哲学的贫困》选读(2)}

\subsection{读书会记录}
\begin{figure}[h]
    \centering
    \includegraphics[width=1\linewidth]{10.23.jpg}
    \caption{第二期读书会实拍图(2023.10.23)}
    \label{fig1}
\end{figure}
2023年10月23日18:00—20:00,我们在东北林业大学奥林学院203教室开展了第二期的读书会。本期读书会我们阅读了《哲学的贫困》第二章第一节中的\textbf{“第五个说明”}与\textbf{“第六个说明”}。我们本期做了非常丰富的讨论,现场情况见\textbf{\nameref{fig1}}\footnote{编者注:在这里非常抱歉,由于本期读书会讨论的过于热情,以至于我在中途拍照的时候漏掉了一些同学与老师。},主要讨论内容可简要地概括如下:
\subsubsection{1.第五个说明}

在“第五个说明”中,马克思首先指出:
\begin{adjustwidth}{2em}{2em}
    \qquad\fangsong
    “如果把辩证运动的全部过程归结为简单地对比善和恶,归结为提出任务来消除恶并且把一个范畴用作另一个范畴的消毒剂,那末范畴就失去自己的独立运动;观念就“不再发生作用”;他就没有内在的生命。它既不能把自己安置为范畴,也不能把自己分解为范畴。范畴的顺序成了一种脚手架。辩证法已不是绝对理性的运动了。辩证法没有了,代替它的至多不过是最纯粹的道德而已。”
\end{adjustwidth}

这里马克思似乎是想表达:“辨证运动”并不是蒲鲁东所认为的那种“简单地对比善和恶,并提出任务来消除恶”。马克思认为蒲鲁东的“辩证运动”不过是借助他自己的搭建出的一种“脚手架”而进行的,这意味着蒲鲁东将黑格尔所认为的“范畴的自我运动”变成了通过他(指蒲鲁东)自己搭建起来的“脚手架”而进行的运动。因此,马克思会说,在蒲鲁东的“辩证运动”下,范畴本身“就没有内在的生命”了。

之后,马克思揭示了蒲鲁东自己的逻辑混乱。马克思认为,蒲鲁东在谈论“历史”的时候,他是从他自认为的“范畴的顺序”\footnote{编者注:根据前文的分析,不难理解,这种“范畴的顺序”是通过蒲鲁东自己构建的“脚手架”而展开的。}去出发的。但是,当蒲鲁东将他自认为的“范畴的顺序”应用到现实中时,便同现实产生了矛盾。因此马克思指出:
\begin{adjustwidth}{2em}{2em}
    \qquad\fangsong
    “……于是蒲鲁东先生只得承认,他用以说明经济范畴的次序和这些经济范畴在其中相互产生的次序是不相适应的。经济的进化不再是理性本身的进化了。”
\end{adjustwidth}

最后,马克思在“第五个说明”中的最后一部分的内容中强调,如果按照蒲鲁东所设想的“现实的历史是观念、范畴和原理在其中出现的那种历史顺序”的话,那末就必须对此进行进一步的追问。也即是说,如果说是某一时代的原理创造了这一时代的历史的话,那末,为什么这一时代的原理没有创造别的时代的历史呢?为什么这一时代的原理恰好地就出现在这一时代了呢?为了回答上述疑问,就必须进一步考察那一时代的人们是怎样进行生产生活的、那一时代的人们之间的关系是怎么样的。但只要进入到对具体时代的人们的生产生活的考察之中,那末就是背弃了那种认为“观念创造历史”的观点,就是回到了历史发生的真正出发点。

因此,马克思在最后是这么论述的:
\begin{adjustwidth}{2em}{2em}
    \qquad\fangsong
“我们暂且和蒲鲁东先生一同假定:现实的历史,适应时间次序的历史是观念、范畴和原理在其中出现的那种历史顺序。

每个原理都有其出现的世纪。例如,与权威原理相适应的是11世纪,与个人主义原理相适应的是18世纪,推其因果,我们应当说,不是原理属于世纪,而是世纪属于原理。换句话说,不是历史创造原理,而是原理创造历史。但是,如果为了顾全原理和历史我们再进一步自问一下,为什么该原理出现在11世纪或者18世纪,而不出现在其它某一世纪,我们就必然要仔细研究一下:11世纪的人们是怎样的,18世纪的人们是怎样的,在每个世纪中,人们的需求、生产力、生产方式以及生产中使用的原料是怎样的;最后,由这一切生存条件所产生的人与人之间的关系是怎样的。难道探讨这一切问题不就是研究每个世纪中人们的现实的、世俗的历史,不就是把这些人既当成剧作者又当成剧中人物吗?但是,只要你们把人们当成他们本身历史的剧中人物和剧作者,你们就是迂回曲折地回到真正的出发点,因为你们抛弃了最初作为出发点的永恒的原理。”
\end{adjustwidth}

\subsubsection{2.第六个说明}
马克思在这部分内容中开篇提到:
\begin{adjustwidth}{2em}{2em}
    \qquad\fangsong
    “我们已经看到,在这一切一成不变的、停滞不动的永恒下面没有历史可言,即使有,至多也只是观念中的历史,即反映在纯理性的辩证运动中的历史。蒲鲁东先生谈到辩证运动中的各种观念不能自相‘区分’时,把运动的一切影子和影子(它们可以造成某种类似历史的东西)的一切运动一概抹熬。”
\end{adjustwidth}

其中,\textbf{“运动的影子”}和\textbf{“影子的运动”}分别代表的是什么意思呢?在这里,我们认为前者指的是“感性历史”的理论表现,后者指的是黑格尔的那种“纯粹理性”的运动。因此,我们可以认为,马克思在这里想表明:蒲鲁东所谈论的“辨证运动”既不是对感性历史发展过程的理论表达,也不是对黑格尔式的纯粹理性的运动的表达。因此,正如前面我们所讨论的那样,事实上,蒲鲁东自己构建了一个范畴的“脚手架”,范畴遵循着蒲鲁东自己构建的“脚手架”向上爬行。而范畴为什么要踏着“脚手架”向上运动呢?因为蒲鲁东认为范畴就得踏着他所构建的“脚手架”向上运动,这似乎是蒲鲁东先生的一厢情愿。

既然蒲鲁东的“辨证运动”既不是对感性历史的理论表达,也不是对黑格尔式的“纯粹理性”运动的表达,那末,蒲鲁东是如何从底层逻辑层面说明他的“辩证运动”的合理性的呢?很简单,既然蒲鲁东自己的“辨证运动”既不是“运动的影子”也不是“影子的运动”,那末,蒲鲁东先生完全可以“大言不惭地”构建出一个新的规律,并用一种“唬人的词句”宣称这种规律的至高性地位。因此,蒲鲁东先生自己构建出了所谓的\textbf{“人类理性”}、\textbf{“社会天才”}等等的概念。然而事实上,我们不难看出,所谓的\textbf{“人类理性”}、\textbf{“社会天才”}等的概念不过是蒲鲁东自己虚构出来的东西,不过是蒲鲁东自己的主观设想。

因此,当蒲鲁东的理论同现实的历史之间发生矛盾的时候,他就将这些\textbf{“矛盾”}看作是\textbf{“人类理性”}、\textbf{“社会天才”}的任务,他认为\textbf{“人类理性”}、\textbf{“社会天才”}就是需要解决这些矛盾的,进而使得范畴朝着\textbf{“人类理性”}、\textbf{“社会天才”}所期待的目标发展。

因而我个人\footnote{编者注:指本书编者,下同。}认为,在马克思的视角下,蒲鲁东的逻辑进路应可分为如下的步骤:


\begin{tcolorbox}[colback=gray!20, colframe=gray!100, sharp corners, leftrule={3pt}, rightrule={0pt}, toprule={0pt}, bottomrule={0pt}, left={2pt}, right={2pt}, top={3pt}, bottom={3pt}] 
\textbf{step1.}假定有一个“先在”的真理:例如“纯粹的平等”、“纯粹的自由”等最终的概念。

\textbf{step2.}假定一种“规则”:可以通过这种“规则”发现“真理”,并把这种“规则”命名为“人类理性”、“社会天才”。

\textbf{step3.}因此,当理论同现实相违背时,蒲鲁东就可以借着“人类理性”、“社会天才”的名义说:这种矛盾正是要通过“辩证运动”被解决的!


\end{tcolorbox}

因此,马克思会说:
\begin{adjustwidth}{2em}{2em}
    \qquad\fangsong
    “假设只是为了某种特定的目的而设立的。通过蒲鲁东先生之口讲话的社会天才首先给自己提出的目的,就是消除每个经济范畴的一切坏的东西,使它只保留好的东西。他认为,好的东西,最高的幸福,真正的实际目的就是平等。为什么社会天才只要平等,而不要不平等或友爱、不要天主教或别的什么原理呢?因为‘人类之所以实现这么多特殊的假设,正是由于考虑到一个最高的假设’,这个最高的假设就是平等。换句话说,因为平等是蒲鲁东先生的理想。他以为分工、信用、工厂,一句话,一切经济关系都仅仅是为了平等的利益才被发明的,但是结果它们往往对平等不利。由于历史和蒲鲁东先生的臆测步步发生矛盾,所以他得出结论说,有矛盾存在。即使是有矛盾存在,那也只存在于他的固定观念和现实运动之间。”
\end{adjustwidth}

在上面马克思的这段论述中,我们可以挑几句关键的句子来看,例如:\begin{fangsong}
    “通过蒲鲁东之口讲话的社会天才”
\end{fangsong},以及\begin{fangsong}
    “为什么社会天才只要平等,而不要不平等或友爱、不要天主教或别的什么原理呢?……因为平等是蒲鲁东先生的理想”
\end{fangsong}
等。不难看出,前面这几句话处处体现出了我在前文所概括的\textbf{“蒲鲁东的逻辑进路”}中的三个步骤。因此,马克思会指出:\begin{fangsong}
    “即使是有矛盾存在,那也只存在于他\footnote{编者注:指蒲鲁东。}的固定观念和现实运动之间。”
\end{fangsong}

因此我们可以看到,马克思在这里最终想要表达的是:\textbf{蒲鲁东的思想既不唯物,也不客观。}在这里,我个人认为,蒲鲁东的思想既没有超越古典经济学家们,也没有超越黑格尔。因为,相较于古典经济学家们而言,蒲鲁东缺少了这些理论家们的唯物主义倾向;相较于黑格尔而言,蒲鲁东也没有领悟“辩证法”或“辨证运动”的实质,因而蒲鲁东的思想更像一种拙劣的“折衷主义”\footnote{编者注:事实上,我认为在“第七个说明”中马克思对蒲鲁东的这种“折衷主义”的批判会体现地更为明显,但本期讨论会并没有讨论到“第七个说明”,因此这里先按下不表。}。

在“第六个说明”的最后,马克思这样说道:
\begin{adjustwidth}{2em}{2em}
    \qquad\fangsong
    “当然,平等趋势是我们这个世纪所特有的。但是,说以往各世纪及其完全不同的需求、生产资料等等都是为实现平等而遵照天命行事,这首先就是把我们这个世纪的人和生产资料当做过去世纪的人和生产资料看待,否认世世代代不断改变前代所获得的成果的历史运动。经济学家们很清楚,同是一件东西对甲说来是成品,对乙说来只是从事另一种生产的原料。
    
    如果你们同蒲鲁东先生一道假定:社会天才制造出,或者更确切些说随兴制造出封建主,是为了达到把耕者变为负有义务的和彼此平等的劳动者这一天命的目的,那末,你们就是把目的和人换了一下,这种做法和为了达到恶意的满足(即羊群赶走人)而在苏格兰确立土地私有制的天命比较起来,毫不逊色。”
\end{adjustwidth}

马克思这两段论述同样地体现出了满满的历史唯物主义原则。马克思承认,“平等”确实是当时那个世纪所特有的思想倾向,但他想进一步表明的是,并不是自古以来就一直存在着这种“平等”的思想倾向的。如果将“平等”这种仅仅在当时那个世纪所特有的思想倾向看作是永恒不变的思想倾向的话,就意味着,当时那个世纪的人的物质生产活动同以往一切世纪的物质生产活动之间并无差别了。而这种观念是反历史唯物主义的,在这里就不赘述了。

\subsubsection{本期小结:以及一些余论}
本期我们讨论的比较热烈,同时一些新同学也在本期加入了我们的讨论会。在本期讨论会的最后几分钟,我们初步探讨了一下古典国民经济学(特别是李嘉图的经济学)同马克思主义经济学之间的差异。我们探讨了“李嘉图难题”的产生,以及马克思是如何解决李嘉图难题的(但我们只是初步地提了一嘴,并没有深入讨论下去)。我们计划在下期的讨论会中留一段时间和大家讨论一下李嘉图的利润理论与马克思之间的差异,以及马克思是如何系统地批判李嘉图经济学的。总之,这是一期收获很满的讨论会。
\newpage
\section{第三期:《哲学的贫困》选读(完结)+《雇佣劳动与资本》(1)}

\subsection{读书会记录}
2023年11月6日18:00—20:00,我们在东北林业大学奥林学院203教室开展了第三期的读书会\footnote{编者注:由于2023年10月30日大家感冒的比较多,我们就将第三期读书会推迟到了11月6日。}。本期读书会我们结束了《哲学的贫困》第二章第一节的讨论,由于前两期的积累,我们对于《哲学的贫困》第二章第一节中\textbf{“第七即最后一个说明”}这部分内容的讨论比较顺利。之后,我们开启了马克思的另一篇政治经济学经典著作《雇佣劳动与资本》的讨论。总之,和之前几期一样,我们进行了充分而深刻的讨论,具体内容可概括如下:

\subsubsection{1.第七即最后一个说明}
“第七个说明”这部分内容更像是对前面六个说明的总结,整体读下来不难理解,不过还是存在一些值得深入讨论的细节之处。
例如,马克思的这两段话是这么说的:

\begin{adjustwidth}{2em}{2em}
    \qquad\fangsong
 “这样,为了正确地判断封建的生产,必须把它当做以对抗为基础的生产方式来考察。必须指出,财富怎样在这种对抗中间形成,生产力怎样和阶级对抗同时发展,这些阶级中一个代表着社会上坏的、否定的方面的阶级怎样不断地成长,直到它求得解放的物质条件最后成熟。这难道不是说,生产方式、生产力在其中发展的那些关系并不是永恒的规律,而是同人们及其生产力发展的一定水平相适应的东西,人们生产力的一切变化必然引起他们的生产关系的变化吗?由于最重要的是不使文明的果实(已经获得的生产力)被剥夺,所以必须粉碎生产力在其中产生的那些传统形式。从此以后,从前的革命阶级将成为保守阶级。

资产阶级开始自己的历史发展时就有一个本身是封建时期无产阶级残存物的无产阶级存在。资产阶级在其历史发展过程中不可避免地要发展它的对抗性质,起初这种性质或多或少是掩饰起来的,只是处于隐蔽状态。随着资产阶级的发展,在它的内部发展着一个新的无产阶级,即现代无产阶级。无产阶级同资产阶级之间展开了斗争,在双方尚未感觉、注意、重视、理解、承认并公开宣告以前,这个斗争最初仅表现为局部的暂时的冲突,表现为一些破坏行为。另一方面,如果说现代资产阶级的全体成员由于组成一个与另一个阶级相对立的阶级而有共同的利益,那末,由于他们互相对立,他们的利益又是对立的,对抗的。这种利益上的对立是由他们的资产阶级生活的经济条件产生的。资产阶级运动在其中进行的那些生产关系的性质绝不是一致的单纯的,而是两重的;在产生财富的那些关系中也产生贫困;在发展生产力的那些关系中也发展一种产生压迫的力量;只有在不断消灭资产阶级个别成员的财富和形成不断壮大的无产阶级的条件下,这些关系才能产生资产者的财富,即资产阶级的财富;这一切都一天比一天明显了。”
\end{adjustwidth}

这里马克思主要是在阐述他的历史唯物主义思想。我们可以看到,在上面的第一段中,马克思说:
\begin{adjustwidth}{2em}{2em}
    \qquad\fangsong
    “这难道不是说,生产方式、生产力在其中发展的那些关系并不是永恒的规律,而是同人们及其生产力发展的一定水平相适应的东西,人们生产力的一切变化必然引起他们的生产关系的变化吗? ”
\end{adjustwidth}

在这里便有一个疑问,什么叫\textbf{“生产方式在其中发展的那些关系”}呢?我们知道,在马克思看来,“生产方式”本身就应包涵着“生产关系”与“生产力”两个维度,但马克思在这里又说“生产方式在其中发展的关系”,这是否有同义反复的嫌疑了呢?我们认为,这是一种很微妙的说法,马克思在这里想表示的应该就是“生产关系”本身,或者认为,马克思这里说的“生产方式”是一种更为具体的“生产形式”,也是生产力的一种表现。总之,这并不影响整段所论述的思想,这一细节之处可留给读者们细细思索。

此外,在上面的第二段中,马克思论述了无产阶级的生成以及无产阶级同资产阶级二者的斗争。马克思在这里对无产阶级的描述同样比较微妙,马克思既不是从经济关系出发,将无产阶级表述为不占有生产资料的阶级,也不是从哲学层面出发,即像在《<黑格尔法哲学批判>导言》中所表述的那样:“一个被戴上彻底的锁链的阶级,一个并非市民社会阶级的市民社会阶级,形成一个表明一切等级解体的等级”\footnote{编者注:见马克思《<黑格尔法哲学批判>导言》。}。

我们认为,马克思在这里更想说明的是无产阶级的自我发展过程,即从自在到自为的过程。马克思在这里首先指明了无产阶级最早生成于资本主义发展的初期,从某种意义上说,无产阶级是伴随着资产阶级的发展而发展的。然而在资本主义发展的前期,无产阶级与资产阶级之间的阶级对抗不明显,因而在那一时期,资本主义社会的对抗性质“或多或少是掩饰起来的,只是处于隐蔽状态。”因此,在这种对抗性不明显的社会下,无产阶级的阶级意识\footnote{编者注:这里借用了卢卡奇的术语“阶级意识”。}同样是不明显的,这便是一种“自在”的状态。而随着资本主义社会的进一步发展,阶级对抗也愈来愈明显,如马克思所言,资产阶级的运动由于资本主义生关系的作用所表现出了双重性的结果,即资产阶级的历史运动一方面制造着大量的财富(但这种财富是个别性的、属于个别资产阶级成员的财富),另一方面同时又制造着大量的贫困(这种贫困是一种普遍性的贫困)。这种直接在物质利益上的对抗促使了无产阶级的阶级意识的提高,也意味着无产阶级从“自在”向“自为”的转变\footnote{编者注:这里用的词是“转变”,并没有明确指出无产阶级必然会随着物质利益的对抗而成为“自为阶级”,因为其实在这里有一些疑问,当代资本主义国家的工人阶级是否可以被称作是“自为阶级”呢?}。

马克思在“第七个说明”后面部分的论述就很像《共产党宣言》了。不过马克思在《共产党宣言》中是对形形色色的社会主义者进行了划分,而在“第七个说明”中则是对政治经济学语境下的形形色色的改良主义学派进行了划分。在这里,按照马克思在“第七个说明”中的划分,我给大家做一个简短的归纳总结:
\begin{tcolorbox}
    \textbf{1.宿命论的经济学家:}斯密、李嘉图等代表的古典政治经济学,可以被看作是一种无批判的实证主义。

    \textbf{2.浪漫派:}对宿命论经济学家的直接继承,但他们甚至认为无产阶级的贫困是“理所应当”的。

    \textbf{3.人道学派:}企图在道德层面约束资产阶级。

    \textbf{4.博爱学派:}人道学派的进一步完善,幻想着人人都能变成资产者。

    \textbf{5.空想社会主义者:}在无产阶级解放的物质存在条件不足的情况下,企图寻得无产阶级解放的理论。
\end{tcolorbox}

最后,回到出发点,马克思对蒲鲁东做了一个总结性的评论,在这里我们就不过多解释了,我仅把原文放出来,根据前面几期的内容读者们应该能够体会到马克思的意思了。

马克思在最后写道:
\begin{adjustwidth}{2em}{2em}
    \qquad\fangsong
    “现在再来谈谈蒲鲁东先生。

每一种经济关系都有其好的一面和坏的一面;只有在这一点上蒲鲁东先生没有背叛自己。他认为好的方面由经济学家来揭示,坏的方面由社会主义者来揭发。他从经济学家那里借用了永恒经济关系的必然性这一看法;从社会主义者那里借用了使他们在贫困中只看到贫困的那种幻想。他对两者都表示赞成,企图拿科学权威当靠山。而科学在他的观念里已成为某种微不足道的科学公式了;他无休止地追逐公式。正因为如此,蒲鲁东先生自以为他既批判了政治经济学,也批判了共产主义;其实他远在这两者之下。说他在经济学家之下,因为他作为一个哲学家,自以为有了神秘的公式就用不着深入纯经济的细节;说他在社会主义者之下,因为他既缺乏勇气,也没有远见,不能超出(哪怕是思辨地也好)资产者的眼界。

他希望成为一种合题,结果只不过是一种总合的错误。

他希望充当科学泰斗,凌驾于资产者和无产者之上,结果只是一个小资产者,经常在资本和劳动、政治经济学和共产主义之间摇来摆去。”
\end{adjustwidth}

\subsubsection{2.雇佣劳动与资本(1)}
本期讨论会我们把《雇佣劳动与资本》开了个头,本期具体讨论的内容在我们所编排的讲义的18—21页。在进一步展开我们的讨论内容之前,我有必要跟读者们交代一下《雇佣劳动与资本》这部著作的写作背景以及后续恩格斯在1891对此做出的一些修改与补充。

\paragraph{一个简要的背景概述}\begin{fangsong}
《雇佣劳动与资本》这部著作是马克思根据1847年12月在布鲁塞尔德意志工人协会发表的演说写成的,最初以社论形式于 1849年4月5—8日和11日在《新莱茵报》陆续发表。后来由于《新莱茵报》被迫停刊,这部著作的连载遂告中断。 
时间来到1891年,为适应工人群众学习科学社会主义理论的需要,在恩格斯的关心下,这部著作的新单行本在柏林印行。 恩格斯根据《资本论》的基本观点和科学论述,对马克思这部著作进行了适当的修改和补充,并为该著作的单行本写了一篇《导言》。 恩格斯在《导言》中指出:“我所作的全部修改,都归结为一点。 在原稿上是,工人为取得工资向资本家出卖自己的劳动,在现在这一版本中则是出卖自己的劳动力。”\footnote{编者注:见马恩选集第一卷(中央编译局2012年版)第318页。}恩格斯阐明了修改的理由,论述了马克思主义政治经济学的科学价值,揭露了资本主义制度的本质,指出工人阶级不仅是社会财富的生产者,而且是新的社会制度的创造者。
\end{fangsong}

事实上,就马克思在1849年在《新莱茵报》发表的原版内容而言,我们不得不承认其中存在着不可忽视的局限性,当时的马克思并没有完成他的政治经济学批判工作,虽然当时他能够在整体性层面洞察出资本主义经济社会的固有局限,但是就一些关键的细节之处,例如对“劳动”与“劳动力”之间的区分,马克思的观点还是处于一个模糊的状态。然而经过恩格斯修改之后的版本就好多了。

 \vspace{0.5cm} %设置垂直距
\textbf{在把背景交代清楚之后,接下来便可以展开我们的讨论了。}
首先,按照原文的顺序,我们跟着马克思的思路讨论了“什么是工资?”的问题。对于这一问题的回答我们首先需要从现实的经验现象中去寻找答案。马克思说,假如去问工人们什么是他们的工资,他们会回答:“工资是工作的一定时间所换来的报酬。”可见,工资似乎表现为是工人同资本家之间进行的交换所获得的价值物,更确切地说,似乎是工人工作的一定时间(或一定量的工作)同资本家之间进行的交换所获得的价值物。但事实上并非如此,因为当我们考虑一下这个“交换”本身是否同其他商品之间的交换存在着差异时,我们便会发现,工人为获取工资同资本家之间进行的交换同他为获取生活资料同卖主进行的交换没有什么不同:这种交换总是遵循着“等价交换”的原则的\footnote{编者注:因为如果不是这样的话,商品经济本身便会出现混乱。}。那为什么工人所获得的工资与工人所生产出的商品的价值量之间存在着不一致呢?事实上,问题的关键则在于,工人同资本家所付给他的工资之间所进行的等价交换的那个“物”并不是劳动本身,而是劳动力商品。事实上,劳动是无法成为商品的,劳动本身没有价值,关于这一点的详细论述,大家可以去看《资本论》第一卷第四章的“劳动力的买和卖”以及第六章的“工资”\footnote{编者注:事实上,在《雇佣劳动与资本》中,即使是恩格斯修改之后的版本,也并没有详细地论述劳动力商品同劳动之间的差异性,因为恩格斯是直接根据《资本论》第一卷对其进行的修改与增补的,因而在其中有些概念的出场会显得缺少一些前提性的阐述,因此,我还是推荐大家直接去看《资本论》第一卷,这是一个苦功夫,需要静下心来好好研读,在这里我就不赘述其具体内容了。}。

\textbf{因此,仅仅就劳动力商品与工资之间的交换而言,这一过程不存在剥削!}\footnote{编者注:请读者们记住这一点,在不考虑生产价格理论的前提下,这一点是完全正确的,我和光玉、许婕老师都是赞成的,但是在后面关于劳动与自由之间的关系时,我们之间发生了一定的分歧。}如果有人固执地认为劳动力商品同工资之间的交换过程发生了剥削,我想他大概率是不懂马克思主义政治经济学的。

接下来的这一段引起了我们的广泛讨论:

\begin{adjustwidth}{2em}{2em}
    \qquad\fangsong
“可是,劳动是工人本身的生命活动,是工人本身的生命的表现。工人正是把这种生命活动出卖给别人,以获得自己所必需的生活资料。可见,工人的生命活动对于他不过是使他能以生存的一种手段而已。他是为生活而工作的。他甚至不认为劳动是自己生活的一部分;相反地,对于他来说,劳动就是牺牲自己的生活。劳动是已由他出卖给别人的一种商品。因此,他的活动的产物也就不是他的活动的目的。工人为自己生产的不是他织成的绸缎,不是他从金矿里开采出的黄金,也不是他盖起的高楼大厦。他为自己生产的是工资,而绸缎、黄金、高楼大厦对于他都变成一定数量的生活资料,也许是变成棉布上衣,变成铜币,变成某处地窖的住所了。一个工人在一昼夜中有十二小时在织布、纺纱、钻孔、研磨、建筑、挖掘、打石子、搬运重物等等,他能不能认为这十二小时的织布、纺纱、钻孔、研磨、建筑、挖掘、打石子是他的生活的表现,是他的生活呢?恰恰相反,对于他来说,在这种活动停止以后,当他坐在饭桌旁,站在酒店柜台前,睡在床上的时候,生活才算开始。在他看来,十二小时劳动的意义并不在于织布、纺纱、钻孔等等,而在于这是挣钱的方法,挣钱使他能吃饭、喝酒、睡觉。假如说蚕儿吐丝作茧是为了维持自己的生存,那末它就可算是一个真正的雇佣工人了。”
\end{adjustwidth}

马克思在这里说:“可见,工人的生命活动对于他不过是使他能以生存的一种手段而已。他是为生活而工作的。他甚至不认为劳动是自己生活的一部分;相反地,对于他来说,劳动就是牺牲自己的生活。”
关于这一方面的论述,可以引申到对\textbf{劳动与自由之间的关系}的讨论。\textbf{在关于这一点的看法上,我们出现了分歧。}下面我将尽我所能还原当时我们争论的情况\footnote{编者注:遗憾的是,正如马克思的好朋友、著名的诗人海涅曾在其所著的《论德国宗教和哲学的历史》一书中所言的那样:“箭一离开弦便不再属于射手了,言论一离开说话人的口……便不再属于他了。”即使我全程极度认真地参与了这场讨论,但我也无法完全准确地描绘当事人的思想观点,请原谅我可能在某些地方对当事人的观点存在着的误解。}:

\paragraph{观点1:}\begin{fangsong}
以往似乎有一种误解,认为自由王国只有在共产主义社会才能达到,但其实在资本主义社会,自由王国与必然王国二者是以一种二分的形式所存在着的。诚然,当工人将自己的劳动力商品出卖给资本家之后,在资本家使用劳动力商品进行价值生产的这一过程中,工人事实上是不自由的,是处在一种“必然王国”之中。但是,一旦工人脱离生产过程,也即是说,当工人跨入到对于其自身的劳动力商品的恢复过程之中时,他实际上是暂时性地脱离了必然王国,进入到自由王国之中了。这应该如何理解呢?可以认为对于“工人恢复其自身劳动力商品”的这一过程而言,这一过程不涉及生产的领域,换句话说,这一过程仅仅是工人的纯粹消费的领域,不涉及价值与剩余价值的创造。因而这一领域不具有任何剥削性,换言之,这一领域是工人真正的生活领域——对于工人而言的仅仅纯粹的消耗使用价值的领域,完全是一个自然的过程。举个例子,比如说某工人一天工作了八个小时,赚了100块钱,他下班了给自己买了一个烤鸭吃,从这一行为中是无法分析出什么样的社会关系的,这一行为仅仅是工人消耗使用价值的过程,因而从这个意义上来说,这一个过程是“自由”的(至少相对于生产过程而言是自由的)。
\end{fangsong}
\paragraph{观点2:}\begin{fangsong}
我们认为“观点1”中所认为的那种“工人的纯消费领域属于自由王国”的看法是存在局限性的。我们同样承认剩余价值产生于生产领域,这即意味着在工人的纯消费领域不存在着剥削。但我们认为对于问题的分析不能止步于此。因为事实上,工人的消费可以被看作是资本主义生产过程的延申。诚然,工人在纯消费领域进行的活动仅仅是对使用价值的消耗,但这并不意味着工人阶级脱离了必然王国,我们认为工人阶级仍然是非自由的。这是因为,工人阶级的消费事实上具有两重性:一方面满足了工人阶级自身的物质需求\footnote{编者注:我们甚至对这种“满足工人阶级自身物质需求”的观点还是持有保留意见的,因为处于贫困水平的工人阶级似乎连这一点也无法被满足。};另一方面,工人阶级的消费事实上为资本主义的再生产创造了条件,而这一方面则是我们批判“观点1”的关键。我们认为,工人阶级的消费实际上处于一种“知其不可而为之”的境遇,工人阶级为了生存(生活)他必须进行消费,但工人阶级的这种消费同时又为下一次资产阶级对其的剥削创造了物质条件。在这里我们做一个简单的政治经济学分析:对于资本家而言,雇佣工人进行生产这一过程诚然是至关重要的,因为这一过程直接涉及到了价值与剩余价值的生成、涉及到了资本家对工人阶级的剥削,但是生产过程的结束对于资本家而言并不是直接的结果,资本家必须要使得剩余价值转换为最终的利润\footnote{编者注:这涉及到马克思的生产价格理论,在这里就不赘述了。有兴趣的读者可以阅读《资本论》第三卷的前10章。},而这种转换恰恰是通过工人阶级的消费而得以实现的。因此,在这个意义上,我们认为工人阶级的纯消费领域依然是受到着资本主义制度的规制,并没有脱离必然王国。    
\end{fangsong}
\vspace{0.5cm} %设置垂直距

我们之间的分歧主要是“观点1”和“观点2”之间的分歧。光玉老师的看法倾向于“观点1”,我和许婕老师的看法倾向于“观点2”。并且,从某种意义上而言,我们似乎都能够理解彼此观点的侧重之处。事实上,我认为,在政治经济学原理部分,我们都能清晰地认识到剩余价值的生产与剩余价值的实现之间的区别,我们也都能理解商品经济的等价交换原则。因此,相较于政治经济学原理而言,我们之间更像是对于“自由”这一范畴的看法出现了分歧,而对于“自由”的理解,在我看来是一个哲学问题,因而我们之间的分歧,确切的说是一个哲学层面的分歧。

事实上,马克思早年就在其博士论文《论德谟克利特的自然哲学与伊壁鸠鲁的自然哲学之差别》中探讨了“自由”的问题。当时的马克思由于受到青年黑格尔派的自我意识哲学的影响,他致力于通过诉诸一种“抽象个别性的自我意识”的形式去对抗当时存在着的普遍的宗教威权。(关于这部分的内容,我就不在这里赘述了,有兴趣的读者可以自行去阅读马克思的博士论文,或者在将来的读书会中继续探讨这部分的内容\footnote{编者注:估计这学期不会讨论到这部分内容了,这学期的主题是马克思的政治经济学。}。)当然,现在我们知道,当时的马克思的自由思想是存在着局限性的,因为对于“抽象个别性的自我意识”的形式而言,其代表的是一种“脱离定在的自由”,马克思甚至在当时也意识到了这种“自由”是无法“在定在之光中发亮”的。

让我们顺着这个思路继续下去,我们会意识到,不存在绝对的自由。进而再次回到我们之间争论的出发点,“工人阶级的纯消费领域”的自由\footnote{编者注:我们暂且假定这种自由存在。}便是一种“定在”之中的自由,而这个“定在”本身就是资本主义社会,更具体地说,是资本主义市场机制。我想关于我在上面的那种诠释,我们之间应该都是认同的。如果大家都能够接受这一点的话,那末争论的焦点便又一次地被转移了,现在的问题变成了\textbf{工人阶级在资本主义市场经济下的“消费自由”是否是“真实的自由”?}关于这一问题,我想我在这里还是不要做出回答的好,这部分留给读者们细细思索了。

\subsubsection{本期小结:以及一些余论}
本期讨论会的人数比较少,但却是讨论的最为热烈的一期。当天哈尔滨下了一天的大雪,不过恶劣的天气却丝毫没有减少大家学习与讨论的热情,我想这就是马克思主义政治经济学的魅力吧!此外,在本期讨论会中,我们还提到了20世纪60年代出现的“斯拉法体系”\footnote{编者注:详见皮埃罗·斯拉法《用商品生产商品》。}、马克思的“地租理论”、恩格斯对农民问题的研究、日本马克思经济学家置盐信雄提出的“置盐定理(Okishio Theorem)”、森岛通夫等人提出的“马克思主义基本定理(Fundamental Marxian Theorem,FMT)”等。这些内容大多是在讨论某一具体点的时候顺便提的一嘴,没有具体的深入展开。事实上,上述提到的每个部分都蕴含着极为丰富的内容,限于篇幅限制,我就不把这些具体内容一一编排到本书中了\footnote{编者注:可能在之后我会以附录的形式编排一些内容。},如果有成员对其中某一方面感兴趣的话,可以在群里提出来,我会给大家发相关的学习资料。总而言之,这是一期收获满满的讨论会。
\newpage
\section{第四期:《雇佣劳动与资本》(2)}
\subsection{读书会记录}
2023年11月13日18:00—20:00,我们在东北林业大学奥林学院203教室开展了第四期的读书会。本期读书会我们继续探讨了马克思《雇佣劳动与资本》中的部分内容。接下来我将对本期讨论内容进行简要概述。

\subsubsection{1.劳动力并不向来就是商品}
我们对马克思的这段话进行了一定的讨论:
\begin{adjustwidth}{2em}{2em}
    \qquad\fangsong
    “劳动\footnote{在1891年的版本中,“劳动”改为“劳动力”。——编者注}并不向来就是商品。劳动并不向来就是雇佣劳动、即自
由劳动。奴隶就不是把他自己的劳动 \footnote{同上}出卖给奴隶主,正如耕牛不
是向农民卖工一样。奴隶连同自己的劳动 \footnote{同上} 一次而永远地卖给自
己的主人了。奴隶是商品,可以从一个所有者手里转到另一个所有
者手里。奴隶本身是商品,但劳动  \footnote{同上} 却不是 他的商品。农奴只出卖
自己的一部分劳动  \footnote{同上} 。不是他从土地所有者方面领得报酬;相反
地,土地所有者从他那里收取贡赋。农奴是土地的附属品,替土地
所有者生产果实。相反地,自由工人自己出卖自己,并且是零碎地
出卖。他每天把自己生命中的八小时、十小时、十二小时、十五小时
拍卖给出钱最多的人,拍卖给原料、劳动工具和生活资料的所有者,即拍卖给资本家。工人既不属于私有者,也不属于土地,但是他
每日生命的八小时、十小时、十二小时、十五小时却属于它的购买
者。工人只要愿意,就可以离开雇用他的资本家,而资本家也可以
随意辞退工人,只要工人使他不能再获得利益或者不能使他获得
预期的利益,他就可以辞退。但是,工人是以出卖劳动 \footnote{同上} 为其工资
的唯一来源的,如果他不愿饿死,就不能离开整个购买者阶级即资
本家阶级。工人不是属于某一个资产者,而是属于整个资产阶
级\footnote{在1891年的版本中,“不是属于某一个资产者,而是属于整个资产阶级”改为“不是属于某一个资本家,而是属于整个资本家阶级”。——编者注};至于工人给自己寻找一个雇主,即在资产阶级\footnote{在1891年的版本中, “资产阶级”改为“资本家阶级”。——编者注} 中间寻找一
个买主,那是工人自己的事情了。”
\end{adjustwidth}

马克思在这里说“劳动力并不向来就是商品。劳动并不向来就是雇佣劳动、即自由劳动。”这种说法实际上便是同资产阶级经济学家所认为的“经济范畴永恒存在”的观点相区别了。我们认为,马克思在这里想表达的还是他的历史唯物主义的哲学观点,即一种透过现象看本质的哲学观。这事实上是不难理解的,因为“劳动力”作为工人自身所内蕴的一种自然属性,从发生学机制层面来看,它最初显然不可能是作为体现着社会关系的“商品”而发生的。因此,对于“劳动力商品”这一规定——即劳动力是商品——而言,这种规定性一定是在历史中生成的。马克思在这里举了个例子,他说,“奴隶向奴隶主出卖自身”实际上与“资本主义社会下的工人向资本家出卖劳动力商品”是截然不同的两种情况。因为在奴隶制社会里,奴隶本身就是商品,但奴隶自身的劳动力却不是奴隶自己的商品,即对于奴隶自身而言,他是丧失了全部自主性(主体性)的;而在资本主义社会里,工人的劳动力却是他自己的商品,即工人在一定程度上具有自主性(主体性),因而这便是这两种情况的差异所在。

我们认为,这两种情况的差异实际上就是由不同历史阶段上的不同社会关系所造成的,资本主义的雇佣劳动形式打破了奴隶社会存在的直接人身束缚,这是一种进步,但是资本主义雇佣劳动本身又是一种对于工人而言的无形的束缚。就像马克思在《<黑格尔法哲学批判>导言》里面带有讽刺意味地谈论马丁·路德的宗教改革时所说的那样:
\begin{adjustwidth}{2em}{2em}
   \qquad\fangsong
   “的确, 路德战胜了虔信造成的奴役制,是因为他用 信念 造成的
奴役制代替了它。 他破除了对权威的信仰,是因为他恢复了信仰
的权威。 他把僧侣变成了世俗人,是因为他把世俗人变成了僧侣。
他把人从外在的宗教笃诚解放出来,是因为他把宗教笃诚变成了
人的内在世界。 他把肉体从锁链中解放出来,是因为他给人的心
灵套上了锁链。”
\end{adjustwidth}

我就不再继续解读了,请读者们自行领会。但总的来说,无论如何,奴隶制社会与资本主义社会由于整个社会的生产方式——进而是生产关系——之间的差异,导致了经济形式的差异。资本主义社会的经济关系并非永恒不变的自然产物,恰恰相反,这种经济关系是在社会历史运动中生成。并且,还需注意的是,由于这些“关系”需要依靠物质载体得以表现自身,因而人们在面对社会现实时,由于面对的是直接的物质载体,因而往往会错把历史中生成的“关系”看作是永恒的自然法则,这便是一种\textbf{遮蔽},事实上,马克思在后面展开他的意识形态批判理论的时候便是要对这种遮蔽进行\textbf{祛魅}。关于这一点,我也不再这里继续赘述了,下期讨论会的时候大家可以线下交流。

\subsubsection{2.商品的价格是由什么决定的?}
说实话,我认为《雇佣劳动与资本》的这部分内容是稍微有些混乱的,尽管恩格斯修改之后的版本会好一点,然而其中的某些概念的论述在现在看来还是有些模糊不清的,特别是马克思在这里没有刻意区分“价值”和“价格”\footnote{编者注:实际上,对于“价值”和“价格”的区分是一件非常重要的事情,甚至不亚于对“劳动力”与“劳动”之间的区分。\textbf{在这里请读者们务必要弄清,“价格”是一个纯粹现象层面的表现,而“价值”则是透过现象看本质的那个“本质”,这是理解马克思主义政治经济学的一个关键!}}。马克思在这部分首先说,商品的价格\footnote{编者注:这里的价格指的就是现象层面的价格。}是由“买主和卖主之间的竞争即供求关系决定的”。转而马克思分析了这一“竞争即供求关系”本身。我们认为,事实上,马克思在这里的分析是处在一个现象的层面的,因而也是较为贴近现实的、令人容易理解的。马克思在这里指出,影响商品价格的竞争涉及三个方面,即卖主之间的竞争、买主之间的竞争、卖主同卖主之间的竞争。

我在这里给大家做一个简要概述。在马克思看来,卖主之间的竞争压低了商品的“供给价格”,买主之间的竞争提高了商品的“需求价格”,卖主与买主之间的竞争促使了最终“市场价格”的形成。看完上面的概述,我想读者们应该就不难理解马克思的这几段话了:

\begin{adjustwidth}{2em}{2em}
    \qquad\fangsong
    “同一种商品,有许多不同的卖主供应。谁以最便宜的价格出卖
同一质量的商品,谁就一定会战胜其他卖主,从而保证自己有最大
的销路。于是,各个卖主彼此间就进行争夺销路、争夺市场的斗争。
他们每一个人都想出卖商品,都想尽量多卖,如果可能,都想由他
一个人独卖,而把其余的卖主排挤掉。因此,一个人就要比另一个
人卖得便宜些。于是卖主之间就发生了竞争,这种竞争降低他们所
供应的商品的价格。

但是买主之间也有竞争,这种竞争反过来提高所供应的商品
的价格。

最后,买主和卖主之间也有竞争。前者想买得尽量便宜些,后
者却想卖得尽着贵些。买主和卖主之间的这种竞争的结果怎样,要
依上述竞争双方的对比关系怎样来决定,就是说要看是买主阵营
里的竞争激烈些呢还是卖主阵营里的竞争激烈些。产业把两支军
队抛到战场上对峙,其中每一支军队内部又发生内讧。战胜敌人
的是内部冲突较少的那支军队。”
\end{adjustwidth}

此外,这里还有一个思考点,即对于劳动力这一特殊商品而言,它的价格(即工资)是否也决定于以上三个层面的竞争呢?我倾向认为也是这样的。影响劳动力商品的这种竞争的具体情况同样可以分为三个方面:资本家之间的竞争会提高工人阶级的整体工资水平(但往往资本家会发现,相较于竞争,他们更倾向于联合),工人阶级内部的竞争会压低工人阶级整体的工资水平(工人阶级的内卷),工人阶级同资本家阶级之间的博弈会影响一定历史时期的工资水平。因而从某种意义上说,我认为,工资的大小在马克思看来似乎是一种外生给定的因素,即一定社会生产力水平下的一定历史时期存在着一个特定的工资水平。

\textbf{但上述那种竞争似乎又不是决定商品价格的根本因素。}\footnote{编者注:竞争当然不是决定商品价格的根本因素,价值才是,但这里马克思对于政治经济学的一些概念的认识还是较为模糊。}我认为马克思在这里是想表明,竞争、供求等因素会导致商品价格的上下波动,但商品的价格不会一直“虚高”也不会一直“虚低”,这里隐约地表明着商品的价格是由另一种因素所决定的,马克思发现了这个因素,在本部分内容中马克思将其称作“生产费用”\footnote{编者注:遗憾的是,马克思在这里的叙述还是处在一个模棱两可的状态,根据他的论述,生产费用似乎指的既是生产成本,又是生产价格,甚至还表示着价值。但生产成本、生产价格、价值是截然不同的东西,因而这是一种矛盾的叙述。}。马克思在这里是这么论述的:

\begin{adjustwidth}{2em}{2em}
    \qquad\fangsong
    “我们刚才说过,需求和供应的波动,每次都把商品的价格引导
到生产费用的水平。固然,商品的实际价格始终不是高于生产费
用,就是低于生产费用;但是,上涨和下降是相互抵销的,因此,在
一定时间内,如果把工业中的资本流入和流出总合起来看,就可看
出各种商品是依其生产费用而互相交换的,所以它们的价格是由
生产费用决定的。

价格由生产费用决定这一点,不应当了解成像经济学家们所
了解的那种意思。经济学家们说,商品的平均价格等于生产费用;
在他们看来,这是一个规律。他们把价格的上涨被价格的下降所抵
销,而下降则被上涨所抵销这种无政府状态的变动看作偶然现象。
那末,同样也可以(另一些经济学家就正是这样做的)把价格的波
动看作规律,而把价格由生产费用决定这一点看作偶然现象。可是
实际上,只有在这种波动的进程中,价格才是由生产费用决定的;
我们细加分析时就可以看出,这种波动起着极可怕的破坏作用,并像地震一样震撼资产阶级社会的基础。这种无秩序状态的总运动
就是它的秩序。在这种产业无政府状态的进程中,在这种循环运转
中,竞争可以说是拿一个极端去抵销另一个极端。

由此可见,商品价格是由生产费用这样来决定的:某些时期,
某种商品的价格超过它的生产费用,另一些时期,该商品的价格却
下跌到它的生产费用以下,而抵销以前超过的时期,反之亦然。当
然,这不是就个别产业的产品来说的,而只是就整个产业部门来说
的。所以,这同样也不是就个别产业家来说的,而只是就整个产业
家阶级来说的。”
\end{adjustwidth}
上面那几段论述中的“生产费用”事实上可以理解为“生产价格”。但随之马克思的这段论述又让人陷入了迷惑:
\begin{adjustwidth}{2em}{2em}
    \qquad\fangsong
    “价格由生产费用决定,就等于说价格由生产商品所必需的劳
动时间决定,因为构成生产费用的是:(1)原料和劳动工具\footnote{在1891年的版本中, “劳动工具”改为“劳动工具损耗部分”。——编者注} ,即产
业产品,它们的生产耗费了一定数量的工作日,因而也就是代表一
定数量的劳动时间;(2)直接劳动,它也是以时间计量的。”
\end{adjustwidth}
上面这段话中的“生产费用”更应该被理解为“价值”。因为马克思用生产商品的“必要劳动时间”代替了“生产费用”,这显然指的是商品的价值。不过值得欣慰的是,若是将上面的“生产费用”理解为“商品的价值”的话,事实上在这里马克思已经超越了斯密教条。因为马克思意识到了,商品的价值的实际上是由两个部分组成的,即生产资料的价值的转移(即死劳动的转移、物化劳动的转移)和活劳动的对象化(或者称做活劳动的物化)。

之后,马克思还论述了工人工资一般会处在一个水平,即:
\begin{adjustwidth}{2em}{2em}
\qquad\fangsong
“单个工人所得,千百万工人所得,不足以维持生存和延续后代,
但整个工人阶级的工资在其波动范围内则是和这个最低额相等
的。”
\end{adjustwidth}
这里可以对照《1844年经济学哲学手稿》中的部分内容来看,这里我就不赘述了。
\subsubsection{3.资本是一种社会生产关系}
马克思说:
\begin{adjustwidth}{2em}{2em}
    \qquad\fangsong
“资本包括原料、劳动工具和各种生活资料,这些东西是用以生产新的原料、新的劳动工具和新的生活资料的。资本的所有这些组成部分都是劳动的创造物,劳动的产品,积累起来的劳动。作为进行新生产的手段的积累起来的劳动就是资本。

经济学家们就是这样说的。

什么是黑奴呢?黑奴就是黑种人。上面的说明和这个说明是一样的。

黑人就是黑人。只有在一定的关系下,他才成为奴隶。纺纱机是纺棉花的机器。只有在一定的关系下,它才成为资本。脱离了这种关系,它也就不是资本了,就像黄金本身并不是货币,沙糖并不是沙糖的价格一样。”
\end{adjustwidth}
马克思在这里是想表明,如果仅仅把资本看作是“作为新生产的手段的积累起来的劳动”是不够的。这种解释同那种认为“黑奴就是黑种人”的观点并无差异。借用复旦大学吴晓明老师的说法,这种解释只能被称作是“不错”,但“不错”不能被称为“真”,譬如我问你水果是什么?你告诉我是苹果和香蕉,那么我应该说“不错”,但如果我是一个生物学教授,我要你回答的是水果的定义,而不要你告诉我苹果、香蕉和橘子是水果,但我也不能说你不对,只能说“不错”。因此,当我们在考察某一事物时,绝不可以仅仅止步于外部的观察层面,必须要将对事物所处的社会关系的考察同对事物本身的考察联系起来。这是进入马克思主义政治经济学研究的一项基本要求。同样,关于这一点,请读者们细细思索,我就不再赘述了。

因此,不难理解,马克思会说:

\begin{adjustwidth}{2em}{2em}
    \qquad\fangsong
    “资本也是一种社会生产关系。这是资产阶级的生产关系,是资产阶级社会的生产关系。构成资本的生活资料、劳动工具和原料,难道不是在一定的社会条件下,不是在一定的社会关系下生产出来和积累起来的吗?难道这一切不是在一定的社会条件下,在一定的社会关系内被用来进行新生产的吗?并且,难道不正是这种一定的社会性质把那些用来进行新生产的产品变为资本的吗?”
\end{adjustwidth}

\subsubsection{本期小结}
本期读书会同样见到了许多新面孔,大家讨论的也是较为热烈的。本期我们讨论的内容比较多,这一方面是因为我们想加快一下阅读的速度,另一方面则是因为马克思的这部分内容的叙述相较于其他文本而言算是较为容易理解的。此外,在本期读书会我们又讨论了一下置盐定理,并且在读书会之后我在群里给大家发了我临时编写的简化版的置盐定理概述,我将在本期读书会记录的附录页中附上我编写的更为严谨的置盐定理概述\footnote{编者注:当然,我依然会省略大部分的数学推导过程。},请大家按需查阅。总之,这同样是一期收获满满的讨论会。

\newpage
\subsection{本期附录}
\subsubsection{1.置盐定理概述(Okishio Theorem)}
\noindent\textbf{这是相较于我之前在群里发的更为严谨的置盐定理概述。}\footnote{编者注:其实也不是特别严谨,因为我省略了大部分的数学推导。} 接下来我将为大家介绍置盐信雄在1961年发表的《技术变革与利润率》\footnote{编者注:该文于2010年被我国学者骆桢、李怡乐和孟捷等人翻译成中文,刊登在《教学与研究》期刊2010年第7期。}一文。

\vspace{0.5cm} %设置垂直距

我们知道,马克思关于\textbf{一般利润率下降趋势规律}的逻辑可以被概括为3个方面的内容:(1)资本家之间的竞争迫使他们引进新的生产技术以提高劳动生产率;(2)劳动生产率的提高通常会提高资本的有机构成;(3)资本有机构成的提高会导致一般利润率的下降,虽然剩余价值率的提高会阻碍一般利润率的下降,但总的来说一般利润率还是处于下降的趋势。

针对以上内容,置盐信雄提出了3方面的疑问:
\begin{tcolorbox}[colback=gray!20, colframe=gray!100, sharp corners, leftrule={3pt}, rightrule={0pt}, toprule={0pt}, bottomrule={0pt}, left={2pt}, right={2pt}, top={3pt}, bottom={3pt}] 
\textbf{(1)}资本家引入的新生产技术一定会提高劳动生产率吗?

\textbf{(2)}提高劳动生产率的生产技术通常会提高资本的有机构成吗?

\textbf{(3)}为什么有机构成提高对一般利润率产生的影响会大于剩余价值率的提高对一般利润率产生的影响呢?

\end{tcolorbox}


对于上述3点疑问,置盐信雄分别做出了回答。

\paragraph{对于问题(1)。}置盐认为资本家引入新生产技术遵循的并不是马克思所认为的生产率准则,而是成本准则。并且他指出,“生产率准则”不同于“成本准则”。置盐用数学的形式对两者做出了区分,首先一单位商品中所蕴涵的劳动量可以表示为:
\begin{equation}\tag{1}
    t_i=\Sigma a_{ij}t_j+\tau_i \quad (i=1,2,...,n)
\end{equation}
其中,$t_i$为生产一单位第$i$种商品所耗费的直接或间接地必要劳动量,$a_{ij}$为生产一单位第$i$种商品所必须的第$j$种商品的直接投入量,$\tau_i$表示生产一单位第$i$种商品所需的直接劳动量。

因此,对于第$k$产业而言,新技术能提高生产第$k$中商品的劳动生产率的条件是:
\begin{equation}\tag{2}
    \Sigma a_{kj}t_j+\tau_k >\Sigma a'_{kj}t_j+\tau'_k
\end{equation}
其中$(a'_{k1},a'_{k2},...,a'_{kn},\tau'_k)$表示第$k$产业中的新技术,\textbf{式(2)即为“生产率准则”}。

另一方面,成本准则可表示为:
\begin{equation}\tag{3}
    \Sigma a_{kj}q_j+\tau_k >\Sigma a'_{kj}q_j+\tau'_k
\end{equation}
其中,$q_j=\frac{p_j}{w}$,$p_j$和$w$分别为第$j$种商品的价格和货币工资率。可见,只有对于所有的$i$而言$q_i=t_i$时,“生产率准则”和“成本准则”才等价。而对于资本主义经济而言,每个行业都必须存在正的利润,从而必须满足下列不等式:
\begin{equation}\tag{4}
    q_i> \Sigma a_{ij}+\tau_i
\end{equation}
因此,对于所有的$i$而言,都有$q_i>t_i$。那末,对于式(2)和式(3)而言,二者不是等价的,也即是说,“生产率准则”不同于“成本准则”。

\paragraph{对于问题(2)。}置盐认为这是一个经验统计意义上的研究,即提高劳动生产率的技术是否会提高有机构成需要通过统计研究来说明,换言之,置盐认为劳动生产率与有机构成之间的关系需要通过实证研究来说明。

\paragraph{对于问题(3)。}置盐认为大多数的结论和通常的回答如下:
\begin{equation}\tag{5}
    \frac{m}{c+v} \leq \frac{m+v}{c}
\end{equation}
只有当$v=0$时,上式才能取等号,换句话说,此时工人完全无偿劳动。可见,一般利润率存在着一个上界,这个上界$\frac{m+v}{c}$便是活劳动与物化劳动的比值。按照马克思的观点,生产的活劳动与物化劳动的比值会逐渐降低,因此虽然平均利润率会上下波动,但它的整体趋势是伴随着其上界的下降而下降的。

但是置盐认为,一般利润率不应该通过\textbf{价值形式}来表示,也即是说他认为$r=\frac{m}{c+v}$这一式子是错误的。
置盐通过斯拉法体系对此进行了修正,置盐认为一般利润率$r$应是由下列方程组决定的:
\begin{equation}\tag{6}
    \begin{cases}
        q_i=(1+r)(\Sigma a_{ij}q_j+\tau_i)\quad(i=1,2,...,n)\\
        1=\Sigma b_iq_i
    \end{cases}
\end{equation}

事实上,对于置盐给出的方程组(6)而言,每一个等式两边同时乘工资率$w$,便是斯拉法体系对于生产价格的规定\footnote{编者注:不过这里和传统斯拉法体系中的后付工资不同,这里的工人的工资被纳入到成本之中了,是预付的。},即:
\begin{equation}\tag{7}
    \begin{cases}
         p_i=(1+r)(\Sigma a_{ij}p_j+w\tau_i)\quad(i=1,2,...,n)\\
        w=\Sigma b_ip_i
    \end{cases}
\end{equation}
这里的$b_i$可以理解为工人消耗一单位的劳动时间所能够以工资形式换来的第$i$种消费资料的实物量。

将方程(7)与方程(1)联立,可得\footnote{编者注:此处省略数学推导过程}:
\begin{equation}\tag{8}
    r<\frac{\tau_i}{\Sigma a_{ij}t_j}
\end{equation}
置盐指出对于某些$i$而言,式(8)的右边表示的含义同式(5)的右边表示的含义相同,即生产过程中活劳动与物化劳动的比值。因此,虽然置盐认为式(5)不正确,但式(5)表示的结论同式(8)相比仍具有一定的合理性,然而置盐认为这种合理性只有对于某些$i$而言才是成立的,因此还需进一步考察资本家引进新生产技术的类型。

置盐认为,资本家在进行新技术引进时所遵循的是“成本准则”。也即是说对于第$k$行业的生产技术而言,新生产技术向量$(a'_{k1},...,a'_{kn},\tau'_k)$和原生产技术向量$(a_{k1},...,a_{kn},\tau_k)$满足式(3)中的不等式,那末便有如下结论\footnote{编者注:此处省略数学推导过程。}:
\begin{tcolorbox}
(1)在实际工资率\footnote{编者注:实际工资率其实就是$\Sigma b_i$,或者把它理解为工人的实物工资量。}不变的情况下,如果引入新技术的行业是“非基本品行业”,则一般利润率不受影响。

(2)在实际工资率不变的情况下,如果引入新技术的行业是“基本品行业”,则一般利润率必然上升。  
\end{tcolorbox}


\vspace{0.5cm} %设置垂直距

\textbf{上述便是置盐定理的的基本内容。}

\vspace{0.5cm} %设置垂直距

这里补充说明一下基本品行业,在置盐看来,基本品行业指的是工资品行业以及与工资品行业不可分的行业,即$b_i>0$的行业。在置盐信雄看来,当给定工资率时,剩余价值率取决于基本品行业的生产技术,这是因为
\begin{equation}\tag{9}
\frac{m}{v}=\frac{\tau_i-\tau_i\Sigma b_jt_j}{\tau_i\Sigma b_jt_j}=\frac{1-\Sigma b_jt_j}{\Sigma b_jt_j}    
\end{equation}
可见,在式(9)中,剩余价值率只受到$b_j>0$的部门(即基本品部门)的生产技术的影响;而对于$b_j=0$的部门(即非基本品部门)而言,无论$t_j$如何变化,对于整体剩余价值率的变化是没有影响的。


因而在置盐看来,一般利润率下降趋势并不是资本主义制度内生的规律,相反,利润率下降在置盐眼里是阶级斗争(工人阶级争取提高实际工资率)的外生结果。

事实上,若是将置盐定理用矩阵的形式表示,那末置盐定理在形式上便是\textbf{Perron-Frobenius定理}的一个推论,这一形式是能够被数学手段严格证明的。但形式逻辑的严密,并不意味与现实情况相契合,因此,对于置盐定理的前提假设及其结论而言,仍然是值得进一步商榷的。
\newpage
\section{第五期:《雇佣劳动与资本》(3)}
\subsection{读书会记录}
2023年11月20日18:00—20:00,我们在东北林业大学奥林学院203教室开展了第五期的读书会。本期读书会我们依然是继续探讨了马克思的《雇佣劳动与资本》中的部分内容。本期讨论我们从老版本《雇佣劳动与资本》第“三”节中的“一些商品即一些交换价值的总和究竟是怎样成为资本的呢?”开始,一直读到老版本中的第“五”节之前。接下来我将对本期讨论会的内容进行简要概述。
\subsubsection{1.“资本的必要前提”与“必要的资本前提”:一种异化的关系}
事实上,马克思在他写作的后期很少用“异化”这个词了,并且在本部分内容\footnote{编者注:老版本的第“三”节的剩余内容,具体在我编排的讲义的29—31页。}中马克思也没有提到“异化”这个词。但是我思来想去,并结合我们本期讨论的记录,我认为用“异化”这个词来概括这部分的核心思想是较为合适的。我们记得,在上期读书会记录的最后一小节中,马克思谈论了“资本之实质是一种社会生产关系”的观点,本期的这部分内容实质上是顺着上述观点深入考察了资本这种“社会生产关系”的具体细节之处。在我看来,即使在这部分的内容中马克思的政治经济学批判理论还未成熟,但是经由这部分的论述所带给我们的哲学思考依然是极为重要的。

马克思在这里说:
\begin{adjustwidth}{2em}{2em}
    \qquad\fangsong
“它成为资本,是由于它作为一种独立的社会力量,即作为一种属于社会一部分的力量,借交换直接的、活的劳动\footnote{在1891年的版本中,“劳动”改为“劳动力”。——编者注}而保存下来并增殖起来。除劳动能力以外一无所有的阶级的存在是资本的必要前提。

只是由于积累起来的、过去的、物化的劳动支配直接的、活的劳动,积累起来的劳动才变为资本。

资本的实质并不在于积累起来的劳动是替活劳动充当进行新生产的手段。它的实质在于活劳动是替积累起来的劳动充当保存自己并增加其交换价值的手段。”
\end{adjustwidth}
我们在这里可以看到,马克思认为“除劳动能力以外一无所有的阶级的存在是资本的必要前提”,这是完全正确的。为什么这么说呢?因为资本的一个关键(本质)属性就是增殖,而这种“增殖”只有通过对劳动力商品的使用(即活劳动的对象化、物化)才能够得以实现。因此,在这个意义上说来,工人阶级的劳动是资本得以发生的必要前提。这即意味着,资本一旦脱离工人阶级的劳动,就无法实现自身。

上述这些观点是不难理解的。然而现实情况却表现出了一种颠倒的形式——似乎资本不再以雇佣劳动为前提,而是雇佣劳动以资本为前提。我们可以看到马克思在之后是这么论述的:
\begin{adjustwidth}{2em}{2em}
    \qquad\fangsong
“资本和雇佣劳动\footnote{在1891年的版本中,“资本和雇佣劳动”改为“资本家和雇佣工人”。——编者注}是怎样进行交换的呢?

工人拿自己的劳动\footnote{在1891年的版本中,“劳动”改为“劳动力”。——编者注}换到生活资料,而资本家拿归他所有的生活资料换到劳动,即工人的生产活动,亦即创造力量。这种力量不仅能补偿工人所消费的东西,并且还使积累起来的劳动具有比以前更大的价值。工人从资本家那里得到一部分现有的生活资料。这些生活资料对工人有什么用处呢?用于直接消费。可是,如果我不把靠这些生活资料维持我的生活的一段时间用来生产新的生活资料,即在消费的同时用我的劳动创造新价值来补偿那些因消费而消失了的价值,那末我一把这些生活资料消费完,它们对于我就算是完全白耗费了。但是,工人为了换到生活资料,正是把这种贵重的再生产力量让给了资本家。因此,对于工人本身来说,这种力量是白耗费了。

举一个例子来说吧。有个农场主每天付给他的一个短工五银格罗申。这个短工为得到这五银格罗申,就整天在农场主的田地上干活,保证农场主能得到十银格罗申的收入。农场主不但收回了他付给短工的价值,并且还把它增加了一倍。可见,他有成效地、生产性地使用和消费了他付给短工的五银格罗申。他拿这五银格罗申买到的正是一个短工的能生产出双倍价值的农产品并把五银格罗申变成十银格罗申的劳动和力量。短工则拿他的生产力(他正是把这个生产力让给了农场主)换到五银格罗申,并用它们换得迟早要消费掉的生活资料。所以,这五银格罗申的消费有两种方法:对资本家来说,是有生产性的,因为他用这五银格罗申换来的劳动力使他得到了十银格罗申;对工人来说,是非生产性的,因为他用这五银格罗申换来的生活资料永远消失了,他只有再和农场主进行同样的交换才能重新取得这些生活资料的价值。这样,资本以雇佣劳动为前提,而雇佣劳动又以资本为前提。两者相互制约;两者相互产生。”
\end{adjustwidth}
上面这一大段话想表达的观点简要说来就是,工人为了生存(即换取他所必须的生活资料),他就必须得把自己的劳动力出卖给资本家(劳动力同资本进行交换),他就不得不使自己被资本支配。这样一来,似乎“雇佣劳动”又是以资本为前提的了。

我们认为,上述那种近乎矛盾的情况是资本主义生产关系所造成的异化现象。关于这一点,实际上我们在讨论会中已经很细致地讨论过了,我在这里就进行一个简要地回顾。“异化”这个范畴我们可以将其理解为这样的一种机制:主体在创造客体的过程中丧失了自己的主体性。工人阶级的劳动使资本得以发生,然而工人阶级的劳动却无时无刻不受制于资本,这种情况便是一种典型的“异化”关系。并且这种“异化”关系又随着资本主义生产方式\footnote{编者注:完全可以认为,资本主义生产方式本身就内蕴着一种异化的关系。}的进一步发展而使其遮蔽性日益得到稳固,以至于马克思带着一种反讽的语气如是说道:
\vspace{0.5cm} %设置垂直距
\begin{adjustwidth}{2em}{2em}
\qquad\fangsong
“千真万确呵!工人若不受雇于资本家就会灭亡。”
\end{adjustwidth}

\vspace{0.5cm} %设置垂直距
然而让我们进行进一步地反思,真的是“工人离开了资本家就会灭亡”吗?若是从现实情况来看,似乎是这样的。等等!我们切不要轻易地做出结论。正如我们在上期读书会所强调的那样,对问题的考察必须要将其置于一定的社会关系之中,正是由于资本主义生产关系所体现出的异化性质,因而建筑于这种生产关系之上的现实表象必然会产生一种颠倒的形式。也即是说,在资本主义生产关系下,仿佛资本家阶级与工人阶级是互为补充、不可分割的:资本家离不开工人,工人也离不开资本家,二者失去了对方都无法过活。然而这是一种非常荒谬的看法,这种看法就好比说:仿佛寄生虫与寄主是不可分割的,仿佛寄主同寄生虫一样,离开了对方就无法过活了!事实上,认为资本家与工人二者无法分离的观点恰恰是由于资本主义生产方式所产生的意识形态之幻象,这是一种资本主义的遮蔽。

因此,马克思最后会说:
\begin{adjustwidth}{2em}{2em}
    \qquad\fangsong
    “断言资本的利益和劳动的利益\footnote{注:在1891年的版本中,“劳动的利益”改为“工人的利益”。——编者注}是一致的,事实上不过是说资本和雇佣劳动是同一种关系的两个方面罢了。一个方面制约着另一个方面,就如同高利贷者和挥霍者相互依存一样。

当雇佣工人仍然是雇佣工人的时候,他的命运是取决于资本的。所谓工人和资本家的利益一致就是这么一回事。”
\end{adjustwidth}
关于这一点,我也同样不再赘述了,留给读者们细细思索了。

\subsubsection{2.名义工资、实际工资与相对工资}
在老版本《雇佣劳动与资本》的第“四”节中,马克思较为细致地探讨了工人阶级的\textbf{名义工资
、实际工资与相对工资}三者的差别,并着重强调了相对工资所具有的不同于前两者的重要性地位。

马克思论述了当名义工资不变时,实际工资可能会发生的变化。具体内容我就不赘述了,归结起来马克思想表达这样一种意思:当名义工资(即劳动力商品的货币价格)不变时,实际工资(即用名义工资换取的实物量)可能会发生变化。因而马克思会说:
\begin{adjustwidth}{2em}{2em}
    \qquad\fangsong
 “因此,我们谈到工资的增加或降低时,不应当仅仅注意到劳动的货币价格,仅仅注意到名义工资。”
\end{adjustwidth}
但这是否意味着工人只需关注他自身的实际工资就好了呢?显然不是这样的,因为在马克思看来,工人的实际工资也是可以随着资本主义生产方式的发展而提高的\footnote{编者注:其实这是一个社会现实,随着劳动生产率的发展,工人所占有的消费资料的绝对量事实上是在不断提升的。}。事实上,我认为马克思在这里是想强调,对于工人阶级而言,真正有原则性意义的“工资”所代表的应是一种“对比关系”。而为什么这种“对比关系”是如此重要的呢?在我看来,是因为这种“对比关系”反映了资本家和工人阶级之间的斗争。

同样,为了更为清晰地说明在工资与利润之间存在着的“斗争因素”,让我们做一个简单的政治经济学分析\footnote{编者注:为了使对问题的分析更加简洁,这里先不考虑马克思的生产价格理论,也即是说,在这里我们认为工资=可变资本,利润=剩余价值。}:\begin{fangsong}
    在资本主义实际生产中,劳动过程与价值增殖过程事实上是不可分割的,二者统一于资本主义生产过程本身。这即是说,在生产过程中创造的新价值事实上是一个整体,这个整体的所有权是归资本家所有的,且这个整体的总量等于可变资本与剩余价值的和。资本家会在生产结束之后抽调这一整体其中的一部分来补偿他预先支付给工人的工资量,而剩下的那部分,便是剩余价值。
\end{fangsong}

可见,工人的工资和资本家的利润之间具有一种对抗性的关系。因此,无论是名义工资还是实际工资,都无法体现出这种对抗关系,它们二者都是处在现象层面,还没有深入到问题的实质。就像我们当时所讨论的那样,当劳动的强度与复杂程度不变时,无论劳动生产率发生怎样的变化,一小时的劳动创造的价值量是不会发生任何变化的\footnote{编者注:在这里有必要提醒一下读者,在进入马克思劳动价值论的过程中,请务必突破实物主义的思考范式,弄清使用价值与价值之间的辩证关系。}。并且这一小时劳动所创造的价值量在量的层面可以被划分为可变资本与剩余价值,而按照何种比例划分,便体现了资本家和工人阶级之间的博弈。

因此马克思会说:
\begin{adjustwidth}{2em}{2em}
    \qquad\fangsong
    “所谓资本迅速增加对工人有好处的论点,实际上不过是说:工
人把他人的财富增殖得愈迅速,落到工人口里的残羹剩饭就愈多,
能够获得工作和生活下去的工人就愈多,依附资本的奴隶人数就
增加得愈多。
这样我们就看出:

即使最有利于工人阶级的情势,即使资本的尽快增加如何改
善了工人的物质生活状况,也不能消灭工人的利益和资产者即资
本家的利益之间的对立状态。利润和工资仍然是互成反比的。

假如资本增加得迅速,工资是可能提高的;可是资本家的利润
增加得更迅速无比。工人的物质生活改善了,然而这是以他们的社
会地位的降低为代价换来的。横在他们和资本家之间的社会鸿沟扩大了。”
\end{adjustwidth}
同样,我在这里就不继续赘述了,请读者们多读几遍原文,细细思索。
\subsubsection{本期小结}
本期读书会人数不是太多,但是讨论的依旧非常热情。本期讨论结束之后,我们本学期的读书计划已经顺利完成一半了。经过这五期的讨论,我想各位师友们应该会认同,马克思主义政治经济学实际上是一门深奥且非常富有研究价值的学问,对于马克思的经典著作,我们是有必要静下心来仔细研读的。在后续几期读书会中我们会尽量完成余下的讨论内容\footnote{编者注:我们还剩《雇佣劳动与资本》的末尾部分、《政治经济学批判》导言与序言、《哥达纲领批判》这些内容。}。总而言之,本期读书会同样是收获满满的。

\newpage
\section{第六期:《雇佣劳动与资本》(完)+《<政治经济学批判>导言》(1)}
\subsection{读书会记录}
2023年11月23日18:00—20:00,我们在东北林业大学奥林学院203教室开展了第六期的读书会。本期读书会我们结束了马克思《雇佣劳动与资本》的讨论,并开启了马克思的另一篇政治经济学的经典著作《<政治经济学批判>导言》的讨论。接下来我将对本期讨论会的内容进行简要概述。

\subsubsection{1.资本主义的“进步强制”}
在老版本《雇佣劳动与资本》第“五”节中,马克思论述了资本家之间的竞争导致社会平均劳动生产率提高的作用。大致的逻辑可以表述为:一个资本家为了将其他资本家逐出市场,他就必须使得本部门商品的价格低于其他部门商品的价格,但是,商品的价格是受到生产成本的制约的,因而他若是要降低本部门商品的价格,就需要降低本部门商品的生产成本,而商品生产成本的降低又意味着本部门劳动生产力(率)的提高。

可见,对于单个资本家而言,他若是要在市场竞争中获得优胜,他就需要将自己内部的个别劳动生产力(率)提高至社会平均劳动生产力(率)之上。在资本主义市场经济中\footnote{编者注:其实这里蕴含着一个假设,即资本主义处于自由竞争时期。},对这个资本家是这样,对其他资本家也是一样的道理,因此从整体来看,资本主义市场竞争会促使社会生产力的进步。因此马克思会说:
\begin{adjustwidth}{2em}{2em}
    \qquad\fangsong
    “一个资本家只有在自己更便宜地出卖商品的情况下,才能把另一个资本家逐出战场,并占有他的资本。可是,要能够贱卖而又不破产,他就必须廉价生产,就是说,必须尽量增加劳动的生产力。而增加劳动的生产力的首要办法是更细地分工,更全面地运用和经常地改进机器。内部实行分工的工人大军愈庞大,应用机器的规模愈广大,生产费用相对地就愈迅速缩减,劳动就更有效率。因此,资本家之间就发生了各方面的竞争:他们竭力设法扩大分工和增加机器,并尽可能大规模地使用机器。”
\end{adjustwidth}

可见,对于资本主义社会制度而言,它似乎内在地包涵着一种运动机制,这一机制的内在驱动力就是\textbf{资本贪婪的本性、资本的逐利性}。并且这种机制表明了:资本只要不增殖,它就会灭亡,因而它必须为自己创造增殖的条件。事实上,我们认为马克思上面的那段论述很像《共产党宣言》中的部分内容,我在这里给大家截取出来:
\begin{adjustwidth}{2em}{2em}
    \qquad\fangsong
    “资产阶级除非对生产工具,从而对生产关系,从而对全部社会关系不断地进行革命,否则就不能生存下去。反之,原封不动地保持旧的生产方式,却是过去的一切工业阶级生存的首要条件。生产的不断变革,一切社会状况不停的动荡,永远的不安定和变动,这就是资产阶级时代不同于过去一切时代的地方。一切固定的僵化的关系以及与之相适应的素被尊崇的观念和见解都被消除了,一切新形成的关系等不到固定下来就陈旧了。一切等级的和固定的东西都烟消云散了,一切神圣的东西都被亵渎了。人们终于不得不用冷静的眼光来看他们的生活地位、他们的相互关系。”\footnote{编者注:见《共产党宣言》。}
\end{adjustwidth}
\vspace{0.5cm}
让我们从这一点继续引申下去,资本主义的这种内在机制表明了一个事实,即资本主义社会存在着一种“进步强制”。即以资本为原则的社会必须不断取得进步以维持价值的增殖。换句话说,资本主义社会的生产力提高\textbf{往往}并不是因为资本家非常高尚、想为人类做出贡献。恰恰相反,社会生产力的提高事实上是资本家们不得已而为之的结果,即资本家们为了追逐自己利益的最大化,必须要提高劳动生产力(率)。

\subsubsection{2.机器排挤工人?资本排挤工人!}
让我们顺着前面的资本主义的“进步强制”继续讨论下去。我们已经知道了,资本家之间的竞争会内在地促进社会生产力的提高,而这首先又表现为机器和分工的扩大化。接下来,马克思探讨了机器和分工的扩大化对于工人阶级而言造成了什么样的影响。我们可以看到马克思在下面是这么说的:
\begin{adjustwidth}{2em}{2em}
    \qquad\fangsong
“其次,分工愈细,劳动就愈简单化。工人的特殊技巧失去任何价值。工人变成了一种简单的、单调的生产力,就不需要体力上或智力上的特别本事和技能了。他的劳动成为人人都能从事的劳动了。因此,工人受到四面八方的排挤;我们还要提醒一下,一种工作愈简单,就愈容易学会,为学会这种工作所需要的生产费用愈少,工资也就愈降低,因为工资像一切商品的价格一样,是由生产费用决定的。

总之,劳动愈是不能给人以乐趣,愈是令人生厌,竞争也就愈激烈,工资也就愈减少。工人想维持自己的工资总额,就得多劳动:多工作几小时或者在一小时内造出更多的产品。这样一来,工人为贫困所迫,就愈加重分工的极危险的后果。结果就是:他工作得愈多,他所得的工资就愈少。这里的原因很简单:他工作得愈多,他给自己的工友们造成的竞争就愈激烈,因而就使自己的工友们变成他自己的竞争者,这些竞争者也像他一样按同样恶劣的条件出卖自己。所以,原因同样很简单:他归根到底是自己给自己,即自己给作为工人阶级一员的自己造成竞争。

机器也发生同样的影响,而且影响的规模更大得多,因为机器用不熟练的工人代替熟练工人,用女工代替男工,用童工代替成年工;因为在最先使用机器的地方,机器就把大批手工工人抛到街头上去,而在机器日益完善、改进或为生产效率更高的机器所替换的地方,机器又把一批一批的工人排挤出去。我们在前面大略地描述了资本家相互间的产业战争。这种战争有一个特点,就是致胜的办法与其说是增加劳动大军,不如说是减少劳动大军。统帅们即资本家们相互竞赛,看谁能解雇更多的产业士兵。”
\end{adjustwidth}
不难看出,马克思在这里认为机器和分工的扩大化导致了工人阶级整体利益的削减。劳动生产率的提高缩减了工人的数量,越来越多的工人沦为无业游民,以前需要10个人的工作现在可能只需要5个人了,剩下的那5个人便成为了资本主义制度下“生产力进步”的牺牲品。此外,在这里还产生了一个现象:工人的竞争愈发激烈了。这在现实情况中表现为\textbf{机器排挤工人,工人之间内卷程度的加深}。

从人类历史发展的脉络来看,机器的使用、生产分工的科学化……总之一句话,劳动生产率的提高显然是一件具有积极意义的事情。然而对于资本主义社会的现实而言,劳动生产率的提高不仅没有令工人感到欣喜,反而让工人变得更加麻木、更加失去自我。在这种状况下,工人感到自己被机器排挤了,进而不得不同自己阶级内部的其他成员之间进行竞争,在这一竞争过程中,个别工人为了取得胜利,则必须要更为努力地向资本家献殷勤,让自己的劳动力商品更为廉价。而这种现象又表明了,束缚着工人阶级的锁链变得更为沉重了。

这种情况下,对于工人而言,他们往往看到的是自身同机器之间的矛盾,早期的卢德运动\footnote{编者注:卢德运动(Luddite Movement)是英国工人以破坏机器为手段反对工厂主压迫和剥削的自发工人运动。首领称为卢德王,故名。相传,莱斯特郡一个名叫卢德(Ludd)的工人,为抗议工厂主的压迫,第一个捣毁织袜机。1811年初卢德运动开始形成高潮。其中心是诺丁汉郡,1811年,诺丁汉郡的袜商不顾行业规矩,生产一种劣质长筒袜,压低袜子价格,严重冲击了织袜工人的正常收入。一些织工秘密组织起来,以“路德将军”的名义捣毁商人的织袜机。1812年,英国国会通过《保障治安法案》,动用军警对付工人。1813年政府颁布《捣毁机器惩治法》,规定可用死刑惩治破坏机器的工人。1813年在约克郡绞死和流放破坏机器者多人。1814年企业主又成立了侦缉机器破坏者协会,残酷迫害工人。但运动仍继续蔓延。1816年这类运动仍时有发生。在当代,“卢德分子”一词用于描述工业化、自动化、数字化或一切新科技的反对者。他们也被称为“新卢德分子”。}便是工人联合起来对机器进行反抗的典型例子。但实际上,让我们深入思考一下便不难发现,工人同机器之间的对立仅仅是一种假象,这只不过是工人同资本之间的矛盾一种外在表现。具体内容就不在这里深入讨论了。总的来说,我们认为,说“机器排挤工人”是不完全正确的,究其实质,应是“资本排挤工人”。事实上,机器的广泛运用、生产力的提高等因素都是无产阶级解放的必要条件,按其本性而言是有利于工人阶级的,只不过资本主义社会的生产关系使得生产力的发展展现出了一种颠倒的迹象,生产力的提高仅仅成为了资本主义实现其自身要求的手段。

因此,总的来说,马克思的观点认为,资本家(其只不过是资本的人格化)与工人之间的利益是对立的。资本的迅速增殖是以更大规模地牺牲工人阶级的整体利益为前提的。资本主义社会无论处在繁荣或萧条时期,受苦的总是工人。因此马克思会在最后说道:
\begin{adjustwidth}{2em}{2em}
    \qquad\fangsong
    “但是,资本不光靠剥削劳动来生活。像显贵的野蛮的奴隶主一样,资本也要他的奴隶们陪葬,即在危机时期要使大批的工人死亡。由此可见:如果说资本增长得迅速,那末工人之间的竞争就增长得更迅速无比,就是说,资本增长得愈迅速,工人阶级的就业手段即生活资料就相对地缩减得愈厉害”
\end{adjustwidth}
\vspace{0.4cm}
到这里,我们对于《雇佣劳动与资本》的讨论就暂告一段落了,读者们若是还觉得意犹未尽的话,就多读几遍原文吧。

\subsubsection{3.《政治经济学批判》导言(1)}
本期讨论会我们把马克思的《政治经济学批判》导言(下文简称《导言》)开了个头,我们一致认为,这部分内容的难度相较于前几期要更高一些。在进一步展开我们的讨论内容之前,我首先跟读者们交代一下《导言》的写作背景与内容概述。

\paragraph{背景及内容概述}\begin{fangsong}
《 导言 》 产生于 1857 年 8 月底 。 这一手稿是
马克思为自己计划中的政治经济学巨著而写的,但后来没有发表。马
克思在《政治经济学批判》第 1 分册的《序言》中说明了这个手稿没
有写完和没有发表的原因:“我把已经起草好的一篇总的导言压下
了,因为仔细想来,我觉得预先说出正要证明的结论总是有妨害的。”

在这篇具有重要科学价值的手稿中 , 马克思比在任何别的地方
都更详细地论述了他关于政治经济学的对象和方法的观点 。 马克思
指出 , 资产阶级经济学家把生产与分配 、 交换 、 消费的内在联系割裂
开来和并列起来 , 认为发生变化的只是分配方式 , 往往把分配提到首
位 , 把它当作政治经济学的研究对象 , 并把资本主义说成是历史上永恒的制度 。 马克思同他们相反 , 说明生产不是某种抽象的永恒不变的
东西 , 它是由特定的社会历史条件决定的 。 他阐明了生产 、 分配 、 交
换 、 消费的辩证统一和相互作用 , 指出它们是一个总体的各个环节 。
他得出结论说 , 生产不仅是这种统一的出发点 , 而且是决定因素 , 而
分配形式不过是生产形式的另一种表现 。 马克思认识到生产是一定
社会性质的生产 , 并把它当作自己的研究对象 。

马克思阐述了政治经济学的从抽象上升到具体的方法 , 指出它
是 “ 科学上正确的方法 ” , 同时批评了黑格尔对这一
方法的唯心主义观点 。 按照马克思对从抽象上升到具体的方法的辩
证唯物主义的解释 , 作为理论分析的出发点的具体 , 在研究的结果中
表现为多样性的统一 、 许多规定的综合 。 马克思理论中的科学抽象是
同作为它们的前提的具体现实不可分割地联系在一起的 , 而从简单
到复杂的抽象思维的进程 , 总的说来是同现实的历史过程相一致的。

马克思从他对政治经济学的对象和方法的见解出发 , 在 《 导言 》
中拟定了他的未来的经济学巨著的结构的最初计划 。 这一结构包括
资产阶级社会一切重要的方面 , 拟分为五篇: “ (1) 一般的抽象的规
定 , 因此它们或多或少属于一切社会形式 … … (2) 形成资产阶级社会
内部结构并且成为基本阶级的依据的范畴。 资本 、 雇佣劳动 、 土地所
有制 。 它们的相互关系 。 城市和乡村 。 三夫社会阶级 。 它们之间的
交换 。 流通 。 信用事业 ( 私人的 ) 。 (3) 资产阶级社会在国家形式上的
概括 。 就它本身来考察 。 ‘ 非生产 ’阶级 。 税 。 国债 。 公共信用 。 人
口 。 殖民地 。 向国外移民 。 (4)生产的国际关系 。 国际分工 。 国际交
换 。 输出和输入 。 汇率 。 (5) 世界市场和危机 。 ”

在 《 导言 》 的最后一节中 , 马克思从社会发展的经济基础出发 , 也
研究了属于政治的和意识形态的上层建筑领域的某些过程 , 探究这些过程对经济基础的依赖关系和反作用 , 论述了艺术作为社会意识
的一种形式的特点 。 他指出 , 物质生产在社会生活中的决定作用并不
排除艺术和文学这样一些上层建筑要素的相对独立性 。 他以古希腊
的艺术和莎士比亚的创作为例 , 说明艺术的兴盛并不是必然同经济
和社会的发展完全一致的 。 这是由错综复杂的情况决定的 。 上层建
筑对基础的依赖关系 , 是不能简单化地加以阐述的 。\footnote{编者注:以上内容摘选自马恩全集第二版第30卷。}
\end{fangsong}

\paragraph{接下来正式进入我们的讨论。}首先,马克思在《导言》的开篇说:

\begin{adjustwidth}{2em}{2em}
    \qquad\fangsong
    “在社会中进行生产的个人,因而,这些个人的一定社会性质的生产,自然是出发点。”
\end{adjustwidth}

我们认为,马克思在这里强调的是社会性。这是一个老生常谈的话题,简言之,不能抽离历史关系(社会关系)去孤立地考察个人。马克思认为鲁滨逊的故事是一种“毫无想象力的虚构”,这仅仅是美学上的错觉。马克思在这里批判了斯密、李嘉图、卢梭等人的自然主义式的错觉,即认为存在一个通过自然形成的、且合乎自然的“人类天性”,并幻想着达到这种“人类天性”。这些人的错误在于他们没有从社会历史关系中去考察他们所认为的“人类本性”,他们没有意识到他们所认为的这种“人类本性”恰恰是历史的结果,而误以为这种所谓的“人类本性”是历史的起点。关于这一点,我就不再继续赘述了。
\vspace{0.2cm}

之后,马克思的这句话引起了我们的讨论:
\vspace{0.2cm}
\begin{adjustwidth}{2em}{2em}
    \qquad\fangsong
    “只有到十八世纪,在‘市民社会’中,社会结合的各种形式,对个人说来,才只是达到他私人目的手段,才是外在的必然性。”
\end{adjustwidth}
\vspace{0.3cm}

如何理解“外在的必然性”呢?我们认为\footnote{编者注:这是我们讨论会的观点,可能不准确。}可以将其理解为“个人的目的只有通过他人才能实现”,也即是说,在资本主义社会中,对于个人而言,他人变成了自己的手段。关于这一点,我同样不在这里过多的阐释了,大家可以联系马克思的《黑格尔法哲学批判》中的内容去理解。这里附上马克思《黑格尔法哲学批判》中有关“外在必然性”的论述:
\begin{adjustwidth}{2em}{2em}
    \qquad\fangsong
“对家庭和市民社 会这两个领域来说,一方面,国家是‘外在必然性’,是一种权力,由于这种权力,‘法律’和‘利益’都‘从属并依存于’国家。国家对家庭和市民社会来说是‘外在必然性’,这已经部分地包含在‘过渡’这一范畴中,部分地包含在家庭和市民社会对国家的被意识到的关系中。对国家的‘从属’完全符合这种‘外在必然性’的关系。”\footnote{编者注:见马恩全集第二版第三卷《黑格尔法哲学批判》。}
    
\end{adjustwidth}

\vspace{0.3cm}
在本期讨论的最后,我们读到了马克思关于“生产一般”的论述。马克思在这里是这么说的:
\begin{adjustwidth}{2em}{2em}
    \qquad\fangsong
    “因此,说到生产,总是指在一定社会发展阶段上的生产——社会个人的生产。因而,好象只要一说到生产,我们或者就要把历史发展过程在它的各个阶段上一一加以研究,或者一开始就要声明,我们只的是某个一定的历史时代,例如,是现代资产阶级生产——这种生产事实上是我们研究的本题。可是,生产的一切时代有某些共同标志,共同规定。生产一般是一个抽象,但是只要它真正把共同点提出来,定下来,免得我们重复,它就是一个合理的抽象。”
\end{adjustwidth}

\vspace{0.8cm}
我们认为,从方法论角度上说,“生产一般”是马克思政治经济学批判的“从抽象上升到具体”辩证法的第一个必要环节,即马克思说的“合理的抽象”。比方说,小孩子在认识水果的过程中,他总是从具体的水果开始认识的,当他面对香蕉的时候,我告诉他这是香蕉,当他面对苹果的时候,我告诉他这是苹果,当他面对梨的时候,我告诉他这是梨……,但是这仅仅是一种感性直观,并没有上升到合理的抽象,只有当小孩子从这些苹果、香蕉、梨等物中抽象出“水果”概念的时候,他才算进入到了“从抽象上升到具体”辩证法的第一个环节,即合理的抽象。

但是,绝不能仅仅停留在最初的抽象(尽管这种抽象对于马克思的政治经济学批判而言是极为重要的一个环节)。我们可以看到马克思接下来是这么说的:
\begin{adjustwidth}{2em}{2em}
    \qquad\fangsong
    “如果说最发达语言的有些规律和规定也是最不发达语言所有的,但是构成语言发展的恰恰是有别于这一般和共同点的差别,那末,对生产一般适用的种种规定所以要抽出来,也正是为了不致因见到统一(主体是人,客体是自然,这总是一样的,这里已经出现了统一)就忘记了本质的差别。而忘记这种差别,正是那些证明现存社会关系永存与和谐的现代经济学家的全部智慧所在。例如,他们说,没有生产工具,哪怕这种生产工具不过是手,任何生产都不可能。没有过去的,累积下来的劳动,哪怕这种劳动不过是由于反复操作而累聚在野蛮人手上的技巧,任何生产都不可能。资本,别的不说,也是生产工具,也是过去的,客体化了的劳动。可见资本是一种一般的,永存的自然关系;这就是说,如果我们恰好抛开了正是使‘生产工具’,‘累积下来的劳动’成为资本的那个特殊的话。因此,生产关系的全部历史,例如在凯里看来,是历代政府的恶意篡改。”
\end{adjustwidth}
马克思在这里想表达的意思是,如果我们对于社会历史的观察仅仅止步于“共有的规定”的话,那末,历史在人们眼里就是不发展的、静止的。这是不难理解的,因为如果我们仅仅从“共有的规定”去考察包涵着历史关系的对象的话,我们便会发现这一对象在不同的历史时代并无什么差异。

比如说,在资本主义社会下,资本表现为“积累起来的劳动”,但是对于积累起来的劳动(客体化的劳动)而言,哪一个历史时代没有“积累起来的劳动”呢?难道说在其他的时代,“积累起来的劳动”也可以称作是资本吗?对于那些仅仅处在那种最初的抽象的理论家们而言,他们难免会认为“资本”同“积累起来的劳动”一样,是一种永存的自然关系,从而认为“生产关系的全部历史”是“历代政府的恶意篡改”。然而,在我们看来,这些理论家们犯了形而上学的错误。关于这一点,我同样不再继续赘述了,请读者们细细思索。

\subsubsection{本期小结}
本期读书会是在周四举办的,这个礼拜一共举办了两次。本期参加读书会的师友人数比上期多一些,都是一些较为熟悉的面孔。本期结束了《雇佣劳动与资本》的讨论,开启了《导言》的讨论。我编排的讲义中《导言》的内容或许存在着一些问题\footnote{编者注:我是从中文马克思主义文库上面摘取的文章,可能部分内容会和马恩全集上面的不一致。},大家可以直接参阅马恩全集第二版第30卷中的内容。《导言》的文本是要比我们之前讨论过的内容难理解一些的,因而我估计之后的几期读书会将会是非常高质量的。此外,最近感冒的比较多,在这里我呼吁大家一定要多注意防护,带好口罩,生病要及时吃药,健康的身体是搞好学术的物质基础。
\newpage
\section{第七期:《<政治经济学批判>导言》(2)}
\subsection{读书会记录}
2023年11月30日18:00—20:00,我们在东北林业大学奥林学院203教室开展了第七期的读书会。本期读书会我们继续了对马克思《<政治经济学批判>导言》的讨论。接下来我将对本期读书会的讨论内容进行简要概述。
\subsubsection{1.生产一般}
在上期读书会中我们讨论了马克思关于“生产一般”的规定,但仅仅是开了个头。我们认为,“生产一般”在马克思的政治经济学理论中属于最初的抽象,且这种抽象是合理的与必要的。但我们又进一步意识到(马克思同样也意识到),仅仅通过这个源初的抽象去考察事物的话,往往是片面的,甚至会得出荒谬的结论。究其原因是由于事物本身是在历史中不断发展着的,且对于事物的发展起关键性作用的是那个“特殊”,而不是普遍的“一般”,因而抽象必须要上升到具体。关于这一点,我不再继续赘述,请读者回顾上期的内容。

在本期的讨论内容中,马克思批判了资产阶级经济学家们的研究范式。马克思是这么说的:
\begin{adjustwidth}{2em}{2em}
    \qquad\fangsong
“现在时髦的做法,是在经济学的开头摆上一个总论部份——就是标题为《生产》的那部份(参看约翰,斯图亚特,穆勒的著作),用来论述一切生产的一般条件。
    
这个总论部份包括或者好像应当包括:

(1)进行生产所必不可缺少的条件。因此,这实际上不过是要说明一切生产的基本要素。可是,我们将会知道,实际上归纳起来不过是几个十分简单的规定,却扩展成浅薄的同义反复。

(2)或多或少促进生产的条件,如像亚当·斯密所说的前进的和停滞的社会状态。要把这些在斯密那里作为提示而具有价值的东西提升到科学意义上来,就得研究各个民族的发展过程终生产率程度不同的各个时期——这种研究超出本题应有的范围,但就属于本题范围来说,在叙述竞争,累积等等时是要谈到的。照一般的提法,答案总是这样一个一般的说法:一个工业民族,当它一般地达到它的历史高峰的时候,也就达到它的生产高峰。实际上,一个民族的工业高峰是在它还不是以既得利益为要务,而是以争取利益为要务的时候。在这一点上,美国人胜过英国人。或者是这样的说法:例如,某一些种族,素质,气候,自然条件如离海远近,土地肥沃程度等等,比另外一些更有利于生产。这又是同义反复,即财富的主客观因素越是在更高的程度上具备,财富就越容易创造。”
\end{adjustwidth}
\vspace{0.3cm}

在这里我们认为,马克思想表达的观点是,资产阶级经济学家们对于生产的考察脱离了一定的社会历史关系,而是仅仅对在不同历史阶段所共有的生产本身(即生产一般)进行考察。这种仅仅对生产一般的物质条件的考察并没有触及到社会历史关系层面,因而这种考察往往会使人们受到遮蔽,即认为资本主义时代的生产关系是一种永恒的“自然产物”。

我个人认为,在这里马克思并没有肯定地说对“生产一般”的物质条件的考察就是不正确的,我更倾向于认为马克思在这里想表达的意思是资产阶级经济学家们仅仅止步于“从抽象上升到具体”的第一个环节\footnote{编者注:即最初的合理抽象。},并在这第一个环节内部兜圈子,以至于他们\footnote{编者注:这里指马克思所批判的资产阶级经济学家们。}看不清历史的发展。

此外,我们认为,在马克思看来,资产阶级经济学家们似乎也并没有完全地领悟“从抽象上升到具体”的第一个环节,即“抽象”本身。我们可以看到马克思随后是这么说的:



\begin{adjustwidth}{2em}{2em}
    \qquad\fangsong
    “但是,经济学家在这个总论部分所真正要谈的并不是这一切。相反,照他们的意见,生产不同于分配等等(参看穆勒的著作),应当被描写成局限在脱离历史而独立的永恒自然规律之内的事情,于是资产阶级关系就被乘机当作社会一般的颠扑不破的自然规律偷偷地塞了进来。这是整套手法的多少有意识的目的。反之,在分配上,好像人们事实上可以随心所欲。”
\end{adjustwidth}
\vspace{0.8cm}

我们认为,马克思在上面那段话中想表达的意思是,资产阶级经济学家认为生产不同于分配。生产和分配之间的区别在资产阶级经济学家们眼里表现为:\textbf{生产是具有普遍性的,而分配则是脱离于生产的纯粹偶然的规定。} 这种说法在现实社会中看起来似乎是这样的,但是让我们继续深入思考一下,就会发现资产阶级理论家们的观点实际上是非常令人费解的。因而马克思随后会说:

\begin{adjustwidth}{2em}{2em}
    \qquad\fangsong
    “即使根本不谈生产和分配的这种粗暴割裂与生产与分配的现实关系,下面这一点总应当是一开始就明白的:无论在不同社会阶段上分配如何不同,总是可以像在生产中那样提出一些共同的规定来,可以把一切历史差别混合和融化在一般人类规律之中。例如,奴隶,农奴,雇佣工人都得到一定量的食物,使他们能够作为奴隶,农奴和雇佣工人来生存。靠贡赋生活的征服者,靠税收生活的官吏,靠地租生活的土地占有者,靠施舍生活的僧侣,或者靠什一税生活的教士,都得到一份社会产品,而决定这一份产品的规律不同于决定奴隶等等那一份产品的规律。”
\end{adjustwidth}
\vspace{0.2cm}
这就是说,当资产阶级经济学家们从不同历史时代的生产中抽象出一个共性的“生产一般”的时候,他们是先验地将“生产”与“分配”割裂开来的。这是因为:\textbf{既然资产阶级经济学家们承认不同历史时代存在着共性的“生产一般”的话,那末资产阶级经济学家们就没有理由不承认不同历史时代同样也存在着一个共性的“分配一般”了。}然而对于资产经济学家们而言,显然,他们只承认共性的“生产一般”,但不承认共性的“分配一般”,这便是理论的前后不一致。因此,对于那种资产阶级经济学家们所认为的将“生产”看作永恒的、而仅仅将“分配”看作偶然的情况而言,是一种割裂。



因此,我们认为,马克思在这里想表达的是,资产阶级经济学家们所考察的“分配”并不是纯粹偶然性的规定,而是处在一定的社会历史关系下的“分配”,换言之,资产阶级经济学家们所认为的那种所谓的具有偶然性的“分配”不过同样是被赋予了一定“特殊”的“分配一般”。

因而在这里不难看出,资产阶级经济学家们是很有趣的。他们看不到生产形式的变化,却能看到分配形式的变化。于是他们会天真地认为生产是永恒的自然规律,而分配是偶然的社会形式。于是乎,资产阶级经济学家们将注意力集中在分配领域——按照他们的理解,生产是一个永恒的自然规律,因而是无法改变的,因而他们只能在分配领域“大显身手”——并企图通过在分配问题上做文章,从而幻想着实现他们所期待的美好愿景。但这是不现实的,关于这一点,马克思在之后会谈到,在这里就不赘述了。

在本节最后,马克思是这么说的:

\begin{adjustwidth}{2em}{2em}
    \qquad\fangsong
    “总之:一切生产阶段所共同的,被思维当作一般规定而确定下来的规定,是存在的,但是所谓一切生产的一般条件,不过是这些抽象要素,用这些抽象要素不可能理解任何一个现实的历史的生产阶段。”
\end{adjustwidth}
这段话在一定程度上印证了我们之前的观点,就是说“生产一般”这个抽象是存在的,但是仅仅通过这个源初的抽象去理解什么东西是不现实的,因而抽象必须要上升到具体。在这里同样不再赘述了。

\subsubsection{2.生产与消费之间的关系}

这里讨论的内容是《导言》的第二节“生产与分配,交换,消费的一般关系”。在这里马克思首先指出,在进一步分析生产之前,有必要观察一下资产阶级经济学家们拿来与生产并列的几个项目,即分配、交换与消费。马克思认为将以上三者(分配、交换与消费)同生产相并列实际上是一种“肤浅的表象”,因而在这里我们能够体会到,生产、分配、交换与消费之间应该存在着某种关联,换句话说,应该存在着某种统一性。

因此在这里,马克思首先论述了生产与消费之间的同一性,为了不引起误解,我尽量用《导言》中的原文来概述这种同一性关系。
马克思在这里是这么说的:

\begin{adjustwidth}{2em}{2em}
    \qquad\fangsong
“生产直接也是消费。双重的消费,主体的和客体的:个人在生产当中发展自己的能力,也在生产行为中支出和消耗这种能力,同自然的生殖是生命力的一种消耗完全一样。第二,生产资料的消费,生产资料被使用,被消耗,一部分(如在燃烧中)重新分解为一般元素。原料的消费也是这样,原料不再保持自己的自然形状和特性,这种自然形状和特性倒是消耗掉了。因此,生产行为本身就它的一切要素来说也是消费行为。不过,这一点是经济学家所承认的,他们把直接与消费同一的生产,直接与生产合一的消费,称作生产的消费。生产和消费的这种同一性,归结起来是斯宾诺莎的命题:“规定即否定”。但是,提出生产的消费这个规定,只是为了把与生产同一的消费跟原来意义上的消费区别开来,后面这种消费被理解为起消灭作用的与生产相对的对立面,我们且观察一下这个原来意义上的消费。

消费直接也是生产,正如自然界中的元素和化学物质的消费是植物的生产一样。例如,吃喝是消费形式之一,人吃喝就生产自己的身体,这是明显的事。而对于以这种或那种形式从某一方面来生产人的其它任何消费形式也都可以这样说。消费的生产。可是,经济学却说,这种与消费同一的生产是第二种生产,是靠消灭第一种生产的产品引起的。在第一种生产中,生产者物化,在第二种生产中,生产者所创造的物人化。因此,这种消费的生产,——虽然它是生产和消费的直接统一——是与原来意义上的生产根本不同的。生产同消费合而为一和消费同生产合而为一的这种直接统一,并不排斥它们的直接两立。”
\end{adjustwidth}
\vspace{0.3cm}
上面这两段话是容易理解的,在这里,我们认为马克思侧重于从哲学层面\footnote{编者注:这里我更倾向于认为马克思是在本体论层面论述了生产与消费之间的同一性。}论述了生产与消费之间的同一性。我来简单概括一下就是:
\begin{tcolorbox}[colback=gray!20, colframe=gray!100, sharp corners, leftrule={3pt}, rightrule={0pt}, toprule={0pt}, bottomrule={0pt}, left={2pt}, right={2pt}, top={3pt}, bottom={3pt}] 
  1.生产过程本身就是参与生产的主体和客体的消费过程。
  
  2.主体的消费过程就是主体自身的生产过程。
  
\end{tcolorbox}

在这之后,马克思又从另外的角度分别论述了“消费生产着生产”与“生产生产着消费”,限于篇幅,我就不做具体说明了,读者可以自行阅读原文。

最后,马克思总结出了生产与消费之间的同一性,并做了三方面的概括,这里我把原文截取出来,大家可以自行体会,我同样不再赘述了。

\begin{adjustwidth}{2em}{2em}
    \qquad\fangsong
“因此,消费和生产之间的同一性表现在三方面:

(1)直接的同一性:生产是消费;消费是生产。消费的生产。生产的消费。政治经济学家把两者都称为生产的消费,可是还做了一个区别。前者表现为再生产,后者表现为生产的消费。关于前者的一切研究是关于生产的劳动或非生产的劳动的研究;关于后者的研究是关于生产的消费或非生产的消费的研究。

(2)每一方表现为对方的手段;以对方为媒介;这表现为他们的相互依存;这是一个运动,它们通过这个运动彼此发生关系,表现为互不可缺,但又各自处于对方之外。生产为消费创造作为外在对象的材料;消费为生产创造作为内在对象,作为目的的需要。没有生产就没有消费;没有消费就没有生产。这在经济学中以多种多样的形式表现出来。

(3)生产不仅直接是消费,消费也不仅直接是生产;而且生产不仅是消费的手段,消费不仅是生产的目的,——就是说,每一方都为对方提供对象,生产为消费提供外在的对象,消费为生产提供想象的对象;两者的每一方不仅直接就是对方,不仅媒介着对方,而且,两者的每一方当自己实现时也就创造对方,把自己当作对方创造出来。消费完成生产行为,只是在消费使产品最后完成其为产品的时候,在消费把它消灭,把它的独立的物体形式毁掉的时候;在消费使得在最初生产行为中发展起来的素质通过反复的需要达到完美的程度的时候;所以,消费不仅是使产品成为产品的最后行为,而且也是使生产者成为生产者的最后行为。另一方面,生产生产出消费,是在生产创造出消费的一定方式的时候,然后是在生产把消费的动力,消费能力本身当作需要创造出来的时候。这和第三项所说的这个最后的同一性,经济学在论述需求和供给,对象和需要,社会创造的需要和自然需要的关系时,曾多次加以解释。
” 
\end{adjustwidth}

\subsubsection{3.生产和分配之间的关系(1)}
在本期读书会中,我们把马克思关于生产与分配之间关系的讨论开了个头。在进行对生产与分配之间关系的讨论之前,马克思提出了一个疑问\footnote{编者注:当然,马克思心里面是已经有了答案的。},马克思是这么说的:“分配是否作为独立的领域,处于生产之旁和生产之外呢?”事实上,根据前文的讨论,我们是可以猜到的,马克思应该不会觉得分配是独立于生产的领域的,它们二者之间应该具有某种联系。

马克思指出对于“普通的经济学著作”\footnote{编者注:这里指的应该就是资产阶级经济学著作。}而言,这些著作里“什么都被提出两次”。这是什么意思呢?马克思给我们举了个例子,比方说对于“地租、工资、利息和利润”而言,它们处在分配领域,而对于“土地、劳动与资本”而言,它们则处在生产领域。这也即是说,在“普通的经济学著作”中,相应的生产要素对应着相应的收入源泉,换句话说,生产的规定对应着分配的规定\footnote{编者注:若是有熟悉马克思在《资本论》中对萨伊经济学的“三位一体”公式批判的读者在这里一定会感到奇怪,按照劳动价值论,劳动是价值的唯一源泉,因而所谓不同的生产要素对应不同的收入源泉这个观点本质上是错误的。但我在这里想提醒读者的是,马克思在这里讨论的生产规定与分配规定之间的“对应”更倾向于在形式层面的“对应”,事实上,它们也确实是在形式层面对应着的,并且如果进行进一步思索的话,这种“形式”似乎表现为一种外在性的强制。}。

因此,马克思会说:
\begin{adjustwidth}{2em}{2em}
    \qquad\fangsong
    “同样,工资也是在另一个项目中被考察的雇佣劳动:在一处作为生产要素的劳动所具有的规定性,在另一处表现为分配的规定。如果劳动不是规定为雇佣劳动,那末,它参与产品分配的方式,也就不表现为工资,如在奴隶制度下就是这样。”
\end{adjustwidth}
我认为,马克思在这里想说明的是,生产与分配似乎是同一个东西的不同侧面。也即是说,生产的性质同时决定了分配的性质。在这里,按照马克思的例子说来就是,工人以“雇佣劳动”的形式参与生产,那末,他就必然以“工资”的形式参与分配。而“雇佣劳动”本身又是资本主义社会关系的表现,关于这一点,我同样不在这里继续深入解读了,读者们可以自行体会。

因此,我们不难理解,马克思会如是说道:
\begin{adjustwidth}{2em}{2em}
    \qquad\fangsong
    “所以,分配关系和分配方式只是表现为生产要素的背面。个人以雇佣劳动的形式参与生产,就以工资形式参与产品,生产成果的分配。分配的结构完全取决于生产的结构,分配本身就是生产的产物,不仅就对象说是如此,而且就形式说也是如此。就对象说,能分配的只是生产的成果,就形式说,参与生产的一定形式决定分配的特定形式,决定参与分配的形式。把土地放在生产上来谈,把地租放在分配上来谈,等等,简直是幻觉。”
\end{adjustwidth}

\vspace{0.5cm}

事实上,我认为上面这段论述可以和“生产资料之间的分配”联系起来来看,但我在这里就不继续赘述了,我在读书会中已经叙述过我的观点了,有感兴趣的同学可以在下期继续探讨。

\subsubsection{本期小结}
本期读书会同样是在周四举办的,参加本期读书会的师友都是比较熟悉的面孔。本期我们讨论的内容非常丰富,我只是挑了其中的一小部分内容记录了下来。事实上,《导言》是学习马克思的政治经济学批判理论的一部非常重要的文本,这里面的许多内容是很难通过短时间的阅读与讨论而完全弄懂的,因而还需反复阅读,在这部分内容中可以深入探究的点有很多。此外,在本期结束之后,我们会暂停两个礼拜,在第16周之后继续举行剩下几期的读书会。在我编写本期读书会记录的时候,已经到12月份,临近放假了,希望大家能够坚持下去,搞好学术、收获知识。总而言之,本期读书会同样是收获满满的。
\newpage
\section{第八期:《<政治经济学批判>导言》(3)}
\subsection{读书会记录}
2023年12月19日14:00—16:00,我们在东北林业大学丹青楼608教室开展了第八期的读书会。本期读书会我们同样地继续进行对马克思《<政治经济学批判>导言》的讨论,本期讨论内容截止到\textbf{“政治经济学的方法”}一章中的开头部分。接下来我将对本期读书会的讨论内容进行简要概述。
\subsubsection{1.生产与分配的关系(2)}
在本期的讨论内容中,我们可以看到,马克思首先指出,无论是对于\textbf{“单个的个人”}还是\textbf{“整个社会”}而言,分配似乎都是先于生产而存在的,并且似乎是分配决定着生产。马克思的叙述逻辑总体上是这样的:

\begin{tcolorbox}
\textbf{1.对于单个的个人。}~\begin{fangsong}
    他在出生之前所具有的分配情况决定了他的生产地位。例如,若是一个人出生下来就不占有资本或地产的话,那末,他一出生就由社会分配指定专门从事雇佣劳动,也即是说,他在生产中充当着劳动力商品的地位。
\end{fangsong}

\textbf{2.对于整体社会。}~\begin{fangsong}
    一个社会整体的生产方式在经验层面往往是决定于这个社会形成之初的资本、地产的分配形式的。例如一个征服者民族在征服者之间分配土地,造成了地产的一定的分配与形式,从而决定了生产。又或者,一个民族经过革命把大地产粉碎成小块,从而通过这种新的地产分配形式使得生产有了一种新的性质。这些例子表明,似乎对于整体社会而言,也是分配决定了生产的。
\end{fangsong}
\end{tcolorbox}

\vspace{0.5cm}
但上述的那种理解仅仅是对于表象的理解。事实上在上期的讨论会中,我们便隐约地意识到了,生产和分配二者之间是有着极为密切的联系的。说分配先于生产,其实就是在无意中将分配与生产之间的关联给割裂开来了。我们可以看到马克思接下来是这么说的:

\vspace{0.3cm}
\begin{adjustwidth}{2em}{2em}
    \qquad\fangsong
    “照最浅薄的理解,分配表现为产品的分配,因而它仿佛离开生产很远,对生产是独立的。但是,在分配是产品的分配之前,它是(1)生产工具的分配,(2)社会成员在各类生产之间的分配(个人从属于一定的生产关系)——这是同上述同一关系的进一步规定。这种分配包含在生产过程本身中并且决定着生产的结构。”
\end{adjustwidth}

马克思的这段话意味着,生产本身便包涵着分配的因素。诚然,分配在一定程度上表现为产品的分配,并且这种“产品的分配”使得分配本身具有了一定的独立形式,但若仅仅将分配看作是产品的分配的话,这便是一种非常浅薄的理解。

事实上,生产本身便内在地包涵着生产资料的分配。换句话说,生产之所以是具体的生产,便是因为其内在地包涵着分配的要素。按照我的理解,这即是说,“生产资料的分配”是生产得以实现其自身的必要条件。我认为马克思下面的这段话很关键,马克思说:
\begin{adjustwidth}{2em}{2em}
    \qquad\fangsong
    “如果在考察生产时把包含在其中的这种分配撇开,生产显然只是一个空洞的抽象;反过来说,有了这种本来构成生产的一个要素的分配,力求在一定的社会结构中来理解现代生产并且主要是研究生产的经济学家李嘉图,不是把生产而是把分配说成现代经济学的本题。从这里,又一次显出了那些把生产当作永恒真理来论述而把历史限制在分配范围之内的经济学家是多么荒诞无稽。”
\end{adjustwidth}

我认为这段话是想表明,当我们在考察生产的过程中,不应该机械地将生产与分配割裂开来,去寻求二者之间的某种“机械式的纽带关系”(即寻求究竟是谁先谁后的关系)。因为真实的情况往往是辩证的而非机械的。因此,我认为在这个意义上便不难理解马克思的这句话:\textbf{“如果在考察生产时把包含在其中的这种分配撇开,生产显然只是一个空洞的抽象”}。

因此,对于马克思而言,分配作为生产的一个内部因素,它们二者之间的关系统一于生产自身内部。马克思是这么说的:
\begin{adjustwidth}{2em}{2em}
    \qquad\fangsong
    “这种决定生产本身的分配究竟和生产处于怎么样的关系,这显然是属于生产本身内部的问题。”
\end{adjustwidth}

但是,或许有人会存在着这样的疑问:既然承认了生产本身包含着分配的要素,即生产必须从生产资料的分配出发才能展开自身,那末,至少在这个意义上我们有理由认为分配是先于生产的,因而在这个意义上说分配先于生产是并无问题的。事实上,马克思是承认生产在\textbf{最初}可能存在着一个自然的分配前提的,但是马克思在这里想强调的是,在这个最初的自然前提之后,通过生产过程本身的作用,影响生产的那种分配便成为了一种历史性的结果。关于这一点,我不再赘述了,在这里我把马克思的原文放上来,大家可以再仔细思考一下。

\begin{adjustwidth}{2em}{2em}
    \qquad\fangsong
    “如果有人说,既然生产必须从生产工具的一定分配出发,至少在这个意义上分配先于生产,成为生产的前提,那末就应该答复他说,生产实际上有它的条件和前提,这些条件和前题构成生产的要素。这些要素最初可能表现为自然发生的东西。通过生产过程本身,它们就从自然发生的东西变成历史的东西了,如果它们对于一个时期表现为生产的自然前提,对于另一个时期就是生产的历史结果了。它们在生产内部不断地改变。例如,机器的应用既改变了生产工具的分配,也改变了产品的分配。现代大土地所有制本身既是现代商业和现代工业的结果,也是现代工业在农业上应用的结果。
    
    ……
    
    例如,蒙古人把俄罗斯弄成一片荒凉,这样做是适合于他们的生产,畜牧的,大片无人居住的地带是畜牧的主要条件。在日耳曼蛮族,用农奴耕作是传统的生产,过的是乡村的孤独生活,他们能够非常容易地让罗马各省服从于这些条件,因为那里发生的土地所有权的集中已经完全推翻了旧的农业关系。

有一种传统的观念,认为在某些时期人们只靠劫掠生活。但是要能够劫掠,就要有可以劫掠的东西,因此就要有生产。而劫掠方式本身又决定生产方式。例如,劫掠一个从事证券投机的民族就不能同劫掠一个游牧民族一样。

奴隶直接被剥夺了生产工具。但是奴隶受到剥夺的国家的生产必须安排得容许奴隶劳动,或者必须建立一种适于使用奴隶的生产方式(如在南美等)。
”
\end{adjustwidth}

\subsubsection{2.交换和流通}
关于交换和流通,我们认为,马克思在这里并没有刻意地给这二者做出一个区分。从某种意义上说,“交换和流通”这部分内容的关注度相较于前面的生产与分配而言略有下降,至少就《导言》中关于“交换和流通”这部分内容的写作篇幅而言,是较少的。造成这种现象的原因可能是因为仅仅就交换和流通这一过程本身而言,它们很难体现出一定的社会关系。就商品经济——进而是资本主义商品经济——而言,一方面,流通过程不涉及价值的增减;另一方面,在交换过程中反而是体现出了某种平等的要素。

但实际上,虽然交换与流通表现为独立于生产过程之外的因素,但其实它们还都是生产的内部因素,都是由生产方式本身所决定着的。马克思这里列举了三点来说明交换与生产之间的密切联系。

第一,如果没有分工,便没有交换。这实际上是很好理解的,分工作为生产方式的一种表现,导致了某一分工领域的群体无法生产出他们所需要的全部商品,这就需要同其他分工领域的产品进行交换。

第二,私人交换以私人生产为前提。这同样不难理解,对于个人而言,他若是要同别人进行交换,便需要进行私人生产,否则他就没有用以交换的东西了。

第三,交换的深度、广度和方式都是由生产的发展和结构决定的。马克思在这里举了城乡之间的交换、乡村中的交换以及城市中的交换等的例子。比如货币出现以前的简单商品交换与资本主义商品交换显然在交换方式上存在着差异,而这种差异恰恰是由于这两个时期的生产方式的差异所造成。

最后,马克思对生产、分配、交换和消费这四者的关系做了一个总结性的论述,同样我不再继续赘述了,我把原文放在这里,请读者们细细思索。

\begin{adjustwidth}{2em}{2em}
    \qquad\fangsong
“我们得到的结论并不是说,生产,分配,交换,消费是同一的东西,而是说,它们构成一个总体的各个环节,一个统一体内部的差别。生产既支配着生产的对立规定上的自身,也支配着其它要素。过程总是从生产重新开始。交换和消费是不能支配作用的东西,那是自明之理。分配,作为产品的分配,也是这样。而作为生产要素的分配,它本身就是生产的一个要素。因此,一定的生产决定一定的消费,分配,交换和这些不同要素相互间的一定关系。当然,生产就其片面形式来说也决定于其它要素。例如,当市场扩大,即交换范围扩大时,生产的规模也就增大,生产也就分得更细。随着分配的变动,例如,随着资本的集中,随着城乡人口的不同的分配等等,生产也就发生变动。最后,消费的需要决定着生产。不同要素之间存在着相互作用。每一个有机整体都是这样。”
\end{adjustwidth}

\subsubsection{3.政治经济学的方法(1)}

这一部分的内容是《导言》的关键之处,马克思的从“抽象上升到具体”的方法论思想理解起来是较为困难的,但同样又是对于整个政治经济学研究而言极为重要的。为了尽可能地减少歧义之处,我尽量多引用马克思原文中的语句来进行说明。

马克思在本部分的开篇说:
\begin{adjustwidth}{2em}{2em}
    \qquad\fangsong
    “从实在和具体开始,从现实的前提开始,因而,例如在经济学上从做为全部社会生产行为的基础和主体开始,似乎是正确的。但是,更仔细地考察起来,这是错误的。如果我抛开构成人口的阶级,人口就是一个抽象。如果我不知道这些阶级所依据的因素,如雇佣劳动,资本等等,阶级又是一句空话。而这些因素是以交换,分工,价格等等为前提的。比如资本,如果没有雇佣劳动,价值,货币,价格等等,它就什么也不是。因此,如果我从人口着手,那末这就是一个混沌的关于整体的表象,经过更切进的规定之后,我就会在分析中达到越来越简单的概念;从表象中的具体达到越来越稀薄的抽象,直到我达到一些最简单的规定。于是行程又得从那里回过头来,直到我最后又回到人口,但是这回人口已不是一个混沌的关于整体的表象,而是一个具有许多规定和关系的丰富的总体了。”
\end{adjustwidth}

我们认为,在这段叙述中,马克思已经阐述了他的“从抽象上升到具体”的哲学方法论思想。马克思说,从实在和具体开始,从现实的前提开始,似乎是正确的,但更仔细地考察起来,这是错误的。\textbf{这句话往往会引起歧义},似乎会觉得马克思是这样认为的:从实在和具体开始考察事物是错误的,也即是说,实在和具体不是考察事物的科学出发点。

我们认为,上面的那种看法是一种误解。事实上,实在和具体、现实的前提作为考察事物的出发点而言是没有问题的。如果对于事物的考察不从感性的实在和具体出发,那末就会陷入唯心主义的错误。那为什么马克思接下来会说“更加仔细地考察起来,这是错误的”呢?我们认为,马克思在这里想表明的是,虽然要从感性的实在和具体出发考察事物,但是对于事物的考察绝对不能止步于此,换句话说,若是要对事物进行科学的考察,便不能仅仅外在地停留于对实在与具体的笼统的抽象概括。

马克思在这里举了人口的例子,他说:“如果我从人口着手,那末这就是一个混沌的关于整体的表象,经过更切进的规定之后,我就会在分析中达到越来越简单的概念;从表象中的具体达到越来越稀薄的抽象,直到我达到一些最简单的规定。”这看似是很合理的对于事物的分析过程,但是我认为,更为关键的是后面的一句话,马克思的叙述进行了一次\textbf{“转折”},他接着说:“于是行程又得从那里\textbf{回过头来},直到我最后又回到人口,但是这回人口已不是一个混沌的关于整体的表象,而是一个具有许多规定和关系的丰富的总体了。”

让我们来分析一下上面这个认识过程。我们要首先要明确一点,这整个认识过程的主体是人,因此只要这个认识过程仅仅是在主体的头脑中进行的话,那末这一过程就并不会对认识对象产生什么实质性的影响。因此,就整个认识过程而言,它的起点是“混沌的关于整体的表象”,它的终点是“具有许多规定和关系的丰富的总体”,而这二者的区别是仅仅对于认识主体而言的区别,这即意味着,对于认识对象而言,其自身是没有什么变化的。按照“人口”的例子来说,这种科学认识过程从总体性层面来看就是从“人口”到“人口”的回归,前一个“人口”是“混沌的关于整体的表象”,后一个“人口”是“具有许多规定和关系的丰富的总体”,但“人口”在现实中所对应着的那个客体是不会发生变化的,变化仅仅存在于对于“人口”这一范畴的认识。

马克思指出,从“混沌的表象”出发得到“最简单的规定”这是“第一条道路”,而从“最简单的规定”回到范畴本身,以达到“许多规定和关系的丰富的总体”,这便是“第二条道路”。马克思指出,“在第一条道路上,完整的表象蒸发为抽象的规定;在第二条道路上,抽象的规定在思维行程中导致具体的再现。”马克思认为,第二条道路是科学上正确的方法。但这并不意味着第一条道路就是毫无意义的。我们认为,第二条道路需要通过第一条道路,换句话说,第二条道路是对第一条道路的进一步扬弃。

\textbf{接下来马克思批判了黑格尔式的唯心主义。}马克思指出:
\begin{adjustwidth}{2em}{2em}
    \qquad\fangsong
    “因而黑格尔陷入幻觉,把实在理解为自我综合,自我深化和自我运动的思维的结果,其实,从抽象上升到具体的方法,只是思维用来掌握具体并把它当作一个精神上的具体再现出来的方式。但决不是具体本身的产生过程。”
\end{adjustwidth}

这便正如我在前面所叙述的那样,认识过程仅仅是一个主体自身的思维过程,就这一过程而言,它是不会影响认识对象自身的,在整个认识过程的起点与终点发生变化的不是认识对象,而是认识主体的思维。而唯心主义则将这一过程颠倒,将主体认识世界的过程等于世界自身的生成过程,把具体思维的生产等同于客观具体事物的生产,这是极其荒谬的。马克思接下来是这么说的:
\begin{adjustwidth}{2em}{2em}
    \qquad\fangsong
“在意识看来(而哲学意识就是被这样规定的:在它看来,正在理解着的思维是现实的人,而被理解了的世界本身才是现实的世界),范畴的运动表现为现实的生产行为(只可惜它从外界取得一种推动),而世界是这种生产行为的结果;这——不过又是一个同义反复——只有在下面这个限度内才是正确的:具体总体作为思想总体、作为思想具体,事实上是思维的、理解的产物;但是,决不是处于直观和表象之外或驾于其上而思维着的、自我产生着的概念的产物,而是把直观和表象加工成概念这一过程的产物。”
\end{adjustwidth}

可见,在黑格尔看来,思维是真正具有“现实性”的主体,思维的运动是“现实”的生产行为。既然黑格尔这类唯心主义的哲学家们认为思维是具有“现实性”的主体,那末,他们便会很自然的认为是这种真正具有“现实性”的思维主体正在塑造着他们所认为的“现实”,“现实世界”便是这种“现实的”思维运动的结果。但他们没有意识到,这种思维主体的运动是受到客观世界的外部推动的,这种思维的运动无非是真正的现实主体通过他的头脑认识世界的一种机制,这种运动本身究其实质是以真正现实的客观世界作为前提的,只要这种思维运动仅仅局限于思维领域,那末它就不会对真正的现实世界产生任何影响,从而现实世界相较于这种思维运动而言便是保持着独立性的。因此马克思会在最后说道:
\begin{adjustwidth}{2em}{2em}
    \qquad\fangsong
“整体,当它在头脑中作为思想整体而出现时,是思维着的头脑的产物,这个头脑用它所专有的方式掌握世界,而这种方式是不同于对于世界的艺术精神的,宗教精神的,实践精神的掌握的。实在主体仍然是在头脑之外保持着它的独立性;只要这个头脑还仅仅是思辨地、理论地活动着。因此,就是在理论方法上,主体,即社会,也必须始终作为前提浮现在表象面前。”
\end{adjustwidth}


\subsubsection{本期小结}
时隔两个礼拜,我们的读书会又再次举办了,参加本期读书会的师友都是非常熟悉的面孔。本期我们讨论的内容较为晦涩难理解,特别是《导言》的“政治经济学的方法”一章中的内容。我们在本期近乎以一种“句读”的形式讨论了马克思的“从抽象上升到具体”的方法论。我在这里浅说一下我个人的看法,关于“从抽象上升到具体”的具体机制是如何的,在本期读书会的内容记录中我已经表达了很多我的观点,大家可以向上翻阅;另一方面,对于这一思想的存在论意义而言,我倾向于认为马克思的“从抽象上升到具体”的哲学方法论思想是一种对于科学认识论的描述性(解释性)的分析,而不是一种规定性的分析。也即是说,我更倾向于认为,马克思只是揭露了人类的科学认识过程的机制究竟是怎么样的,而并不是提供了一种“放之四海而皆准”的方法,因而对于具体问题还需具体分析,关于这一点,限于篇幅,我就不在这里充分展开论述了。总之,本期的讨论是热情而充实的,这是一期收获满满的读书会。
\newpage
\section{第九期:《<政治经济学批判>导言》(4)}
\subsection{读书会记录}
2023年12月21日14:00—16:00,我们在东北林业大学丹青楼608教室开展了第九期的读书会。本期读书会我们继续进行了对马克思《<政治经济学批判>导言》的“政治经济学的方法”一章的讨论。接下来我将对本期读书会的讨论内容进行简要概述。
\subsubsection{1.简单范畴有时需要以具体范畴为前提}
本期读书会的内容是较为抽象难理解的。我们花了很多的时间句读了马克思下面的这段话:
\begin{adjustwidth}{2em}{2em}
    \qquad\fangsong
“但是,这些简单的范畴在比较具体的范畴以前是否也有一种独立的历史存在或自然存在呢?要看情况而定。比如,黑格尔论法哲学,是从主体的最简单的法的关系即占有开始的,这是对的。但是,在家庭或主奴关系这些具体的多的关系之前,占有并不存在。相反,如果说有这样的家庭和氏族,它们还只是占有,而没有所有权,这倒是对的。所以,这种比较简单的范畴,表现为简单的家庭或氏族的公社在所有权方面的关系。它在比较高级的社会中表现为一个发达的组织的比较简单的关系。但是那个以占有为关系的具体的基础总是前提。可以设想一个孤独的野人占有东西,但是在这种情况下,占有并不是法的关系。说占有在历史上发展为家庭,是错误的。占有倒总是以这个“比较具体的法的范畴”为前提的。但是,不管怎样总可以说,简单范畴是这样一些关系的表现,在这些关系中,不发展的具体可以已经实现,而那些通过较具体的范畴在精神上表现出来的较多方面的联系和关系还没有产生;而比较发展的具体则把这个范畴当作一种从属关系保存下来。在资本存在之前,银行存在之前,雇佣劳动存在之前,货币能够存在,而且在历史上存在过。因此,从这一方面看来,可以说,比较简单的范畴可以表现一个比较不发展的整体的处于支配地位的关系,或者可以表现一个比较发展的整体的从属关系,后面这些关系,在整体向着一个比较具体的范畴表现出来的方面发展之前,在历史上已经存在。在这个限度内,从最简单上升到复杂这个抽象思维的进程符合现实的历史过程。”
\end{adjustwidth}

在上面的那段论述中,马克思认为“简单的范畴”不一定在“具体的范畴”以前就具有一种独立的历史或自然存在。这是什么意思呢?我认为,马克思在这里想表明的意思是简单范畴或具体范畴都只不过是人类头脑的产物,是人类通过自己的思维对真实对象的把握。当我们通过思维生成了对于认识对象的“具体范畴”之时,通过我们理性的回溯会很自然地寻得相较于这种“具体范畴”而言的更为简单的“简单范畴”。但这并不意味着通过我们理性的回溯所寻找到的“简单范畴”就一定对应着一个在先前的历史中存在着的自然客体(对象)。马克思举了黑格尔论法哲学的例子,他说黑格尔从“占有”开始,将“占有”看作“主体的最简单的法的关系”,这是正确的。马克思为什么会认为这是正确的呢?按照我的理解,\textbf{因为这符合已经形成了的具体范畴的思维逻辑}。也即是说,对于已经由较为复杂的“法的关系”确定下来的具体抽象而言,若是按照这种“法的关系”去研究这种“具体抽象”的话,那末这种“具体抽象”便生成于最简单的“法的关系”的抽象,即生成于在“法的关系”层面的“占有”这一简单范畴。这种思考逻辑对于已经形成了的“法的关系”而言是没有什么问题的,但问题的关键恰恰在于这种“法的关系”并不是先验的存在,换句话说,“法的关系”本身同样是要在历史中生成的。因此,在这个意义上,我认为我们可以在一定程度上理解马克思为什么会说“占有(即主体的最简单的法的关系)\footnote{编者注:括号里的内容是编者添加的。}倒总是以这个‘比较具体的法的范畴’为前提的”。简言之,\textbf{按我的理解},只要承认这种“简单范畴”是通过理性逻辑对“具体范畴”的回溯性反思的结果的话,那末,这种“简单范畴”便恰恰是以“具体范畴”为前提基础的。但究竟要在多大程度上(或在何种意义上)承认这种回溯性反思机制,则仍然需要进行进一步的深入思考。

\subsubsection{2.劳动一般:理论的出发点与历史的结果}
马克思接下来说:
\begin{adjustwidth}{2em}{2em}
    \qquad\fangsong
 “劳动似乎是一个十分简单的范畴。它在这种一般性——作为劳动一般——上的表象也是古老的。但是,在经济学上从这种简单性上来把握的‘劳动’,和产生这个简单抽象的那些关系一样,是现代的范畴。”
\end{adjustwidth}

可见,马克思认为,劳动似乎是一个十分简单的范畴,且对于“劳动一般”而言,这种表象也是显地十分古老的。这当然是正确的,因为对于“劳动”而言,我们完全可以认为这是人类社会自古以来一直存在着的东西。人类社会若是要发展,就必须依靠“劳动”实现人与自然之间的物质变换。因而这种具有一般性意义的“劳动”(即劳动一般)是具有久远的历史存在的。然而,马克思紧接着进行了一个转折,他说从经济学意义上来把握的这种“劳动”,却是现代的范畴。马克思为什么会这么说呢?我认为这是同经济学家们对于劳动与价值之间关系的认识历史相对应的,我们可以看到马克思接下来举的例子:

\begin{adjustwidth}{2em}{2em}
    \qquad\fangsong
“货币主义把财富看成还是完全客观的东西,看成存在于货币中的物。同这个观点相比,重工主义或重商主义把财富的源泉从对象转到主体的活动——商业劳动和工业劳动,已经是很大的进步,但是,他们仍然只是局限地把这种活动本身理解为取得货币的活动。同这个学派相对立的重农学派把劳动的一定形式——农业——看作创造财富的劳动,不再把对象本身看做裹在货币的外衣之中,而是看做产品一般,看做劳动的一般成果了。这种产品还与活动的局限性相应而仍然被看做自然规定的产品——农业的产品,主要还是土地的产品。

亚当·斯密大大地前进了一步,他抛开了创造财富的活动的一切规定性,——干脆就是劳动,既不是工业劳动,又不是商业劳动,也不是农业劳动,而既是这种劳动,又是那种劳动,有了创造财富的活动的抽象一般性,也就有了被规定为财富的对象的一般性,这就是产品一般,或者说又是劳动一般,然而是作为过去的,物化的劳动。这一步跨得多么艰难,多么远,只要看看连亚当·斯密本人还时时要回到重农学派的观点上去,就可想见了。这会造成一种看法,好象由此只是替人——不论在哪种社会形式下——做为生产者在其中出现的那种最简单,最原始的关系找到了一个抽象表现。从这一方面来看这是对的,从另一方面看来就不是这样。”
\end{adjustwidth}

上面这段论述中,可能是由于手稿的原因,我认为马克思的用词在一些细节之处有点不太恰当,例如上面论述中的“财富”一词而言,我认为应将其理解为“价值”的含义。为了方便叙述,让我们姑且将“财富”理解为“价值”\footnote{编者注:事实上,这二者是完全不同的概念,马克思在《哥达纲领批判》中就对此做了相应的批判,不过按照我们的进度,估计这学期的读书会我们不会读到《哥达纲领批判》了。}。现在我们知道,在质的层面来看,价值的本质事实上是对象化劳动、物化劳动,且这种劳动是无差别的一般人类劳动。这种价值的本质即“无差别的一般人类劳动”事实上是一种非常简单的范畴,但是从经济学的意义上把握这一简单范畴的过程却经历了漫长的曲折。我认为,这种漫长且曲折的认识过程便是从“抽象上升到具体”的第一阶段,即从感性的具体到稀薄的抽象。

早期的经济学派如货币主义学派直接的将价值看作是某种脱离人的对象物,看作是某种纯粹客观的、脱离人的主体性的物。之后的经济学派逐渐意识到了劳动的重要性,但只不过他们并没有认识到一般性的劳动范畴,而仅仅是看到了特殊的具体劳动对于价值形成的意义。而到了古典政治经济学,他们终于发现了劳动的一般性,他们正确地意识到了决定价值的并不是什么特殊劳动,而是“劳动一般”。

事实上,经济学家们对于“劳动一般”的认识过程是与现实的社会物质生产发展密切联系着的。在社会生产力水平不太发达的阶段,具体劳动的种类是较少的,以至于人们往往会直接关注到个别种类的具体劳动对于价值创造的重要意义,从而对于价值的创造具有决定性意义的抽象“劳动一般”便处在一种被遮蔽了的状态。从这个意义上来说,我们可以认为,人们认识不到存在于不同种类具体劳动中的抽象一般,恰恰是由于具体劳动的形式并不丰富,从而个别劳动形式的突出掩盖了蕴含在不同劳动形式中的普遍性。因此马克思会说:

\begin{adjustwidth}{2em}{2em}
    \qquad\fangsong
    “对任何种类劳动的同样看待,以一个十分发达的实在劳动种类的总体为前提,在这些劳动种类中,任何一种劳动都不再是支配一切的劳动。所以,最一般的抽象只产生在最丰富的具体的发展的地方,在那里,一种东西为许多东西所共有,为一切所共有。”
\end{adjustwidth}

以上论述表明了经济学家们对于劳动范畴的认识恰恰是一个历史的过程,也即是说,作为最简单的“劳动一般”这一概念恰恰是在现实的劳动形式发展到最为丰富多样的阶段上被确定(认识)的。因此我认为\footnote{编者注:我理解的可能存在着偏差,仅提供一种看法。},马克思想告诉我们的是,这一最简单的“劳动一般”范畴,在理论层面作为出发点,而在现实层面却作为某种历史发展的结果。

\subsubsection{3.资本主义并不是历史的目的与最终形态}
在本期读书会的最后,我们对马克思下面的这句话进行了一定的讨论,马克思说:

\begin{adjustwidth}{2em}{2em}
    \qquad\fangsong
“因为资产阶级社会本身只是发展的一种对立的形式,所以,那些早期形式的各种关系,在它里面常常只以十分萎缩的或者完全歪曲的形式出现。公社所有制就是个例子。因此,如果说资产阶级经济的范畴适用于一切其他社会形式这种说法是对的,那么,这也只能在一定意义上来理解。这些范畴可以在发展了的、萎缩了的、漫画式的种种形式上,总是在有本质区别的形式 上,包含着这些社会形式。所说的历史发展总是建立在这样的基础上的:最后的形式总是把过去的形式看成是向着自己发展的各个阶段,并且因为它很少而且只是在特定条件下才能够进行自我批判,——这里当然不是指作为崩溃时期出现的那样的历史时期,——所以总是对过去的形式作片面的理解。基督教只有在它的自我批判在一定程度上,可说是在可能范围内完成时,才有助于对早期神话作客观的理解。同样,资产阶级经济学只有在资产阶级社会的自我批判已经开始时,才能理解封建的、古代的和东方的经济。在资产阶级经济学没有用编造神话的办法把自己同过去的经济完全等同起来时,它对于以前的经济,特别是它曾经还不得不与之直接斗争的封建经济的批判,是与基督教对异教的批判或者新教对旧教的批判相似的。”
\end{adjustwidth}

我们可以看到,马克思说资产阶级社会本身只是发展的一种对抗的形式,这即意味着资产阶级社会同样只是社会历史运动的产物,它并不是什么历史的终极目的。因此,认为资本主义社会的经济范畴包涵着“适用于其他一切社会形式的真理”的这种观点从根本上来说是错误的,然而如果从很狭隘的角度来考察的话,这种观点又存在着一定的合理性,正如新事物总是体现着旧事物的影子与未来事物的萌芽一样,资产阶级社会的经济范畴同样体现着旧社会形态的影子与未来的社会形态的萌芽,但这些“影子”和“萌芽”并不占着主导地位。

资产阶级经济学家们并没有看到历史发展的真实形式,即历史发展的唯物主义因素。资产阶级经济学家们天真地将历史发展过程仅仅看作是“过去的形式”向着自己发展的各个阶段。于是,他们便无批判的将资本主义社会当作中心,认为历史的发展无非是资本主义社会的自我展开过程罢了。因此,他们总是对过去的历史形式做出片面的理解,即通过一种形而上学式的传统,自以为资本主义社会是最美好的,最符合人之本性的社会,历史的发展便是要趋向于资本主义社会自身。因此,在这个意义上,我们不难理解马克思为什么会说“资产阶级经济学只有在资产阶级社会的自我批判已经开始时,才能理解封建的、古代的和东方的经济。”这是因为只有当资产阶级经济学家们对资产阶级社会本身的生成进行批判性分析之时,资产阶级经济学家们才会发现资产阶级社会同以前的种种社会经济形态是相类似的,它们都只不过是历史发展的一定阶段的产物,从而意识到资产阶级社会并不是历史的最终目的或最终社会形态,这种社会形态同样是要在历史的发展中被扬弃的。
\subsubsection{本期小结}

本期读书会大家讨论的较为热烈,参加本期读书会的同样都是非常熟悉的面孔。在本期读书会中我们还讨论了关于劳动二重性的相关问题,限于篇幅,我并没有一一记录下来。我们估计会在下期结束《导言》的“政治经济学的方法”一章中的剩余部分内容的讨论,然后我们这学期的读书会就算是落下帷幕了。希望坚持下来的师友们能真正的有所收获,在这里也预祝下周最后一期的读书会顺利完成。最后,本期的读书会依然是收获满满的。
\newpage
\section{第十期:《<政治经济学批判>导言》(5)}
\subsection{读书会记录}
2023年12月29日14:00—16:00,我们在东北林业大学丹青楼608教室开展了第十期的读书会,本期读书会是本学期最后一次的读书会。我们顺利结束了对马克思《<政治经济学批判>导言》的讨论。接下来我将对本期读书会的讨论内容进行简要概述。
\subsubsection{1.“普照的光”与“特殊的以太”}
我们首先对马克思下面这段话进行了一定的探讨:

\begin{adjustwidth}{2em}{2em}
    \qquad\fangsong
    “在研究经济范畴的发展时,正如在研究任何历史科学,社会科学时一样,应当时刻把握住:无论在现实中或在头脑中,主体——这里是现代资产阶级社会——都是既与的;因而范畴表现这一定社会的,这个主体的存在形式,存在规定,常常只是个别的侧面;因此,这个一定社会在科学上也决不是把它当作这样一个社会来谈论的时候才开始存在的。这必须把握住,因为这对于分篇直接具有决定的意义。”
\end{adjustwidth}

我们认为,马克思在这里想表明的是,一方面,现实主体在人们通过范畴将其表示出来以前就已经存在了;另一方面,现实主体是一个具有整体性的主体,人们在把握这个现实主体的时候,往往只能在侧面把握住它的某一部分。因此从这个意义上讲,我们认为,人们既不能盲目地将对象仅仅看作是一个混沌的整体去研究,也不能仅仅抓住其中某个不要紧的方面去研究,而是要将其合理的进行抽象,然后再从抽象上升到具体,并且这种具体是在人类思维中重组的具体,按照马克思的话来说,这种具体是“许多规定的综合,因而是多样性的统一”。关于这一点,我不再继续赘述了。

接下来,马克思说:
\begin{adjustwidth}{2em}{2em}
    \qquad\fangsong
    “例如,从地租开始,从土地所有制开始,似乎是再自然不过的,因为它是同土地结合着的,而土地是一切生产的源泉,并且它又是同农业结合着的,而农业是一切多少固定的社会的最初的生产方式。但是,这是最错误不过的了。在一切社会形式中都有一种一定的生产支配着其它一切生产的地位和影响。这是一种普照的光,一切其它色彩都隐没其中,它使它们的特点变了样。这是一种特殊的以太,它决定着它里面显露出来的一切存在的比重。”
\end{adjustwidth}

这里出现了马克思的经典名言“普照的光”与“特殊的以太”。我们认为,“普照的光”与“特殊的以太”表达的是一个意思,都是对一定历史阶段上占主导地位的生产关系的比喻。我认为,马克思在这里想表示的是,每一个历史阶段都会存在着一个具有主导地位的生产关系,这是由当时的社会生产力水平以及生产力同生产关系之间的矛盾斗争程度所决定的。例如,对于前资本主义时期的社会而言,农业是对于整个社会的生产而言起主导地位的支柱性产业,但是到了资本主义社会,农业仅仅变成了社会生产中的一个工业部门。这种差异说明了,在某一个历史阶段作为“普遍的光”存在着的东西,在另一个历史时代可能就会失去它的光芒。

我们在下文可以看到马克思是这么论述的:
\begin{adjustwidth}{2em}{2em}
    \qquad\fangsong
    “在资产阶级社会中情况则相反。农业越来越变成仅仅是一个工业部门,完全由资本支配。地租也是如此。在土地所有制居于支配地位的一切社会形式中,自然联系还占优势。在资本居于支配地位的社会形式中,社会,历史所创造的因素占优势。不懂资本便不能懂地租。不懂地租却完全可以懂资本。资本是资产阶级社会的支配一切的经济权力。它必须成为起点又成为终点,必须放在土地所有制之前来说明。分别考察了两者之后,必须考察它们的相互关系。”
\end{adjustwidth}

马克思说,在资本居于支配地位的社会形式(即资本主义社会)中,不懂资本便不能懂地租,而不懂地租却完全可以懂资本。这是不难理解的,因为在资本主义社会中,“普照的光”是资本主义生产关系,是资本主义的经济理性,因而资本主义社会的地租只能被看作是资本主义社会关系的一种特殊表现形式,即对超额利润的瓜分。然而对于前资本主义社会而言,由于“资本主义生产关系”并不是那一历史时期的“普照的光”,因而,同资本主义社会的地租本质恰恰相反,那一历史时期的地租往往表现为一种超经济的强制。

\subsubsection{2.经济范畴的“次序”需要置于特定的历史时期来考察}

让我们顺着资本和地租之间的关系继续说起。在前面我们谈到了,在资产阶级社会,要理解“地租”就必须要首先理解“资本”,这是由于资本主义生产关系作为“普照的光”所导致的,地租在资本主义社会中已经成为了资本主义生产关系的一种特殊表现形式。但我们知道,地租这个经济范畴在资本主义社会形成之前就已然存在了,换句话说,若是按照历史发生的次序排列的话,地租这一经济范畴是先于资本而存在的。但历史发生的次序就是真正地符合“从抽象上升到具体”的逻辑顺序\footnote{编者注:我暂时想不到一个更好的词语来形容马克思在这里面想表述的那种“次序”,请读者把这种“次序”理解为符合对资本主义社会进行科学考察的逻辑顺序,即符合“从抽象上升到具体”的次序。}吗?我们可以看到马克思是这么说的:
\begin{adjustwidth}{2em}{2em}
    \qquad\fangsong
    “因此,把经济范畴按它们在历史上起作用的先后次序来安排是不行的,错误的。它们的次序倒是由他们在现代资产阶级社会中的相互关系决定的,这种关系同看来是它们的合乎自然次序或者符合历史发展次序的东西恰好相反。问题不在于各种经济关系在不同社会形式的相继更替的序列中在历史上占有什么地位,更不在于它们在“观念上”(蒲鲁东)(在历史运动的一个模糊表象中)的次序。而在于它们在现代资产阶级社会内部的结构。”
\end{adjustwidth}

我们可以看到马克思指出,经济范畴的次序一方面不能按照它们在历史上起作用的先后次序来安排,也不能像蒲鲁东一样从“观念上”安排它们的次序。对于后者(即对于蒲鲁东经济思想的批判)而言,我们已经在前面几期读书会中详细讨论过了,读者们可以回头去看我们关于《哲学的贫困》中的讨论记录。在这里我们着重讨论一下为什么马克思认为在对某一社会的考察中,经济范畴的展开次序并不能按照“它们在历史上起作用的先后次序来安排”。

我们知道,若是按照历史发生的顺序来说的话,“地租”是先于“资本”而存在的,那末,对资本主义社会经济范畴的考察似乎需要先从“地租”开始,然后通过“地租”去理解“资本”,即先理解“地租”,才有办法理解“资本”。但事实上,这种看法将经济范畴的历史性维度抹去了,因而是片面与错误的。我们知道,前资本主义时代的地租是一种超经济的强制,而资本主义时代的地租却是一种符合资本原则的“契约关系”,不同时代的经济范畴似乎具有一定的独立性,但其实它们的本质是截然不同的,它们都只不过是特定历史阶段的社会整体的不同侧面。因此,不难理解,在历史上率先出现的“地租”这一经济范畴,在资本主义时代却只能作为资本主义生产方式的结果而存在。换句话说,前资本主义社会的“地租”和资本主义社会的“地租”是本质完全不同的经济范畴,因而绝不能因前资本主义社会的“地租”这一经济范畴在历史上先于资本主义社会的“资本”而产生,就说资本主义社会的“地租”同样是先于资本主义社会的“资本”的。

因此总的来说,对于诸多的经济范畴而言,必须将其置于一定的社会历史阶段中进行考察。不同历史时期的经济范畴事实上往往是具有截然不同的本质的,地租就是一个例子。关于这一点,读者们可以细细思索一下,我在这里就不继续赘述了。

\subsubsection{3.艺术与物质生产的发展}

可能是手稿的原因,《导言》最后一部分内容马克思突然转向了对于艺术与物质生产发展之间不平衡关系的探讨。马克思认为,艺术的繁盛并不是同社会的一般发展成比例的。这事实上对于我们而言是好理解的,因为从经验层面来看,在社会生产力不发展的时期,人们对于很多自然现象往往是无法解释的,这样就会使得人们通过自己的想象来解释难以用科学解释的自然现象,这样就生成了许多具有丰富想象力的艺术形式。正如马克思所说的那样:
\begin{adjustwidth}{2em}{2em}
    \qquad\fangsong
    “希腊艺术的前提是希腊神话,也就是已经通过人民的幻想用一种不自觉的艺术方式加工过的自然和社会形式本身。”
\end{adjustwidth}

在这种神话诞生的时期,社会物质生产还是不发达的,人类与自然之间往往处在一种直接依附性的关系层面,人们对于自然的了解是不够的,科学的不在场为人类幻想的发挥提供了充分的空间。从某种意义上说,物质生产力的低下恰恰是这种神话得以诞生的必要条件。

马克思将人类艺术繁荣发展的时期比喻成人类发展的“儿童”时期,他在《导言》全文的最后是这么说的:
\begin{adjustwidth}{2em}{2em}
    \qquad\fangsong
    “一个成人不能再变成儿童,否则就变得稚气了。但是,儿童的天真不使它感到愉快吗?他自己不该努力在一个更高的阶梯上把自己的真实再现出来吗?在每一个时代,它的固有的性格不是在儿童的天性中纯真地复活着吗?为什么历史上的人类童年时代,在它发展的最完美的地方,不该作为永不复返的阶段而显示出永久的魅力呢?有粗野的儿童,有早熟的儿童。古代民族中有许多是属于这一类的。希腊人是正常的儿童。他们的艺术对我们所产生的魅力,同它在其中生长的那个不发达的社会并不矛盾。它倒是这个社会阶段的结果,并且是同它在其中产生而且只能在其中产生的那些未成熟的社会条件永远不能复返这一点分不开的。”
\end{adjustwidth}

当然,现代人不能再用神话故事去解释自然了,但这种神话故事却依旧是人类文明的宝贵成果。艺术的魅力往往在于人类本性中的那种纯真。然而令人较为感到伤感的是,现代社会生产力的迅速发展似乎逐渐排挤掉了人类源初的那种纯真性,这体现了一种关于现代性的困境。
\subsubsection{本期小结}
本期是我们这学期举办的最后一期的读书会,大家讨论的也是非常热烈。时间总是过得飞快,一转眼两个半月就过去了。在本学期的读书会中,我们先后讨论了马克思的《哲学的贫困》第二章“政治经济学的形而上学”、《雇佣劳动与资本》与《<政治经济学批判>导言》等的内容。此外,在本学期的读书会中,我们还讨论了一系列关于马克思主义政治经济学的基本理论问题,以及国内外相关的理论研究,并在一定程度上探讨了马克思的价值与生产价格理论、地租理论以及政治经济学研究的科学方法论,能够在一定程度上运用马克思主义政治经济学原理对资产阶级古典政治经济学中的一些基本理论观点进行批判。最后,非常感谢各位老师与同学对本读书会的支持,总的来说,本学期的读书会是收获满满的。






\newpage
\section{读书会书目汇总}
\subsection{哲学的贫困(节选)}
\subsubsection{第二章~政治经济学的形而上学}
\begin{center}
    \textbf{第一节~方法}
\end{center}

现在我们已在德国中心!我们一方面谈论政治经济学,同时又要谈形而上学。这一次,我们也只是跟着蒲鲁东先生的“矛盾”走。刚才他迫使我们讲英国话,使我们在相当程度上变成了英国人。现在场面变了。蒲鲁东先生使我们回到我们亲爱的祖国,使我们不由得又变成了德国人。

如果说有一个英国人把人变成帽子,那末,有一个德国人就把帽子变成了观念。这个英国人就是李嘉图,一位银行巨子,杰出的经济学家;这个德国人就是黑格尔,柏林大学的一位专任哲学教授。法国最末一个专制君主和法兰西王朝没落的代表者路易十五有一个御医,这个人同时又是法国的第一个经济学家。这位御医,这位经济学家是预言法国资产阶级必然要取得胜利的先知。魁奈医生使政治经济学成为一门科学;他在自己的名著“经济表”中概括地叙述了这门科学。除了已经有的对该表的一千零一个注解以外,我们还找到医生本人做的一个注解。这就是附有“七个重要说明”的“经济表的分析”。

蒲鲁东先生是魁奈医生第二,他是政治经济学的形而上学方面的魁奈。

但是在黑格尔看来,形而上学同整个哲学一样,可以概括在方法里面。所以我们必须设法弄清楚蒲鲁东先生那套至少同“经济表”一样含糊不清的方法。因此,我们做了七个比较重要的说明。如果蒲鲁东博士不满意我们的说明,那没关系,他可以扮演修道院长勃多的角色,亲自写一篇“经济学——形而上学方法解说”。

\begin{center}
\textbf{第一个说明}    
\end{center}

\begin{fangsong}
    “这里我们论述的不是适应时间次序的历史,而是适应观念顺序的历史。各经济阶段或范畴有时候是同时出现,有时候又是颠倒的……不过,经济理论有它自己的逻辑顺序和理性中的一定系列,经济理论的这种次序,如所预期的那样,已被我们发现了。”(蒲鲁东,“贫困的哲学”第一卷第145和146页)
\end{fangsong}

蒲鲁东先生把这些冒牌的黑格尔词句扔向法国人,毫无问题是想吓唬他们一下。这样一来,我们就要同两个人打交道:首先是蒲鲁东先生,其次是黑格尔。蒲鲁东先生和其它经济学家有什么不同呢?黑格尔在蒲鲁东先生的政治经济学中又起什么作用呢?

经济学家们都把分工、信用、货币等资产阶级生产关系说成是固定不变的、永恒的范畴。蒲鲁东先生有了这些完全形成的范畴,他想给我们说明所有这些范畴、原理、规律、观念、思想的形成情况和来历。

经济学家们向我们解释了生产怎样在上述关系下进行,但是没有说明这些关系本身是怎样产生的,也就是说,没有说明产生这些关系的历史运动。由于蒲鲁东先生把这些关系看成原理、范畴和抽象的思想,所以他只要把这些思想(它们在每一篇政治经济学论文末尾已经按字母表排好)编一下次序就行了。经济学家的材料是人的生动活泼的生活;蒲鲁东先生的材料则是经济学家的教条。但是,既然我们忽略了生产关系(范畴只是它在理论上的表现)的历史发展,既然我们只希望在这些范畴中看到观念、不依赖实际关系而自生的思想,那末,我们就只得到纯理性的运动中去找寻这些思想的来历了。纯粹的、永恒的、无人身的理性怎样产生这些思想呢?它是怎样造成这些思想的呢?

假如在黑格尔主义方面我们具有蒲鲁东先生那种大无畏精神,我们就会说,理性在自身中把自己和自身区分开来。这是什么意思呢?因为无人身的理性在自身之外既没有可以安置自己的地盘,又没有可与自己对置的客体,也没有自己可与之结合的主体,所以它只得把自己颠来倒去:安置自己,把自己跟自己对置起来,自己跟自己结合——安置、对置、结合。用希腊语来说,这就是:正题、反题、合题。对于不懂黑格尔语言的读者,我们将告诉他们一个神圣的公式:肯定、否定、否定的否定。这就是措词的含意。固然这不是天书(请蒲鲁东先生不要见怪),然而却是脱离了个体的纯理性的语言。这里看到的不是一个用普通方式说话和思考的普通个体,而是没有个体的纯粹普通方式。

在抽象的最后阶段(因为这里谈的是抽象,而不是分析),一切事物都成为逻辑范畴,这用得着奇怪吗?如果我们抽掉构成某座房屋特性的一切,抽掉建筑这座房屋所用的材料和构成这座房屋特点的形式,结果只剩下一个一般的物体;如果把这一物体的界限也抽去,结果就只有空间了;如果再把这个空间的向度抽去,最后我们就只有同纯粹的数量,即数量的逻辑范畴打交道了,这用得着奇怪吗?用这种方法把每一个物体的一切所谓偶性(有生命的或无生命的,人类的或物类的)抽去,我们就有理由说,在抽象的最后阶段,作为实体的将是一些逻辑范畴。所以形而上学者认为进行抽象就是进行分析,越远离物体就是日益接近物体和深入事物。这些形而上学者说,我们世界上的事物只不过是逻辑范畴这种底布上的花彩;在他们自己看来,这种说法是正确的。哲学家和基督教徒不同之处正是在于:基督徒只知道逻各斯的化身,不管什么逻辑不逻辑;而哲学家那里则有无数这种化身。既然如此,那末一切存在物,一切生活在地上和水中的东西经过抽象都可以归结为逻辑范畴,因而整个现实世界都淹没在抽象世界之中,即淹没在逻辑范畴的世界之中,这又有什么奇怪呢?

一切存在物,一切生活在地上和水中的东西,只是由于某种运动才得以存在、生活。例如,历史的运动创造了社会关系,工业的运动给我们提供了工业产品,等等。

正如我们通过抽象把一切事物变成逻辑范畴一样,我们只要抽去各种各样的运动的一切特征,就可得到抽象形态的运动,纯粹形式上的运动,运动的纯粹逻辑公式。既然我们把逻辑范畴看做一切事物的实体,那末也就不难设想,我们在运动的逻辑公式中已找到了一种绝对方法,它不仅说明每一个事物,而且本身就包含每个事物的运动。

关于这种绝对方法,黑格尔这样说过:

\begin{fangsong}
    “方法是任何对象所不能抗拒的一种绝对的、唯一的、最高的、无限的力量;这是理性企图在每一个事物中发现和认识自己的意向。”(“逻辑学”第三卷[60])
\end{fangsong}

既然把任何一种事物都归结为逻辑范畴,任何一个运动、任何一种生产行为都归结为方法,那末,由此自然得出一个结论,产品和生产、对象和运动的任何总和都可以归结为应用的形而上学。黑格尔为宗教、法等做过的事情,蒲鲁东先生也想在政治经济学上如法泡制。

那末,这种绝对方法到底是什么呢?是运动的抽象。运动的抽象是什么呢?是抽象形态的运动。抽象形态的运动是什么呢?是运动的纯粹逻辑公式或者纯理性的运动。纯理性的运动又是怎么回事呢?就是它安置自己,把自己跟自己对置,自相结合,就是它把自己规定为正题、反题、合题,或者就是它自我肯定、自我否定和否定自我否定。

理性怎样进行自我肯定,或者它怎样把自己形成这种或那种特定的范畴呢?这已经是理性自己及其辩护人的事情了。

但是理性一旦把自己作为正题安置下来,这个正题、这个思想就会自相对置,分为两个互相矛盾的思想,即肯定和否定,“是”和“否”。这两个包含在反题中的对抗因素的斗争,形成辩证运动。“是”转化为“否”,“否”转化为“是”。“是”同时成为“是”和“否”,“否”同时成为“否”和“是”。对立面就是通过这种方式互相均衡,互相中和,互相抵消。这两个彼此矛盾的思想的融合,就形成一个新的思想,即它们的合题。这个新的思想又分为两个彼此矛盾的思想,而这两个思想又融合成新的合题。这种增殖过程就构成思想群。同简单的范畴一样,思想群也遵循这个辩证运动,它也有另一个与自己矛盾的群为自己的反题。从这两个思想群中产生出新的思想群,即它们的合题。

正如从简单范畴的辩证运动中产生群一样,从群的辩证运动中产生系列,从系列的辩证运动中又产生整个体系。

把这个方法运用到政治经济学的范畴上面,就会得出政治经济学的逻辑学和形而上学,换句话说,就会把人所共知的经济范畴翻译成人们不大知道的语言,这种语言使人觉得这些范畴似乎是刚从充满纯粹理性的头脑中产生的,好象这些范畴单凭辩证运动才互相产生、互相联系、互相交织。请读者不要害怕这个形而上学以及它那一大堆范畴、群、系列和体系,尽管蒲鲁东先生费了九牛二虎之力想爬上矛盾体系的顶峰,可是他从来没有超越过头两级即简单的正题和反题,而且这两级他仅仅爬上过两次,其中有一次还跌了下来。

在这以前我们谈的只是黑格尔的辩证法。下面我们要看到蒲鲁东先生怎样把它降低到极可怜的程度。黑格尔认为,世界上过去发生的一切和现在还在发生的一切,就是他自己的思维中发生的一切。因此,历史的哲学仅仅是哲学的历史,即他自己的哲学的历史。没有“适应时间次序的历史”,只有“观念在理性中的顺序”。他以为他是在通过思想的运动建设世界;其实,他只是根据自己的绝对方法把所有人们头脑中的思想加以系统的改组和排列而已。

\begin{center}
    \textbf{第二个说明}
\end{center}

经济范畴只不过是生产方面社会关系的理论表现,即其抽象。真正的哲学家蒲鲁东先生对事物的理解是颠倒的,他认为现实关系只是睡在“人类的无人身的理性”怀抱里(正如这位哲学家蒲鲁东先生告诉我们的)的一些原理和范畴的化身。

经济学家蒲鲁东先生非常明白,人们是在一定的生产关系范围内制造呢绒、麻布和丝织品的。但是他不明白,这些一定的社会关系同麻布、亚麻等一样,也是人们生产出来的。社会关系和生产力密切相联。随着新生产力的获得,人们改变自己的生产方式,随着生产方式即保证自己生活的方式的改变,人们也就会改变自己的一切社会关系。手工磨产生的是封建主为首的社会,蒸汽磨产生的是工业资本家为首的社会。

人们按照自己的物质生产的发展建立相应的社会关系,正是这些人又按照自己的社会关系创造了相应的原理、观念和范畴。

所以,这些观念、范畴也同它们所表现的关系一样,不是永恒的。它们是历史的暂时的产物。

生产力的增长、社会关系的破坏、思想的产生都是不断变动的,只有运动的抽象即“不死的死”才是停滞不动的。

\begin{center}
    \textbf{第三个说明}
\end{center}

每一个社会中的生产关系都形成一个统一的整体。蒲鲁东先生把种种经济关系看做同等数量的社会阶段,认为这些阶段一个产生一个,一个来自一个,正如反题来自正题一样;认为这些阶段在自己的逻辑顺序中实现着人类的无人身的理性。

这个方法的唯一短处就是:蒲鲁东先生在考察其中任何一个阶段时,都不能不靠其它一些社会关系来说明,可是当时这些社会关系尚未被他用辩证运动产生出来。当蒲鲁东先生后来借助纯粹理性使其它阶段产生出来时,却又把它们当成初生的婴儿,忘记它们和第一个阶段是同样年老了。

因此,要构成被他看做一切经济发展基础的价值,就非有分工、竞争等等不可。然而当时这些关系在一定的系列中、在蒲鲁东先生的理性中以及逻辑顺序中根本还不存在。

谁用政治经济学的范畴构筑某种思想体系的大厦,谁就是把社会体系的各个环节割裂开来,就是把社会的各个环节变成同等数量的互相连接的单个社会。其实,单凭运动、顺序和时间的逻辑公式怎能向我们说明一切关系同时存在而又互相依存的社会机体呢?

\begin{center}
    \textbf{第四个说明}
\end{center}

现在我们看一看蒲鲁东先生把黑格尔的辩证法应用到政治经济学上去的时候,把它变成了什么样子。

蒲鲁东先生认为,任何经济范畴都有好坏两个方面。他看范畴就象小资产者看历史伟人一样:拿破仑是一个大人物;他行了许多善,但是也作了许多恶。

蒲鲁东先生认为,好的方面和坏的方面,益处和害处加在一起就构成每个经济范畴所固有的矛盾。

应当作的是:保存好的方面,消除坏的方面。

奴隶制是同其它任何经济范畴一样的一个经济范畴。因此,它也有两个方面。我们抛开奴隶制的坏的方面不谈,且来看看它的好的方面。自然,这里谈的只是直接奴隶制,即苏里南、巴西和北美南部各州的黑人奴隶制。

同机器、信用等等一样,直接奴隶制是资产阶级工业的基础。没有奴隶制就没有棉花;没有棉花现代工业就不可设想。奴隶制使殖民地具有价值,殖民地产生了世界贸易,世界贸易是大工业的必备条件。可见,奴隶制是一个极重要的经济范畴。

没有奴隶制,北美这个进步最快的国家就会变成宗法式的国家。如果从世界地图上把北美划掉,结果看到的是一片无政府状态,现代贸易和现代文明十分衰落的情景。消灭奴隶制就等于从世界地图上抹掉美洲\footnote{注:这对1847年说来是完全正确的。当时美国的对外贸易主要限于输入移民和工业产品,输出棉花和烟草,即南部奴隶劳动的产物。北部各州主要是为奴隶占有制各州生产谷物和肉类。直至北部开始生产供输出用的谷物和肉类,并且成为工业国,而美洲棉花的垄断又遇到印度、埃及、巴西等国的激烈竞争的时候,奴隶制才有可能废除。而且当时,奴隶制的废除曾引起南部的破产,因为南部还没有以印度和中国隐蔽的苦力奴隶制代替公开的黑人奴隶制。——弗·恩·(恩格斯在1885年德文版上加的注)}。

因为奴隶制是一个经济范畴,所以它总是列入各民族的社会制度中。现代各民族只是在本国内把奴隶制掩饰一下,而在新大陆却赤裸裸地公开推行奴隶制。

蒲鲁东先生将用什么办法挽救奴隶制呢?他提出的任务是:保存这个经济范畴的好的方面,消除其坏的方面。

黑格尔没有需要提出任务。他只有辩证法。蒲鲁东先生从黑格尔的辩证法那里只学得了术语。而蒲鲁东先生自己的辩证运动只不过是机械地划分出好、坏两面而已。

我们暂且把蒲鲁东先生当做一个范畴看待,看一看他的好的方面和坏的方面,他的长处和短处。

如果说,与黑格尔比较,他的长处是提出任务并且保留为人类最大幸福而解决这些任务的权利,那末,他也有一个短处:当他想用辩证法引出一个新范畴时,却毫无所获。两个矛盾方面的共存、斗争以及融合成一个新范畴,就是辩证运动的实质。谁要给自己提出消除坏的方面的任务,就是立即使辩证运动终结。我们看到的已经不是由于矛盾本性而自我安置和自相对置的范畴,而是在范畴的两个方面中间激动、挣扎和冲撞的蒲鲁东先生。

这样,蒲鲁东先生就陷入了用正当方法难以摆脱的困境,于是他用尽全力一跳,便跳到一个新范畴的领域中。这时在他那惊异的目光面前便出现了理性中的一定系列。

他抓住第一个到手的范畴,随心所欲地给它一种特性:把应该清除的范畴的缺陷消除。例如,如果相信蒲鲁东先生的话,捐税可以消除垄断的缺陷,贸易差额可以消除捐税的缺陷,土地所有权可以消除信用的缺陷。

这样,蒲鲁东先生把所有经济范畴逐一取来,把一个范畴用作另一个范畴的消毒剂,用矛盾和矛盾的消毒剂的混合物写成两卷矛盾,并且恰当地称为“经济矛盾的体系”。

\begin{center}
    \textbf{第五个说明}
\end{center}

\begin{fangsong}
    “在绝对理性中,所有这些观念……是同样简单和普遍的……实际上我们只有靠我们的观念搭成的一种脚手架才能达到科学境地。但是,真理本身并不以这些辩证的图形为转移,而且不受我们智能的种种组合的束缚。”(蒲鲁东,第二卷第97页)
\end{fangsong}

这样,一个急转弯(现在我们才知道其中奥妙)就使政治经济学的形而上学变成了幻想!蒲鲁东先生的话从来没有说得这样公正。当然,如果把辩证运动的全部过程归结为简单地对比善和恶,归结为提出任务来消除恶并且把一个范畴用作另一个范畴的消毒剂,那末范畴就失去自己的独立运动;观念就“不再发生作用”;它就没有内在的生命。它既不能把自己安置为范畴,也不能把自己分解为范畴。范畴的顺序成了一种脚手架。辩证法已不是绝对理性的运动了。辩证法没有了,代替它的至多不过是最纯粹的道德而已。

当蒲鲁东先生谈到理性中的一定系列即范畴的逻辑顺序的时候,他肯定地说,他不是想论述适应时间次序的历史,即蒲鲁东先生所认为的范畴在其中出现的历史顺序。他认为那时一切都在理性的纯粹以太中进行。一切都应当通过辩证法从这种以太中产生。现在当实际应用这种辩证法的时候,理性却背叛了他。蒲鲁东先生的辩证法背弃了黑格尔的辩证法,于是蒲鲁东先生只得承认,他用以说明经济范畴的次序和这些经济范畴在其中相互产生的次序是不相适应的。经济的进化不再是理性本身的进化了。

那末,蒲鲁东先生给了我们什么呢?是现实的历史、即蒲鲁东先生所认为的范畴适应着时间次序在其中出现的那种顺序吗?不是。是在观念本身中进行的历史吗?更不是。这就是说,他既没有给我们范畴的世俗历史,也没有给我们范畴的神圣历史!那末,到底他给了我们什么历史呢?是他本身矛盾的历史。让我们来看看这些矛盾怎样行进以及它们怎样拖着蒲鲁东先生吧。

在未研究这一点(这是第六个重要说明的引子)之前,我们应当再作一个比较次要的说明。

我们暂且和蒲鲁东先生一同假定:现实的历史,适应时间次序的历史是观念、范畴和原理在其中出现的那种历史顺序。

每个原理都有其出现的世纪。例如,与权威原理相适应的是11世纪,与个人主义原理相适应的是18世纪,推其因果,我们应当说,不是原理属于世纪,而是世纪属于原理。换句话说,不是历史创造原理,而是原理创造历史。但是,如果为了顾全原理和历史我们再进一步自问一下,为什么该原理出现在11世纪或者18世纪,而不出现在其它某一世纪,我们就必然要仔细研究一下:11世纪的人们是怎样的,18世纪的人们是怎样的,在每个世纪中,人们的需求、生产力、生产方式以及生产中使用的原料是怎样的;最后,由这一切生存条件所产生的人与人之间的关系是怎样的。难道探讨这一切问题不就是研究每个世纪中人们的现实的、世俗的历史,不就是把这些人既当成剧作者又当成剧中人物吗?但是,只要你们把人们当成他们本身历史的剧中人物和剧作者,你们就是迂回曲折地回到真正的出发点,因为你们抛弃了最初作为出发点的永恒的原理。

至于蒲鲁东先生,他一直还在思想家所走的这条迂回曲折的道路上缓慢行进,离开历史的康庄大道还有一大段路程。

\begin{center}
    \textbf{第六个说明}
\end{center}

我们且沿着这条迂回曲折的道路跟蒲鲁东先生走下去。

假定被当做不变规律、永恒原理、理想范畴的经济关系先于人们的生动活跃的生活而存在;再假定这些规律、这些原理、这些范畴自古以来就睡在“人类的无人身的理性”的怀抱里。我们已经看到,在这一切一成不变的、停滞不动的永恒下面没有历史可言,即使有,至多也只是观念中的历史,即反映在纯理性的辩证运动中的历史。蒲鲁东先生谈到辩证运动中的各种观念不能自相“区分”时,把运动的一切影子和影子(它们可以造成某种类似历史的东西)的一切运动一概抹熬。他没有这样做,反而把自己的无能归罪于历史,埋怨一切,甚至连法国话也埋怨起来。

哲学家蒲鲁东先生告诉我们:“我们说什么东西出现或者什么东西产生,这种说法是不确切的,无论是在文明里还是在宇宙中,自古以来一切就存在着、活动着……整个社会经济也是如此。”(蒲鲁东,第二卷第102页)

在蒲鲁东先生的体系中起作用并且使蒲鲁东先生本人也起作用的矛盾的实力竟大到这样程度,以至他本想说明历史,但却不得不否定历史;本想说明社会关系的顺次出现,但却根本否定某种东西可以出现;本想说明生产及其一切阶段,但却否定某种东西可以生产出来。

这样,在蒲鲁东先生看来,再没有什么历史,也没有什么观念的顺序了;可是,他那本自称为“适应观念顺序的历史”的大作却继续存在。怎样才能找到一个公式(因为蒲鲁东先生就是公式的人物)帮助他一跳就越过这一切矛盾呢?

为了做到这一点,他发明了一种新理性,这既不是绝对的、纯粹的和纯真的理性,也不是生活在不同历史时期的活跃的人们的普通的理性;这是一种十分特殊的理性,是作为人的社会的理性,是称为人类的这种主体的理性,这种理性在蒲鲁东先生的笔下有时也被写为“社会天才”、“普遍理性”以及“人类理性”。然而这种名目繁多的理性都是蒲鲁东先生的个人理性,它有一切好的和坏的方面,有消毒剂也有任务。

“人类理性不创造真理”,真理蕴藏在绝对的永恒的理性的深处。它只能发现真理。但是直到现在它所发现的真理是不完备的,不充足的,而且是矛盾的。经济范畴是人类理性、社会天才所发现和揭示出来的真理,所以也是不完备的并含有矛盾的萌芽。在蒲鲁东先生以前,社会天才只看见对抗因素而未发现综合公式,虽然两者同时潜藏在绝对理性里面。既然经济关系只是这些不充足的真理、这些不完备的范畴、这些矛盾的概念在人世间的实现,因此,它们本身就包含着矛盾,并且有好坏两个方面。

社会天才的任务是发现完备的真理、完整无缺的概念、排除二律背反的综合公式。这就再一次说明,为什么蒲鲁东先生想象中的这个社会天才不得不从一个范畴跑到另一个范畴,尽管已经有了一整套范畴,但是直到现在还不能从上帝那里,从绝对理性那里得到一个综合公式:

\begin{fangsong}
    “首先,社会(社会天才)\footnote{括号里的话是马克思的。——译者注}假定一个原始的事实,提出一个假设……一个真正的二律背反,它的对抗性结果在社会经济中展开就象它们作为后果可以在精神上被推论出来一样,所以工业运动在各方面随着观念的演绎分为两道洪流:一道是有益行为的洪流,一道是有害结果的洪流……为了和谐地构成这个两重性的原理和解决这个二律背反,社会就产生第二个二律背反,随后很快地又产生第三个二律背反;社会天才将一直这样行进,直到它用尽自己的全部矛盾(尽管未曾得到证实,但是我料想,人类固有的矛盾是有止境的),一跳而回到它自己原来的各种论点并在唯一的公式中将自己的全部任务加以解决时为止。”(第一卷第133页)
\end{fangsong}

正如以前反题变成消毒剂一样,现在正题将变成假设。但是,蒲鲁东先生这种术语上的交换现在再也不能使我们感到惊奇了。人类理性最不纯洁,因为它只具有不完备的见解,每走一步都要遇到新的待解决的任务。人类理性在绝对理性中发现的以及作为第一个正题的否定的每一个新的正题,对它说来都是一个合题,并且被它相当天真地当做一个任务的解决。这个理性就这样在不断变换的矛盾中乱窜,直至它达到了矛盾的终点,发觉这一切正题和合题不过是相互矛盾的假设时为止。在极度混乱的状态下,“人类理性、社会天才一跳而回到它自己原来的各种论点并在唯一的公式中将自己的全部任务加以解决”。

假设只是为了某种特定的目的而设立的。通过蒲鲁东先生之口讲话的社会天才首先给自己提出的目的,就是消除每个经济范畴的一切坏的东西,使它只保留好的东西。他认为,好的东西,最高的幸福,真正的实际目的就是平等。为什么社会天才只要平等,而不要不平等或友爱、不要天主教或别的什么原理呢?因为“人类之所以实现这么多特殊的假设,正是由于考虑到一个最高的假设”,这个最高的假设就是平等。换句话说,因为平等是蒲鲁东先生的理想。他以为分工、信用、工厂,一句话,一切经济关系都仅仅是为了平等的利益才被发明的,但是结果它们往往对平等不利。由于历史和蒲鲁东先生的臆测步步发生矛盾,所以他得出结论说,有矛盾存在。即使是有矛盾存在,那也只存在于他的固定观念和现实运动之间。

从此以后,肯定平等的就是每个经济关系的好的方面,否定平等和肯定不平等的就是坏的方面。每一个新的范畴都是社会天才为了消除前一个假设所产生的不平等而作的假设。总之,平等是原始的意向、神秘的趋势、天命的目的,社会天才在经济矛盾的圈子里旋转时从来没有忽略过它。因此,天命是一个火车头,用它拖蒲鲁东先生的全部经济行囊前进远比用他那走了气的纯粹理性要好得多。我们这位著者在论捐税一章之后,用了整整一章来写天命。

天命,天命的目的,这是当前用以说明历史进程的一个响亮字眼。其实这个字眼不说明任何问题。它至多不过是一种修辞形式,是冗长地重述事实的若干方式之一。

大家知道,英国工业的发展提高了苏格兰地产的价值。英国工业为羊毛开辟了新的销售市场。要生产大量的羊毛,必须把耕地变成牧场。要这样做就必须集中地产。要集中地产就必须消灭世袭租佃者的小农庄,使成千上万的租佃者离开家园,让放牧几百万只羊的少数牧羊人来居住。这样,由于耕地接连不断地变成牧场,结果苏格兰的地产使羊群赶走了人。如果现在你们说,羊群赶走人就是苏格兰土地私有制度的天命的目的,那末,你们就会得到天命的历史。

当然,平等趋势是我们这个世纪所特有的。但是,说以往各世纪及其完全不同的需求、生产数据等等都是为实现平等而遵照天命行事,这首先就是把我们这个世纪的人和生产数据当做过去世纪的人和生产数据看待,否认世世代代不断改变前代所获得的成果的历史运动。经济学家们很清楚,同是一件东西对甲说来是成品,对乙说来只是从事另一种生产的原料。

如果你们同蒲鲁东先生一道假定:社会天才制造出,或者更确切些说随兴制造出封建主,是为了达到把耕者变为负有义务的和彼此平等的劳动者这一天命的目的,那末,你们就是把目的和人换了一下,这种做法和为了达到恶意的满足(即羊群赶走人)而在苏格兰确立土地私有制的天命比较起来,毫不逊色。

可是,蒲鲁东先生既然对天命表现出那样亲切的关怀,我们就介绍他看一看维尔纽夫-巴尔热蒙的“政治经济学历史”,此人也是追求天命的目的。但他这个目的已经不是平等,而是天主教了。

\begin{center}
    \textbf{第七个即最后一个说明}
\end{center}

经济学家们在论断中采用的方式是非常奇怪的。他们认为只有两种制度:一种是人为的,一种是天然的。封建制度是人为的,资产阶级制度是天然的。在这方面,经济学家很象那些把宗教也分为两类的神学家。一切异教都是人们臆造的,而他们自己的教则是神的启示。经济学家所以说现存的关系(资产阶级生产关系)是天然的,是想以此说明,这些关系正是使生产财富和发展生产力得以按照自然规律进行的那些关系。因此,这些关系是不受时间影响的自然规律。这是应当永远支配社会的永恒规律。于是,以前是有历史的,现在再也没有历史了。以前所以有历史,是由于有过封建制度,由于在这些封建制度中有一种和经济学家称为自然的、因而是永恒的资产阶级社会生产关系完全不同的生产关系。

封建主义也有过自己的无产阶级,即包含着资产阶级的一切萌芽的农奴等级。封建的生产也有两个对抗的因素,人们称为封建主义的好的方面和坏的方面,可是,却没想到结果总是坏的方面占优势。正是坏的方面引起斗争,产生形成历史的运动。假如在封建主义统治时代,经济学家看到骑士的德行、权利和义务之间美妙的协调、城市中的宗法式的生活、乡村中家庭手工业的繁荣、各同业公会、商会和行会中所组织的工业的发展,总而言之,看到封建主义的这一切好的方面而深受感动,抱定目的要消除这幅图画上的一切阴暗面(农奴状况、特权、无政府状态),那末结果会怎样呢?引起斗争的一切因素就会灭绝,资产阶级的发展在萌芽时就会被切断。经济学家就会给自己提出把历史一笔勾销的荒唐任务。

资产阶级得势以后,也就谈不到封建主义的好的方面和坏的方面了。资产阶级把它在封建主义统治下发展起来的生产力掌握起来。一切旧的经济形式、一切与之相适应的市民关系以及作为旧日市民社会的正式表现的政治制度都被粉碎了。

这样,为了正确地判断封建的生产,必须把它当做以对抗为基础的生产方式来考察。必须指出,财富怎样在这种对抗中间形成,生产力怎样和阶级对抗同时发展,这些阶级中一个代表着社会上坏的、否定的方面的阶级怎样不断地成长,直到它求得解放的物质条件最后成熟。这难道不是说,生产方式、生产力在其中发展的那些关系并不是永恒的规律,而是同人们及其生产力发展的一定水平相适应的东西,人们生产力的一切变化必然引起他们的生产关系的变化吗?由于最重要的是不使文明的果实(已经获得的生产力)被剥夺,所以必须粉碎生产力在其中产生的那些传统形式。从此以后,从前的革命阶级将成为保守阶级。

资产阶级开始自己的历史发展时就有一个本身是封建时期无产阶级残存物的无产阶级存在。资产阶级在其历史发展过程中不可避免地要发展它的对抗性质,起初这种性质或多或少是掩饰起来的,只是处于隐蔽状态。随着资产阶级的发展,在它的内部发展着一个新的无产阶级,即现代无产阶级。无产阶级同资产阶级之间展开了斗争,在双方尚未感觉、注意、重视、理解、承认并公开宣告以前,这个斗争最初仅表现为局部的暂时的冲突,表现为一些破坏行为。另一方面,如果说现代资产阶级的全体成员由于组成一个与另一个阶级相对立的阶级而有共同的利益,那末,由于他们互相对立,他们的利益又是对立的,对抗的。这种利益上的对立是由他们的资产阶级生活的经济条件产生的。资产阶级运动在其中进行的那些生产关系的性质绝不是一致的单纯的,而是两重的;在产生财富的那些关系中也产生贫困;在发展生产力的那些关系中也发展一种产生压迫的力量;只有在不断消灭资产阶级个别成员的财富和形成不断壮大的无产阶级的条件下,这些关系才能产生资产者的财富,即资产阶级的财富;这一切都一天比一天明显了。

这种对抗性质表现得越明显,经济学家们,这些资产阶级生产的学术代表就越和他们自己的理论发生分歧,于是形成了各种学派。

宿命论的经济学家,在理论上对他们所谓的资产阶级生产的否定方面采取漠不关心的态度,正如资产者在实践中对他们赖以取得财富的无产者的疾苦漠不关心一样。这个宿命论学派有古典派和浪漫派两种。古典派如亚当·斯密和李嘉图,他们代表着一个还在同封建社会的残余进行斗争、力图清洗经济关系上的封建残污、扩大生产力、使工商业具有新的规模的资产阶级。从他们的观点看来,参加这一斗争并专心致力于这一狂热活动的无产阶级只是经受着暂时的偶然的苦难,并且它自己也把这些苦难当做暂时的。亚当·斯密和李嘉图这样的经济学家是当代的历史学家,他们的使命只是表明在资产阶级生产关系下如何获得财富,只是将这些关系表述为范畴和规律并证明这些规律和范畴比封建社会的规律和范畴更便于进行财富的生产。在他们看来,贫困只不过是一种暂时的病痛,正如自然界中新生出东西来和工业上新东西出现时的情况一样。

浪漫派属于我们这个时代,这时资产阶级同无产阶级处于直接对立状态,贫困象财富那样大量产生。这时,经济学家便以饱食的宿命论者的姿态出现,他们自命高尚、蔑视那些用劳动创造财富的活人机器。他们的一言一语都仿照他们的前辈,可是,前辈们的漠不关心只是出于天真,而他们的漠不关心却已成为卖弄风情了。

其次是人道学派,这个学派对现时生产关系的坏的方面倒是放在心上的。为了不受良心的责备,这个学派想尽量缓和现有的对比;他们对无产者的苦难以及资产者之间的剧烈竞争表示真诚的痛心;他们劝工人安分守己,好好工作,少生孩子;他们建议资产阶级节制一下生产热情。这个学派的全部理论建立在理论和实践、原理和结果、观念和应用、内容和形式、本质和现实、法和事实、好的方面和坏的方面之间无限的区别上面。

博爱学派是完善的人道学派。他们否认对抗的必然性;他们愿意把一切人都变成资产者;他们愿意实现理论,因为这种理论与实践不同而且本身不会包含对抗。毫无疑问,在理论上把现实中每一步都要遇到的矛盾撇开不管并不困难。那样一来,这种理论就会变成理想化的现实。因此,博爱论者愿意保存那些表现资产阶级关系的范畴,而不要那种构成这些范畴的实质并且同这些范畴分不开的对抗。博爱论者以为,他们是在严肃地反对资产者的实践,其实,他们自己比任何人都更象资产者。

正如经济学家是资产阶级的学术代表一样,社会主义者和共产主义者是无产者阶级的理论家。在无产阶级尚未发展到足以确立为一个阶级,因而无产阶级同资产阶级的斗争尚未带政治性以前,在生产力在资产阶级本身的怀抱里尚未发展到足以使人看到解放无产阶级和建立新社会必备的物质条件以前,这些理论家不过是一些空想主义者,他们为了满足被压迫阶级的需求,想出各种各样的体系并且力求探寻一种革新的科学。但是随着历史的演进以及无产阶级斗争的日益明显,他们在自己头脑里找寻科学真理的做法便成为多余的了;他们只要注意眼前发生的事情,并且有意识地把这些事情表达出来就行了。当他们还在探寻科学和只是创立体系的时候,当他们的斗争才开始的时候,他们认为贫困不过是贫困,他们看不出它能够推翻旧社会的革命的破坏的一面。但是一旦看到这一面,这个由历史运动产生并且充分自觉地参与历史运动的科学就不再是空论,而是革命的科学了。

现在再来谈谈蒲鲁东先生。

每一种经济关系都有其好的一面和坏的一面;只有在这一点上蒲鲁东先生没有背叛自己。他认为好的方面由经济学家来揭示,坏的方面由社会主义者来揭发。他从经济学家那里借用了永恒经济关系的必然性这一看法;从社会主义者那里借用了使他们在贫困中只看到贫困的那种幻想。他对两者都表示赞成,企图拿科学权威当靠山。而科学在他的观念里已成为某种微不足道的科学公式了;他无休止地追逐公式。正因为如此,蒲鲁东先生自以为他既批判了政治经济学,也批判了共产主义;其实他远在这两者之下。说他在经济学家之下,因为他作为一个哲学家,自以为有了神秘的公式就用不着深入纯经济的细节;说他在社会主义者之下,因为他既缺乏勇气,也没有远见,不能超出(哪怕是思辨地也好)资产者的眼界。

他希望成为一种合题,结果只不过是一种总合的错误。

他希望充当科学泰斗,凌驾于资产者和无产者之上,结果只是一个小资产者,经常在资本和劳动、政治经济学和共产主义之间摇来摆去。


\newpage

\subsection{雇佣劳动与资本}
\subsubsection{一}
科伦4月4日\footnote{“雇佣劳动与资本”的某些版本中(包括1891年版在内)去掉了每篇文章开首的日期。——编者注}。我们听到了各方面的责难,说我们没有详述
构成现代阶级斗争和民族斗争的物质基础的经济关系。我们只是
当这些关系在政治冲突中直接表现出来的时候,才有意地讲到过
这些关系。

过去首先必须从日常历史进程中去考察阶级斗争,并根据已
有的和每天新出现的历史材料来从经验上证明:当实现了二月革
命和三月革命的工人阶级遭到失败的时候,它的敌人(在法国是资
产阶级共和派,在整个欧洲大陆则是反对过封建专制制度的资产
阶级和农民阶级)也同时被战胜了;法国“正直的共和国”的胜利,
同时也就是以争取独立的英勇战争响应了二月革命的那些民族的
失败;最后,随着革命工人的失败,欧洲又落到了过去那种受双重
奴役即受英俄两国奴役的地位。巴黎的六月斗争,维也纳的陷落,
柏林十一月\footnote{在1891年的版本中, “柏林十一月”前面加有“1848年”等字。——编者注}的悲喜剧,波兰、意大利和匈牙利的拚命努力,爱尔
兰的严重饥荒——这些就是那些集中表现了欧洲资产阶级和工人阶级之间的阶级斗争的主要事件。我们曾经根据这些实例证明过:
任何一次革命起义,不论它的目的仿佛距离阶级斗争多么远,在革
命的工人阶级没有获得胜利以前,都是不免要失败的;任何一种社
会改革,在无产阶级革命和封建反革命没有在世界战争中较量一
下以前,都是要成为空想的。在我们的阐述中,也如在现实中一样,
比利时和瑞士都是大历史画中的两幅悲喜剧式的、漫画式的世俗
画:前者是资产阶级君主制的典型国家,后者是资产阶级共和制的
典型国家,两者都自以为既跟阶级斗争无关,又跟欧洲革命无关。

现在,在我们的读者看到了1848年以波澜壮阔的政治形式展
开的阶级斗争以后,我们想更切近地考察一下资产阶级的生存及
其阶级统治和工人的奴役地位所依为基础的经济关系本身,也就
适当其时了。

我们分三大部分来加以说明:(1)雇佣劳动对资本的关系,工
人的奴役地位,资本家的统治;(2)中等资产阶级和农民等级\footnote{在1891年的版本中, “农民等级”改为“所谓的市民等级”。——编者注}在
现存制度下必然发生的灭亡过程;(3)欧洲各国资产者阶级在商业
上受世界市场霸主英国奴役和剥削的情形。

我们力求说得尽量简单和通俗,我们就当读者连最起码的政
治经济学概念也没有。我们希望工人能明白我们的解说。加之,在
德国到处对一些最简单的经济关系的了解都极端无知和十分混
乱,从特许的现存制度的辩护者到冒牌的社会主义者和末被承认
的政治天才都莫不如此,这种人在四分五裂的德国比“诸侯王爷”
还多。

我们首先来讲第一个问题:什么是工资?它是怎样决定的?

假如问工人们:“你们的工资是多少?”,那末一个工人回答说:
“我做一天工从资产者那里得到一法郎\footnote{一法郎等于八普鲁士银格罗申。[在1891年的版本中,“法郎”完全改为“马克”。——编者注}。”;另一个工人回答说:
“我得到两法郎”,等等。由于他们工作的劳动部门不同,他们每一
个人因劳动了一定的时间或 \footnote{在1891年的版本中,“劳动了一定的时间或”删去了。——编者注} 做了一定的工作(譬如,织成一尺麻
布或排好一个印张的字)而从各自的资产者那里得到的货币数量
也不同。尽管他们得到的货币数量不同,但是有一点是一致的:就
是工资是资产者为了偿付劳动一定的时间或完成一定的工作而支
出的一笔货币。

可见\footnote{在1891年的版本中,“可见”后面加有“看起来仿佛”。——编者注},资产者是用货币购买工人的劳动。工人是为了货币而
向资产者出卖自己的劳动\footnote{在1891年的版本中, “劳动”后面加有: “但这只是表面情形。实际上,他们是为
了货币向资本家出卖自己的劳动力的。资本家以一日、一周、一月等等为期购
买这个劳动力。而当他把劳动力买到手以后,他就使用它,迫使工人在约定的
期间内劳动。”——编者注}。资产者用以购买工人劳动 \footnote{在1891年的版本中,“资产者用以购买工人劳动”改为“资本家用以购买工人
劳动力”。——编者注}的那个货
币量,譬如说两法郎,也可以买到两磅糖或一定数量的其他某种商
品。他用以购买两磅糖的两法郎,就是两磅糖的价格。他用以购买
十二小时劳动\footnote{在1891年的版本中, “劳动”改为“劳动力使用”。——编者注}的两法郎,就是十二小时劳动的价格。可见,劳动 \footnote{在1891年的版本中, “劳动”改为“劳动力”。——编者注}
是一种商品,是和沙糖一模一样的商品。前者是用钟点来计量的:
后者是用重量来计量的。

工人拿自己的商品即劳动\footnote{在1891年的版本中, “劳动”改为“劳动力”。——编者注} 去换得资本家的商品,即换得货
币,并且这种交换是按一定的比率进行的。一定量的货币交换一定
量的劳动 \footnote{在1891年的版本中, “劳动”改为“劳动力使用”。——编者注}时间。织布工人的十二小时劳动交换两法郎。但是,难
道这两法郎不是代表其他一切可以用两法郎买到的商品吗?可见,
实质上工人是拿他自己的商品即劳动交换各种各样的其他商
品 \footnote{在1891年的版本中,“劳动交换各种各样的其他商品”改为“劳动力交换各种各样的商品”。——编者注} ,并且是按一定的比率变换的。资本家付给他两法郎,就是为
交换他的劳动日而付给了他一定量的肉,一定量的衣服,一定量的
劈柴,一定量的灯光,等等。可见,这两法郎是表现劳动\footnote{在1891年的版本中, “劳动”改为“劳动力”。——编者注} 跟其他商
品相交换的比例,即表现他的劳动 \footnote{在1891年的版本中, “劳动”改为“劳动力”。——编者注}的交换价值。商品通过货币表
现出来的交换价值,也就称为商品的价格。所以,工资只是劳动价
格\footnote{在 1891年的版本中,“劳动价格”改为“通常被称为劳动价格的劳动力价
格”。——编者注}的特种名称,是只能存在于人的血肉中的这种特殊商品价格
的特种名称。

拿任何一个工人来说,比如拿一个织布工人来说吧。资产者\footnote{在1891年的版本中,“资产者”改为“资本家”。——编者注}
供给他一架织布机和一些纱。织布工人动手工作,纱变成了布。资
产者 \footnote{同上}把布拿去,卖了——比方说——二十法郎。织布工人的工资
是不是这块布中的一份,二十法郎中的一份,他的劳动产品中的一
份呢?绝对不是。因为这个织布工人是在布还没有卖出很久以前,
甚至可能是在布还没有织成很久以前就得到了自己的工资的。可
见,资本家支付的这笔工资并不是来自他卖布所赚的那些货币,而是来自他原来储备的货币。织布工人从资产者那里领来使用的织
布机和纱不是他的产品,同样,他用自己的商品即劳动\footnote{在1891年的版本中,“劳动”改为“劳动力”。——编者注}交换所得
的那些商品也不是他的产品。可能有这样的情形:资产者给自己的
布找不到一个买主。他出卖布所赚的钱,也许甚至不能捞回他用于
开销工资的款项。也有可能他出卖布所得的钱,比他付给织布工人
的工资数目大得多。这一切都与织布工人毫不相干。资本家拿自
己的一部分现有财产即一部分资本去购买织布工人的劳动\footnote{同上},这
就同他拿他的另一部分资本去购买原料(纱)和劳动工具(织布机)
完全一样。购买了这些东西(其中包括生产布所必需的劳动 \footnote{同上})以
后,资本家就用只是属于他的原料和劳动工具进行生产。当然,我
们这位善良的织布工人现在也属于劳动工具之列,他也做织布机
一样在产品中或在产品价格中是没有份的。

所以,工资不是工人在他所生产的商品中占有的一份。工资是
原有商品中由资本家用以购买一定量的生产劳动\footnote{同上}的那一部分。

总之,劳动\footnote{同上}是一种商品,是由其所有者即雇佣工人出卖给资
本的一种商品。他为什么出卖它呢?为了生活。

可是\footnote{在1891年的版本中, “可是”后面加有“劳动力在动作中的表现”。——编者注},劳动是工人本身的生命活动,是工人本身的生命的表
现。工人正是把这种生命活动出卖给别人,以获得自己所必需的
生活资料。可见,工人的生命活动对于他不过是使他能以生存的
一种手段而已。他是为生活而工作的。他甚至不认为劳动是自己
生活的一部分;相反地,对于他来说,劳动就是牺牲自己的生活。
劳动是已由他出卖给别人的一种商品。因此,他的活动的产物也就不是他的活动的目的。工人为自己生产的不是他织成的绸缎,不
是他从金矿里开采出的黄金,也不是他盖起的高楼大厦。他为自
己生产的是工资,而绸缎、黄金、高楼大厦对于他都变成一定数
量的生活资料,也许是变成棉布上衣,变成铜币,变成某处地窖
的住所了。一个工人在一昼夜中有十二小时在织布、纺纱、钻孔、
研磨、建筑、挖掘、打石子、搬运重物等等,他能不能认为这十
二小时的织布、纺纱、钻孔、研磨、建筑、挖掘、打石子是他的
生活的表现,是他的生活呢?恰恰相反,对于他来说,在这种活
动停止以后,当他坐在饭桌旁,站在酒店柜台前,睡在床上的时
候,生活才算开始。在他看来,十二小时劳动的意义并不在于织
布、纺纱、钻孔等等,而在于这是挣钱的方法,挣钱使他能吃饭、
喝酒、睡觉。假如说蚕儿吐丝作茧是为了维持自己的生存,那末
它就可算是一个真正的雇佣工人了。

劳动\footnote{在1891年的版本中,“劳动”改为“劳动力”。——编者注}并不向来就是商品。劳动并不向来就是雇佣劳动、即自
由劳动。奴隶就不是把他自己的劳动 \footnote{同上}出卖给奴隶主,正如耕牛不
是向农民卖工一样。奴隶连同自己的劳动 \footnote{同上} 一次而永远地卖给自
己的主人了。奴隶是商品,可以从一个所有者手里转到另一个所有
者手里。奴隶本身是商品,但劳动  \footnote{同上} 却不是 他的商品。农奴只出卖
自己的一部分劳动  \footnote{同上} 。不是他从土地所有者方面领得报酬;相反
地,土地所有者从他那里收取贡赋。农奴是土地的附属品,替土地
所有者生产果实。相反地,自由工人自己出卖自己,并且是零碎地
出卖。他每天把自己生命中的八小时、十小时、十二小时、十五小时
拍卖给出钱最多的人,拍卖给原料、劳动工具和生活资料的所有者,即拍卖给资本家。工人既不属于私有者,也不属于土地,但是他
每日生命的八小时、十小时、十二小时、十五小时却属于它的购买
者。工人只要愿意,就可以离开雇用他的资本家,而资本家也可以
随意辞退工人,只要工人使他不能再获得利益或者不能使他获得
预期的利益,他就可以辞退。但是,工人是以出卖劳动 \footnote{同上} 为其工资
的唯一来源的,如果他不愿饿死,就不能离开整个购买者阶级即资
本家阶级。工人不是属于某一个资产者,而是属于整个资产阶
级\footnote{在1891年的版本中,“不是属于某一个资产者,而是属于整个资产阶级”改为“不是属于某一个资本家,而是属于整个资本家阶级”。——编者注};至于工人给自己寻找一个雇主,即在资产阶级\footnote{在1891年的版本中, “资产阶级”改为“资本家阶级”。——编者注} 中间寻找一
个买主,那是工人自己的事情了。

现在,在更详细地谈论资本和雇佣劳动之间的关系以前,我们
先简短地叙述一下在决定工资时起作用的一些最一般的条件。

我们已经说过,工资是一定商品——劳动 \footnote{在1891年的版本中,“劳动”改为“劳动力”。——编者注}的价格。所以,工
资是由那些决定其他一切商品价格的规律决定的。

那末,试问:商品的价格是怎样决定的呢?

\subsubsection{二}
科伦4月5日。商品的价格是由什么决定的?

它是由买主和卖主之间的竞争即供求关系决定的。决定商品
价格的竞争是三方面的。

同一种商品,有许多不同的卖主供应。谁以最便宜的价格出卖
同一质量的商品,谁就一定会战胜其他卖主,从而保证自己有最大
的销路。于是,各个卖主彼此间就进行争夺销路、争夺市场的斗争。
他们每一个人都想出卖商品,都想尽量多卖,如果可能,都想由他
一个人独卖,而把其余的卖主排挤掉。因此,一个人就要比另一个
人卖得便宜些。于是卖主之间就发生了竞争,这种竞争降低他们所
供应的商品的价格。

但是买主之间也有竞争,这种竞争反过来提高所供应的商品
的价格。

最后,买主和卖主之间也有竞争。前者想买得尽量便宜些,后
者却想卖得尽着贵些。买主和卖主之间的这种竞争的结果怎样,要
依上述竞争双方的对比关系怎样来决定,就是说要看是买主阵营
里的竞争激烈些呢还是卖主阵营里的竞争激烈些。产业把两支军
队抛到战场上对峙,其中每一支军队内部又发生内讧。战胜敌人
的是内部冲突较少的那支军队。

假定说,市场上有一百包棉花,而买主们却需要一千包。在
这种情形下,需求比供应大十倍,因而买主之间的竞争就会很激烈,他们中间的每一个人都竭力设法至少也要搞到一包,如果可
能,就把一百包全都搞到手里。这个例子并不是随意虚构的。在
商业史上有过这样一些棉花歉收的时期,那时几个资本家彼此结
成联盟,不只想把一百包棉花都收买下来,而且想把世界上的全
部存棉都收买下来。这样,在我们前述的情形下,每一个买主都
力图排挤掉另一个买主,出较高的价格收买每包棉花。棉花的卖
主们看见敌军队伍里发生极剧烈的内讧、并完全相信他们的一百
包棉花都能卖掉,因此他们就严防自己内部打起架来,以免在敌
人竞相抬高价格的时候降低自己商品的价格。于是卖主阵营里忽
然奠定了和平。他们冷静地叉着双手,像一个人似地对抗买主;只
要那些最热衷的买主出价又没有一定的限度,那卖主的贪图也就
会没有止境了。

可见,某种商品的供应低于需求,那末这种商品的卖主中间的
竞争就会很弱,甚至于完全没有竞争。卖主中间的竞争在某种程度
上减弱,买主中间的竞争就会在某种程度上加剧。结果便是商品价
格或多或少显著地上涨。

大家知道,较为常见的是产生相反后果的相反情形:供应大大
超过需求,卖主中间拚命竞争,买主少,商品贱价拍卖。

但是,价格上涨和下跌是什么意思呢?高价和低价是什么意思
呢?沙粒在显微镜下就显得高,宝塔比起山岳来就显得低了。既然
价格是由供求关系决定的,那末供求关系又是由什么决定的呢?

让我们随便问一个资产者吧。他会像新即位的亚历山大大帝
一样,马上毫不犹豫地利用乘法表来解开这个形而上学的纽结。他
会对我们说,假如我生产我出卖的这个商品的费用是一百法郎,而
我把它卖了一百一十法郎(自然是在一年期满后),那末这是一种普通的、老实的、正当的利润。假如我在进行交换时得到了一百二
十或一百三十法郎,那就是高额利润了。假如我得到了整整二百法
郎,那就会是特高的巨额利润了。对于这个资产者来说,究竟什么
是衡量利润的尺度呢?这就是他的商品的生产费用。假如他拿自
己的商品换来的一定数量的别种商品,其生产费用少于他的商品
的生产费用,那他就算亏本了。可是假如他拿自己的商品换来的一
定数量的别种商品,其生产费用大于他的商品的生产费用,那他就
算赢利了。他是以生产费用作为零度,根据他的商品的交换价值在
零度上下的度数来测定他的利润的升降的。

由此可见,供求关系的改变,引起价格的上涨或下跌,引起高
价或低价。

假如某一种商品的价格,由于供应不足或需求剧增而大大上
涨,那末另一种商品的价格就不免要相应地下跌,因为商品的价格
不过是以货币来表示的别种商品和它交换的比率。举例说,假如一
尺绸缎的价格从五法郎上涨到六法郎,那末白银的价格对于绸缎
来讲就下跌了,其他一切商品也都是这样,它们的价格虽然没有改
变,但比起绸缎来却是跌价了。这时若要交换得原来那么多的绸
缎,就必须拿出更多的商品。

商品价格上涨会产生什么后果呢?大量资本将涌向繁荣的产
业部门中去,而这种资本流入较为有利的产业部门中去的现象,要
继续到该部门的利润跌落到普通水平时为止,或者更确切些说,要
继续到该部门产品的价格由于生产过剩而跌落到生产费用以下时
为止。

反之,假如某一种商品的价格跌落到它的生产费用以下,那末
资本就要从该种商品生产部门中流出去了。除了该工业部门已经不合时代要求,因而必然衰亡以外,该商品的生产,即该商品的供
应,就要因为资本的这种外流而缩减,直到该商品的供应和需求相
适应为止,就是说,直到该商品的价格重新上涨到它的生产费用水
平,或者更确切些说,直到供应低于需求,即直到商品价格又上涨
到它的生产费用以上为止,因为商品的现时价格总是高于或低于
它的生产费用。

由此可见,资本是不断地从一个生产部门向另一个生产部门
流出或流入的。价格高就引起资本的过分猛烈的流入,价格低就引
起资本的过分猛烈的流出。

我们从另一个角度来研究问题时能够证明:不仅供应,连需求
也是由生产费用决定的。可是,这样一来,我们就未免离题太远了。

我们刚才说过,需求和供应的波动,每次都把商品的价格引导
到生产费用的水平。固然,商品的实际价格始终不是高于生产费
用,就是低于生产费用;但是,上涨和下降是相互抵销的,因此,在
一定时间内,如果把工业中的资本流入和流出总合起来看,就可看
出各种商品是依其生产费用而互相交换的,所以它们的价格是由
生产费用决定的。

价格由生产费用决定这一点,不应当了解成像经济学家们所
了解的那种意思。经济学家们说,商品的平均价格等于生产费用;
在他们看来,这是一个规律。他们把价格的上涨被价格的下降所抵
销,而下降则被上涨所抵销这种无政府状态的变动看作偶然现象。
那末,同样也可以(另一些经济学家就正是这样做的)把价格的波
动看作规律,而把价格由生产费用决定这一点看作偶然现象。可是
实际上,只有在这种波动的进程中,价格才是由生产费用决定的;
我们细加分析时就可以看出,这种波动起着极可怕的破坏作用,并像地震一样震撼资产阶级社会的基础。这种无秩序状态的总运动
就是它的秩序。在这种产业无政府状态的进程中,在这种循环运转
中,竞争可以说是拿一个极端去抵销另一个极端。

由此可见,商品价格是由生产费用这样来决定的:某些时期,
某种商品的价格超过它的生产费用,另一些时期,该商品的价格却
下跌到它的生产费用以下,而抵销以前超过的时期,反之亦然。当
然,这不是就个别产业的产品来说的,而只是就整个产业部门来说
的。所以,这同样也不是就个别产业家来说的,而只是就整个产业
家阶级来说的。

价格由生产费用决定,就等于说价格由生产商品所必需的劳
动时间决定,因为构成生产费用的是:(1)原料和劳动工具\footnote{在1891年的版本中, “劳动工具”改为“劳动工具损耗部分”。——编者注} ,即产
业产品,它们的生产耗费了一定数量的工作日,因而也就是代表一
定数量的劳动时间;(2)直接劳动,它也是以时间计量的。

一般调节商品价格的那些最一般的规律,当然也调节工资,即
调节劳动价格。

劳动报酬忽而提高,忽而降低,是依供求关系为转移的,依购
买劳动 \footnote{在1891年的版本中, “劳动”改为“劳动力”。——编者注}的资本家和出卖劳动\footnote{同上}的工人之间的竞争情形为转移的。
工资的波动一般是和商品价格的波动相适应的。可是,在这种波动
的范围内,劳动的价格是由生产费用即为创造劳动\footnote{同上} 这一商品所
需要的劳动时间来决定的。

那末,劳动 \footnote{同上} 本身的生产费用究竟是什么呢?

这就是为了使工人保持其为工人并把他训练成为工人所需要
的费用。

因此,某一种劳动所需要的训练时间愈少,工人的生产费用也
就愈少,他的劳动的价格即他的工资也就愈低。在那些几乎不需要
任何训练时间,只要有工人的肉体存在就行的产业部门里,为造成
工人所需要的生产费用,几乎只归结为维持工人生存\footnote{在1891年的版本中, “生存”后面加有“及其劳动能力”。——编者注}所需要的
商品。因此,工人的劳动的价格是由必需生活资料的价格决定的。

可是,这里还应该注意到一种情况。

工厂主在计算自己的生产费用,并根据生产费用计算产品的
价格的时候,是把劳动工具的损耗也计算在内的。譬如说,假如一
架机器值一千法郎,使用期限为十年,那末他每年就要往商品价格
中加进一百法郎,以便在十年期满时有可能用新机器来更换用坏
的机器。同样,简单劳动\footnote{在1891年的版本中,“劳动”改为“劳动力”。——编者注}的生产费用中也应加入延缓工人后代的
费用,即使工人阶级能够繁殖后代并用新工人来代替失去劳动能
力的工人的费用。所以,工人的损耗也和机器的损耗一样,是要计
算进去的。

总之,简单劳动\footnote{同上}的生产费用就是 维持工人生存和延续工人
后代的费用。这种维持生存和延续后代的费用的价格就是工资。这
样决定的工资就叫做最低工资。这种最低工资,也和商品价格一般
由生产费用决定一样,不是就单个人来说的,而是就整个种属来说
的。单个工人所得,千百万工人所得,不足以维持生存和延续后代,
但整个工人阶级的工资在其波动范围内则是和这个最低额相等
的。

现在,我们既已讲明了调节工资以及其他任何商品的价格的
最一般规律,我们就能更切近地研究我们的本题了。

\subsubsection{三}

科伦4月6日。资本包括原料、劳动工具和各种生活资料,这些东西是用以生产新的原料、新的劳动工具和新的生活资料的。资本的所有这些组成部分都是劳动的创造物,劳动的产品,积累起来的劳动。作为进行新生产的手段的积累起来的劳动就是资本。

经济学家们就是这样说的。

什么是黑奴呢?黑奴就是黑种人。上面的说明和这个说明是一样的。

黑人就是黑人。只有在一定的关系下,他才成为奴隶。纺纱机是纺棉花的机器。只有在一定的关系下,它才成为资本。脱离了这种关系,它也就不是资本了,就像黄金本身并不是货币,沙糖并不是沙糖的价格一样。

人们在生产中不仅仅同自然界发生关系\footnote{在1891年的版本中,“不仅仅与自然界发生关系”改为“不仅仅影响自然界,而且也互相影响”。——编者注}。他们如果不以一定方式结合起来共同活动和互相交换其活动,便不能进行生产。为了进行生产,人们便发生一定的联系和关系;只有在这些社会联系和社会关系的范围内,才会有他们对自然界的关系,\footnote{在1891年的版本中,“对自然界的关系”改为“对自然界的影响”。——编者注}才会有生产。

生产者相互发生的这些社会关系,他们借以互相交换其活动和参于共同生产的条件,当然依照生产资料的性质而有所不同。随着新作战工具即射击火器的发明,军队的整个内部组织就必然改变了,各个人借以组成军队并能作为军队行动的那些关系就改变了,各个军队相互间的关系也发生了变化。

总之,各个人借以进行生产的社会关系,即社会生产关系,是随着物质生产资料、生产力的变化和发展而变化和改变的。生产关系总合起来就构成为所谓社会关系,构成为所谓社会,并且是构成为一个处于一定历史发展阶段上的社会,具有独特的特征的社会。古代社会、封建社会和资产阶级社会都是这样的生产关系的总和,而其中每一个生产关系的总和同时又标志着人类历史发展中的一个特殊阶段。

资本也是一种社会生产关系。这是资产阶级的生产关系,是资产阶级社会的生产关系。构成资本的生活资料、劳动工具和原料,难道不是在一定的社会条件下,不是在一定的社会关系下生产出来和积累起来的吗?难道这一切不是在一定的社会条件下,在一定的社会关系内被用来进行新生产的吗?并且,难道不正是这种一定的社会性质把那些用来进行新生产的产品变为资本的吗?

资本不仅包括生活资料、劳动工具和原料,不仅包括物质产品。并且还包括交换价值。资本所包括的一切产品都是商品。所以,资本不仅是若干物质产品的总和,并且也是若干商品或若干交换价值或若干社会定量的总和。

不论我们是以棉花代替羊毛也好,是以米代替小麦也好,是以轮船代替铁路也好,只要这些体现资本的棉花、米和轮船同原先体现资本的羊毛、小麦和铁路具有同样的交换价值即同样的价格,那末资本依然还是资本。资本的肉体可以经常改变,但不会使资本性质有丝毫改变。

虽然任何资本都是一些商品即交换价值的总和,然而远不是任何一些商品即交换价值的总和都是资本。

任何一些交换价值的总和都是一个交换价值。任何单个交换价值都是一些交换价值的总和。例如,值一千法郎的一座房子是一千法郎的交换价值。值一生丁[注:在1891年的版本中,“生丁”改为“分尼”。——编者注]的一印张纸是100/100生丁的交换价值的总和。能同别的产品交换的产品就是商品。这些产品由以交换的一定比率就是它们的交换价值,如果这种比率是用货币来表示的,就是它们的价格。这些产品的数量多少丝毫不能改变它们成为商品,或者表现交换价值,或者具有一定价格的这种性能。一株树不论其大小如何,终究是一株树。我们拿铁同别的产品交换时不是以两为单位,而是以公担为单位,难道铁作为商品,作为交换价值的性能竟会因此而改变吗?铁作为一种商品,只是依其数量多少而具有大小不同的价值,高低不同的价格。

一些商品即一些交换价值的总和究竟是怎样成为资本的呢?

它成为资本,是由于它作为一种独立的社会力量,即作为一种属于社会一部分的力量,借交换直接的、活的劳动[注:在1891年的版本中,“劳动”改为“劳动力”。——编者注]而保存下来并增殖起来。除劳动能力以外一无所有的阶级的存在是资本的必要前提。

只是由于积累起来的、过去的、物化的劳动支配直接的、活的劳动,积累起来的劳动才变为资本。

资本的实质并不在于积累起来的劳动是替活劳动充当进行新生产的手段。它的实质在于活劳动是替积累起来的劳动充当保存自己并增加其交换价值的手段。

资本和雇佣劳动\footnote{在1891年的版本中,“资本和雇佣劳动”改为“资本家和雇佣工人”。——编者注}是怎样进行交换的呢?

工人拿自己的劳动\footnote{在1891年的版本中,“劳动”改为“劳动力”。——编者注}换到生活资料,而资本家拿归他所有的生活资料换到劳动,即工人的生产活动,亦即创造力量。这种力量不仅能补偿工人所消费的东西,并且还使积累起来的劳动具有比以前更大的价值。工人从资本家那里得到一部分现有的生活资料。这些生活资料对工人有什么用处呢?用于直接消费。可是,如果我不把靠这些生活资料维持我的生活的一段时间用来生产新的生活资料,即在消费的同时用我的劳动创造新价值来补偿那些因消费而消失了的价值,那末我一把这些生活资料消费完,它们对于我就算是完全白耗费了。但是,工人为了换到生活资料,正是把这种贵重的再生产力量让给了资本家。因此,对于工人本身来说,这种力量是白耗费了。

举一个例子来说吧。有个农场主每天付给他的一个短工五银格罗申。这个短工为得到这五银格罗申,就整天在农场主的田地上干活,保证农场主能得到十银格罗申的收入。农场主不但收回了他付给短工的价值,并且还把它增加了一倍。可见,他有成效地、生产性地使用和消费了他付给短工的五银格罗申。他拿这五银格罗申买到的正是一个短工的能生产出双倍价值的农产品并把五银格罗申变成十银格罗申的劳动和力量。短工则拿他的生产力(他正是把这个生产力让给了农场主)换到五银格罗申,并用它们换得迟早要消费掉的生活资料。所以,这五银格罗申的消费有两种方法:对资本家来说,是有生产性的,因为他用这五银格罗申换来的劳动力使他得到了十银格罗申;对工人来说,是非生产性的,因为他用这五银格罗申换来的生活资料永远消失了,他只有再和农场主进行同样的交换才能重新取得这些生活资料的价值。这样,资本以雇佣劳动为前提,而雇佣劳动又以资本为前提。两者相互制约;两者相互产生。

一个棉纺织厂的工人是不是只生产棉织品呢?不是,他生产资本。他生产重新供人利用去支配他的劳动并借他的劳动创造新价值的价值。

资本只有同劳动[注:在1891年的版本中,“劳动”改为“劳动力”。——编者注]交换,只有引起雇佣劳动的产生,才能增加起来。雇佣劳动[注:在1891年的版本中,“雇佣劳动”改为“雇佣工人的劳动力”。——编者注]只有在它增加资本,使奴役它的那种权力加强时,才能和资本交换。因此,资本的增加就是无产阶级即工人阶级的增加。

所以,资产者及其经济学家们断言,资本家和工人的利益是一致的。千真万确呵!工人若不受雇于资本家就会灭亡。资本若不剥削劳动[注:在1891年的版本中,“劳动”改为“劳动力”。——编者注]就会灭亡,而要剥削劳动[注:在1891年的版本中,“劳动”改为“劳动力”。——编者注],资本就得购买劳动[注:在1891年的版本中,“劳动”改为“劳动力”。——编者注]。投入生产的资本即生产资本增殖愈快,也就是说,产业愈繁荣,资产阶级愈发财,生意愈兴隆,资本家需要的工人也就愈多,工人出卖自己的价格也就愈高。

原来,生产资本的尽快增加竟是工人能勉强过活的必要条件。

但是,生产资本的增加又是什么意思呢?就是积累起来的劳动对活劳动的支配权力的增加,就是资产阶级对工人阶级的统治力量的增加。雇佣劳动生产着对它起支配作用的他人财富,也就是说生产着同它敌对的力量——资本,而它从资本那里取得就业手段,即取得生活资料,是以雇佣劳动又会变成资本的一部分,又会变成使资本加速增殖的杠杆为条件的。

断言资本的利益和劳动的利益[注:在1891年的版本中,“劳动的利益”改为“工人的利益”。——编者注]是一致的,事实上不过是说资本和雇佣劳动是同一种关系的两个方面罢了。一个方面制约着另一个方面,就如同高利贷者和挥霍者相互依存一样。

当雇佣工人仍然是雇佣工人的时候,他的命运是取决于资本的。所谓工人和资本家的利益一致就是这么一回事。

\subsubsection{四}
科伦4月7日。资本愈增长,雇佣劳动量就愈增长,雇佣工人
人数就愈增加,一句说,受资本支配的人数就愈增多。我们且假定
有这样一种最适意的情形:随着生产资本的增加,对劳动的需求也
增加了,因而劳动价格即工资也提高了。

一座小房子不管怎样小,在周围的房屋都是这样小的时候,它
是能满足社会对住房的一切要求的。但是,一旦在这座小房子近旁
耸立起一座宫殿,这座小房子就缩成可怜的茅舍模样了。这时,狭
小的房子证明它的居住者毫不讲究或者要求很低;并且,不管小房
子的规模怎样随着文明的进步而扩大起来,但是,只要近旁的宫殿
以同样的或更大的程度扩大起来,那末较小房子的居住者就会在
那四壁之内越发觉得不舒适,越发不满意,越发被人轻视。

工资的任何显著的增加是以生产资本的迅速增加为前提的。
生产资本的迅速增加,就要引起财富、奢侈,社会需要和社会享受
等同样迅速的增长。所以,工人可以得到的享受纵然增长了,但是,
比起资本家的那些为工人所得不到的大为增加的享受来,比起一
般社会发展水平来,工人所得到的社会满足的程度反而降低了。我
们的需要和享受是由社会产生的,因此,我们对于需要和享受是以
社会的尺度,而不是以满足它们的物品去衡量的。因为我们的需要
和享受具有社会性质,所以它们是相对的。

工资一般不仅是由我能够用它交换到的商品数量来决定的。工资包含着各种对比关系。

首先,工人靠出卖自己的劳动\footnote{在1891年的版本中, “劳动”改为“劳动力”。——编者注}取得一定数量的货币。工资是
不是单由这个货币价格来决定的呢?

在十六世纪,由于美洲的发现\footnote{在1891年的版本中,“美洲的发现”改为“在美洲发现了更丰富和更易于开采
的金矿”。——编者注},欧洲流通的黄金和白银的数
量增加了。因此,黄金和白银的价值和其他各种商品比较起来就降
低了。但是,工人们出卖自己的劳动\footnote{在1891年的版本中, “劳动”改为“劳动力”。——编者注} 所得到的银币数仍和从前一
样。他们的劳动的货币价格仍然如旧,然而他们的工资毕竟是降低
了,因为他们拿同样数量的银币所交换到的别种商品比以前少了。
这是促成十六世纪资本增殖和资产阶级兴盛的原因之一。

我们再举一个别的例子。1847年冬,由于歉收,最必需的生活
资料(面包、肉类、黄油、千酪等等)大大涨价了。假定工人靠出卖自
己的劳动 \footnote{同上} 所得的货币量仍和以前一样。难道他们的工资没有降
低吗?当然是降低了。他们拿同样多的货币所能换到的面包、肉类
等等东西比从前少了。他们的工资降低并不是因为白银的价值减
低了,而是因为生活资料的价值增高了。

我们最后再假定,劳动的货币价格仍然未变,可是一切农产品
和工业品由于使用新机器、年成好等等原因而降低了价格。这时,
工人拿同样多的货币可以买到更多的各种商品。所以,他们的工资
正因为工资的货币价值仍然未变而提高了。

总之,劳动的货币价格即名义工资,是和实际工资即用工资
实际交换所得的商品量并不一致的。因此,我们谈到工资的增加
或降低时,不应当仅仅注意到劳动的货币价格,仅仅注意到名义工资。

但是,无论名义工资,即工人把自己卖给资本家所得到的货币
量,无论实际工资,即工人用这些货币所能买到的商品量,都不能
把工资所包含的各种对比关系完全表示出来。

此外,工资首先是由它和资本家的赢利即利润的对比关系来
决定的。这就是比较工资、相对工资。
实际工资所表示的是同其他商品的价格相比的劳动价格,反
之,相对工资所表示的则是同积累起来的劳动的价格相比的直接
劳动价格,是雇佣劳动和资本的相对价值,是资本家和工人的相互
价值\footnote{在1891年的版本中,恩格斯把自“反之,相对工资”以下的一段话改为:“而相
对工资所表示的,则是直接劳动从劳动新创造出的价值中所获得的那个同积
累起来的劳动即资本从这种价值中所取得的份额相比的份额。

上面,在第14页[见本卷第477页。—— 编者注]上,我们说过:‘工资不
是工人在他所生产的商品中占有的一份。工资是原有商品中由资本家用以购
买一定量的生产劳动力的那一部分。’但是,资本家要从卖出由工人创造的产
品所得的进款中再补偿这笔工资。资本家在补偿这笔工资时,照例要在扣除
生产费用后,还有若干剩余,即还有利润。工人所生产的商品的销售价格,对
资本家来说可分为三部分:第一,补偿他所垫支的原料价格和他所垫支的工
具、机器及其他劳动资料的损耗;第二,补偿资本家所垫支的工资;第三,这些
费用以外的余额,即资本家的利润。第一部分只是补偿原已存在的价值;很清
楚,补偿工资的那一部分和构成资本家利润的余额完全是从工人劳动所创造
出来的并附加到原料价值上去的新价值中得来的。而在这个意义上说,为了
把工资和利润加以比较,我们可以把两者都看成是工人生产的产品中的份
额。”——编者注}。

实际工资可能仍然未变,甚至可能增加了,但是相对工资却可
能降低了。假定说,一切生活资料跌价三分之二,而日工资只降低
了三分之一,比方由三法郎降低到两法郎。这时,虽然工人拿这两法郎可以买到比从前拿三法郎买到的更多的商品,但是和资本家
的利润比较起来,工人的工资终究是降低了。资本家(比如,一个工
厂主)的利润增加了一法郎,换句话说,资本家拿比以前少的交换
价值付给工人,而工人却得替资本家生产出比以前多的交换价俄。
资本的价值比劳动的价值提高了\footnote{在1891年的版本中,“资本的价值比劳动的价值提高了”改为“资本所得的份
额比劳动所得的份额提高了”。——编者注}。社会财富在资本和劳动之间
的分配更不平衡了。资本家用同样多的资本支配着更多的劳动。资
本家阶级支配工人阶级的权力增加了,工人的社会地位更坏了,比
起资本家的地位来又降低了一级。

决定工资和利润在其相互关系上的降低和增加的一般规律究
竟是怎样的呢?

工资和利润是互成反比的。资本的交换价值\footnote{在1891年的版本中, “资本的交换价值”,改为“资本的所得份额”。—— 编者
注} 即利润愈增加,
则劳动的交换价值\footnote{在1891年的版本中, “劳动的交换价值”改为“劳动的所得份额”。——编者注} 即按日工资就愈降低;反之亦然。利润增加多
少,工资就降低多少;而利润降低多少,则工资就增加多少。

也说有人会驳斥说:资本家赢利可能是由于他拿自己的产品
同其他资本家进行了有利的交换,可能是由于开辟了新的市场或
者原有市场上的需要骤然增加等等,从而对他的商品的需求量大
为增加;所以,一个资本家所得利润的增加可能是由于损害了其他
资本家的利益,而与工资即劳动 \footnote{在1891年的版本中, “劳动”改为“劳动力”。——编者注} 的交换价值的涨落无关;或者,
资本家所得利润的增加也可能是由于改进了劳动工具,采用了利
用自然力的新方法等等。

首先必须承认,所得出的结果依然是一样的,只不过这是经过
相反的途径得出的。固然,利润的增加不是由于工资的降低,但是
工资的降低却是由于利润的增加。资本家用同一数量的劳动\footnote{在1891年的版本中,“劳动”改为“别人的劳动”。——编者注} ,购
得了更多的交换价值,而对这个劳动却没有多付一文。这就是说,
劳动所得的报酬同它使资本家得到的纯收入相比却减少了。

此外,我们还应提醒,无论商品价格如何波动,每一种商品的
平均价格,即它同别种商品相交换的比率,总是由它的生产费用决
定的。因此,资本家相互间的盈亏得失必定在整个资本家阶级范围
内互相抵销。改进机器,在生产中采取利用自然力的新方法,使得
在一定的劳动时间内,用同样数量的劳动和资本可以创造出更多
的产品,但绝不是创造出更多的交换价值。如果我用纺纱机能够在
一小时内生产出比未发明这种机器以前的产量多一倍的纱,比方
从前为五十磅,现在为一百磅,那末我用这一百磅纱交换所得 \footnote{在1891年的版本中,在“所得”之后还有“平均起来在一个相当长的时间内”一句话。——编者注} 的
商品,并不比以前用五十磅交换到的多些,因为纱的生产费用减低
了一半,或者说,因为现在我用同样多的生产费用能够生产出比以
前多一倍的产品。

最后,不管资本家阶级即资产阶级(一个国家的也好,整个世
界市场的也好)相互之间分配生产所得的纯收入的比率如何,这个
纯收入的总额归根到底只是活劳动\footnote{在1891年的版本中, “活劳动”改为“直接劳动”。——编者注} 加到全部积累起来的劳动上
去的那个数额。所以,这个总额是按劳动增殖资本的比率,即按利
润比工资增加的比率增长的。

可见,即使我们单只在资本和雇佣劳动的关系这个范围内观
察问题,也可以知道资本的利益和雇佣劳动的利益是截然对立的。

资本的迅速增加就等于利润的迅速增加。而利润的迅速增加
只有在劳动的交换价值\footnote{在1891年的版本中, “劳动的交换价值”改为“劳动价格”。——编者注}同样迅速下降,相对工资同样迅速下降
的条件下才是可能的。即使在实际工资同名义工资即劳动的货币
价值同时增加的情况下,只要实际工资不是和利润同等地增加,相
对工资还是可能下降的。比如说,在经济兴旺的时期,工资提高
5%,而利润却提高30%,那末比较工资即相对工资不是增加,而
是减少了。

所以,一方面工人的收入在资本迅速增加的情况下也有所增
加:可是另一方面横在资本家和工人之间的社会鸿沟也同时扩大,
而资本支配劳动的权力,劳动对资本的依赖程度也随着增大。

所谓资本迅速增加对工人有好处的论点,实际上不过是说:工
人把他人的财富增殖得愈迅速,落到工人口里的残羹剩饭就愈多,
能够获得工作和生活下去的工人就愈多,依附资本的奴隶人数就
增加得愈多。
这样我们就看出:

即使最有利于工人阶级的情势,即使资本的尽快增加如何改
善了工人的物质生活状况,也不能消灭工人的利益和资产者即资
本家的利益之间的对立状态。利润和工资仍然是互成反比的。

假如资本增加得迅速,工资是可能提高的;可是资本家的利润
增加得更迅速无比。工人的物质生活改善了,然而这是以他们的社
会地位的降低为代价换来的。横在他们和资本家之间的社会鸿沟扩大了。

最后:

所谓生产资本的尽快增加是对雇佣劳动最有利的条件这种论
点,实际上不过是说:工人阶级愈迅速地扩大和增加敌对它的力
量,即愈迅速地扩大和增加支配它的他人财富,它就能在愈加有利
的条件下重新为资产阶级增殖财富、重新为资本加强权力而工作
—— 这样的工作无非是它本身在铸造金锁链,让资产阶级用来牵
着它走罢了。

\subsubsection{五}

科伦4月10日。然而,是不是像资产阶级的经济学家们所说的那样,生产资本的增加真的和工资的提高密不可分呢?我们不应当听信他们的话。我们甚至于不能相信他们的这种说法:似乎资本长得越肥,它的奴隶也吃得越好。资产阶级太开明了,太会打算了,它没有封建主的那种以奴仆的衣着华丽夸耀于人的偏见。资产阶级的生存条件迫使它锱铢必较。

因此我们就应当更仔细地研究一个问题:

生产资本的增长是怎样影响工资的?

如果资产阶级社会的生产资本整个说来是在不断增长,那末劳动的积累就是更多方面的了。资本的数目和资本的数额\footnote{在1891年的版本中,“资本的数目和资本的数额”改为“资本家的数目和他们的资本的数额”。——编者注}日益增加。资本的增殖加剧资本家之间的竞争。资本数额的增加,就使得有可能把装备着火力更猛烈的斗争武器的更强大的工人大军抛入产业战场。

一个资本家只有在自己更便宜地出卖商品的情况下,才能把另一个资本家逐出战场,并占有他的资本。可是,要能够贱卖而又不破产,他就必须廉价生产,就是说,必须尽量增加劳动的生产力。而增加劳动的生产力的首要办法是更细地分工,更全面地运用和经常地改进机器。内部实行分工的工人大军愈庞大,应用机器的规模愈广大,生产费用相对地就愈迅速缩减,劳动就更有效率。因此,资本家之间就发生了各方面的竞争:他们竭力设法扩大分工和增加机器,并尽可能大规模地使用机器。

可是,假如某一个资本家由于更细地分工、更多地采用新机器并改进新机器,由于更有利和更广泛地利用自然力,因而有可能用同样多的劳动或积累起来的劳动生产出比他的竞争者更多的产品(即商品),比如说,在同一劳动时间内,他的竞争者只能织出半尺麻布,他却能织出一尺麻布,那末他会怎样办呢?

他可以继续按照原来的市场价格出卖每半尺麻布,但是这样他就不能把自己的敌人逐出战场,就不能扩大自己的销路。可是随着他的生产的扩大,他对销路的需要也增加了。固然,他所采用的这些更有力更贵重的生产资料使他能够廉价出卖商品,但是这种生产资料又使他不得不出卖更多的商品,为自己的商品争夺更大得多的市场。因此,这个资本家出卖半尺麻布的价格就要比他的竞争者便宜些。

虽然这个资本家生产一尺麻布的费用并不比他的竞争者生产半尺麻布的费用多,但是他不会以他的竞争者出卖半尺麻布的价格来出卖一尺麻布。不然他就占不到任何便宜,而只是通过交换把自己的生产费用收回罢了。如果他的收入终究还是增加了,那只是因为他动用了更多的资本,而不是因为他的资本比别人的资本更多地增加了自己的价值。而且只要他把他的商品价格定得比他的竞争者低百分之几,他追求的目的也就达到了。他压低价格就能把他的竞争者挤出市场,或者至少也能夺取他的竞争者的一部分销路。最后,我们再提一下,现时价格总是高于或低于生产费用,这取决于该种商品是在产业的旺季出卖的还是在淡季出卖的。一个采用了生产效能更高的新生产资料的资本家所能得到的超出他的实际生产费用的百分率,是依每尺麻布的市场价格高于或低于迄今的一般生产费用为转移的。

可是这个资本家的特权不会长久,因为同他竞争的资本家也会采用同样的机器,实行同样的分工,并以同样的或更大的规模采用这些机器的分工。这些新措施将得到广泛的推广,直到麻布价格不仅跌到原先的生产费用以下,而且跌到新的生产费用以下为止。

这样,资本家的相互关系又会像采用新生产资料以前那样了;如果说他们由于采用这种生产资料曾经能够用以前的价格供给加倍的产品,那末现在他们已不得不按低于以前的价格出卖加倍的产品了。在这种新生产费用的水平上,同样一场钩心斗角的斗争又重新开始。又有人实行更细的分工,又有人增加机器数量,利用这种分工的范围和采用这些机器的规模日益扩大。而竞争又对这个结果发生反作用。

由此可见,生产方式和生产资料总在不断变更,不断革命化;分工必然要引起更进一步的分工;机器的采用必然要引起机器的更广泛的采用;大规模的生产必然要引起更大规模的生产。

这是一个规律,这个规律一次又一次地把资产阶级的生产甩出原先的轨道,并迫使资本加强劳动的生产力,因为它以前就加强过劳动的生产力;这个规律不让资本有片刻的停息,老是在它耳边催促说:前进!前进!

这个规律正就是那个在商业的周期性波动中必然使商品价格和商品生产费用趋于一致的规律。

不管一个资本家运用了效率多么高的生产资料,竞争总使这种生产资料的采用成为普遍的现象,而当这种生产资料的采用一旦成为普遍的现象时,他的资本具有更大效率的唯一后果就只能是:要取得原有的价格,他就必须供给比以前多十倍、二十倍、一百倍的商品。可是,因为现在他必须售出也许比以前多一千倍的商品,才能靠增加所售产品数量的办法来弥补由于售价降低所受的损失;因为他现在必须卖出更多的商品不仅是为了得到利润\footnote{在1891年的版本中,“得到利润”改为“得到更多的利润”。——编者注},并且也是为了抵补生产费用(我们已经说过,生产工具本身也日益昂贵);因为此时这种大量出卖不仅对于他而且对于他的竞争对方都成了生死问题,所以先前的斗争就因已经发明的生产资料的生产效率愈大而愈残酷无情地激烈起来。所以,分工和机器的采用又将以更大得无比的规模发展起来。

不管已被采用的生产资料的力量多么强大,竞争总是要把资本从这种强大力量中得到的黄金果实夺去,使商品的价格降低到生产费用的水平;也就是说,只要有可能更廉价的生产,即有可能用同一数量的劳动生产更多的产品,竞争就使廉价生产即按原先价格供给日益增多的产品数量成为确定不移的规律。可见,资本家努力的结果,除了必须在同一劳动时间内生产出更多的商品以外,换句话说,除了使他的资本的价值增殖的条件恶化以外,并没有得到任何好处。因此,虽然竞争经常以其生产费用的规律迫使资本家坐卧不宁,把他制造出来对付竞争者的一切武器倒转来针对着他自己,但资本家总是想方设法在竞争中取胜,孜孜不倦地采用价钱较贵但能进行廉价生产的新机器,实行新分工,以代替旧机器和旧分工,并且不等到竞争使这些新措施过时,就这样做了。

现在我们若是想像一下这种狂热的激发状态同时笼罩了整个世界市场,那我们就会明白,资本增殖、积累和集聚的结果,如何导向了不断地、日新月异地、更大规模地实行分工,采用新机器,改进旧机器。

这些同生产资本的增殖分不开的情况又怎样影响工资的确定呢?

更进一步的分工使一个工人能做五个、十个乃至二十个人的工作,因而就使工人之间的竞争加剧五倍、十倍乃至二十倍。工人中间的竞争不只表现于一个工人把自己出卖得比另一个工人贱些,而且还表现于一个工人做五个、十个乃至二十个人的工作。而资本所实行的和经常扩展的分工就迫使工人进行这种竞争。

其次,分工愈细,劳动就愈简单化。工人的特殊技巧失去任何价值。工人变成了一种简单的、单调的生产力,就不需要体力上或智力上的特别本事和技能了。他的劳动成为人人都能从事的劳动了。因此,工人受到四面八方的排挤;我们还要提醒一下,一种工作愈简单,就愈容易学会,为学会这种工作所需要的生产费用愈少,工资也就愈降低,因为工资像一切商品的价格一样,是由生产费用决定的。

总之,劳动愈是不能给人以乐趣,愈是令人生厌,竞争也就愈激烈,工资也就愈减少。工人想维持自己的工资总额,就得多劳动:多工作几小时或者在一小时内造出更多的产品。这样一来,工人为贫困所迫,就愈加重分工的极危险的后果。结果就是:他工作得愈多,他所得的工资就愈少。这里的原因很简单:他工作得愈多,他给自己的工友们造成的竞争就愈激烈,因而就使自己的工友们变成他自己的竞争者,这些竞争者也像他一样按同样恶劣的条件出卖自己。所以,原因同样很简单:他归根到底是自己给自己,即自己给作为工人阶级一员的自己造成竞争。

机器也发生同样的影响,而且影响的规模更大得多,因为机器用不熟练的工人代替熟练工人,用女工代替男工,用童工代替成年工;因为在最先使用机器的地方,机器就把大批手工工人抛到街头上去,而在机器日益完善、改进或为生产效率更高的机器所替换的地方,机器又把一批一批的工人排挤出去。我们在前面大略地描述了资本家相互间的产业战争。这种战争有一个特点,就是致胜的办法与其说是增加劳动大军,不如说是减少劳动大军。统帅们即资本家们相互竞赛,看谁能解雇更多的产业士兵。

不错,经济学家们告诉我们说,似乎因采用机器而成为多余的工人可以在新的工业部门里找到工作。

他们不敢干脆地肯定说,在新的劳动部门中找到栖身之所的就是那些被解雇的工人。事实最无情地粉碎了这种谎言。其实,他们不过是肯定说,在工人阶级的其他组成部分面前,譬如说,在一部分已准备进入那种衰亡的产业部门的青年工人面前,出现了新的就业门路。这对于不幸的工人当然是一个很大的安慰。资本家老爷们是不会缺少可供剥削的新鲜血肉的,于是他们就让死人们去埋葬自己的尸体。这种安慰,与其说是对工人的安慰,不如说是对资本家本身的安慰。要知道,假若机器消灭了整个雇佣工人阶级,那末资本的最可怕的时刻就会到来,因为资本没有雇佣劳动就不再成为资本了!

就假定那些直接被机器从一个产业部门排挤出去的工人以及原已指望受雇于该产业部门的那一部分青年工人都能找到新工作。是否可以相信新工作的报酬会和已失去的工作的报酬同样高呢?要是这样,那就是违反了一切经济规律。我们说过,现代产业经常是用简单的和低级的工作来代替较复杂和较高级的工作的。

既是这样,被机器从一个产业部门排挤出去的一大批工人若不甘愿领取更低更坏的报酬,又怎能在别的部门找到栖身之所呢?

有人说制造机器本身的工人是一种例外。他们说,既然产业需要并使用更多的机器,机器的数量就必然增加,因而机器的生产也必然增加,而在这个生产部门中工作的工人人数也必然随之增加;况且这个产业部门的工人是熟练工人,而且还是受过教育的工人。

从1840年起,这种原先也只有一半正确的论点已经毫无正确的影子了,因为机器生产部门也完全和棉纱生产部门一样,日益多方面地采用机器,而机器生产部门的工人,比起极完善的机器来,只能起着极不完善的机器的作用。

可是,在一个男工被机器排挤出去以后,工厂方面也许会雇佣三个童工和一个女工!难道先前一个男工的工资不是应该足够养活三个孩子和一个妻子吗?难道先前最低工资不是应该足够维持工人生活和繁殖工人后代吗?资产阶级爱说的这些话在这里究竟证明了什么呢?只证明了一点:现在要得到维持一个工人家庭生活的工资,就得消耗比以前多三倍的工人生命。

总括起来说:生产资本愈增加,分工和采用机器的范围就愈扩大。分工和采用机器的范围愈扩大,工人之间的竞争就愈剧烈,他们的工资就愈减少。

加之,工人阶级还从较高的社会阶层中得到补充;降落到无产阶级队伍里来的有大批小产业家和小食利者,他们除了赶快跟工人一起伸手乞求工作,毫无别的办法。这样,伸出来乞求工作的手像森林似地愈来愈稠密,而这些手本身则愈来愈消瘦。

不言而喻,小产业家是支持不住这种战争\footnote{在1891年的版本中,“战争”改为“斗争”。——编者注}的:这种战争的首要条件之一就是生产的规模经常扩大,也就是说必须要做大产业家而绝不能做一个小产业家。

当然,还有一点也是用不着进一步说明的:资本愈增殖,资本的总量和数目愈增加,资本的利息也就愈减少;因此,小食利者就不可能再依靠利息来维持生活,必须投到产业方面去,即补充小产业家的队伍,从而增加无产者的候补人数。

最后,上述发展进程愈迫使资本家以日益扩大的规模使用既有的巨大的生产资料,并为此而动用一切信贷机构,而“地震”\footnote{在1891年的版本中,“地震”改为“产业方面的地震”。——编者注}也来得愈来愈频繁,在每次地震中,商业界只是由于埋葬一部分财富、产品以至生产力才维持下去,——也就是说,危机来得愈益剧烈了。这种危机之所以来得愈频繁和愈剧烈,就是因为随着产品总量的增加,亦即随着对扩大市场的需要的增长,世界市场变得愈加狭窄了,剩下可供榨取的市场\footnote{在1891年的版本中,“市场”改为“新市场”。——编者注}愈益减少了,因为先前发生的每一次危机都把一些新市场或以前只被微微榨取过的市场卷入了世界贸易。但是,资本不光靠剥削劳动来生活。像显贵的野蛮的奴隶主一样,资本也要他的奴隶们陪葬,即在危机时期要使大批的工人死亡。由此可见:如果说资本增长得迅速,那末工人之间的竞争就增长得更迅速无比,就是说,资本增长得愈迅速,工人阶级的就业手段即生活资料就相对地缩减得愈厉害;虽然如此,资本的迅速增长对雇佣劳动却是最有利的条件。

\newpage

\subsection{《政治经济学批判》导言}
\begin{center}
 I生产,消费,分配,交换(流通)    
\end{center}
\subsubsection{1~生产}

\textbf{(a)面前的对象,首先是物质生产。}

在社会中进行生产的个人,因而,这些个人的一定社会性质的生产,自然是出发点。被斯密和李嘉图当作出发点的单个的孤立的猎人和渔夫,应归入18世纪鲁宾逊故事的毫无想象力的虚构,鲁宾逊故事决不像文化史家设想的那样,仅仅是对极度文明的反动和想要回到被误解了的自然生活中去。同样,卢梭的通过契约来建立天生独立的主体之间的相互关系和联系的社会契约论,也不是奠定在这种自然主义的基础上的,这是错觉,只是美学上大大小小的鲁宾逊故事的错觉。这倒是对于16世纪以来就进行准备,而在18世纪大踏步走向成熟的”市民社会”的预感。在这个自由竞争的社会里,单个的人表现为了摆脱了自然联系等等,后者在过去历史时代使他成为一定的狭隘人群的附属物。这种18世纪的个人,一方面是封建社会形式解体的产物,另一方面是16世纪以来新兴生产力的产物,而在18世纪的预言家看来(斯密和李嘉图还完全以这些预言家为依据),这种个人是一种理想,他的存在是过去的事;在他们看来,这种个人不是历史的结果,而是历史的起点。因为,按照他们关于人类天性的看法,合乎自然的个人并不是从历史中产生的,而是由自然造成的。这样的错觉是到现在为止的每个新时代所具有的。斯图亚特在许多方面同18世纪对立并做为贵族比较多地站在历史上,从而避免了这种局限性。

我们愈往前追溯历史,个人,也就是进行生产的个人,就显得愈不独立,愈从属于一个更大的整体:最初还是十分自然地在家庭和扩大成为氏族的家庭中;后来是在由氏族间的冲突和融合而产生的各种形式的公社中。只有到十八世纪,在”市民社会”中,社会结合的各种形式,对个人说来,才只是达到他私人目的手段,才是外在的必然性。但是,产生这种孤立的个人的观点的时代,正是具有迄今为止最发达的社会关系(从这种观点来看是一般关系)的时代。人是最名符其实的社会动物,不仅是一种合群的动物,而且是只有在社会中才能独立的动物。孤立的一个人在社会之外进行生产-这是罕见的事,’偶然落到荒野中的已经内在地具有社会力量的文明人或许能做到-就像许多个人不再一起生活和彼此交谈而竟有语言发展一样,是不可思议的。在这方面无须多说。十八世纪的人们有这种荒诞无稽的看法本是可以理解的,如果不是巴师夏,凯里和蒲鲁东等人又把这种看法郑重其事地引进最新的经济学中来,这一点本来可以完全不提。蒲鲁东等人自然乐于用编造神话的办法,来对一种他不知道历史来源的经济关系做历史哲学的说明,说什么这种观念对亚当及普罗米修斯已经是现成的,后来他就被付诸实行等等。再没有比这类想入非非的陈腔滥调更加乏味的了。

因此,说到生产,总是指在一定社会发展阶段上的生产-社会个人的生产。因而,好象只要一说到生产,我们或者就要把历史发展过程在它的各个阶段上一一加以研究,或者一开始就要声明,我们只的是某个一定的历史时代,例如,是现代资产阶级生产-这种生产事实上是我们研究的本题。可是,生产的一切时代有某些共同标志,共同规定。生产一般是一个抽象,但是只要它真正把共同点提出来,定下来,免得我们重复,它就是一个合理的抽象。不过,这个一般,或者说,经过比较而抽出来的共同点,本身就是有另一些是几个时代共有的,[有些]规定是最新时代和最古时代共有的,没有它们,任何生产都无从设想;如果说最发达语言的有些规律和规定也是最不发达语言所有的,但是构成语言发展的恰恰是有别于这一般和共同点的差别,那末,对生产一般适用的种种规定所以要抽出来,也正是为了不致因见到统一(主体是人,客体是自然,这总是一样的,这里已经出现了统一)就忘记了本质的差别。而忘记这种差别,正是那些证明现存社会关系永存与和谐的现代经济学家的全部智慧所在。例如,他们说,没有生产工具,哪怕这种生产工具不过是手,任何生产都不可能。没有过去的,累积下来的劳动,哪怕这种劳动不过是由于反复操作而累聚在野蛮人手上的技巧,任何生产都不可能。资本,别的不说,也是生产工具,也是过去的,客体化了的劳动。可见资本是一种一般的,永存的自然关系;这就是说,如果我们恰好抛开了正是使”生产工具”,”累积下来的劳动”成为资本的那个特殊的话。因此,生产关系的全部历史,例如在凯里看来,是历代政府的恶意篡改。

如果没有生产一般,也就没有一般的生产。生产总是一个特殊的生产部门-如农业,畜牧业,制造业等,或者是他们的总体。可是,政治经济学不是工艺学。生产的一般规定在一定社会阶段上对特殊生产形式的关系,留待别处(后面)再说。

最后,生产也不只是特殊的生产,而始终是一定的社会体及社会的主体在或广或窄由各生产部门组成的总体中活动着。科学的叙述对现实运动的关系,也还不是这里所要说的。生产一般。特殊生产部门。生产的总体。

现在时髦的做法,是在经济学的开头摆上一个总论部份-就是标题为《生产》的那部份(参看约翰,斯图亚特,穆勒的著作),用来论述一切生产的一般条件。

这个总论部份包括或者好像应当包括:

(1)进行生产所必不可缺少的条件。因此,这实际上不过是要说明一切生产的基本要素。可是,我们将会知道,实际上归纳起来不过是几个十分简单的规定,却扩展成浅薄的同义反复。

(2)或多或少促进生产的条件,如像亚当。斯密所说的前进的和停滞的社会状态。要把这些在斯密那里作为提示而具有价值的东西提升到科学意义上来,就得研究各个民族的发展过程终生产率程度不同的各个时期-这种研究超出本题应有的范围,但就属于本题范围来说,在叙述竞争,累积等等时是要谈到的。照一般的提法,答案总是这样一个一般的说法:一个工业民族,当它一般地达到它的历史高峰的时候,也就达到它的生产高峰。实际上,一个民族的工业高峰是在它还不是以既得利益为要务,而是以争取利益为要务的时候。在这一点上,美国人胜过英国人。或者是这样的说法:例如,某一些种族,素质,气候,自然条件如离海远近,土地肥沃程度等等,比另外一些更有利于生产。这又是同义反复,即财富的主客观因素越是在更高的程度上具备,财富就越容易创造。

但是,经济学家在这个总论部份所真正要谈的并不是这一切。相反,照他们的意见,生产不同于分配等等(参看穆勒的著作),应当被描写成局限在脱离历史而独立的永恒自然规律之内的事情,于是资产阶级关系就被乘机当作社会一般的颠扑不破的自然规律偷偷地塞了进来。这是整套手法的多少有意识的目的。反之,在分配上,好象人们事实上可以随心所欲。即使根本不谈生产和分配的这种粗暴割裂与生产与分配的现实关系,下面这一点总应当是一开始就明白的:无论在不同社会阶段上分配如何不同,总是可以像在生产中那样提出一些共同的规定来,可以把一切历史差别混合和融化在一般人类规律之中。例如,奴隶,农奴,雇佣工人都得到一定量的食物,使他们能够作为奴隶,农奴和雇佣工人来生存。靠贡赋生活的征服者,靠税收生活的官吏,靠地租生活的土地占有者,靠施舍生活的僧侣,或者靠什一税生活的教士,都得到一份社会产品,而决定这一份产品的规律不同于决定奴隶等等那一份产品的规律。一切经济学家在这个项目下提出的两个要点是:(1)所有制,(2)司法,警察等对所有制的保护,对此要极简单地答复一下:

关于第一点,一切生产都是个人在一定社会形式中并藉这种社会形式而进行的对自然的占有。在这个意义上,说所有制(占有)是生产的一个条件,那是同义反复。但是,可笑的是从这里一步就跳到所有制的一定形式,如私有制。(而且还把对立的形式即无所有作为条件。)历史却表明,公有制是原始形式(如印度人,斯拉夫人,古克尔特人等等),这种形式在公社所有制形式下还长期起着显着的作用。至于财富在这种还是那种所有制形式下能更好地发展的问题,还根本不是这里所要谈的。可是,如果说在任何所有制都不存在的地方,就谈不到任何生产,因此也就谈不到任何社会,那末,这是同义反复。什么也不据为己有的占有,是自相矛盾。

关于第二点,对既得物的保护等等。如果把这些滥调还原为它们的实际内容,它们所表示的就比它们的说教者所知道的还多。就是说,每种生产形式都产生出它所特有的法权关系,统治形式等等。粗率和无知之处正在于把有机地联系着的东西看成是彼此偶然发生关系的,纯粹反射联系中的东西,资产阶级经济学家只模糊地感到,在现代警察制度下,比在例如强权下能更好地进行生产,他们只是忘记了,强权也是一种法权,而且强者的法权也以另一种形式继续存在于他们的”法治国家”中。

当与生产的一定阶段相应的社会状态刚刚产生或者已经衰亡的时候,自然会出现生产上的紊乱,虽然程度和影响有所不同。

总之:一切生产阶段所共同的,被思维当作一般规定而确定下来的规定,是存在的,但是所谓一切生产的一般条件,不过是这些抽象要素,用这些抽象要素不可能理解任何一个现实的历史的生产阶段。

\subsubsection{2~生产与分配,交换,消费的一般关系}

在进一步分析生产之前,必须观察一下经济学家拿来与生产并列的几个项目。

敷浅的表象是:在生产中,社会成员占有(开发,改造)自然产品供人类需要;分配决定个人分取这些产品的比例;交换给个人带来它享用分配给他的一份去换取的那些特殊产品;最后,在消费中,产品变成享受的对象,个人占有的对象。生产创造出适合需要的对象;分配依照社会规律把它们分配;交换依照个人需要把已经分配的东西再分配;最后,在消费中,产品脱离这种社会运动,直接变成个人需要的对象和仆役,被享受而满足个人需要。因而,生产表现为起点,消费表现为终点,分配和交换表现为中间环节,这中间环节又是二重的,因为分配被规定为从社会出发的要素,交换被规定为从个人出发的要素。在生产中,人客体化,在人中,物主体化;在分配中,社会以一般的,居于支配地位的规定的形式,担任生产和消费之间的媒介;在交换中,生产和消费由偶然的个人的规定性来媒介。

分配决定产品归个人的比例(分量);交换决定个人对于分配给自己的一份所要求的产品。

生产,分配,交换,消费因此形成一个正归的三段论法;生产是一般,分配和交换是特殊,消费是个别,全体由此结合在一起。这当然是一种联系,然而是一种敷浅的联系。生产决定于一般的自然规律,分配决定于社会的偶然情况,因此它能够或多或少地对生产起促进作用;交换作为形式上的社会运动介于两者之间;而消费这个不仅被看成终点而且被看成最后目地的结束行为,除了它又反过来作用于起点并重新引起整个过程之外,本来不属于经济学的范围。

反对政治经济学家的人们,-不论这些反对者是不是他们的同行,-责备他们把联系着的东西粗野地割裂了,这些反对者或者是同他们站在同一个基础上,或者是在他们之下。最庸俗不过的责备就是,说政治经计学家过于重视生产,把它当作目的本身。说分配也是同样重要的。这种责备的立足点恰恰是那种把分配当作与生产并列的独立自主的领域的经济见解。或者是这样的责备,说媒有把这些要素放在其统一中来理解。好象这种割裂不是从现实中进到教科书中去的,而相反地是从教科书进到现实中去的,好像这里的问题是要把概念作辩证的平衡,而不是解释现实的关系!

\textbf{(a)[生产和消费]}

生产直接也是消费。双重的消费,主体的和客体的:个人在生产当中发展自己的能力,也在生产行为中支出和消耗这种能力,同自然的生殖是生命力的一种消耗完全一样。第二,生产资料的消费,生产资料被使用,被消耗,一部分(如在燃烧中)重新分解为一般元素。原料的消费也是这样,原料不再保持自己的自然形状和特性,这种自然形状和特性倒是消耗掉了。因此,生产行为本身就它的一切要素来说也是消费行为。不过,这一点是经济学家所承认的,他们把直接与消费同一的生产,直接与生产合一的消费,称作生产的消费。生产和消费的这种同一性,归结起来是斯宾诺莎的命题:”规定即否定”。但是,提出生产的消费这个规定,只是为了把与生产同一的消费跟原来意义上的消费区别开来,后面这种消费被理解为起消灭作用的与生产相对的对立面,我们且观察一下这个原来意义上的消费。

消费直接也是生产,正如自然界中的元素和化学物质的消费是植物的生产一样。例如,吃喝是消费形式之一,人吃喝就生产自己的身体,这是明显的事。而对于以这种或那种形式从某一方面来生产人的其它任何消费形式也都可以这样说。消费的生产。可是,经济学却说,这种与消费同一的生产是第二种生产,是靠消灭第一种生产的产品引起的。在第一种生产中,生产者物化,在第二种生产中,生产者所创造的物人化。因此,这种消费的生产,-虽然它是生产和消费的直接统一-是与原来意义上的生产根本不同的。生产同消费合而为一和消费同生产合而为一的这种直接统一,并不排斥它们的直接两立。

可见,生产直接是消费,消费直接是生产。每一方直接是它的对方。可是同时在两者之间存在着一种媒介运动。生产媒介着消费,它创造出消费的材料,没有生产,消费就没有对象。但是消费也媒介着生产,因为正式消费替产品创造了主体,产品对这个主体才是产品。产品在消费中才得到最后完成。一条铁路,如果没有通车,不被磨损,不被消费,它只是可能性的铁路,不是现实的铁路。没有生产,就没有消费,但是,没有消费,也就没有生产,因为如果这样,生产就没有目的。消费从两方面生产着生产。

(1)因为只是在消费中产品才成为现实的产品,例如,一件衣服由于穿的行为才现实地成为衣服;一间房屋无人居住,事实上就不成为现实的房屋;因此,产品不同于单纯的自然对象,它在消费中才证实自己是产品,才成为产品。消费是在把产品消灭的时候才使产品最后完成,因为产品之所以是产品,不是它做为物化了的活动,而只是做为活动着的主体的对象。

(2)因为消费创造出新的生产的需要,因而创造出生产的观念上的内在动机,后者是生产的前提。消费创造出生产的动力;它也创造出在生产中做为决定目的的东西而发生作用的对象。如果说,生产在外部提供消费的对象是显而易见的,那末,同样显而易见的是,消费在观念上提出生产的对象,做为内心的意象,作为需要,做为动力和目的。消费创造出还是在主观形式上的生产对象。没有需要,就没有生产。而消费则把需要再生产出来。

与此相应,就生产方面来说:

(1)它为消费提供材料,对象。消费而无对象,不成其为消费;因而,生产在这方面创造出,生产出消费。

(2)但是,生产为消费创造的不只是对象。它也给予消费以消费的规定性,消费的性质,使消费得以完成。正如消费使产品得以完成其为产品一样,生产使消费得以完成。首先,对象不是一般的对象,而是一定的对象,是必须用一定的而又是由生产本身所媒介的方式来消费的。饥饿总是饥饿,但是用刀叉吃熟肉来解除的饥饿不同于用手,指甲和牙齿啃生肉来解除的饥饿。因此,不仅消费的对象,而且消费的方式,不仅客体方面,而且主体方面,都是生产所生产的。所以,生产创造消费者。

(3)生产不仅为需要提供材料,而且它也为材料提供需要。在消费脱离了它最初的自然粗陋状态和直接状态之后,-如果停留在这种状态,那也是生产停滞在自然粗陋状态的结果,-消费本身做为动力是靠对象做媒介的。消费对于对象所感到的需要,是对于对象的知觉所创造的。艺术对象创造出懂得艺术和能够欣赏美的大众,-任何其它产品也都是这样。因此,生产不仅做为主体生产对象,而且也为对象生产主体。

因此,生产生产着消费:(1)是由于生产为消费创造材料,(2)是由于生产决定消费的方式,(3)是由于生产靠它起初当作对象生产出来的产品在在消费者身上引起需要。因而,它生产出消费的对象,消费的方式和消费的动力。同样,消费生产出生产者的素质,因为它在生产者身上引起追求一定目的的需要。

因此,消费和生产之间的同一性表现在三方面:

(1)直接的同一性:生产是消费;消费是生产。消费的生产。生产的消费。政治经济学家把两者都称为生产的消费,可是还做了一个区别。前者表现为再生产,后者表现为生产的消费。关于前者的一切研究是关于生产的劳动或非生产的劳动的研究;关于后者的研究是关于生产的消费或非生产的消费的研究。

(2)每一方表现为对方的手段;以对方为媒介;这表现为他们的相互依存;这是一个运动,它们通过这个运动彼此发生关系,表现为互不可缺,但又各自处于对方之外。生产为消费创造作为外在对象的材料;消费为生产创造作为内在对象,作为目的的需要。没有生产就没有消费;没有消费就没有生产。这在经济学中以多种多样的形式表现出来。

(3)生产不仅直接是消费,消费也不仅直接是生产;而且生产不仅是消费的手段,消费不仅是生产的目的,-就是说,每一方都为对方提供对象,生产为消费提供外在的对象,消费为生产提供想象的对象;两者的每一方不仅直接就是对方,不仅媒介着对方,而且,两者的每一方当自己实现时也就创造对方,把自己当作对方创造出来。消费完成生产行为,只是在消费使产品最后完成其为产品的时候,在消费把它消灭,把它的独立的物体形式毁掉的时候;在消费使得在最初生产行为中发展起来的素质通过反复的需要达到完美的程度的时候;所以,消费不仅是使产品成为产品的最后行为,而且也是使生产者成为生产者的最后行为。另一方面,生产生产出消费,是在生产创造出消费的一定方式的时候,然后是在生产把消费的动力,消费能力本身当作需要创造出来的时候。这和第三项所说的这个最后的同一性,经济学在论述需求和供给,对象和需要,社会创造的需要和自然需要的关系时,曾多次加以解释。

这样看来,对于一个黑格尔主义者来说,把生产和消费同一起来,是最简单不过的事。不仅社会主义美文学家这样做过,而且平庸的经济学家也这样做过,萨伊就是个例子;他的说法是,就一个民族来说,它的生产也就是它的消费。或者,就人类一般来说,也是这样。施托尔希指出过萨伊的错误,因为例如一个民族,不是把自己的产品全部消费掉,而是还要创造生产资料等等,固定资本等等。此外,把社会当作一个单独的主体来观察,是对它做了不正确的观察,思辨式的观察。就一个主体来说,生产和消费表现为一个行为的两个要素。这里要强调的主要之点是:如果我们把生产和消费看做一个主体的或者许多单个个人的活动,它们无论如何表现为一个过程的两个要素,在这个过程中,生产是实际的起点,因而也是居于支配地位的要素。消费,做为必需,做为需要,本身就是生产活动的一个内在要素。但是生产活动是实现起点,因而也是实现的居于支配地位的要素,是整个过程借以从新进行的行为。个人生产出一个对象,因消费了它而再回到自己身上,然而,他是作为生产的个人,把自己再生产的个人。所以,消费表现为生产的要素。

但是,在社会中,产品一经完成,生产者对产品的关系就是一种外在的关系,产品回到主体,取决于主体对其它个人的关系。他不是直接获得产品。如果说他是在社会中生产,那末直接占有产品也不是他的目的。在产品和生产者之间插进了分配,分配借社会规律决定生产者在产品世界中的份额,因而插在生产和消费之间。

那末,分配是否作唯独立的领域,处于生产之旁和生产之外呢?

\textbf{(b)[生产和分配]}

如果看看普通的经济学著作,首先令人注目的是,在这些著作里什么都被提出两次。举例来说,在分配上出现的是地租,工资,利息和利润,而在生产上做为生产要素出现的是土地,劳动,资本。说到资本,一看就清楚,它被提出了两次:(1)当作生产要素;(2)当作收入源泉,当作决定一定的分配形式的东西。利息和利润,就它们做为资本增殖和扩大的形式,因而做为资本自身的生产的要素来说,本身也出现在生产中。利息和利润作为分配形式,是以资本作为生产要素为前提的。他们是以资本作为生产要素为前提的分配方式。它们又是资本的再生产方式。

同样,工资也是在另一个项目中被考察的雇佣劳动:在一处作为生产要素的劳动所具有的规定性,在另一处表现为分配的规定。如果劳动不是规定为雇佣劳动,那末,它参与产品分配的方式,也就不表现为工资,如在奴隶制度下就是这样。最后,地租-我们直接地来看地产参与产品分配的最发达形式-的前提,是作为生产要素的大地产(其实是大农业),而不是通常的土地,就像工资的前提不是通常的劳动一样。所以,分配关系和分配方式只是表现为生产要素的背面。个人以雇佣劳动的形式参与生产,就以工资形式参与产品,生产成果的分配。分配的结构完全取决于生产的结构,分配本身就是生产的产物,不仅就对象说是如此,而且就形式说也是如此。就对象说,能分配的只是生产的成果,就形式说,参与生产的一定形式决定分配的特定形式,决定参与分配的形式。把土地放在生产上来谈,把地租放在分配上来谈,等等,简直是幻觉。

因此,像李嘉图那样的经济学家,最受责备的就是他们眼中只有生产,他们却专门把分配规定为经济学的对象,因为他们本能地把分配形式看成是一定社会中的生产要素得以确定的最确切的表现。

在单个的个人面前,分配自然表现为一种社会规律,这种规律决定他在生产中-指他在其中进行生产的那个生产-的地位,因而分配先于生产。这个个人一开始就没有资本,也没有地产。他一出生就由社会分配指定专门从事雇佣劳动。但是这种指定本身是资本和地产作为独立的生产要素存在的结果。

就整个社会来看,从一方面说,分配似乎先于生产,并且决定生产,似忽是先经济的事实。一个征服者民族在征服者之间分配土地,因而造成了地产的一定的分配和形式,由此决定了生产。或者,它使被征服的民族成为奴隶,于是使奴隶劳动成为生产的基础。或者,一个民族经过革命把大地产粉碎成小块,从而通过这种新的分配使生产有了一种新的性质。或者,立法使地产永远属于一定的家庭,或者,把劳动[当作]世袭的特权来分配,因而把它像等级一样地固定下来。在所有这些历史上有过的情况下,似乎不是生产安排和决定分配,而相反地是分配安排和决定生产。

照最浅薄的理解,分配表现为产品的分配,因此它彷佛离开生产很远,对生产是独立的。但是,在分配是产品的分配之前,它是(1)生产工具的分配,(2)社会成员在各类生产之间的分配(个人从属于一定的生产关系)-这是上述同一关系的进一步规定。这种分配包含在生产过程本身中并且决定生产的结构,产品的分配显然只是这种分配的结果。如果在考察生产时把包含在其中的这种分配撇开,生产显然只是一个空洞的抽象;反过来说,有了这种本来构成生产的一个要素的分配,力求在一定的社会结构中来理解现代生产并且主要是研究生产的经济学家李嘉图,不是把生产而是把分配说成现代经济学的本题。从这里,又一次显出了那些把生产当作永恒真理来论述而把历史限制在分配范围之内的经济学家是多么荒诞无稽。

这种决定生产本身的分配究竟和生产处于怎么样的关系,这显然是属于生产本身内部的问题。如果有人说,既然生产必须从生产工具的一定分配出发,至少在这个意义上分配先于生产,成为生产的前提,那末就应该答复他说,生产实际上有它的条件和前提,这些条件和前题构成生产的要素。这些要素最初可能表现为自然发生的东西。通过生产过程本身,它们就从自然发生的东西变成历史的东西了,如果它们对于一个时期表现为生产的自然前提,对于另一个时期就是生产的历史结果了。它们在生产内部不断地改变。例如,机器的应用既改变了生产工具的分配,也改变了产品的分配。现代大土地所有制本身既是现代商业和现代工业的结果,也是现代工业在农业上应用的结果。

上面提出的一些问题,归根到底就是:一般历史条件在生产上是怎样起作用的,生产和一般历史运动的关系又是怎样的。这个问题显然属于对生产本身的讨论和分析。

然而,这些问题即使照上面那样平庸的提法,也可以同样给予简短的回答。所有的征服有三种可能。征服民族把自己的生产方式强加于被征服的民族(例如,本世纪英国人在爱尔兰所做的,部份地在印度所做的);或者是征服民族让旧生产方式维持下去,自己满足于征收贡赋(如土耳其人和罗马人);或者是发生一种相互作用,产生一种新的,综合的生产方式(日耳曼人的征服中一部分就是这样)。在所有的情况下,生产方式,不论是征服民族的,被征服民族的,还是两者混合形成的,总是决定新出现的分配。因此,虽然这种分配对于新的生产时期表现为前提,但它本身又是生产的产物,不仅是一般历史生产的产物,而且是一定历史生产的产物。

例如,蒙古人把俄罗斯弄成一片荒凉,这样做是适合于他们的生产,畜牧的,大片无人居住的地带是畜牧的主要条件。在日耳曼蛮族,用农奴耕作是传统的生产,过的是乡村的孤独生活,他们能够非常容易地让罗马各省服从于这些条件,因为那里发生的土地所有权的集中已经完全推翻了旧的农业关系。

有一种传统的观念,认为在某些时期人们只靠劫掠生活。但是要能够劫掠,就要有可以劫掠的东西,因此就要有生产。而劫掠方式本身又决定生产方式。例如,劫掠一个从事证券投机的民族就不能同劫掠一个游牧民族一样。

奴隶直接被剥夺了生产工具。但是奴隶受到剥夺的国家的生产必须安排得容许奴隶劳动,或者必须建立一种适于使用奴隶的生产方式(如在南美等)。

法律可以使一种生产资料,例如土地,永远属于一定家庭。这些法律,只有当大土地所有权适合于社会生产的时候,如像在英国那样,才有经济意义。在法国,尽管有大土地所有权,但经营的是小土地农业,因而大土地所有权就被革命摧毁了。但是,土地析分的状态是否例如通过法律永远固定下来了呢?尽管有这种法律,土地的所有权却又集中起来了。法律在巩固分配关系方面的影响和它们由此对生产发生的作用,要专门加以确定。

\textbf{(c)最后,交换和流通}

流通本身只是交换的一定要素,或者也是从总体上看的交换。

既然交换只是生产以及由生产决定的分配一方和消费一方之间的媒介要素,而消费本身又表现为生产的一个要素,交换当然也就当做生产的要素包含在生产之内。

首先很明显,在生产本身之中发生的各种活动和各种能力的交换,直接属于生产,并且从本质上组成生产。第二,这同样适用于产品交换,只要产品交换是用来制造供直接消费的成品的手段。在这个限度内,交换本身是包含在生产之中的行为。第三,所谓企业家之间的交换,从它的组织方面看,既完全决定于生产,且本身也是生产行为。只有在最后阶段上,当产品直接为了消费而交换的时候,交换才表现为独立于生产之外,与生产漠不相干。但是,(1)如果没有分工,不论这种分工是自然发生的或者本身已经是历史的成果,也就没有交换;(2)私的交换以私的生产为前提;(3)交换的深度,广度和方式都是由生产的发展和结构决定的。例如,城乡之间的交换,乡村中的交换,城市中的交换等等。可见,交换就其一切要素来说,或者是直接包含在生产当中,或者是由生产决定。

我们得到的结论并不是说,生产,分配,交换,消费是同一的东西,而是说,它们构成一个总体的各个环节,一个统一体内部的差别。生产既支配着生产的对立规定上的自身,也支配着其它要素。过程总是从生产重新开始。交换和消费是不能支配作用的东西,那是自明之理。分配,作为产品的分配,也是这样。而作为生产要素的分配,它本身就是生产的一个要素。因此,一定的生产决定一定的消费,分配,交换和这些不同要素相互间的一定关系。当然,生产就其片面形式来说也决定于其它要素。例如,当市场扩大,即交换范围扩大时,生产的规模也就增大,生产也就分得更细。随着分配的变动,例如,随着资本的集中,随着城乡人口的不同的分配等等,生产也就发生变动。最后,消费的需要决定着生产。不同要素之间存在着相互作用。每一个有机整体都是这样。

\subsubsection{3~政治经济学的方法}

当我们从政治经济学方面观察某一国家的时候,我们从该国的人口,人口的阶级划分,人口在城乡海洋的分布,在不同生产部门的分布,输入和输出,全年的生产和消费,商品价格等等开始。

从实在和具体开始,从现实的前提开始,因而,例如在经济学上从做为全部社会生产行为的基础和主体开始,似乎是正确的。但是,更仔细地考察起来,这是错误的。如果我抛开构成人口的阶级,人口就是一个抽象。如果我不知道这些阶级所依据的因素,如雇佣劳动,资本等等,阶级又是一句空话。而这些因素是以交换,分工,价格等等为前提的。比如资本,如果没有雇佣劳动,价值,货币,价格等等,它就什么也不是。因此,如果我从人口着手,那末这就是一个混沌的关于整体的表象,经过更切进的规定之后,我就会在分析中达到越来越简单的概念;从表象中的具体达到越来越稀薄的抽象,直到我达到一些最简单的规定。于是行程又得从那里回过头来,直到我最后又回到人口,但是这回人口已不是一个混沌的关于整体的表象,而是一个具有许多规定和关系的丰富的总体了。第一条道路是经济学在它产生时期在历史上走的道路。例如,十七世纪的经济学家总是从生动的整体,从人口,民族,国家,若干国家等等开始;但是他们最后总是从分析中找出一些具有决定意义的抽象的一般的关系,如分工,货币,价值等等。这些个别要素一旦多少确定下来和抽象出来,从劳动,分工,需要交换价值等等这些简单的东西上升到国家,国际交换和世界市场的各种经济学体系就开始出现了。后一种显然是科学上正确的方法。具体之所以具体,因为它是许多规定的综合,因而是多样性的统一。因此它在思维中表现为综合的过程,表现为结果,而不是表现为起点,虽然它是现实中的起点,因而也是直观和表象的起点。在第一条道路上,完整的表象蒸发为抽象的规定;在第二条道路上,抽象的规定在思维行程中导致具体的再现。因而黑格尔陷入幻觉,把实在理解为自我综合,自我深化和自我运动的思维的结果,其实,从抽象上升到具体的方法,只是思维用来掌握具体并把它当作一个精神上的具体再现出来的方式。但决不是具体本身的产生过程。举例来说,最简单的经济范畴,如交换价值,是以人口,以在一定关系中进行生产的人口为前提的;也是以某种形式的家庭,公社或国家等为前提的。它只能做为一个既与的,具体的,生动的整体的抽象片面的关系而存在。相反,做为范畴,交换价值却有一种洪水期前的存在。因此,在意识看来-而哲学意识就是被这样规定的:在它看来,正在理解着的思维是现实的人,因而,被理解的世界本身才是现实的世界-范畴的运动表现为现实的生产行为(只可惜它从外界取得一种推动),而世界是这种生产行为的结果;这-不过又是一个同义反复-只有在下面这个限度内才是正确的:具体总体做为思维总体,做为思维具体,事实上是思维的,理解的产物;但是,决不是处于直观和表象之外或驾乎其上而思维着的,自我产生着的概念的产物,而是把直观和表象加工成概念这一过程的产物。整体,当它在头脑中作为被思维的整体而出现时,是思维着的头脑的产物,这个头脑用它所专有的方式掌握世界,而这种方式是不同于对世界的艺术的,宗教的,实践-精神的掌握的。实在主体仍然是在头脑之外保持着它的独立性;只要这个头脑还仅仅是思辨地,理论地活动着。因此,就是在理论方法上,主体,即社会,也一定要经常作为前提浮现在表象面前。

但是,这些简单的范畴在比较具体的范畴以前是否也有一种独立的历史存在或自然存在呢?要看情况而定。比如,黑格尔论法哲学,是从主体的最简单的法的关系即占有开始的,这是对的。但是,在家庭或主奴关系这些具体的多的关系之前,占有并不存在。相反,如果说有这样的家庭和氏族,它们还只是占有,而没有所有权,这倒是对的。所以,这种比较简单的范畴,表现为简单的家庭或氏族的公社在所有权方面的关系。它在比较高级的社会中表现为一个发达的组织的比较简单的关系。但是那个以占有为关系的具体的基础总是前提。可以设想一个孤独的野人占有东西,但是在这种情况下,占有并不是法的关系。说占有在历史上发展为家庭,是错误的。占有倒总是以这个”比较具体的法的范畴”为前提的。但是,不管怎样总可以说,简单范畴是这样一些关系的表现,在这些关系中,不发展的具体可以已经实现,而那些通过较具体的范畴在精神上表现出来的较多方面的联系和关系还没有产生;而比较发展的具体则把这个范畴当作一种从属关系保存下来。在资本存在之前,银行存在之前,雇佣劳动存在之前,货币能够存在,而且在历史上存在过。因此,从这一方面看来,可以说,比较简单的范畴可以表现一个比较不发展的整体的处于支配地位的关系,或者可以表现一个比较发展的整体的从属关系,后面这些关系,在整体向着一个比较具体的范畴表现出来的方面发展之前,在历史上已经存在。在这个限度内,从最简单上升到复杂这个抽象思维的进程符合现实的历史过程。

另一方面,可以说,有一些十分发展的,但在历史上还不成熟的社会形式,其中有最高级的经济形式,如协作,发达的分工等等,却不存在任何货币,秘鲁就是一个例子。就在斯拉夫公社中,货币以及作为货币的条件的交换,也不是或者很少是出现在个别公社内部,而是出现在它的边界上,出现在与其它公社的交往中,因此,把同一公社内部的交换当作原始构成因素,是完全错误的。相反地,与其说它起初发生在同一公社内部的成员间,不如说它发生在不同公社的相互关系中。其次,虽然货币很早就全面地发生作用,但是在古代它只是片面发展的民族即商业民族中才是处于支配地位的因素。甚至在最文明的古代,在希腊人和罗马人那里,货币的充份发展-在现代的资产阶级社会中这是前提-只是在他们解体的时期。因此,这个十分简单的范畴,在历史上只有在最发达的社会状态下才表现出它的充份的力量。它决没有历尽一切经济关系。例如,在罗马帝国,在它最发达的时期,实物税和实物租仍然是基础。那里,货币制度原来只是在军队中得到充份发展。它也从来没有掌握劳动的整个领域。可见,比较简单的范畴,虽然在历史上可以在比较具体的范畴之前存在,但是,它的充分深入而广泛的发展恰恰只能属于一个复杂的社会形式,而比较具体的范畴在一个比较不发达的社会形式中有过比较充份的发展。

劳动似乎是一个十分简单的范畴。它在这种一般性-作为劳动一般-上的表象也是古老的。但是,在经济学上从这种简单性上来把握的”劳动”,和产生这个简单抽象的那些关系一样,是现代的范畴。例如,货币主义把财富看成还是完全客观的东西,看成存在于货币中的物。同这个观点相比,重工主义或重商主义把财富的源泉从对象转到主体的活动-商业劳动和工业劳动,已经是很大的进步,但是,他们仍然只是局限地把这种活动本身理解为取得货币的活动。同这个学派相对立的重农学派把劳动的一定形式-农业-看作创造财富的劳动,不再把对象本身看做裹在货币的外衣之中,而是看做产品一般,看做劳动的一般成果了。这种产品还与活动的局限性相应而仍然被看做自然规定的产品-农业的产品,主要还是土地的产品。

亚当.斯密大大地前进了一步,他抛开了创造财富的活动的一切规定性,-干脆就是劳动,既不是工业劳动,又不是商业劳动,也不是农业劳动,而既是这种劳动,又是那种劳动,有了创造财富的活动的抽象一般性,也就有了被规定为财富的对象的一般性,这就是产品一般,或者说又是劳动一般,然而是作为过去的,物化的劳动。这一步跨得多么艰难,多么远,只要看看连亚当.斯密本人还时时要回到重农学派的观点上去,就可想见了。这会造成一种看法,好象由此只是替人-不论在哪种社会形式下-做为生产者在其中出现的那种最简单,最原始的关系找到了一个抽象表现。从这一方面来看这是对的,从另一方面看来就不是这样。

对任何种类劳动的同样看待,以一个十分发达的实在劳动种类的总体为前提,在这些劳动种类中,任何一种劳动都不再是支配一切的劳动。所以,最一般的抽象只产生在最丰富的具体的发展的地方,在那里,一种东西为许多东西所共有,为一切所共有。这样一来,它就不再只是再特殊形式上才能加以思考了。另一方面,劳动一般这个抽象,不仅仅是具体的劳动总体的精神结果。对任何种类的劳动的同样看待,适合于这样一种社会形式,在这种社会形式中,个人很容易从一种劳动转到另一种劳动,一定种类的劳动对他们来说是偶然的,因而是无差别的。这里,劳动不仅在范畴上,而且在现实中都是创造财富一般的手段,它不再是在一种特殊性上同个人结合在一起的规定了。在资产阶级社会的最现代的存在形式-美国,这种情况最为发达。所以,在这里,”劳动”,”劳动一般”,直截了当的劳动这个范畴的抽象,这个现代经济学的起点,才成为实际真实的东西。人们也许会说,在美国表现为历史产物的东西-对任何劳动同样看待-在俄罗斯人那里,比如说,就表现为天生的素质了。但是,首先,是野蛮人具有适应一切的素质还是文明人自动去适应一切,是大有区别的。并且,在俄罗斯人那里,实际上同对任何种类劳动同样看待这一点相适应的,是传统地固定在一种十分确定的劳动上的状态,他们只是由于外来的影响才从这种状态中解放出来。

劳动这个例子确切地表明,哪怕是最抽象的范畴,虽然正是由于它们的抽象而适用于一切时代,但是就这个抽象的规定性本身来说,同样是历史关系的产物,而且只有对这于些关系并在这些关系之内才具有充份的意义。

资产阶级社会是历史上最发达的和最复杂的生产组织。因此,那些表现它的各种关系的范畴以及对于它的结构的理解,同时也能使我们透视一切已经覆灭的社会形式的结构和生产关系。资产阶级借这些社会形式的残片和因素建立起来,其中一部分是还未克服的遗物,继续在这里存留着,一部分原来只是征兆的东西,发展到具有充份意义,等等。人体解剖对于猴类解剖是一把钥匙。低等动物身上表露的高等动物的征兆,反而只有在高等动物本身已被认识之后才能理解。因此,资产阶级经济为古代经济等等提供了钥匙,但是,决不是像那些抹杀一切历史差别,把一切社会形式都看成资产阶级社会形式的经济学家所理解的那样。人们认识了地租,什一税等等。但是不应当把它们等同起来。

其次,因为资产阶级社会本身只是发展的一种对抗的形式,所以,那些早期形式的各种关系,在它里面常常只以十分萎缩的或者漫画式的形式出现。公社所有制就是个例子。因此,如果说资产阶级经济的范畴包含着一种适用于一切其它社会形式的真理这种说法是对的,那末,这也只能在一定意义上来理解。这些范畴可以在发展了的,萎缩的了,漫画式的种种形式上,然而总是在有本质区别的形式上,包含着这些社会形式。所谓的历史发展总是建立在这样的基础上的:最后的形式总是把过去的形式看成是向着自己发展的各个阶段,并且因为它很少而且只是在特定条件下才能够进行自我批判,-这里当然不是指做为崩溃时期出现的那样的历史时期,-所以总是对过去的形式做片面的理解。基督教只有在它的自我批判在一定程度上,所谓在可能范围内准备好时,才有助于对早期神话作客观的理解。同样,资产阶级经济只有在资产阶级社会的自我批判已经开始时,才能理解封建社会,古代社会和东方社会.在资产阶级经济没有把自己神话化而同过去完全等同起来时。它对于前一个社会,即它还得与之直接斗争的封建社会的批判,是与基督教对异教的批判或者新教对旧教的批判相似的。

在研究经济范畴的发展时,正如在研究任何历史科学,社会科学时一样,应当时刻把握住:无论在现实中或在头脑中,主体-这里是现代资产阶级社会-都是既与的;因而范畴表现这一定社会的,这个主体的存在形式,存在规定,常常只是个别的侧面;因此,这个一定社会在科学上也决不是把它当作这样一个社会来谈论的时候才开始存在的。这必须把握住,因为这对于分篇直接具有决定的意义。

例如,从地租开始,从土地所有制开始,似乎是再自然不过的,因为它是同土地结合着的,而土地是一切生产的源泉,并且它又是同农业结合着的,而农业是一切多少固定的社会的最初的生产方式。但是,这是最错误不过的了。在一切社会形式中都有一种一定的生产支配着其它一切生产的地位和影响。这是一种普照的光,一切其它色彩都隐没其中,它使它们的特点变了样。这是一种特殊的以太,它决定着它里面显露出来的一切存在的比重。

以畜牧民族为例(纯粹的渔猎民族还处于真正发展的起点之外)。在他们中间出现一定形式的,即偶然的耕作。土地所有制由此决定了。它是公有的,这种形式依这些民族保持传统的多少而或多或少地遗留下来,斯拉夫人中的公社所有制就是个例子。而在从事定居耕作-这种定居已是一大进步-的民族那里,像在古代社会和封建社会,耕作处于支配地位,那里连工业,工业的组织以及与工业相应的所有制形式都或多或少带着土地所有制的性质;或者像在古代罗马人中那样工业完全附属于耕作;或者像中世纪那样工业在城市中和在城市的各种关系上摹仿着乡村的组织。在中世纪,甚至资本-只要不是纯粹的货币资本-做为传统的手工工具等等,也带着这种土地所有制的性质。

在资产阶级社会中情况则相反。农业越来越变成仅仅是一个工业部门,完全由资本支配。地租也是如此。在土地所有制居于支配地位的一切社会形式中,自然联系还占优势。在资本居于支配地位的社会形式中,社会,历史所创造的因素占优势。不懂资本便不能懂地租。不懂地租却完全可以懂资本。资本是资产阶级社会的支配一切的经济权力。它必须成为起点又成为终点,必须放在土地所有制之前来说明。分别考察了两者之后,必须考察它们的相互关系。

因此,把经济范畴按它们在历史上起作用的先后次序来安排是不行的,错误的。它们的次序倒是由他们在现代资产阶级社会中的相互关系决定的,这种关系同看来是它们的合乎自然次序或者符合历史发展次序的东西恰好相反。问题不在于各种经济关系在不同社会形式的相继更替的序列中在历史上占有什么地位,更不在于它们在“观念上”(蒲鲁东)(在历史运动的一个模糊表象中)的次序。而在于它们在现代资产阶级社会内部的结构。

古代世界中的商业民族-腓尼基人,迦太基人-表现的单纯性(抽象规定性);正是由农业民族占优势这种情况决定的。做为商业资本和货币资本的资本,在资本还没有成为社会的支配因素的地方,正是在这种抽象中表现出来。伦巴第人和犹太人对于经营农业的中世纪社会,也是处于这种地位。

还有一个例子,说明同一些范畴在不同的社会阶段有不同的地位,这就是资产阶级社会的最新形式之一:股份公司。但是,它还在资产阶级社会初期就曾以特权的,有垄断权的大公司的形式出现。

国民财富这个概念,在十七世纪经济学家看来,无形中是说财富的创造仅仅是为了国家,而国家的实力是与这种财富成比例的,-这种观念在十八世纪的经济学家中还部份地保留着。这是一种不自觉的伪善形式,在这种形式下财富本身和财富的生产被宣布为现代国家的目的,而现代国家被看成只是生产财富的手段。

显然,应当这样来分篇:

(1)一般的抽象的规定,因此它们或多或少属于一切社会形式,不过是在上面所分析过的意义上。

(2)形成资产阶级社会内部结构并且成为基本阶级的依据的范畴。资本,雇佣劳动,土地所有制。它们相互之间的关系。城市和乡村。三大社会阶级。它们之间的交换。流通。信用事业(私的)。

(3)资产阶级社会在国家形式上的概括。就它本身来考察。”非生产”阶级。税。国债。公的信用。人口。殖民地。向外国移民。

(4)生产的国际关系。国际分工,国际交换。输出和输入。汇率。

(5)世界市场和危机。

\subsubsection{4~生产、生产资料和生产关系。生产关系和交往关系。国家形式和意识形式同生产关系和交往关系的关系。法的关系,家庭关系。}

注意:应该在这里提到而不该忘记的各点:

(1)战争比和平发达的早;某些经济关系,如雇佣劳动,机器等等,怎样在战争和军队等等中比在资产阶级社会内部发展的早。生产力和交往关系的关系在军队中也特别显着。

(2)历来的观念的历史编纂法同现实的历史编纂法的关系。特别是所谓文化史,旧时的宗教使和政治史。(顺便也可以说一下历来的历史编纂法的各种不同方式。所谓客观的,主观的(伦理的等等)。哲学的。)

(3)第二级的和第三级的东西,总之,派生的,转移来的,非原生的生产关系。国际关系在这里的影响。

(4)对这种见解中的唯物主义的种种非难;同自然唯物主义的关系。

(5)生产力(生产资料)的概念和生产关系的概念的辨证法,这样一种辨证法,它的界限应当确定,它不抹杀现实差别。

(6)物质生产的发展例如同艺术生产的不平衡关系。进步这个概念决不能在通常的抽象意义上去理解。现代艺术等等。这种不平衡在理解上还不是像在实际社会关系本身内部那样如此重要和如此困难。例如教育。美国同欧洲的关系。可是,这里要说明的真正困难之点是:生产关系作为法的关系怎样进入了不平衡的发展。例如罗马私法(在刑法和公法中这种情形较少)同现代生产的关系。

(7)这种见解表现为必然的发展。但承认偶然。怎样。(对自由等也是如此。)(交通工具的影响。世界史不是过去一直存在的;作为世界史的历史是结果。)

(8)出发点当然是自然规定性;主观地和客观地。部落,种族等。

关于艺术,大家知道,它的一定繁盛时期决不是同社会的一般发展成比例的,因而也决不是同彷佛是社会组织的骨骼的物质基础的一般发展成比例的。例如,拿希腊人或莎士比亚同现代人相比。就某些艺术形式,例如史诗来说,甚至谁都承认:当艺术生产一旦作为艺术生产出现,他们就再不能以那种在世界史上画时代的,古典的形式创造出来;因此,在艺术本身的领域内,某些有重大意义的艺术形式只有在艺术发展的不发达阶段上才是可能的。如果说在艺术本身的领域内部的不同艺术种类的关系中有这种情形,那末,在整个艺术领域同社会一般发展的关系上有这情形,就不足为奇了。困难只在于对于这些矛盾作一般的表述。一旦它们的特殊性被确定了,它们也就被解释明白了。

我们先拿希腊艺术同现代的关系作例子,然后再说莎士比亚同现代的关系。大家知道,希腊神话不只是希腊艺术的武库,而且是它的土壤。成为希腊人的幻想的基础,从而成为希腊[神话]的基础的那种对自然的观点和对社会关系的观点,能够同自动纺机,铁道,机车和电报并存吗?在罗伯茨公司面前,武尔坎又在哪里?在避雷针面前,邱必特又在哪里?在动产信用公司面前,海尔梅斯又在哪里?任何神话都是用想象和借助想象以征服自然力,支配自然力,把自然力加以形象化;因而,随着这些自然力之实际上被支配,神话也就消失了。在印刷所广场旁边,法玛还成什么?希腊艺术的前提是希腊神话,也就是已经通过人民的幻想用一种不自觉的艺术方式加工过的自然和社会形式本身。这是希腊艺术的素材。不是随便一种神话,就是说,不是对自然(这里指一切对象,包括社会在内)的随便一种不自觉的艺术加工。埃及神话决不能成为希腊艺术的土壤和母胎。但是无论如何总得是一种神话。因此,决不是这样一种社会发展,这种发展排斥一切神话地对待自然的态度和一切把自然神话化的态度;并因而要求艺术家具备一种与神话无关的幻想。

从另一方面看:阿基利斯能同火药和弹丸并存吗?或者,《伊利亚特》能够同活字盘甚至印刷机并存吗?随着印刷机的出现,歌谣,传说和诗神谬斯岂不是必然要绝迹,因而史诗的必要条件岂不是要消失吗?

但是,困难不在于理解希腊艺术和史诗同一定社会发展形式结合在一起。困难的是,他们何以仍然能够给我们以艺术享受,而且就某方面说还是一种规范和高不可及的范本。

一个成人不能再变成儿童,否则就变得稚气了。但是,儿童的天真不使它感到愉快吗?他自己不该努力在一个更高的阶梯上把自己的真实再现出来吗?在每一个时代,它的固有的性格不是在儿童的天性中纯真地复活着吗?为什么历史上的人类童年时代,在它发展的最完美的地方,不该作为永不复返的阶段而显示出永久的魅力呢?有粗野的儿童,有早熟的儿童。古代民族中有许多是属于这一类的。希腊人是正常的儿童。他们的艺术对我们所产生的魅力,同它在其中生长的那个不发达的社会并不矛盾。它倒是这个社会阶段的结果,并且是同它在其中产生而且只能在其中产生的那些未成熟的社会条件永远不能复返这一点分不开的。

\newpage


\end{document}
